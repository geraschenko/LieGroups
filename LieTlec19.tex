 \stepcounter{lecture}
 \setcounter{lecture}{19}
 \sektion{Lecture 19 - The Weyl character formula}

 If $\lambda\in P^+$ (i.e.\ $(\lambda, \check \alpha_i)\in \ZZ_{\ge 0}$ for all $i$),
 then we can construct an irreducible representation with highest weight $\lambda$,
 which we called $V(\lambda)$. We define the \emph{fundamental
 weights}\index{weight!fundamental|idxbf} $\w_1,\dots, \w_n$ of a Lie algebra to be
 those weights for which $(\w_i,\check \alpha_j)=\delta_{ij}$. It is clear that any
 dominant integral weight can be written as $\lambda = \lambda_1\w_1+\cdots +
 \lambda_n \w_n$ for $\lambda_i\ge 0$, so people often talk about $V(\lambda)$ by
 drawing the Dynkin diagram with the the $i$-th vertex labelled by $\lambda_i$.

 With this notation, the first fundamental representation $V(\w_1)$ for $\sl(n)$ is written
  $\begin{xy}
   (0,-.15) *+!D{1} *\cir<2pt>{};
   p+(1,0) *+!D{0} *\cir<2pt>{} **@{-};
   p+(.5,0) **@{-};
   p+(.6,0) **{\hspace{1pt}.\hspace{1pt}};
   p+(.5,0) *+!D{0} *\cir<2pt>{} **@{-};
   p+(1,0) *+!D{0} *\cir<2pt>{} **@{-};
 \end{xy}$, which happens to be the standard representation
 (see Example \ref{lec19Eg:Fund_sl(n+1)} below).
 Similarly, the adjoint representation is
 $\begin{xy}
   (0,-.15) *+!D{1} *\cir<2pt>{};
   p+(1,0) *+!D{0} *\cir<2pt>{} **@{-};
   p+(.5,0) **@{-};
   p+(.6,0) **{\hspace{1pt}.\hspace{1pt}};
   p+(.5,0) *+!D{0} *\cir<2pt>{} **@{-};
   p+(1,0) *+!D{1} *\cir<2pt>{} **@{-};
 \end{xy}$.
 \begin{warning}
   Another common notation (incompatible with this one) is to write $\lambda = \sum
   k_i\alpha_i$ and label the $i$-th vertex by $k_i$. In this notation, the standard
   representation is
   $\begin{xy}
   (0,0) *+{1};
   p+(1,0) *+{0} **@{-};
   p+(.5,0) **@{-};
   p+(.6,0) **{\hspace{1pt}.\hspace{1pt}};
   p+(.5,0) *+{0} **@{-};
   p+(1,0) *+{0} **@{-};
 \end{xy}$
 and the adjoint representation is
 $\begin{xy}
   (0,0) *+{1};
   p+(1,0) *+{1} **@{-};
   p+(.5,0) **@{-};
   p+(.6,0) **{\hspace{1pt}.\hspace{1pt}};
   p+(.5,0) *+{1} **@{-};
   p+(1,0) *+{1} **@{-};
 \end{xy}$. In these notes, we will draw the diagram differently to distinguish
 between the two notations.
 \end{warning}

 Observe that if $v\in V$ a highest vector of weight $\lambda$, and $w\in W$ another
 highest weight vector of weight $\mu$ in another representation, then $v\otimes w\in
 V\otimes W$ is a highest weight vector of weight $\lambda+\mu$. It follows that every
 finite dimensional irreducible representation can be realized as a subrepresentation
 of a tensor product of fundamental representations.
 \begin{example}\label{lec19Eg:Fund_sl(n+1)}\index{sl(n)@$\sl(n)$|idxit}
   Let's calculate the fundamental weights for $\sl(n+1)$. Recall that we have simple
   roots $\e_1-\e_2, \dots, \e_n-\e_{n+1}$, and they are equal to their coroots (since
   they have length $\sqrt 2$). It follows that $\w_i=\e_1+\cdots +\e_i$ for
   $i=1,\dots, n$.

   Let $E$ be the standard $(n+1)$-dimensional representation of $\sl(n+1)$. Let
   $e_1,\dots, e_{n+1}$ be a basis for $E$. Note that $e_i$ has weight $\e_i$, and
   $\e_i-\e_j$ can be written as a non-negative sum of positive roots exactly when
   $i\le j$. Thus, the weights of $E$, in decreasing order, are $\e_1$, $\e_2$, \dots,
   $\e_{n+1}$.

   Consider the representation $\Lambda^k E$. We'd like to write down its weights.
   Note that $\Lambda^k E$ is spanned by the vectors $e_{i_1}\wedge \cdots \wedge
   e_{i_k}$, which have weights $\e_{i_1}+\cdots +\e_{i_k}$. Thus, the highest weight
   is $\e_1+\cdots + \e_k=\w_k$, so we know that $V(\w_k)\subseteq \Lambda^k E$.

   Note also that $\weyl \cong S_{n+1}$ acts by permutation of the $e_i$, so it can
   take any weight space to any other weight space. Such a representation (where all
   the weight spaces form a single orbit of the Weyl group) is called
   \emph{minuscule}\index{minuscule representation|idxbf}. Since the character of any
   subrepresentation must be $\weyl$-invariant, minuscule representations are always
   irreducible. So $\Lambda^k E= V(\w_k)$ is a fundamental representation.
 \end{example}
 \begin{remark}[Highest weights of duals]
   One of the weights of $V(\lambda)^*$ is $-\lambda$, but to compute the highest
   weight, we need to get back into $P^+$, so we apply the longest word $w$ in the
   Weyl group. Thus, $-w(\lambda)$ is the highest weight of $V(\lambda)^*$. This means
   that there is a fixed involution of the Weyl chamber (namely, $-w$) which takes the
   highest weight of a representation to the highest weight of its dual. It is clear
   that $-w$ preserves the set of simple roots and preserves inner products, so it
   corresponds to an involution of the Dynkin diagram.

   In the case of $\sl(n+1)$, the involution is
   \begin{xy}
     (0,0)="1" *\cir<2pt>{};
     (1,0)="2"  *\cir<2pt>{} **@{-};
     p+(.6,0) **@{-};
     p+(.8,0) **{\,.\,};
     p+(.6,0)="22" *\cir<2pt>{} **@{-};
     p+(1,0)="11" *\cir<2pt>{} **@{-};
     "1" *+{\ };"11" *+{\ } **\crv{(2,.5)} ?<*@{<} ?>*@{>},
     "2" *+{\ };"22" *+{\ } **\crv{(2,.3)} ?<*@{<} ?>*@{>},
   \end{xy}. In particular, the dual of the standard representation $V(\w_1)$ is
   $V(\w_n)$.
 \end{remark}
% \begin{exercise}
%   Compute these involutions for the other simple Lie algebras.
%   \begin{solution}
%     $B_n$ and $C_n$ have no non-trivial diagram involutions, so
%   \end{solution}
% \end{exercise}

% How do you calculate $ch\, V$? Recall that $ch\, V = \sum m_{\mu} e^{\mu}$, where
% $m_\mu$ is the multiplicity of the weight $\mu$. You can do it with Verma modules.
% Remember that for Verma modules, you still get a weight decomposition.

 The key to computing the character of $V(\lambda)$ is to write it as a linear
 combination of characters of Verma modules, as in the following example.

 \index{Weyl character formula|(idxbf}
 \begin{example} \label{lec19Eg:2w1+w2}
   Let $\g = \sl(3)$ and let $\lambda = 2\w_1+\w_2$. We try to write $ch\, V(\lambda)$
   as a linear combination of characters of Verma modules in the na\"\i ve way. We
   know  that $M(\lambda)$ must appear once and that $ch\, V(\lambda)$ must end up
   symmetric with respect to the Weyl group. We must subtract off two Verma modules to
   keep the symmetry. Then we find that we must add back two more and subtract one in
   order to get zeros outside of the hexagon. In the picture below, each dot can be
   read as a zero.
 \[\newcommand\buff{.2} \newcommand\cbuff{.8}
   \renewcommand\latticebody{\drop{\raisebox{-8.6pt}{\kern-.1pt \mbox{\LARGE $\cdot$}}}}
   \begin{xy}<2em,0em>:a(30):a(60)::
     (6,0);(-8,0) **@{--},
     (0,5);(0,-7) **@{--},
     (6,-6);(-8,8) **@{--},
     {(2\buff,1\buff);p+(-1,.5):a(120)::
      (10\buff,0);(0,0) **@{-};(0,8\buff) **@{-}},
     0 *{\gdef\buff{.15} \gdef\cbuff{.85}},
     {(4\buff,-2\cbuff);p+(-1,.5):a(120)::
      (12\buff,0);(0,0) **@{-};(0,4\buff) **@{-}},
     {(-3\cbuff,4\buff);p+(-1,.5):a(120)::
      ( 4\buff,0);(0,0) **@{-};(0,12\buff) **@{-}},
     0 *{\gdef\buff{.1} \gdef\cbuff{.9}},
     {(-5\cbuff,2\buff);p+(-1,.5):a(120)::
      ( 2\buff,0);(0,0) **@{-};(0,8\buff) **@{-}},
     {(1\buff,-5\cbuff);p+(-1,.5):a(120)::
      ( 9\buff,0);(0,0) **@{-};(0,2\buff) **@{-}},
     0 *{\gdef\buff{.15} \gdef\cbuff{.85}},
     {(-2\cbuff,-3\cbuff);p+(-1,.5):a(120)::
      ( 5\buff,0);(0,0) **@{-};(0,2\buff) **@{-}},
     {0;0 \croplattice{-10}{10}{-10}{10} {-8.1}{6.1}{-7.1}{5.1}},
     @={(2,1),(-2,3),(-3,2),(-1,-2),(1,-3),(3,-1),(0,2),(-2,0),(2,-2)}
     @@{*{\psframe*[linecolor=white](-.1,-.15)(.1,.15)} *{1}},
     @i @={(1,0),(0,-1),(-1,1)}
     @@{*{\psframe*[linecolor=white](-.15,-.17)(.15,.17)} *{2}},
%      (0,2)  *{\psframe*[linecolor=white](-.05,-.14)(.05,.141)} *{1},
%      (-2,0) *{\pspolygon*[linecolor=white](.05,-.05)(.05,.05)(.03,.15)(-.08,.15)(-.08,-.05)} *{1},
%     (2,-2)  *{\psframe*[linecolor=white](-.07,-.14)(.07,.14)} *{1},
     @i @={(2,1),(-4,4),(-6,2),(-3,-4),(1,-6),(4,-3)}
     @@{*+++{\,};p+(1,1) *+\cir<4pt>{} **@{-}, {?<>(1)*@{>}},},
     @i @={(-4,4),(-3,-4),(4,-3)}
     @@{*!UR{-}}
     @i @={(-6,2),(1,-6)}
     @@{ *!UR{+}}
   \end{xy}
 \]
 For now, just observe that if we shift the weights that appear (by something we will
 call the Weyl vector), we get an orbit of the Weyl group, with signs alternating
 according to the length of the element of the Weyl group.
 \end{example}
% \[\def\buff{.3} \def\cbuff{.7}
%    \begin{xy}<-2.5em,0em>:a(120)::
%    (0,0) *+{1};(0,1) *+{1}, (0,2) *+{0} *\cir{};(0,3) *+{0};(0,4) *+{0};
%    (1,0) *+{1};(1,1) *+{2};(1,2) *+{1};(1,3) *+{0};(1,4) *+{0};
%    (2,0) *+{1};(2,1) *+{2};(2,2) *+{2};(2,3) *+{1};(2,4) *+{0};
%    (3,0) *+{0} *\cir{};(3,1) *+{1};(3,2) *+{1};(3,3) *+{1};(3,4) *+{0};
%    (4,0) *+{0};(4,1) *+{0};(4,2) *+{0};(4,3) *+{0};(4,4) *+{0};
%    @={(-\buff,-\buff),(2,-\buff),(3\buff,1),(3\buff,3\buff),(2,3\buff),(-\buff,1)},
%    s0="prev" @@{;"prev";**@{-}="prev"},
%    (4,-\buff); (2\cbuff,-\buff) **@{-}; (2\cbuff,4) **@{-};
%    (4,1\cbuff); (-\buff,1\cbuff) **@{-}; (-\buff,4\buff) **@{-};
% \end{xy}
% \]
% \mpar{Explain this picture better}
%   In the quotient, the shown weights disappear. What about the one below it? Its
%   multiplicity in the Verma module is 2. The multiplicity in the Verma module
%   centered at $2\w_1$ is 1, so in the quotient, you get multiplicity 1. How about
%   right below it? You had 3 from the big Verma module, and you remove 1 and 1, so you
%   get 1 left. Furthur down and to the left, you get all zeros. Similarly, we can fill
%   in the other numbers. You get the leftover hexagon.

 Some notation: if $w\in \weyl$, we define $(-1)^w := \det(w)$. Since each simple
 reflection has determinant $-1$, this is the same as $(-1)^{\text{length}(w)}$. Note
 that $(-1)^{w'w}=(-1)^{w'}(-1)^w$.

 The \emph{Weyl vector}\index{Weyl vector|idxbf} is $\rho = \frac{1}{2} \sum_{\alpha
 \in \Delta^+}\alpha$. Note that $r_i(\rho) = \rho-\alpha_i$ by Lemma
 \ref{lec14L:key}. On  the other hand, $r_i(\rho)= \rho - (\rho,\check
 \alpha_i)\alpha_i$, so we know that $(\rho,\check \alpha_i)=1$ for all $i$. Thus,
 $\rho$ is the sum of all the fundamental weights.

 \begin{theorem}[Weyl Character Formula]
   For $\lambda\in P^+$, the character of the irreducible finite dimensional
   representation with highest weight $\lambda$ is\footnote{ This formula may look
  ugly, but it is \emph{sweet}. It says that you can compute the character of
  $V(\lambda)$ in the following way. Translate the Weyl vector $\rho$ around by the
  Weyl group; this will form some polytope. Make a piece of cardboard shaped like this
  polytope (ok, so maybe this is only practical for rank 2), and put $(-1)^w$ at the
  vertex $w(\rho)$. This is your \emph{cardboard denominator}.\index{cardboard
  denominator|see{Weyl denominator}} Now the formula tells you that when you center
  your cardboard denominator around any weight, and then add the multiplicities of the
  weights of $V(\lambda)$ at the vertices with the appropriate sign, you'll get zero
  (unless you centered your cardboard
  \begin{window}[1,r,%
      {\begin{xy}<-1.75em,0em>:a(120)::
       (-1,0)*+{0};(-1,1)*+{0};(0,2) *+{0};
       (0,0) *+{1};(0,1) *+{1};
       (1,0) *+{1};(1,1) *+{?};(1,2) *+{1};
       (2,0) *+{1};(2,1) *+{?};(2,2) *+{?};(2,3) *+{1};
       (3,0)      ;(3,1) *+{1};(3,2) *+{1};(3,3) *+{1};
       @={(0,0),(1,1),(1,2),(0,2),(-1,1),(-1,0)},
       s0="prev" @@{;"prev" *=<2.5mm>{\ }; *=<2.5mm>{\ } **@{-}="prev"},
%       (1.65,1.35);(-1,0) **@{.}, (1,0) **@{.}
     \end{xy}},]
  \noindent at $w(\lambda+\rho)$, in which case only one non-zero multiplicity shows
  up in the sum, so you'll get $\pm 1$). Since we know that the highest weight has
  multiplicity 1, we can use this to compute the rest of the character.\\
  For $\sl(3)$, your cardboard denominator will be a hexagon, and one step of
  computing the character of $V_{2\w_1+\w_2}$ might look like:\\
  $0=0-1+\,?-1+0-0$, so $?=2$. Since $ch\, V$ is symmetric with respect to $\weyl$,
  all three of the $?$s must be 2.
  \end{window}}
 \[
  ch\, V(\lambda) = \frac{\sum_{w\in \weyl} (-1)^w e^{w(\lambda+\rho)}}{\sum_{w\in \weyl}
    (-1)^w e^{w(\rho)}}.
 \]
 \end{theorem}
 The denominator is called the \emph{Weyl denominator}\index{Weyl denominator|idxbf}.
 It is not yet obvious that the Weyl denominator divides the numerator (as formal
 sums), so one may prefer to rewrite the equation as $  ch\, V(\lambda)\cdot
 \sum_{w\in \weyl}(-1)^w e^{w(\rho)} = \sum_{w\in \weyl} (-1)^w e^{w(\lambda+\rho)}$.
 \begin{proof}
 %\begin{trivlist}\item \end{trivlist}
  \underline{Step 1. Compute $ch\,M(\gamma)$}: Recall from the previous lecture that
  the multiplicity of $\mu$ in $M(\gamma)$ is the number of ways $\gamma-\mu$ can be
  written as a sum of positive roots. Thus, it is easy to see that $ch\,M(\gamma)$
  is given by the following generating function.
  \begin{align*}
    ch\,M(\gamma) &= e^\gamma \prod_{\alpha\in
    \Delta^+}(1+e^{-\alpha}+e^{-2\alpha}+\cdots)\\
    &= \frac{e^\gamma}{\prod_{\alpha\in \Delta^+}(1-e^{-\alpha})} \\
    &= \frac{e^{\gamma+\rho}}{\prod_{\alpha\in \Delta^+}(e^{\alpha/2}-e^{-\alpha/2})}
    & (\textstyle\prod_{\alpha\in \Delta^+}e^{\alpha/2} = e^\rho )
  \end{align*}

  \underline{Step 2. The action of the Casimir operator}: Recall the Casimir operator
  \index{Casimir operator|idxit} from the proof of Whitehead's Theorem (Theorem
  \ref{lec12Whitehead}). If $\{e_i\}$ is a basis for $\g$, and $\{f_i\}$ is the dual
  basis (with respect to the Killing form), then $\W = \sum e_if_i\in U\g$. We showed
  that $\W$ is in the center of $U\g$ (i.e.\ that $\W x = x\W$ for all $x\in \g$).
  \begin{claim}
    $\W$ acts on $M(\gamma)$ as $(\gamma,\gamma+2\rho)\id$.
  \end{claim}
  \begin{proof}[Proof of Claim]  \renewcommand\qedsymbol{$\Box_\text{Claim}$}
  Since $\W$ is in the center of $U\g$, it is enough to show that
  $\W v=(\gamma+2\rho,\gamma) v$ for a highest weight vector $v\in V_\gamma$.

  Let $\{u_i\}$ be an orthonormal basis for $\h$, and let $\{X_\alpha\}_{\alpha\in
  \Delta}$ be a basis for the rest of $\g$. The dual basis is
  $\Big\{\frac{Y_\alpha}{(X_\alpha,Y_\alpha)}\Big\}$. Then we get
  \begin{align*}
    \W &= \sum_{i=1}^n u_i^2 + \sum_{\alpha\in \Delta}\frac{X_\alpha
    Y\alpha}{(X_\alpha,Y_\alpha)}\\
    &= \sum_{i=1}^n u_i^2 + \sum_{\alpha\in \Delta^+}\frac{X_\alpha
    Y\alpha}{(X_\alpha,Y_\alpha)}+\frac{Y_\alpha X_\alpha}{(X_\alpha,Y_\alpha)}
    & (X_{-\alpha} Y_{-\alpha} = Y_\alpha X_\alpha) .
  \end{align*}
  Using the equalities
  \begin{gather*}
   \begin{array}{rl}
     u_i v &\!\!\!= \gamma(u_i) v,\\
     X_\alpha v&\!\!\!=0,\\
     X_\alpha Y_\alpha v &\!\!\!= H_\alpha v - Y_\alpha X_\alpha v \\
       &\!\!\!= \gamma(H_\alpha) v,\\
     \gamma(H_\alpha) &\!\!\!= \frac{2(\gamma,\alpha)}{(\alpha,\alpha)},
   \end{array}\qquad
   \begin{array}{rl}
     (\gamma,\gamma) &\!\!\!= \sum_{i=1}^n \gamma(u_i)^2,\text{ and}\\
     (X_\alpha,Y_\alpha)&\!\!\!= \half ([H_\alpha,X_\alpha],Y_\alpha) \\
     &\!\!\!= \half(H_\alpha,[X_\alpha,Y_\alpha])\\
     &\!\!\!= \half (H_\alpha, H_\alpha) = \half \cdot \frac{2\alpha(H_\alpha)}{(\alpha,\alpha)}\\
     &\!\!\!= \frac{2}{(\alpha,\alpha)}
   \end{array}
  \end{gather*}
  we get
  \begin{align*}
    \W v &= \Bigl(\sum_{i=1}^n \gamma(u_i)^2\Bigr) v + \sum_{\alpha\in \Delta^+}
    \frac{\gamma(H_\alpha)}{(X_\alpha,Y_\alpha)} v\\
    &= (\gamma,\gamma) v + \sum_{\alpha\in \Delta^+}(\gamma,\alpha)v
    = (\gamma,\gamma+2\rho) v \qedhere
  \end{align*}
  \end{proof}

  Note that the universal property of Verma modules implies that the action of $\W$ on
  any representation generated by a highest vector of weight $\gamma$ is given by
  $(\gamma,\gamma+2\rho)\id$.
  \smallskip

  Finally, consider the set
  \[
    \W^\gamma = \{\mu\in P| (\mu+\rho,\mu+\rho) = (\gamma+\rho,\gamma+\rho)\}.
  \]
  This is the intersection of the weight lattice $P$ with the sphere of radius
  $\|\gamma+\rho\|$ centered at $-\rho$. In particular, it is a \emph{finite set}. On
  the other hand, since $(\gamma,\gamma+2\rho) =
  (\gamma+\rho,\gamma+\rho)-(\rho,\rho)$, it is also the set of weights $\mu$ such
  that $\W$ acts on $M(\mu)$ in the same way it acts on $M(\gamma)$.

  \underline{Step 3. Filter $M(\gamma)$ for another formula}: We say that a weight
  vector $v$ is a \emph{singular vector} if $\n^+v=0$. If a representation is
  generated by some highest vector $v$, and if all singular vectors are proportional to
  $v$, then the representation is irreducible. To see this, note that a highest weight
  vector of any proper subrepresentation must be singular, and it cannot be
  proportional to $v$, lest it generate the whole representation.

  Now let $w$ be a singular vector of weight $\mu$ in $M(\gamma)$. Then $w$ generates
  a subrepresentation which is a quotient of $M(\mu)$. By the claim in Step 2, $\W$
  acts on this subrepresentation by $(\mu,\mu+2\rho)$. On the other hand, since we are
  in $M(\gamma)$, $\W$ must act by $(\gamma,\gamma+2\rho)$. It follows that $\mu\in
  \W^\gamma$.

  In particular, since $\W^\gamma$ is finite, there is a minimal singular vector $w$,
  which generates some irreducible subrepresentation; we will call that representation
  $F_1M(\gamma)$. Mod out my $F_1M(\gamma)$ and repeat the process. Any singular
  vector in $M(\gamma)/F_iM(\gamma)$ must be in $\W^\gamma$, so there is a minimal
  one, $w$, which generates an irreducible subrepresentation. Define
  $F_{i+1}M(\gamma)\subseteq M(\gamma)$ to be the pre-image of that representation.
  Since $\W^\gamma$ is finite and each $V_\mu$ is finite dimensional, the process
  terminates. The result is a filtration
  \[
    0=F_0M(\gamma) \subseteq F_1M(\gamma)\subseteq \cdots \subseteq
    F_kM(\gamma)=M(\gamma)
  \]
  such that $F_iM(\gamma)/F_{i+1}M(\gamma)$ is isomorphic to the irreducible
  representation $V(\mu)$ for some $\mu\in \W^\gamma$.\footnote{We showed in Lecture
  18 that for every $\mu\in \h^*$, there is a unique irreducible representation
  $V(\mu)$ with highest weight $\mu$. However, we only showed that $V(\mu)$ is finite
  dimensional when $\mu\in P^+$. In general, it is infinite dimensional. In fact,
  sometimes it happens that $V(\mu)=M(\mu)$.\anton{Is it true that $V(\mu)=M(\mu)$ if
  and only if for all $i$, $(\mu,\check\alpha_i)\not\in \ZZ$?}} We also know that each
  $\mu$ that appears is less than or equal to $\lambda$.

  This gives us the nice formula
  \[
    ch\, M(\gamma) = \sum_{\makebox[0pt]{$\scriptstyle \mu\le \gamma,\, \mu\in
    \W^\gamma$}} b_{\gamma\mu}  ch\, V(\mu)
  \]
  for some non-negative integers $b_{\gamma\mu}$.\footnote{These $b_{\gamma\mu}$ are
  called \emph{Kazhdan-Luztig multiplicities}\index{Kazhdan-Luztig multiplicities},
  and they are hard to compute for general $\gamma$ and $\mu$.} Moreover, $V(\gamma)$
  appears as a quotient exactly once, so $b_{\gamma\gamma}=1$.

  \underline{Step 4. Invert and simplify the equation}: We've shown that the matrix
  $(b_{\gamma\mu})_{\gamma,\mu\in \W^\lambda}$ is lower triangular with ones on the
  diagonal, so it has a lower triangular inverse $(c_{\gamma\mu})_{\gamma,\mu\in
  \W^\lambda}$ with ones on the diagonal.\footnote{We will prove that each non-zero
  $c_{\gamma\mu}$ is $\pm 1$. It was once conjectured that even if $\lambda\not\in
  P^+$, each non-zero $c_{\gamma\mu}$ is $\pm 1$, but this is false.} This gives us the
  formula
  \[
    ch\, V(\lambda) = \sum_{\makebox[0pt]{$\scriptstyle \mu\le \lambda,\, \mu\in
    \W^\lambda$}} c_{\lambda\mu} ch\, M(\mu).
  \]
  Using Step 1, we can rewrite this as
  \[
    ch\,V(\lambda) \cdot \prod_{\alpha\in \Delta^+}(e^{\alpha/2}-e^{-\alpha/2}) =
    \sum_{\makebox[0pt]{$\scriptstyle \mu\le \lambda,\, \mu\in \W^\lambda$}} c_{\lambda\mu}e^{\mu+\rho}.
  \]
  For any element $w$ of the Weyl group, we know that $w($LHS$)=(-1)^w$LHS, so the
  same must be true of the RHS, i.e.
  \[
    \sum c_{\lambda\mu} e^{w(\mu+\rho)} = \sum (-1)^w c_{\lambda\mu} e^{\mu+\rho}.
  \]
  This is equivalent to the condition $c_{\lambda, w(\mu+\rho)-\rho}=c_{\lambda\mu}$.
  Since $P^+$ is the fundamental domain of $\weyl$, and since $c_{\lambda\lambda}=1$,
  we get
  \[
    ch\,V(\lambda) \cdot \prod_{\alpha\in \Delta^+}(e^{\alpha/2}-e^{-\alpha/2}) =
        \sum_{w\in \weyl}(-1)^w e^{w(\lambda+\rho)} +
        \sum_{\makebox[0pt]{$\substack{\mu< \lambda,\, \mu\in \W^\lambda\\ \mu+\rho
        \in P^+}$}} (-1)^w c_{\lambda\mu} e^{w(\mu+\rho)}.
  \]

  We would like to eliminate the second sum on the right hand side. The following
  claim does that nicely by showing that the sum is empty.
  \begin{claim}
    If  $\mu\le \lambda$, $\mu\in \W^\lambda$, and $\mu+\rho \ge 0$, then $\mu=\lambda$.
  \end{claim}
  \begin{proof}
  We assume that $(\mu+\rho,\mu+\rho)=(\lambda+\rho,\lambda+\rho)$ and $\lambda-\mu =
  \sum_{i=1}^n k_i \alpha_i$ for some non-negative $k_i$. Then we get
  \begin{align*}
    0 &= \big( (\lambda+\rho)-(\mu+\rho),(\lambda+\rho)+(\mu+\rho) \big)\\
      &= (\lambda-\mu, \lambda+\mu+2\rho)\\
      &= \sum_{i=1}^n k_i (\alpha_i, \lambda+\mu+2\rho)
  \end{align*}
  But $\lambda\ge 0$ and $\mu+\rho\ge 0$, so $(\alpha_i,\lambda+\mu+\rho)\ge 0$. Also,
  $(\alpha_i,\rho)>0$ for each $i$, so $(\alpha_i, \lambda+\mu+2\rho)>0$. It follows
  that each $k_i$ is zero.
  \renewcommand{\qedsymbol}{$\Box_\text{Claim}$\quad}
  \end{proof}

  Now we have
  \[
    ch\, V(\lambda) \cdot \prod_{\alpha\in \Delta^+}(e^{\alpha/2}-e^{-\alpha/2}) =
    \sum_{w\in \weyl}(-1)^w e^{w(\lambda+\rho)}.
  \]
  Specializing to the case $\lambda=0$, we know that $V(0)$ is the trivial
  representation, so $ch\, V(0)=1$. This tells us that
  \begin{equation}
    \prod_{\alpha\in \Delta^+}(e^{\alpha/2}-e^{-\alpha/2}) = \sum_{w\in \weyl}(-1)^w
    e^{w(\rho)}, \label{lec19WeylD}
  \end{equation}
  so we get the desired
  \[
    ch\, V(\lambda) =\frac{\sum_{w\in \weyl}(-1)^w e^{w(\lambda+\rho)}}{\sum_{w\in
    \weyl}(-1)^w e^{w(\rho)}}. \qedhere
  \]\index{Verma module|)}
  \end{proof}
 \begin{corollary}[Weyl dimension formula]\index{Weyl dimension formula|idxbf}
   $\displaystyle \dim V(\lambda) = \prod_{\alpha\in \Delta^+} \frac{(\lambda+\rho,
    \alpha)}{(\rho,\alpha)}$.
 \end{corollary}
 \begin{proof}\newcommand\f{6}
   The point is that $e^\mu$ is a formal expression. The only property that we use is
   $e^\mu e^\gamma = e^{\mu+\gamma}$, so everything we've ever done with characters
   works if we replace $e^\mu$ by any other expression satisfying that relation. In
   particular, if replace $e^\mu$ with $\f^{t(\gamma+\rho,\mu)}$, where $t$ is a real
   number,\footnote{Obviously, there is nothing special about the base $\f$; just
   about any number would work. It is important to understand that for any $\mu$,
   $t\mapsto \f^{t(\gamma+\rho,\mu)}$ is an honest real-valued function in $t$.
   Equation \ref{lec19ast} is an equality of \emph{real-valued functions} in $t$! Similarly,
   $ch\, V(\lambda)$ becomes a real-valued function in $t$.} then
   Equation \ref{lec19WeylD} says
   \begin{align}
     \prod_{\alpha\in \Delta^+} \Bigl(\f^{t(\gamma+\rho,\alpha/2)} -
     \f^{-t(\gamma+\rho,\alpha/2)}\Bigr) &=
     \sum_{w\in \weyl}(-1)^w \f^{t(\gamma+\rho,w(\rho))}\notag\\
     &=\sum_{w\in \weyl}(-1)^w \f^{t(w(\gamma+\rho),\rho)} \label{lec19ast}
   \end{align}
   where the second equality is obtained by replacing $w$ by $w^{-1}$ and observing
   that $(x,w^{-1}y)=(w\,x,y)$ and that $(-1)^{w^{-1}}=(-1)^w$.

   Now we switch things up and replace $e^\mu$ by $\f^{t(\mu,\rho)}$, so the character
   formula becomes
   \[
     ch\, V(\lambda) =\frac{\sum_{w\in \weyl}(-1)^w
     \f^{t(w(\lambda+\rho),\rho)}}{\sum_{w\in \weyl}(-1)^w
     \f^{t(w(\rho),\rho)}}.
   \]
   Applying Equation \ref{lec19ast} to the numerator (with $\gamma=\lambda$) and to the
   denominator (with $\gamma=0$), we get
   \[
     ch\, V(\lambda) = \prod_{\alpha\in \Delta^+}
     \frac{\bigl(\f^{t(\lambda+\rho,\alpha/2)} -
     \f^{-t(\lambda+\rho,\alpha/2)}\bigr)}{\bigl(\f^{t(\rho,\alpha/2)} -
     \f^{-t(\rho,\alpha/2)}\bigr)}.
   \]
   The dimension of $V(\lambda)$ is equal to the expression $ch\, V(\lambda)$ with
   $e^\mu$ replaced by $1$. We can obtain this by letting $t$ tend to zero in
   $\f^{t(\mu,\rho)}$. This gives
   \begin{align*}
     \dim V(\lambda) &= \lim_{t\to 0}\prod_{\alpha\in \Delta^+}
     \frac{\bigl(\f^{t(\lambda+\rho,\alpha/2)} -
     \f^{-t(\lambda+\rho,\alpha/2)}\bigr)}{\bigl(\f^{t(\rho,\alpha/2)}
     - \f^{-t(\rho,\alpha/2)}\bigr)}\\
     &= \prod_{\alpha\in \Delta^+}\frac{(\lambda+\rho,\alpha)}{(\rho,\alpha)}. &
     \text{(By l'H\^{o}pital's rule)\qquad} \qedhere
   \end{align*}
 \end{proof}
 \begin{example}
   Let $\g = \sl(n+1)$.\index{sl(n)@$\sl(n)$|idxit} We choose the standard set of
   simple roots $\Pi=\{\alpha_1,\dots, \alpha_n\}$ so that $\Delta^+ =
   \{\alpha_i+\alpha_{i+1}+\cdots+\alpha_j\}_{1\le i\le j\le n}$. Recall that
   $(\rho,\alpha_i)=1$ for $1\le i\le n$ and that $(\w_i,\alpha_j)=\delta_{ij}$. If
   $\lambda+\rho = \sum_{i=1}^n a_i\w_i$, the dimension formula tells us that
   \begin{align*}
     \dim V(\lambda) &= \prod_{\alpha\in \Delta^+}
            \frac{(\lambda+\rho,\alpha)}{(\rho,\alpha)}\\
         &= \prod_{1\le i\le j\le n}\frac{a_i+a_{i+1}+\cdots+a_{j-1}+a_j}{j-i+1}\\
         &= \frac{1}{n!!}\prod_{1\le i\le j\le n} \sum_{k=i}^j a_k
   \end{align*}
   where $n!!:= n!\,(n-1)!\cdots 3!\,2!\,1!$.

   If $\g=\sl(3)$, and if $\lambda+\rho=3\w_1+2\w_2$, we get $\dim V(\lambda) =
   \frac{1}{2!!}\cdot 2\cdot 3\cdot (2+3) =15$, computing the dimension of the
   representation in Example \ref{lec19Eg:2w1+w2}.
   This formula is nice because the calculation does not get big as $\lambda$ gets
   big. If $\lambda+\rho=20\w_1 + 91\w_2$, it would be really annoying to compute
   $ch\, V(\lambda)$ completely, but we can get $\dim V(\lambda) = \half
   20\cdot 91\cdot 111 = 101010$ easily.

   Even for larger $n$, this formula is pretty good. Say we want the dimension of
   $\begin{xy}
    (0,-.15) *+!D{1} *\cir<2pt>{};
    p+(1,0) *+!D{2} *\cir<2pt>{} **@{-};
    p+(1,0) *+!D{0} *\cir<2pt>{} **@{-};
    p+(1,0) *+!D{6} *\cir<2pt>{} **@{-};
  \end{xy}$, then $\lambda+\rho = 2\w_1+ 3\w_2+ 1\w_3+ 7\w_4$, so we get
  {\small \[\frac{1}{4!!}2\cdot 3\cdot 1\cdot 7\cdot
  (2+3)(3+1)(1+7)(2+3+1)(3+1+7)(2+3+1+7)=20020.\]}
 \end{example}
% \begin{exercise}
%   Find similar formulas for the representations of the other classical algebras.
%   \begin{solution}
%     We computed the root systems explicitly in Lecture 15. Choose $\alpha_1,\dots,
%     \alpha_{n-1}$ to look like the simple roots of an $A_{n-1}$, and let $\alpha_n$
%     be the strange simple root. Fix the notation $\lambda+\rho = \sum_{i=1}^n
%     a_i\w_i$.
%
%     \underline{$B_n$}: The positive roots are
%     \begin{gather*}
%     \begin{align*}
%       \{\e_i-\e_j\}_{i<j}\cup\{\e_i\} \cup \{\e_i+\e_j\}_{i<j} &=
%       \{\alpha_i+\cdots+\alpha_{j-1}\}_{i<j}\cup \{\alpha_i+\cdots +\alpha_n\}\cup\\
%            &\cup \{(\alpha_i+\cdots +\alpha_n)+(\alpha_j+\cdots +\alpha_n)\}_{i<j}
%            ,\text{ so}
%     \end{align*}\\
%     \begin{align*}
%       \dim V(\lambda) &= \prod_{1\le i<j\le n} \frac{\sum_{k=i}^{j-1} a_k}{j-i}
%       \times \prod_{1\le i\le n} \frac{\sum_{k=i}^n a_k}{n-i+1}
%       \times \prod_{1\le i<j\le n} \frac{\sum_{k=i}^n a_k + \sum_{k=j}^n a_k}{2n-i-j+2}\\
%       &= \frac{1}{n!!} \Bigl(\prod_{1\le i<j\le n} \sum_{k=i}^{j-1} a_k\Bigr)
%       \times \Bigl(\prod_{1\le i\le n}\sum_{k=i}^n a_k\Bigr).
%     \end{align*}
%     \end{gather*}
%
%     \underline{$C_n$}: The positive roots are
%     \[
%        \{\e_i-\e_j\}_{i<j}\cup \{2\e_i\} = \{\alpha_i+\cdots+\alpha_{j-1}\}_{i< j}\cup
%        \{2\alpha_i+\cdots+2\alpha_{n-1}+\alpha_n\},\text{ so}
%     \]
%     \begin{align*}
%       \dim V(\lambda) &= \prod_{1\le i<j\le n} \frac{\sum_{k=i}^{j-1} a_k}{j-1-i}
%       \times \prod_{1\le i\le n} \frac{a_n+\sum_{k=i}^{n-1} 2a_k}{2n-2i+1}\\
%       &= \frac{2^{n-2}\, (n-2)!}{n!!\, (2n-3)!} \Bigl(\prod_{1\le i<j\le n}
%       \sum_{k=i}^{j-1} a_k\Bigr) \times \biggl(\prod_{1\le i\le n}\Bigl(a_n+2\sum_{k=i}^{n-1}
%       a_k\Bigr)\biggr).
%     \end{align*}
%
%     \underline{$D_n$}: The positive roots are
%     \begin{align*}
%        \{\e_i-\e_j\}_{i<j}\cup \{\e_i&+\e_j\}_{i<j} =
%        \{\alpha_i+\cdots+\alpha_{j-1}\}_{i< j}\quad \cup\\
%           &\cup \quad \bigl\{(\alpha_i+\cdots+\alpha_{n-2}+\alpha_n)+(\alpha_j+\cdots
%           +\alpha_{n-1}) \bigr\},\text{ so}
%     \end{align*}
%     \begin{align*}
%       \dim& V(\lambda) = \prod_{1\le i<j\le n} \frac{\sum_{k=i}^{j-1} a_k}{j-1-i}
%       \times \prod_{1\le i\le n} \frac{a_n+\sum_{k=i}^{n-2} a_k +\sum_{k=j}^{n-1} a_k}{2n-i-j}\\
%       &= \frac{1}{\prod_{k=1}^{n-1} (2k-1)!} \Bigl(\prod_{1\le i<j\le n} \sum_{k=i}^{j-1}
%       a_k\Bigr) \times \biggl(\prod_{1\le i<j\le n}\Bigl(a_n+\sum_{k=i}^{n-2} a_k +\sum_{k=j}^{n-1}
%       a_k\Bigr)\biggr).
%     \end{align*}
%   \end{solution}
% \end{exercise}

% In the $\sl(n)$ case, you get $\e_1+\cdots+\e_n=0$, and you get $\rho = stuff$, so
% you can compute dimension as
% \[
%    \prod_{\alpha\in \Delta^+}(\rho,\alpha) = (n-1)!(n-2)!\cdots = (n-1)!!
% \]
%
% If $\lambda = \sum a_i\e_i$, then we get
% \[
%    \prod_{i< j} (a_i+a_j + j-i)
% \]
% We can  prove this from the character formula? We have to take a limit as $e\to 1$
% (since we get $0/0$ if we take $e=1$).
%
% \[
%    \lim_{u\to 0} \frac{\sum (-1)^w e^{(w(\lambda+\rho),u)}}{\sum (-1)^w e^{(w(\rho),u)}}
% \]
% Let $u=\rho t$ for $t\to 0$, then we get
% \[
%    \frac{\prod_{\alpha\in \Delta^+} e^{t(\lambda+\rho,\alpha/2)}-e^{-t(\lambda+\rho,\alpha/2)}}{\prod_{\alpha\in \Delta^+} e^{t(\rho,\alpha/2)} - e^{-t(\rho,\alpha/2)}}
% \]
% using $(w(\lambda+\rho),u) = (\lambda+\rho, w^{-1}(u))$, and from here it is just an
% exercise in L'hopital.

 \begin{remark}
 Given complete reducibility, knowing the characters of all irreducible
 representations allows you to decompose tensor products, just like in representation
 theory of finite groups. That is, we can now compute the coefficients in
 $V(\lambda)\otimes V(\mu) = \bigoplus b_{\lambda \mu}^\nu V(\nu)$. In the finite
 group case, we make this easier by choosing an inner product on class functions so
 that characters of irreducible representations form an orthonormal basis. Now we
 would like to come up with an inner product on formal expressions $\sum m_\mu e^\mu$
 so that characters of irreducible representations are orthonormal.

 The obvious inner product is $\langle e^\lambda, e^\mu\rangle =
 \delta_{\lambda,\mu}$, under which the $e^\mu$ are an orthonormal basis. There is no
 hope for the $ch\, V(\lambda)$ to be orthogonal, but we can tweak it. Another inner
 product is
 \[
    (e^\lambda,e^\mu) = \frac{1}{|\weyl|} \langle D\cdot e^\lambda, D\cdot
    e^\mu\rangle
 \]
 where $D$ is the Weyl denominator. The character formula tells us that under this
 inner product, the $ch\, V(\lambda)$ are orthonormal, and form a basis for
 \emph{$\weyl$-symmetric} expressions where $m_\mu=0$ for $\mu\not\in P$.

 As with the character formula, this may not look so impressive, but it makes
 decomposing tensor products very fast. We want to compute
 \[
   \bigl(ch\, V(\lambda) \cdot ch\, V(\mu),ch\, V(\gamma)\bigr) =
   \frac{1}{|\weyl|}\bigl\langle
   \underbrace{D\cdot ch\, V(\lambda)}_{
   \makebox[0pt]{\scriptsize $\sum (-1)^w e^{w(\lambda+\rho)}$}}
   \cdot ch\, V(\mu),
   \underbrace{D\cdot ch\,V(\gamma)}_{
   \makebox[0pt]{\scriptsize $\sum (-1)^w e^{w(\gamma+\rho)}$}}
   \bigr\rangle
 \]
 for all $\gamma\in P^+$. Since we know that the result must be $\weyl$-symmetric, we
 can remove the $\frac{1}{|\weyl|}$ and restrict our attention to the Weyl chamber.
 That is, we can just compute $\bigl\langle \sum (-1)^w e^{w(\lambda+\rho)}\cdot ch\,
 V(\mu),e^{\gamma+\rho}\bigr\rangle$, which is the multiplicity of $\gamma$ in
 $\sum (-1)^w e^{w(\lambda+\rho)-\rho}ch\, V(\mu)$. In practice, we choose $|\mu|\le
 |\lambda|$, so most of the summands lie outside of the Weyl chamber, so
 we can ignore them.
 \end{remark}

 \begin{example}[For those who know about $\gl(n)$]\index{gl(n)@$\gl(n)$|idxit} We
 know that $\gl(n+1)$ is the direct sum (as a Lie algebra) of its center, $k\cdot
 \id$, and $\sl(n+1)$.\footnote{ In general, a Lie algebra which is the the direct sum
 of its center and its semisimple part is called \emph{reductive}\index{reductive}.}
 \anton{what does that tell you about its irreps?} Let $\{\e_1,\dots, \e_{n+1}\}$
 be the image of an orthonormal basis of $k^{n+1}$ in $k^n$ (under the usual
 projection, so that $\sum \e_i=0$). Let $z_i = e^{\e_i}$, so $z_1\cdots z_{n+1}=1$.
 The Weyl group $W\simeq S_{n+1}$ acts on the $z_i$ by permutation. We have that
 \begin{align*}
   \rho &= \frac{1}{2} \sum_{i< j} \e_i-\e_j \\
   &= \frac{n}{2}\e_1 + \frac{n-2}{2}\e_2+\cdots + \frac{-n}{2}\e_{n+1} \\
   &= n\e_1 + (n-1)\e_2 + \cdots + 2\e_{n-1} + \e_n + 0\e_{n+1} &
      \bigl(\textstyle\sum \e_i=0\bigr)
 \end{align*}
 so
 \[
   e^\rho = z_1^n z_2^{n-1}\cdots z_n^1 z_{n+1}^0.
 \]
 If $\lambda = \sum_{i=1}^{n+1} a_i\e_i$ (with $\sum a_i=0$) is a dominant integral
 weight, we have $(\lambda, \check \alpha_i)=a_i-a_{i+1} \ge 0$. The character formula
 says that
 \begin{align*}
   ch\, V(\lambda) &= \frac{\sum_{\sigma\in S_{n+1}} (-1)^\sigma
  z_{\sigma(1)}^{a_1+n} \cdots z_{\sigma(n)}^{a_n +1} z_{\sigma(n+1)}^{a_{n+1}}}{\sum_{\sigma\in S_{n+1}}
  (-1)^\sigma z_{\sigma(1)}^n\cdots z_{\sigma(n)}^1 z_{\sigma(n+1)}^0}
 \end{align*}
 The denominator (call it $D$) is the famous Vandermonde determinant,\index{Vandermonde determinant}
 \[
  \renewcommand\arraystretch{1.3}
    \det\left( \begin{array}{cccc}
    z_1^n & z_2^n &\cdots & z_{n+1}^n\\
    z_1^{n-1} & z_2^{n-1} & \cdots & z_{n+1}^{n-1}\\
    \vdots & \vdots & \ddots & \vdots\\
    z_1 & z_2& \cdots & z_{n+1}\\
    1 & 1 & \cdots & 1
  \end{array}\right)
 \renewcommand\arraystretch{3}
   \raisebox{1.5ex}{\mbox{$\begin{array}{l}
    =\displaystyle\sum_{\sigma\in S_{n+1}} (-1)^\sigma z_{\sigma(1)}^n\cdots z_{\sigma(n)}^1
      z_{\sigma(n+1)}^0 \\
    =\displaystyle\prod_{1\le i<j\le n+1}(z_j-z_i)\\
  \end{array}$}}
 \]
 The numerator is
 \[
   D_\lambda =   \renewcommand\arraystretch{1.3}
    \det\left( \begin{array}{cccc}
    z_1^{a_1+n} & z_2^{a_1+n} &\cdots & z_{n+1}^{a_1+n}\\
    z_1^{a_2+n-1} & z_2^{a_2+n-1} & \cdots & z_{n-1}^{a_2+n+1}\\
    \vdots & \vdots & \ddots & \vdots\\
    z_1^{a_n+1} & z_2^{a_n+1}& \cdots & z_{n+1}^{a_n+1}\\
    z_1^{a_{n+1}} & z_1^{a_{n+1}} & \cdots & z_1^{a_{n+1}}
  \end{array}\right)
 \]
 So the character is the Schur polynomial.\index{Schur polynomial}

 Usually, the representations are encoded as Young diagrams. The marks on the dynkin
 diagram are the differences in consecutive rows in the young diagram.\anton{what does
 this part mean?}
 \end{example}
 \index{Weyl character formula|)idxbf}
