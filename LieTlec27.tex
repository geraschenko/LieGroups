 \stepcounter{lecture}
 \setcounter{lecture}{27}
 \sektion{Lecture 27}

 Last week we talked about $\hat e^L$, which was a double cover of $e^L$. $L$ is the
 root lattice of $E_8$. We had the sequence
 \[
    1\to \pm 1\to \hat e^L\to e^L\to 1.
 \]
 The Lie algebra structure on $\hat e^L$ was given by
 \begin{align*}
 [\alpha, \beta]     &= 0\\
 [\alpha, e^\beta]   &= (\alpha, \beta) e^\beta \\
 [e^\alpha, e^\beta] &= \begin{cases}
                       0 & \text{if $(\alpha, \beta) \ge 0$}\\
                       e^\alpha e^\beta & \text{if $(\alpha, \beta) = -1$}\\
                       \alpha & \text{if $(\alpha, \beta) = -2$}
                     \end{cases}
 \end{align*}
 The Lie algebra is $L\oplus \bigoplus_{\alpha^2=2} \hat e^\alpha$.

 Let's finish checking the Jacobi identity. We had two cases left:
 \[
 [[e^\alpha,e^\beta],e^\gamma] + [[e^\beta,e^\gamma],e^\alpha] + [[e^\gamma,e^\alpha],e^\beta]=0
 \]
 \begin{itemize}
   \item[$-$] $(\alpha,\beta)=(\beta,\gamma)=(\gamma,\alpha)=-1$, in which case
   $\alpha+\beta+\gamma=0$. then $[[e^\alpha,e^\beta],e^\gamma] = [e^\alpha
   e^\beta,e^\gamma] = \alpha+\beta$. By symmetry, the other two terms are
   $\beta+\gamma$ and $\gamma+\alpha$;the sum of all three terms is
   $2(\alpha+\beta+\gamma)=0$.

   \item[$-$] $(\alpha,\beta)=(\beta,\gamma)=-1$, $(\alpha,\gamma)=0$, in which case
   $[e^\alpha,e^\gamma]=0$. We check that $[[e^\alpha,e^\beta],e^\alpha]=[e^\alpha
   e^\beta, e^\gamma] = e^\alpha e^\beta e^\gamma$ (since
   $(\alpha+\beta,\gamma)=-1$).
   Similarly, we have $[[e^\beta,
   e^\gamma],e^\alpha] = [e^\beta e^\gamma,e^\alpha] = e^\beta e^\gamma e^\alpha$.
   We notice that $e^\alpha e^\beta = -e^\beta e^\alpha$
   and $e^\gamma e^\alpha = e^\alpha e^\gamma$ so
   $e^\alpha e^\beta e^\gamma = -e^\beta e^\gamma e^\alpha$; again, the sum
   of all three terms in the Jacobi identity is 0.
 \end{itemize}
 This concludes the verification of the Jacobi identity, so we have a Lie algebra.

 Is there a proof avoiding case-by-case check? Good news: yes! Bad news: it's actually
 more work. We really have functors as follows:
 \[\xymatrix @R=3.75em @C=3.75em{
   \txt{Dynkin\\ diagrams} \ar[r] \ar[dr] & \txt{Double\\ cover $\hat L$} \ar[d]
   \ar[rr]^{\mbox{\scriptsize\txt{elementary,\\ but tedious}}}_{\mbox{\scriptsize\txt{only for positive\\ definite lattices}}}
   \ar@/_1.25em/[rrd]_{}="a"
   & & \text{Lie algebras}\\
   & \txt{Root lattice $L$} & & \text{Vertex algebras} \ar[u]^{}="b"\\
   & & & {}\save[]*\txt<6pc>{these work\\ for any\\ \emph{even} lattice} \ar@/^1.25em/
 "a" \ar@/_4.75em/ "b" \restore }
 \]
 where $\hat L$ is generated by $\hat e^{\alpha_i}$
 (the $i$'s are the dots in your Dynkin diagram), with $\hat e^{\alpha_i}\hat
 e^{\alpha_j}=(-1)^{(\alpha_i,\alpha_j)}\hat e^{\alpha_j}\hat e^{\alpha_i}$, and $-1$
 is central of order 2.

 Unfortunately, you have to spend several weeks learning vertex algebras. In fact, the
 construction we did was the vertex algebra approach, with all the vertex algebras
 removed. So there is a more general construction which gives a much larger class of
 infinite dimensional Lie algebras.

 Now we should study the double cover $\hat L$, and in particular prove its existence.
 Given a Dynkin diagram, we can construct $\hat L$ as generated by the elements
 $e^{\alpha_i}$ for $\alpha_i$ simple roots with the given relations. It is easy to
 check that we get a surjective homomorphism $\hat L \to L$ with kernel
 generated by $z$ with $z^2=1$.  What's a little harder to show is that
 $z\neq 1$ (i.e., show that $\hat L\neq L$).
 The easiest way to do it is to use cohomology of groups, but since we have such an
 explicit case, we'll do it bare hands:\\
 Problem: Given $Z$, $H$ groups with $Z$ abelian, construct central extensions
 \[
    1\to Z\to G\to H\to 1
 \]
 (where $Z$ lands in the center of $G$). Let $G$ be the set of pairs $(z,h)$, and set
 the product $(z_1,h_1)(z_2,h_2) = (z_1z_2 c(h_1,h_2),h_1h_2)$, where $c(h_1,h_2)\in
 Z$ ($c(h_1,h_2)$ will be a cocycle in group cohomology). We obviously get a
 homomorphism by mapping $(z,h)\mapsto h$. If $c(1,h)=c(h,1)=1$ (normalization), then
 $z\mapsto (z,1)$ is a homomorphism mapping $Z$ to the center of $G$. In particular,
 $(1,1)$ is the identity. We'll leave it as an exercise to figure out what
 the inverses are. When is this thing \emph{associative}?
 Let's just write everything out:
 \begin{align*}
   \big( (z_1,h_1)(z_2,h_2)\big)(z_3,h_3) &= (z_1z_2z_3 c(h_1,h_2)c(h_1h_2,h_3),
   h_1h_2h_3)\\
   (z_1,h_1)\big( (z_2,h_2)(z_3,h_3)\big) &= (z_1z_2z_3 c(h_1,h_2h_3)c(h_2,h_3),
   h_1h_2h_3)\\
 \end{align*}
 so we must have
 \[
    c(h_1,h_2)c(h_1h_2,h_3) = c(h_1h_2,h_3)c(h_2,h_3).
 \]
 This identity is actually very easy to satisfy in one particular case: when $c$ is
 bimultiplicative: $c(h_1,h_2h_3)=c(h_1,h_2)c(h_1,h_3)$ and
 $c(h_1h_2,h_3)=c(h_1,h_3)c(h_2,h_3)$. That is, we have a map $H\times H\to Z$. Not
 all cocycles come from such maps, but this is the case we care about.

 To construct the double cover, let $Z=\pm 1$ and $H=L$ (free abelian). If we write
 $H$ additively, we want $c$ to be a bilinear map $L\times L \to \pm 1$. It is really
 easy to construct bilinear maps on free abelian groups. Just take any basis
 $\alpha_1,\dots, \alpha_n$ of $L$, choose $c(\alpha_1,\alpha_j)$ arbitrarily for each $i,j$
 and extend $c$ via bilinearity to $L\times L$. In our case, we want to find a double
 cover $\hat L$ satisfying $\hat e^\alpha \hat e^\beta = (-1)^{(\alpha,\beta)} \hat
 e^\beta \hat e^\alpha$ where $\hat e^\alpha$ is a lift of $e^\alpha$. This just means
 that $c(\alpha,\beta) = (-1)^{(\alpha,\beta)} c(\beta,\alpha)$. To satisfy this, just
 choose $c(\alpha_i,\alpha_j)$ on the basis $\{\alpha_i\}$ so that
 $c(\alpha_i,\alpha_j) = (-1)^{(\alpha_i,\alpha_j)} c(\alpha_j,\alpha_i)$. This is
 trivial to do as $(-1)^{(\alpha_i,\alpha_i)}=1$. Notice that this uses the fact that
 the lattice is even. There is no canonical way to choose this 2-cocycle (otherwise,
 the central extension would split as a product), but all the different double covers
 are isomorphic because we can specify $\hat L$ by generators and relations. Thus, we
 have constructed $\hat L$ (or rather, verified that the kernel of $\hat L \to L$ has
 order 2, not 1).

 Let's now look at lifts of automorphisms of $L$ to $\hat L$.
 \begin{exercise}
 Any automorphism of $L$ preserving $(\ ,\,)$ lifts to an automorphism
 of $\hat L$
 \end{exercise}
 There are two special cases:
 \begin{enumerate}
   \item $-1$ is an automorphism of $L$, and we want to lift it to $\hat L$
   explicitly. First attempt: try sending $\hat e^\alpha$ to $\hat e^{-\alpha}:=(\hat
   e^\alpha)^{-1}$, which doesn't work because $a\mapsto a^{-1}$ is not an
   automorphism on non-abelian groups.

   Better: $\w: \hat e^\alpha \mapsto (-1)^{\alpha^2/2}(\hat e^\alpha)^{-1}$ is an
   automorphism of $\hat L$. To see this, check
   \begin{align*}
     \w(\hat e^\alpha) \w(\hat e^\beta) &= (-1)^{(\alpha^2+\beta^2)/2}(\hat
     e^\alpha)^{-1}(\hat e^\beta)^{-1}\\
     \w(\hat e^\alpha \hat e^\beta) &= (-1)^{(\alpha+\beta)^2/2} (\hat e^\beta)^{-1}
     (\hat e^\alpha)^{-1}
   \end{align*}
   which work out just right

   \item If $r^2=2$, then $\alpha\mapsto \alpha - (\alpha,r)r$ is an automorphism of
   $L$ (reflection through $r^\perp$). You can lift this by $\hat e^\alpha \mapsto
   \hat e^\alpha (\hat e^r)^{-(\alpha,r)} \times (-1)^{\binom{(\alpha,r)}{2}}$. This
   is a homomorphism (check it!) of order (usually) 4!
   \begin{remark}
     Although automorphisms of $L$ lift to automorphisms of $\hat L$, the lift might
     have larger order.
   \end{remark}
 \end{enumerate}

 This construction works for the root lattices of $A_n$, $D_n$, $E_6$, $E_7$, and
 $E_8$; these are the lattices which are even, positive definite, and generated by
 vectors of norm 2 (in fact, all such lattices are sums of the given ones). What about
 $B_n$, $C_n$, $F_4$ and $G_2$? The reason the construction doesn't work for these
 cases is because there are roots of different lengths. These all occur as fixed
 points of diagram automorphisms of $A_n$, $D_n$ and $E_6$. In fact, we have a
 \emph{functor}
 from Dynkin diagrams to Lie algebras, so and automorphism of the diagram gives an
 automorphism of the algebra
 \[\begin{tabular}{cc|cc}
  Involution & Fixed points & Involution & Fixed Points \\
  \begin{xy}<1.75em,0em>:
     (0,0)="1" *\cir<2pt>{};
     p+(1,0)="2"  *\cir<2pt>{} **@{-};
     p+(.6,0) **@{-};
     p+(.8,0) **{\hspace{.7pt}.\hspace{.7pt}};
     p+(.6,0)="22" *\cir<2pt>{} **@{-};
     p+(1,0)="11" *\cir<2pt>{} **@{-};
     "1" *+{\ };"11" *+{\ } **\crv{(2,1.5)} ?<*@{<} ?>*@{>},
     "2" *+{\ };"22" *+{\ } **\crv{(2,1)} ?<*@{<} ?>*@{>},
     (2,-.5) *{=A_{2n+1}}
   \end{xy} &
   \begin{xy}<1.75em,0em>:
   (0,2) *\cir<2pt>{};
   p-(0,1) *\cir<2pt>{} **@{-};
   p-(0,.5) **@{-};
   p-(0,.6) **+<-3.4pt,2pt>{.};
   p-(0,.5) *\cir<2pt>{} **@{-};
   p-(0,1) *\cir<2pt>{} **@{=}?(.5)*@{>};
  \end{xy} $= C_{n+1}$ &
  \begin{xy}<1.75em,0em>:
     (0,0) *\cir<2pt>{};
     a(60)="1" *\cir<2pt>{} **@{-},
     a(180)="2" *++!R{D_4=} *\cir<2pt>{} **@{-},
     a(-60)="3" *\cir<2pt>{} **@{-},
     \ar@/_1.4ex/ "1" *+{\ };"2" *+{\ }
     \ar@/_1.4ex/ "2" *+{\ };"3" *+{\ }
     \ar@/_1.4ex/ "3" *+{\ };"1" *+{\ }
   \end{xy} &
   \begin{xy}
   (.5,0) *++!U{=G_2};
   (0,0)="1" *\cir<2pt>{};
   (1,0)="2" *\cir<2pt>{} **@{-}?*@{>},
   \ar@{-} "1" *{\hspace{3pt}};"2" *{\hspace{3pt}} <1.5pt>
   \ar@{-} "1" *{\hspace{3pt}};"2" *{\hspace{3pt}} <-1.5pt>
   \end{xy}\\
   \begin{xy}<1.75em,0em>:
   (0,0) *++!R{D_n=} *\cir<2pt>{};
   p+(.5,0) **@{-};
   p+(.6,0) **{.};
   p+(.5,0)  *\cir<2pt>{} **@{-};
   p+a(45)="a"  *\cir<2pt>{} **@{-},
   p+a(-45)="b"  *\cir<2pt>{} **@{-};
   \ar@{<->}@/^/ "a" *+{\,};"b" *+{\,}
 \end{xy} &
 \begin{xy}<1.75em,0em>:
   (0,0) *\cir<2pt>{};
%   (1,0) *\cir<2pt>{} **@{-};
   p+(.5,0) **@{-};
   p+(.6,0) **{.};
   p+(.5,0) *\cir<2pt>{} **@{-};
   p+(1,0) *\cir<2pt>{} **@{=}?(.5)*@{<};
   (1.3,0) *++!U{=B_n}
  \end{xy} &
  \begin{xy}<1.75em,0em>:
   (0,.8) *{E_6=};
   (0,0)="1" *\cir<2pt>{};
   (1,0)="2"  *\cir<2pt>{} **@{-};
   p+(1,0) *\cir<2pt>{} **@{-};
   p+(0,-1) *\cir<2pt>{} **@{-},
   p+(1,0)="22" *\cir<2pt>{} **@{-};
   p+(1,0)="11" *\cir<2pt>{} **@{-};
   "1" *+{\ };"11" *+{\ } **\crv{(2,1.5)} ?<*@{<} ?>*@{>},
   "2" *+{\ };"22" *+{\ } **\crv{(2,1)} ?<*@{<} ?>*@{>},
 \end{xy} &
 \begin{xy} <1.75em,0em>:
   (0,-1.5) *\cir<2pt>{};
   p+(0,1)  *\cir<2pt>{} **@{-};
   p+(0,1) *\cir<2pt>{} **@{=};
   p+(0,1)  *\cir<2pt>{} **@{-};
 \end{xy} $= F_4$
 \end{tabular}\]

 $A_{2n}$ doesn't really give you a new algebra: it corresponds to some
 superalgebra stuff.

 \subsektion{Construction of the Lie group of \texorpdfstring{$E_8$}{E8}} It is the
 group of automorphisms of the Lie algebra generated by the elements $\exp(\lambda
 Ad(\hat e^\alpha))$, where $\lambda$ is some real number, $\hat e^\alpha$ is one of
 the basis elements of the Lie algebra corresponding to the root $\alpha$, and
 $Ad(\hat e^\alpha)(a) = [\hat e^\alpha, a]$. In other words,
 \begin{align*}
   \exp(\lambda Ad(\hat e^\alpha))(a) &= 1+ \lambda [\hat e^\alpha, a] +
   \frac{\lambda^2}{2} [\hat e^\alpha, [\hat e^\alpha, a]].
 \end{align*}
 and all the higher terms are zero. To see that $Ad(\hat e^\alpha)^3 = 0$, note that
 if $\beta$ is a root, then $\beta+3\alpha$ is not a root (or 0).
 \begin{warning}
   In general, the group generated by these automorphisms is NOT the whole
   automorphism group of the Lie algebra. There might be extra diagram automorphisms,
   for example.
 \end{warning}

 We get some other things from this construction. We can get simple groups over finite
 fields: note that the construction of a Lie algebra above works over any commutative
 ring (e.g.\ over $\ZZ$). The only place we used division is in $\exp(\lambda Ad(\hat
 e^\alpha))$ (where we divided by 2). The only time this term is non-zero is when we
 apply $\exp (\lambda Ad(\hat e^\alpha))$ to $\hat e^{-\alpha}$, in which case we find
 that $[\hat e^\alpha,[\hat e^\alpha,\hat e^{-\alpha}]]= [\hat e^\alpha, \alpha] =
 -(\alpha,\alpha) \hat e^\alpha$, and the fact that $(\alpha,\alpha)=2$ cancels the
 division by 2. So we can in fact construct the $E_8$ group over \emph{any} commutative ring.
 You can mumble something about group schemes over $\ZZ$ at this point. In particular,
 we have groups of type $E_8$ over \emph{finite fields}, which are actually finite
 simple groups (these are called Chevalley groups; it takes work to show that they are
 simple, there is a book by Carter called \textsl{Finite Simple Groups} which you
 can look at).

 \subsektion{Real forms}So we've constructed the Lie group and Lie algebra of type
 $E_8$. There are in fact several \emph{different} groups of type $E_8$. There is one
 \emph{complex} Lie algebra of type $E_8$, which corresponds to several different real
 Lie algebras of type $E_8$.

 Let's look at some smaller groups:
 \begin{example}
   $\sl_2(\RR) = \matrix abcd$ with $a,b,c,d$ real $a+d=0$; this is not compact.
   On the other hand, $\mathfrak{su}_2(\RR) = \matrix abcd$ with $d=-a$ imaginary $b=-\bar c$, is compact.
   These have the same Lie algebra over $\CC$.
 \end{example}

 Let's look at what happens for $E_8$. In general, suppose $L$ is a Lie algebra with
 complexification $L\otimes \CC$. How can we find another Lie algebra $M$ with the
 same complexification? $L\otimes \CC$ has an anti-linear involution $\w_L: l\otimes
 z\mapsto l\otimes \bar z$. Similarly, it has an anti-linear involution $\w_M$. Notice
 that $\w_L\w_M$ is a linear involution of $L\otimes \CC$. Conversely, if we know this
 involution, we can reconstruct $M$ from it. Given an involution $\w$ of $L\otimes
 \CC$, we can get $M$ as the fixed points of the map $a\mapsto \w_L \w(a)$``=''
 $\overline{\w(a)}$. Another way is to put $L=L^+\oplus L^-$, which are the $+1$ and
 $-1$ eigenspaces, then $M=L^+\oplus iL^-$.

 Thus, to find other real forms, we have to study the involutions of the
 complexification of $L$. The exact relation is kind of subtle, but this is a good way
 to go.

 \begin{example}
   Let $L=\sl_2(\RR)$. It has an involution $\w(m) = -m^T$. $\mathfrak{su}_2(\RR)$ is
   the set of fixed points of the involution $\w$ times complex conjugation on
   $\sl_2(\CC)$, by definition.
 \end{example}

 So to construct real forms of $E_8$, we want some involutions of the Lie algebra
 $E_8$ which we constructed. What involutions do we know about? There are two obvious
 ways to construct involutions:
 \begin{enumerate}
   \item Lift $-1$ on $L$ to $\hat e^\alpha \mapsto (-1)^{\alpha^2/2}(\hat
   e^\alpha)^{-1}$, which induces an involution on the Lie algebra.

   \item Take $\beta\in L/2L$, and look at the involution $\hat e^\alpha\mapsto
   (-1)^{(\alpha,\beta)} \hat e^\alpha$.
 \end{enumerate}
 (2) gives nothing new ... you get the Lie algebra you started with. (1) only gives
 you one real form. To get all real forms, you multiply these two kinds of involutions
 together.

 Recall that $L/2L$ has 3 orbits under the action of the Weyl group, of size 1, 120,
 and 135. These will correspond to the three real forms of $E_8$. How do we
 distinguish different real forms? The answer was found by Cartan: look at the
 signature of an invariant quadratic form on the Lie algebra!

 A bilinear form $(\ ,\,)$ on a Lie algebra is called \emph{invariant} if
 $([a,b],c)+(b[a,c])=0$ for all $a,b,c$. This is called invariant because it
 corresponds to the form being invariant under the corresponding group action. Now we
 can construct an invariant bilinear form on $E_8$ as follows:
 \begin{enumerate}
   \item $(\alpha,\beta)_\text{in the Lie algebra} = (\alpha,\beta)_\text{in the lattice}$
   \item $(\hat e^\alpha,(\hat e^\alpha)^{-1}) = 1$
   \item $(a,b)=0$ if $a$ and $b$ are in root spaces $\alpha$ and $\beta$ with
   $\alpha+\beta \neq 0$.
 \end{enumerate}
 This gives an invariant inner product on $E_8$, which you prove by case-by-case check
 \begin{exercise}
   do these checks
 \end{exercise}

 Next time, we'll use this to produce bilinear forms on all the real forms and then
 we'll calculate the signatures.
