  \sektion{How these notes came to be} Among the Berkeley professors, there was once
  Allen Knutson\index{Knutson, Allen}, who would teach Math 261. But it happened that
  professor Knutson was on sabbatical at UCLA, and eventually went there for good. During
  this turbulent time, Maths 261AB were cancelled two years in a row. The last of these
  four semesters (Spring 2006), some graduate students gathered together and asked
  Nicolai Reshetikhin to teach them Lie theory in a giant reading course. When the dust
  settled, there were two other professors willing to help in the instruction of Math
  261A, Vera Serganova and Richard Borcherds. Thus Tag Team 261A was born.

  After a few lectures, professor Reshetikhin suggested that the students write up the
  lecture notes for the benefit of future generations. The first four lectures were
  produced entirely by the ``editors''. The remaining lectures were \LaTeX ed
  \index{LaTeX@\LaTeX} by Anton Geraschenko in class and then edited by the people in
  the following table. The columns are sorted by lecturer.

  \bigskip

  \hbox{\ifthenelse{\boolean{lilbook}}{
         \footnotesize \hskip -25pt % This is about half the ``overflow'' of the table
         }{
         \small \hskip -15pt }
   \index{Reshetikhin, Nicolai}
   \index{Serganova, Vera}
   \index{Borcherds, Richard E.}
 %  \index{Geraschenko, Anton}
 %  \index{George, Nathan}
 %  \index{Christianson, Hans}
 %  \index{Peters, Emily}
 %  \index{Mkrtchyan, Sevak}
 %  \index{Cimasoni, David}
 %  \index{Chen, Qingtau}
 %  \index{Blasiak, Jonah}
 %  \index{Thiel, Hannes}
 %  \index{Martirosyan, Lilit}
 %  \index{Canez, Santiago}
 %  \index{Liesinger, Katie}
 %  \index{McMillan, Aaron}
 %  \index{Do, Hanh Duc}
 %  \index{Huang, An}
 %  \index{Vito-Cruz, Martin}
   \begin{tabular}{clcclccl}
     \multicolumn{2}{c}{Nicolai Reshetikhin} &&
     \multicolumn{2}{c}{Vera Serganova} &&
     \multicolumn{2}{c}{Richard Borcherds} \\
     \cline{1-2} \cline{4-5} \cline{7-8} \vspace{-3mm}\\
     1 & Anton Geraschenko   && 11 & Sevak Mkrtchyan      && 21 & Hanh Duc Do       \\
     2 & Anton Geraschenko   && 12 & Jonah Blasiak        && 22 & An Huang          \\
     3 & Nathan George       && 13 & Hannes Thiel         && 23 & Santiago Canez    \\
     4 & Hans Christianson   && 14 & Anton Geraschenko    && 24 & Lilit Martirosyan \\
     5 & Emily Peters        && 15 & Lilit Martirosyan    && 25 & Emily Peters      \\
     6 & Sevak Mkrtchyan     && 16 & Santiago Canez       && 26 & Santiago Canez   \\
     7 & Lilit Martirosyan   && 17 & Katie Liesinger      && 27 & Martin Vito-Cruz \\
     8 & David Cimasoni      && 18 & Aaron McMillan       && 28 & Martin Vito-Cruz \\
     9 & Emily Peters        && 19 & Anton Geraschenko    && 29 & Anton Geraschenko  \\
     10 & Qingtau Chen       && 20 & Hanh Duc Do          && 30 & Lilit Martirosyan  \\
       &                     &&    &                      && 31 & Sevak Mkrtchyan \\
   \end{tabular}
  }
  \smallskip
  Richard Borcherds then edited the last third of the notes. The notes were further
  edited (and often expanded or rearranged) by Crystal Hoyt, Sevak Mkrtchyan, and Anton
  Geraschenko.

  \bigskip

  \ifthenelse{\boolean{proofmode}}{%
  If you are reading this, then you have the version of the notes that are for
  proofreading. You will see things written in the upper righthand margins. Those are
  index entries. Please note things that should be indexed, but aren't. You will also
  see comments on things that need to be fixed, which are in double square brackets
  with three big stars, \anton{like this}. Please let Anton know if you have a solution
  for one of these.}{}
  Send corrections and comments to
  \href{mailto:geraschenko@gmail.com}{\path{geraschenko@gmail.com}}.

 \sektion{Dependence of results and other information}
 \anton{make a flow chart of dependence of results}

 Within a lecture, everything uses the same counter, with the exception of exercises.
 Thus, item $a$.$b$ is the $b$-th item in Lecture $a$, whether it is a theorem, lemma,
 example, equation, or anything else that deserves a number and isn't an exercise.
