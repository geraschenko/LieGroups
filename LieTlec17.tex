 \stepcounter{lecture}
 \setcounter{lecture}{17}
 \sektion{Lecture 17 - Constructions of Exceptional simple Lie Algebras}\anton{We
 skipped much of this section during editing ... come back to it}

 We'll begin with the construction of $G_2$.

 \anton{remark about octonians goes here. see FH pp 363}
 We saw here that $G_2$ is isomorphic to the Lie algebra of automorphisms of a generic 3-form in
 7 dimensional space. The picture of the projective plane is related to Cayley
 numbers,
%% If you try to come up with an 8 dimensional algebra which is not
%% associative, the multiplication is given by the lines in the plane. What is
%% interesting is that any line plus the 1 gives a copy of the quaternions. This is
 an important nonassociative division algebra, of which  $G_2$ is the algebra of automorphisms.


 Consider the picture
 \[ \newcommand\thirdrootthree{.57735027}
 \begin{xy}<3.75em,0em>:
   (0,0) ="v1"  *+!R{v_1} *{\bullet},
   (1,0) ="w2"  *+!U{w_2} *{\bullet},
   (2,0) ="v3"  *+!L{v_3} *{\bullet},
   a(60) ="w3"  *+!DR{w_3} *{\bullet},
   "w3"+a(60) ="v2" *+!D{v_2} *{\bullet},
   "v3"+a(120)="w1" *+!DL{w_1} *{\bullet},
   (1,\thirdrootthree)="u" *+!(-.15,-.27){u} *{\bullet},
   "v1";"v2" **@{-};{?(.75)*@{>}},
   "v3" **@{-};{?(.75)*@{>}},
   "v1" **@{-};{?(.75)*@{>}},
   "w1" **@{-}, {?(.5)*@{>}},
   "w3";"v3" **@{-}, {?(.5)*@{<}},
   "w2";"v2" **@{-}, {?(.5)*@{<}},
   "u", *\xycircle(\thirdrootthree,\thirdrootthree){},
    *{\rotatebox{20}{\xycircle(\thirdrootthree,\thirdrootthree){=<5em>@{<}}}}
 \end{xy}\]
 This is the projective plane over $\FF_2$.

 Consider the standard 7 dimensional representation of $\gl(7)$, call it
 $E$. Take a basis given by points in the projective plane above.

 \begin{exercise} \label{lec17Ex1}
    Consider the following element of $\Lambda^3 E$.
    \[
    \w=v_1\wedge v_2\wedge v_3 +
    w_1\wedge w_2\wedge w_3 + u\wedge v_1\wedge w_1 + u\wedge v_2 \wedge w_2 + u\wedge
    v_3\wedge w_3
    \]
    Prove that $\gl(7)\w=\Lambda^3 E$.
    \begin{solution}
      It is enough to show that each basis vector of $\Lambda^3 E$ is in the orbit of
      $\w$. Let $p_u$, $p_v$, and $p_1$ be the projections onto $span\{u\}$,
      $span\{v_1,v_2,v_3\}$, and $span\{v_1,w_1\}$ respectively. For $x,y\in
      S:=\{u,v_1,v_2,v_3,w_1,w_2,w_3\}$, let $\phi_{x\to y}$ be the element of
      $\gl(7)$ sending $x$ to $y$, and sending the rest of $S$ to zero. Then a little
      messing around produces
      \[\renewcommand\arraystretch{1.2}
      \begin{array}{cc|cc}
        x & x\cdot \w & x & x\cdot \w\\ \hline
        \frac{1}{3}(p_v-p_u) & v_1\wedge v_2\wedge v_3 & \phi_{v_1\to u} & u\wedge v_2\wedge v_3\\
        \half p_1 + \half p_u - \frac{1}{6}\id & u\wedge v_1\wedge w_1 & \phi_{w_1\to w_2} & u\wedge v_1\wedge w_2\\
        \phi_{v_3\to w_1}+\phi_{w_2\to u} & v_1\wedge v_2\wedge w_1 & \phi_{v_3\to w_3} & v_1\wedge v_2\wedge w_3
      \end{array}\]
      Any other basis vector can be obtained from one of these (up to a sign) by
      permuting indices and/or swapping $v$'s and $w$'s, so we can get all of
      them.\footnote{Since $\w$ is not quite invariant under permutations of the
      indices or swapping of $v$'s and $w$'s, you sometimes have to tweak a sign
      (e.g.\ to get $w_1\wedge w_2\wedge v_1$, take $x=\phi_{w_3\to v_1}-\phi_{v_2\to
      u}$).}
    \end{solution}
 \end{exercise}
 \begin{warning}
   Don't forget that $\gl(7)$ acts on $\Lambda^3 E$ as a \emph{Lie algebra}, not as an
   associative algebra. That is, $x(a\wedge b\wedge c)=(xa)\wedge b\wedge c + a\wedge
   xb\wedge c + a\wedge b\wedge xc$. In particular, the action of $x$ followed by the
   action of $y$ is \emph{not} the same as the action of $yx$.
 \end{warning}

 \begin{claim}
   $\g = \{x\in \gl(7)|x\w=0\}$ is a simple Lie algebra with root system $G_2$.
 \end{claim}
 \begin{proof}
   It is immediate that $\g$ is a Lie algebra\anton{How do we know it is simple?}.
   Let's pick a candidate for the Cartan subalgebra. Consider linear operators which
   are diagonal in the given basis $u,v_1,v_2,v_3,w_1,w_2,w_3$,
   take $h = diag(c,a_1,a_2,a_3,b_1,b_2,b_3)$. If we want $h\in \g$, we must have
   \begin{align*}
    h\w &= (a_1+a_2+a_3)v_1v_2v_3 + (b_1+b_2+b_3)w_1w_2w_3 + \\
        & \quad + (c+a_1+b_1)uv_1w_1 + (c+a_2+b_2)uv_2w_2 + (c+a_3+b_3)uv_3w_3 = 0
   \end{align*}
   which is equivalent to $c=0$, $a_1+a_2+a_3=0$,
   and $b_i=-a_i$. So our Cartan subalgebra is two dimensional.

   If you consider the root diagram for $G_2$ and look at only the long roots, you get
   a copy of $A_2$. This should mean that you have an embedding $A_2\subseteq
   G_2$,\footnote{Note that this is not true of the short roots because the bracket of
   elements in ``adjacent'' short root spaces produces an element in a long root
   space, so the short root spaces will generate all of $G_2$.} so we should look for
   a copy of $\sl(3)$ in our $\g$. We can write $E = ku\oplus V\oplus W$, where
   $V=\langle v_1,v_2,v_3\rangle$ and $W=\langle w_1,w_2,w_3\rangle$. Let's consider
   the subalgebra which not only kills $\w$, but also kills $u$. Let $\g_0 = \{x\in
   \g| xu=0\}$.

   \anton{begin my poor explanation}

   Say $x\in \g_0$ is of the form
   \[
    x = \left(\begin{array}{c|cc}
      0 & a & b\\ \hline
      0 & A & B\\
      0 & C & D\\
    \end{array}\right),
   \]
   where $a,b$ are row vectors, then
   \[
     0 = x\cdot \w =
       \underset{\shortstack{$vvv$\\
                      $\fbox{$uvv$}_3$\\
                      $\fbox{$wvv$}_1$}}{x(v_1v_2v_3)}
     + \underset{\shortstack{$www$\\
                      $\fbox{$uww$}_3$\\
                      $\fbox{$vww$}_1$}}{x(w_1w_2w_3)}
     + \underset{\shortstack{$\fbox{$uvw$}_4$\\
        \rlap{\rule[2.5pt]{4.5ex}{.4pt}}$uuw$ \\
        \rlap{\rule[2.5pt]{4ex}{.4pt}}$uuv$ \\
                 $\fbox{\shortstack{$uvv$ \\
                                    $uww$}}_{2}$}}{u\wedge x(v_1w_1+ v_2w_2+v_3w_3)}
   \]
   where each term lies in the span of the basis vectors below it. Since the terms in
   boxes labelled 1 appear in only one way, we must have $B=C=0$. From that, it
   follows that the terms boxed an labelled $2$ cannot appear. Thus, the terms in
   boxes labelled 3 only appear in one way, so we must have $a=b=0$. Since the terms
   in boxes labelled 2 appear in only one place (though in two ways), we must have
   $D=-A^t$. Finally, since $vvv$ only appears in one place (in three different ways),
   we must have $tr\, A=0$.

   \anton{end poor explanation. Is there a better way?}

   For $x\in \g_0$ we have $x(v_1\wedge v_2\wedge v_3)=0$ and
   $x(w_1\wedge w_2\wedge w_3)=0$, so $x$ preserves $V$ and $W$. It also must kill the
   2-form $\alpha = v_1\wedge w_1+ v_2\wedge w_2+ v_3\wedge w_3$, since
   $0=x(u\wedge\alpha)=xu\wedge\alpha+u\wedge x\alpha=u\wedge x\alpha$ forces $x\alpha
   =0$. This 2-form gives a pairing, so that $V^* \simeq W$. We can compute exactly
   what the pairing is, $\langle v_i,w_j\rangle =\delta_{ij}$. Therefore the operator
   $x$ must be of the form
   \[
    x = \left(\begin{array}{c|cc}
      0 & 0 & 0\\ \hline
      0 & A & 0\\
      0 & 0 & -A^t\\
    \end{array}\right), \text{ where }tr(A)=0.
   \]

  %% Why does $x$ preserve the form? It is a non-degenerate 2-form on $V\oplus W$, so it
  %% identifies it with $V^*\oplus W^*$. And $V$ and $W$ are isotropic with respect to
  %% this form (lets call the form $\alpha$).  Then we have that $\langle w,v\rangle =
  %% \alpha(v,w)$. $x$ preserves $\alpha$ ... to show that, we need
  %% $\alpha(Av,w)+\alpha(v,-A^t w)=0$.
  %% this is showing that anyone of that block matrix form preserves \alpha %%


   The total dimension of $G_2$ is 14, which we also know by the exercise is the
   dimension of $\g$. We saw the Cartan subalgebra has dimension 2, and this $\g_0$
   piece has dimension 8 (two of which are the Cartan). So we still need another 6
   dimensional piece.

   For each $v\in V$, define a linear operator $X_v$ which acts by $X_v(u)=v$.
   $\Lambda^2 V\simeq_\gamma W$ is dual to $V$ (since $\Lambda^3 V=\CC$). Then $X_v$
   acts by $X_v(v')= \gamma(v\wedge v')$ and $X_v(w)=2\langle v,w\rangle u$. Check
   that this kills $\w$, and hence is in $\g$.\anton{clarify how you'd think if this
   $X_v$\dots what constraints does being in a root space put on $X_v$?}

   Similarly, you can define $X_w$ for each $w\in W$ by $X_w(u)=w,
   X_w(w')=\gamma(w\wedge w'), X_w(v) = 2\langle w,v\rangle u$.

   If you think about a linear operator which takes $u\mapsto v_i$, it must be
   in some root space, this tells you about how it should act
   on $V$ and $W$. This is how we constructed $X_v$ and $X_w$, so we know that
   $X_{v_i}$ and $X_{w_i}$ are in some root spaces. We can check that their roots
   are the short roots in the diagram,

   \[\begin{xy}
   (0,0);(0,0);
   \ar a(30) ="a" *+!L{X_{v_1}},
   \ar a(150)-"a"="b",
   \ar "b"+"a" *+!R{X_{v_2}},
   \ar "b"+"a"+"a" *+!D{X_{w_3}},
   \ar "b"+"a"+"a"+"a",
   \ar "b"+"b"+"a"+"a"+"a",
   \ar -"a" *+!R{X_{w_1}},
   \ar -"b",
   \ar -"b"-"a" *+!L{X_{w_2}},
   \ar -"b"-"a"-"a" *+!U{X_{v_3}},
   \ar -"b"-"a"-"a"-"a",
   \ar -"b"-"b"-"a"-"a"-"a",
 \end{xy}\]
   and so they span the remaining 6 dimensions of $G_2$.
   To properly complete this construction, we should check that this is
   semisimple, but we're not going to.
 \end{proof}

 Let's analyze what we did with $G_2$, so that we can do a similar thing to construct $E_8$.
 We discovered certain phenomena,
 we can write $\g = \underbrace{\g_0}_{\sl(3)}\oplus \underbrace{\g_1}_V\oplus
 \underbrace{\g_2}_W$. This gives us a $\ZZ/3$-grading: $[\g_i,\g_j]\subseteq
 \g_{i+j (mod\ 3)}$. As an $\sl(3)$ representation, it has three components: $ad$, standard,
 and the dual to the standard. We get that $W\cong V^*\simeq \Lambda^2 V$. Similarly,
 $V\simeq \Lambda^2 W$. This is called \emph{Triality}\index{triality|idxbf}.

 More generally, say we have $\g_0$ a semisimple Lie algebra, and $V,W$ representations of
 $\g_0$, with intertwining maps
 \begin{align*}
   \alpha:\Lambda^2 V&\to W \\
   \beta: \Lambda^2 W&\to V \\
   V&\simeq W^* . \\
 \end{align*}
  We also have $\gamma: V\otimes W\simeq V\otimes V^*\to \g_0$ (representations are semisimple,
  so the map $\g_0\to \gl(V)\simeq V\otimes V^*$ splits).
  We normalize $\gamma$ in the following way. Let $B$ be the Killing form, and normalize $\gamma$ so that
  $B(\gamma(v\otimes w),X) = \langle w,Xv\rangle$. Make a Lie algebra $\g =
  \g_0\oplus V\oplus W$ by defining $[X,v]=Xv$ and $[X,w]=Xw$ for $X\in \g_0, v\in V,w\in
  W$. We also need to define  $[\ ,\,]$ on $V$,$W$ and between $V$ and $W$. These are actually
  forced, up to coefficients:
  \[ [v_1,v_2] = a\alpha(v_1\wedge v_2) \]
  \[ [w_1,w_2] = b\beta(w_1\wedge w_2) \]
  \[ [v,w] = c\gamma(v\otimes w). \]
 There are some conditions on the coefficients $a,b,c$
 imposed by the Jacobi identity;
 $[x,[y,z]] = [[x,y],z]+[y,[x,z]]$. Suppose $x\in \g_0$, with $y,z\in \g_i$ for
 $i=0,1,2$, then there is nothing to check, these identities come for free because
 $\alpha,\beta, \gamma$ are $\g_0$-invariant maps. There are only a few more cases to
 check, and only one of them gives you a condition. Look at
 \[
    [v_0,[v_1,v_2]] = ca\gamma(v_0\otimes \alpha(v_1\wedge v_2)) \tag{RHS}
 \]
 and it must be equal to
 \begin{align*}
   [[v_0,v_1],v_2]+&[v_1,[v_0,v_2]] = \\ & -ac\gamma(v_2\otimes \alpha(v_0\wedge v_1)) +
   ac\gamma(v_1\otimes \alpha(v_0\wedge v_1)) \tag{LHS}
 \end{align*}
 This doesn't give a condition on $ac$, but we need to check that it is satisfied.
 It suffices to check that $B(RHS,X)=B(LHS,X)$ for any $X\in \g_0$. This
 gives us the following condition:
 \begin{align*}
   \langle \alpha ( v_1\wedge v_2),Xv_0\rangle &= \langle \alpha(v_0\wedge
   v_2),Xv_1\rangle - \langle \alpha(v_0\wedge v_1),Xv_2\rangle
 \end{align*}
 The fact that $\alpha$ is an intertwining map for $\g_0$ gives us the identity:
 \begin{align*}
   \langle \alpha ( v_1\wedge v_2),Xv_0\rangle &= \langle \alpha(Xv_1\wedge
   v_2),v_0\rangle - \langle \alpha(v_1\wedge Xv_2),v_0\rangle
 \end{align*}
 and we also have that
 \[
   \langle \alpha ( v_1\wedge v_2),v_0\rangle = \langle \alpha(v_0\wedge
   v_2),v_1\rangle = \langle \alpha(v_0\wedge v_1),v_2\rangle
 \]
 With these two identities it is easy to show that the equation
 (and hence this Jacobi identity) is satisfied.

 We also get the Jacobi identity on $[w,[v_1,v_2]]$, which is equivalent to:
 \[ ab\beta (w\wedge \alpha(v_1\wedge v_2)) = c(\gamma(v_1\otimes w)v_2-\gamma(v_2\otimes w)v_1)
 \]
 It suffices to show for any $w'\in W$ that the pairings of each side with $w'$ are equal,
 \begin{align*}
   ab\langle w',\beta(w\wedge \alpha(v_1\wedge v_2)\rangle = c B&(\gamma(v_1\otimes
   w),\gamma(v_2\otimes w')) \\ &- c B(\gamma(v_2\otimes w),\gamma(v_1\otimes w'))
 \end{align*}
 This time we will get a condition on $a$, $b$, and $c$.
 You can check that any of the other cases of the Jacobi identity give you the same conditions.

 Now we will use this to construct $E_8$. Write $\g = \g_0\oplus V\oplus W$, where we
 take $\g_0=\sl(9)$. Let $E$ be the 9 dimensional representation of $\g_0$. Then take
 $V=\Lambda^3 E$ and $W=\Lambda^3 E^* \simeq \Lambda^6 E$. We have a pairing
 $\Lambda^3 E \otimes \Lambda^6 E\to k$, so we have $V\simeq W^*$. We would like to
 construct $\alpha: \Lambda^2 \to W$, but this is just given by $v_1\wedge v_2$
 including into $\Lambda^6 E\simeq W$. Similarly, we get $\beta:\Lambda^2 W\to V$. You
 get that the rank of $\g$ is 8 (= rank of $\g_0$). Notice that $\dim V =
 \binom{9}{3} = 84$, which is the same as $\dim W$, and $\dim \g _0 = \dim \sl(9)=80$.
Thus, we have that $\dim \g = 84 + 84+80=248$, which is the dimension of $E_8$, and
this is indeed $E_8$.

 Remember that we previously got $E_7$ and $E_6$ from $E_8$. Look at the diagram for $E_8$:
 \[ \begin{xy}
   (0,0) *+!D{\varepsilon_1-\varepsilon_2} *\cir<2pt>{};
   (1,0) *+!U{\varepsilon_2-\varepsilon_3} *\cir<2pt>{} **@{-};
   p+(1,0) *+!D{\varepsilon_3-\varepsilon_4} *\cir<2pt>{} **@{-};
   p+(1,0) *+!U{\varepsilon_4-\varepsilon_5} *\cir<2pt>{} **@{-};
   p+(1,0) *+!D{\varepsilon_5-\varepsilon_6} *\cir<2pt>{} **@{-};
   p+(0,-1) *+!U{\varepsilon_6+\varepsilon_7+\varepsilon_8} *\cir<2pt>{} **@{-},
   p+(1,0) *+!U{\varepsilon_6-\varepsilon_7} *\cir<2pt>{} **@{-};
   p+(1,0) *+!D{\varepsilon_7-\varepsilon_8} *\cir<2pt>{} **@{-};
 \end{xy}\]
 The extra guy, $\varepsilon_6+\varepsilon_7+\varepsilon_8$, corresponds to the 3-form. When you cut out
 $\varepsilon_1-\varepsilon_2$, you can figure out what is left and you get $E_7$.
 Then you can additionally cut out $\varepsilon_2-\varepsilon_3$ and get $E_6$.

 Fianally, we construct $F_4$:\qquad $\begin{xy}
   (0,0) *\cir<2pt>{};
   (1,0)  *\cir<2pt>{} **@{-};
   p+(1,0)="x" *\cir<2pt>{} **@{=} ?*@{>};
   "x" *{\hspace{4pt}};"x"+(1,0)  *\cir<2pt>{} **@{-};
 \end{xy}$

 We know that any simple algebra can be determined by generators and relations, with a
 $X_i,Y_i,H_i$ for each node $i$. But sometimes our diagram has a symmetry, like
 switching the horns on a $D_n$, which induces an automorphism of the Lie algebra
 given by $\gamma(X_i)=X_i$ for $i<n-1$ and switches $X_{n-1}$ and $X_n$. Because the arrows
 are preserved, you can check that the Serre relations still hold. Thus, in general,
 an automorphism of the diagram induces an automorphism of the Lie algebra (in a very
 concrete way).

 \begin{theorem}
   $(\aut \g)/(\aut_0 \g) = \aut \Gamma$. So the connected component of the identity
   gives some automorphisms, and the connected components are parameterized by
   automorphisms of the diagram.
 \end{theorem}

 $D_n$ is the diagram for $SO(2n)$. We have that $SO(2n)\subset O(2n)$, and the group
 of automorphisms of $SO(2n)$ is $O(2n)$.
 This isn't true of  $SO(2n+1)$, because the automorphisms given by
 $O(2n+1)$ are the same as those from $SO(2n+1)$. This corresponds the the fact that $D_n$
 has a nontrivial automorphism, but $B_n$ doesn't.

 Notice that $E_6$ has a symmetry; the involution:
 \[\begin{xy}
   (0,0)="1" *\cir<2pt>{};
   (1,0)="2"  *\cir<2pt>{} **@{-};
   p+(1,0) *\cir<2pt>{} **@{-};
   p+(0,-1) *\cir<2pt>{} **@{-},
   p+(1,0)="22" *\cir<2pt>{} **@{-};
   p+(1,0)="11" *\cir<2pt>{} **@{-};
   "1" *+{\ };"11" *+{\ } **\crv{(2,1.5)} ?<*@{<} ?>*@{>},
   "2" *+{\ };"22" *+{\ } **\crv{(2,1)} ?<*@{<} ?>*@{>},
 \end{xy}\]

 Define $X_1'=X_1+X_5, X_2' = X_2+X_4, X_3' = X_3, X_6'=X_6$, and the same with $Y$'s,
 the fixed elements of this automorphism. We
 have that $H_1'=H_1+H_5$ (you have to check that this works), and similarly for the
 other $H$'s. As the set of fixed elements, you get an algebra of rank 4 (which must
 be our $F_4$). You can check that
 $\alpha_1'(H_2')=-1,\alpha_2'(H_1')=-1,\alpha_3'(H_1')=0,\alpha_3'(H_2')=-2,\alpha_2'(H_3')=-1$,
 so this is indeed $F_4$ as desired. In fact, any diagram with multiple edges can be
 obtained as the fixed algebra of some automorphism:
 \begin{exercise}
   Check that $G_2$ is the fixed algebra of the automorphism of $D_4$:
   \[\begin{xy}
     (0,0) *\cir<2pt>{};
     a(60)="1" *\cir<2pt>{} **@{-},
     a(180)="2" *\cir<2pt>{} **@{-},
     a(-60)="3" *\cir<2pt>{} **@{-},
     \ar@/_2ex/ "1" *+{\ };"2" *+{\ }
     \ar@/_2ex/ "2" *+{\ };"3" *+{\ }
     \ar@/_2ex/ "3" *+{\ };"1" *+{\ }
   \end{xy}\]
   Check that $B_n,C_n$ can be obtained from $A_{2n}, A_{2n+1}$
   \[\begin{xy}
     (0,0)="1" *\cir<2pt>{};
     (1,0)="2"  *\cir<2pt>{} **@{-};
     p+(.6,0) **@{-};
     p+(.8,0) **{\,.\,};
     p+(.6,0)="22" *\cir<2pt>{} **@{-};
     p+(1,0)="11" *\cir<2pt>{} **@{-};
     "1" *+{\ };"11" *+{\ } **\crv{(2,1.5)} ?<*@{<} ?>*@{>},
     "2" *+{\ };"22" *+{\ } **\crv{(2,1)} ?<*@{<} ?>*@{>},
   \end{xy}\]
 \end{exercise}
