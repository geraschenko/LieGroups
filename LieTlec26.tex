 \stepcounter{lecture}
 \setcounter{lecture}{26}
 \sektion{Lecture 26}

 Today we'll finish looking at $W(E_8)$, then we'll construct $E_8$.

 Remember that we still have a fourth method of finding the order of $W(E_8)$. Let $L$
 be the $E_8$ lattice. Look at $L/2L$, which has 256 elements. Look at this as a set
 acted on by $W(E_8)$. There is an orbit of size 1 (represented by 0). There is an
 orbit of size $240/2=120$, which are the roots (a root is congruent mod $2L$ to it's
 negative). Left over are 135 elements. Let's look at norm 4 vectors. Each norm 4
 vector, $r$, satisfies $r\equiv -r \mod 2$, and there are $240\cdot 9$ of them, which
 is a lot, so norm 4 vectors must be congruent to a bunch of stuff. Let's look at
 $r=(2,0,0,0,0,0,0,0)$. Notice that it is congruent to vectors of the form $(0\dots
 \pm 2\dots 0)$, of which there are 16. It is easy to check that these are the only
 norm 4 vectors congruent to $r$ mod 2. So we can partition the norm 4 vectors into
 $240\cdot 9/16=135$ subsets of 16 elements. So $L/2L$ has 1+120+135 elements, where
 $1$ is the zero, 120 is represented by 2 elements of norm 2, and 135 is represented
 by 16 elements of norm 4. A set of 16 elements of norm 4 which are all congruent is
 called a FRAME. It consists of elements $\pm e_1,\dots, \pm e_8$, where $e_i^2=4$ and
 $(e_i,e_j)=1$ for $i\neq j$, so up to sign it is an orthogonal basis.

 Then we have
 \[
    |W(E_8)| = (\text{\# frames})\times |\text{subgroup fixing a frame}|
 \]
 because we know that $W(E_8)$ acts transitively on frames. So we need to know what
 the automorphisms of an orthogonal base are. A frame is 8 subsets of the form
 $(r,-r)$, and isometries of a frame form the group $(\ZZ/2\ZZ)^8\cdot S_8$, but these
 are not all in the Weyl group. In the Weyl group, we found a $(\ZZ/2\ZZ)^7\cdot S_8$,
 where the first part is the group of sign changes of an EVEN number of coordinates.
 So the subgroup fixing a frame must be in between these two groups, and since these
 groups differ by a factor of 2, it must be one of them. Observe that changing an odd
 number of signs doesn't preserve the $E_8$ lattice, so it must be the group
 $(\ZZ/2\ZZ)^7\cdot S_8$, which has order $2^7\cdot 8!$. So the order of the Weyl
 group is
 \[
    135\cdot 2^7\cdot 8! = |2^7\cdot S_8| \times \frac{\text{\# norm 4
    elements}}{2\times \dim L}
 \]
 \begin{remark}
 Similarly, if $\Lambda$ is the Leech lattice, you actually get the order of Conway's
 group to be
 \[
    |2^{12}\cdot M_{24}|\cdot \frac{\text{\# norm 8 elements}}{2\times \dim\Lambda}
 \]
 where $M_{24}$ is the Mathieu group (one of the sporadic simple groups). The Leech
 lattice seems very much to be trying to be the root lattice of the monster group, or
 something like that. There are a lot of analogies, but nobody can make sense of it.
 \end{remark}

 $W(E_8)$ acts on $(\ZZ/2\ZZ)^8$, which is a vector space over $\FF_2$, with quadratic
 form $N(a)=\frac{(a,a)}{2} \mod 2$, so you get a map
 \[
    \pm 1\to W(E_8) \to O^+_8(\FF_2)
 \]
 which has kernel $\pm 1$ and is surjective. $O^+_8$ is one of the $8$ dimensional
 orthogonal groups over $\FF_2$. So the Weyl group is very close to being an
 orthogonal group of a vector space over $\FF_2$.

 What is inside the root lattice/Lie algebra/Lie group $E_8$? One obvious way to find
 things inside is to cover nodes of the $E_8$ diagram:
 \[\begin{xy}
   (0,0) *\cir<2pt>{};
   p+(1,0) *\cir<2pt>{} **@{-};
   p+(1,0) *{\times} *\cir<2pt>{} **@{-};
   p+(1,0) *\cir<2pt>{} **@{-};
   p+(1,0) *\cir<2pt>{} **@{-};
   p+(0,1) *\cir<2pt>{} **@{-},
   p+(1,0) *\cir<2pt>{} **@{-};
   p+(1,0) *\cir<2pt>{} **@{-};
 \end{xy} \]
 If we remove the shown node, you see that $E_8$ contains $A_2\times D_5$. We can do
 better by showing that we can embed the affine $\tilde E_8$ in the $E_8$ lattice.
 \[\begin{xy}
   (0,0) *+!UR{-\text{highest root}} *\cir<2pt>{};
   p+(1,0) *\cir<2pt>{} **@{-};
   p+(1,0) *\cir<2pt>{} **@{-};
   p+(1,0) *\cir<2pt>{} **@{-};
   p+(1,0) *\cir<2pt>{} **@{-};
   p+(1,0) *\cir<2pt>{} **@{-};
   p+(0,1) *\cir<2pt>{} **@{-},
   p+(1,0) *\cir<2pt>{} **@{-};
   p+(1,0) *\cir<2pt>{} **@{-};
   (4,-.3) *=(6.3,0)\frm{_\}} *+!U{\text{simple roots}}
 \end{xy} \]
 Now you can remove nodes here and get some bigger sub-diagrams. For example, if we
 cover
 \[\begin{xy}
   (0,0) *\cir<2pt>{};
   p+(1,0) *{\times} *\cir<2pt>{} **@{-};
   p+(1,0) *\cir<2pt>{} **@{-};
   p+(1,0) *\cir<2pt>{} **@{-};
   p+(1,0) *\cir<2pt>{} **@{-};
   p+(1,0) *\cir<2pt>{} **@{-};
   p+(0,1) *\cir<2pt>{} **@{-},
   p+(1,0) *\cir<2pt>{} **@{-};
   p+(1,0) *\cir<2pt>{} **@{-};
 \end{xy} \]
 you get that an $A_1\times E_7$ in $E_8$. The $E_7$ consisted of 126
 roots orthogonal to a given root. This gives an easy construction of $E_7$ root
 system, as all the elements of the $E_8$ lattice perpendicular to $(1,-1,0\dots)$

 We can cover
 \[\begin{xy}
   (0,0) *\cir<2pt>{};
   p+(1,0) *\cir<2pt>{} **@{-};
   p+(1,0) *{\times} *\cir<2pt>{} **@{-};
   p+(1,0) *\cir<2pt>{} **@{-};
   p+(1,0) *\cir<2pt>{} **@{-};
   p+(1,0) *\cir<2pt>{} **@{-};
   p+(0,1) *\cir<2pt>{} **@{-},
   p+(1,0) *\cir<2pt>{} **@{-};
   p+(1,0) *\cir<2pt>{} **@{-};
 \end{xy} \]
 Then we get an $A_2\times E_6$, where the $E_6$ are all the vectors with the first 3
 coordinates equal. So we get the $E_6$ lattice for free too.

 If you cover
 \[\begin{xy}
   (0,0) *\cir<2pt>{};
   p+(1,0) *\cir<2pt>{} **@{-};
   p+(1,0) *\cir<2pt>{} **@{-};
   p+(1,0) *\cir<2pt>{} **@{-};
   p+(1,0) *\cir<2pt>{} **@{-};
   p+(1,0) *\cir<2pt>{} **@{-};
   p+(0,1) *\cir<2pt>{} **@{-},
   p+(1,0) *\cir<2pt>{} **@{-};
   p+(1,0) *{\times} *\cir<2pt>{} **@{-};
 \end{xy} \]
 you see that there is a $D_8$ in $E_8$, which is all vectors of the $E_8$ lattice
 with integer coordinates. We sort of constructed the $E_8$ lattice this way in the
 first place.

 We can ask questions like: What is the $E_8$ Lie algebra as a representation of
 $D_8$? To answer this, we look at the weights of the $E_8$ algebra, considered as a
 module over $D_8$, which are the 112 roots of the form $(\dots\pm 1\dots \pm 1\dots)$
 and the 128 roots of the form $(\pm 1/2,\dots)$ and 1 vector 0, with multiplicity 8.
 These give you the Lie algebra of $D_8$. Recall that $D_8$ is the Lie algebra of
 $SO_{16}$. The double cover has a half spin representation of dimension $2^{16/2
 -1}=128$. So $E_8$ decomposes as a representation of $D_8$ as the adjoint
 representation (of dimension 120) plus a half spin representation of dimension 128.
 This is often used to construct the Lie algebra $E_8$. We'll do a better construction
 in a little while.

 We've found that the Lie algebra of $D_8$, which is the Lie algebra of $SO_{16}$, is
 contained in the Lie algebra of $E_8$. Which \emph{group} is contained in the the compact
 form of the $E_8$? We found that there were groups
 \[\xymatrix @R=.75em {
   & \spin_{16}(\RR) \ar@{-}[dl] \ar@{-}[d] \ar@{-}[dr]\\
   SO_{16}(\RR) \ar@{-}[dr]& \spin_{16}(\RR)/(\ZZ/2\ZZ) \ar@{-}[d] \ar@{}[r]|{\cong}
   & *+[F-:<10pt>]{\spin_{16}(\RR)/(\ZZ/2\ZZ)}  \ar@{-}[dl]\\
   & PSO_{16}(\RR)
 }\]
 corresponding to subgroups of the center $(\ZZ/2\ZZ)^2$:
 \[\xymatrix @R=.75em {
   & 1 \ar@{-}[dl] \ar@{-}[d] \ar@{-}[dr]\\
   \ZZ/2\ZZ \ar@{-}[dr]& \ZZ/2\ZZ \ar@{-}[d]& \ZZ/2\ZZ \ar@{-}[dl]\\
   & (\ZZ/2\ZZ)^2
 }\]
 We have a homomorphism $\spin_{16}(\RR)\to E_8$(compact). What is the kernel? The
 kernel are elements which act trivially on the Lie algebra of $E_8$, which is equal
 to the Lie algebra $D_8$ plus the half spin representation. On the Lie algebra of
 $D_8$, everything in the center is trivial, and on the half spin representation, one
 of the elements of order 2 is trivial. So the subgroup that you get is the circled
 one.

 \begin{exercise}
   Show $SU(2)\times E_7$(compact)$/(-1,-1)$ is a subgroup of $E_8$ (compact).
   Similarly, show that $SU(9)/(\ZZ/3\ZZ)$ is also. These are similar to the example
   above.
 \end{exercise}

 \subsektion{Construction of \texorpdfstring{$E_8$}{E8}} Earlier in the course, we had some constructions:
 \begin{enumerate}
   \item using the Serre relations, but you don't really have an idea of what it looks
   like
   \item Take $D_8$ plus a half spin representation
 \end{enumerate}
 Today, we'll try to find a natural map from root lattices to Lie algebras.
 The idea is as follows: Take a basis element $e^\alpha$ (as a formal symbol)
 for each root $\alpha$; then take the Lie algebra to be the direct sum of
   1 dimensional spaces generated by each $e^\alpha$ and $ L$ ($L$ root lattice
   $\cong$ Cartan subalgebra) . Then we have to define the Lie bracket by setting
   $[e^\alpha,e^\beta]=e^{\alpha+\beta}$, but then we have a sign problem because
   $[e^\alpha,e^\beta]\neq -[e^\beta,e^\alpha]$.  Is there some way to resolve the
   sign problem? The answer is that there is no good way to solve this problem (not
   true, but whatever). Suppose we had a nice functor from root lattices to Lie
   algebras. Then we would get that the automorphism group of the lattice has to be
   contained in the automorphism group of the Lie algebra (which is contained in the
   Lie group), and the automorphism group of the Lattice contains the Weyl group of
   the lattice. But the Weyl group is NOT usually a subgroup of the Lie group.

 We can see this going wrong even in the case of $\sl_2(\RR)$. Remember that the Weyl
 group is $N(T)/T$ where $T=\matrix a00{a^{-1}}$ and $N(T)=T\cup \matrix
 0b{-b^{-1}}0$, and this second part is stuff having order 4, so you cannot possibly
 write this as a semi-direct product of $T$ and the Weyl group.

 So the Weyl group is not usually a subgroup of $N(T)$. The best we can do is to find
 a group of the form $2^n\cdot W\subseteq N(T)$ where $n$ is the rank. For example,
 let's do it for $SL(n+1,\RR)$ Then $T = diag(a_1,\dots, a_n)$ with $a_1\cdots a_n=1$.
 Then we take the normalizer of the torus to be $N(T)=$all permutation matrices with
 $\pm 1$'s with determinant 1, so this is $2^n\cdot S_n$, and it does not split. The
 problem we had with signs can be traced back to the fact that this group doesn't
 split.

 We can construct the Lie algebra from something acted on by $2^n\cdot W$ (but not
 from something acted on by $W$). We take a CENTRAL EXTENSION\index{central extension}
 of the lattice by a group of order 2. Notation is a pain because the lattice is
 written additively and the extension is nonabelian, so you want it to be written
 multiplicatively. Write elements of the lattice in the form $e^\alpha$ formally, so
 we have converted the lattice operation to multiplication. We will use the central
 extension
 \[
    1\to \pm 1 \to \hat e^L\to \underbrace{e^L}_{\cong L}\to 1
 \]
 We want $\hat e^L$ to have the property that $\hat e^\alpha \hat e^\beta =
 (-1)^{(\alpha,\beta)} \hat e^\beta \hat e^\alpha$, where $\hat e^\alpha$ is something
 mapping to $e^\alpha$. What do the automorphisms of $\hat e^L$ look like? We get
 \[
    1\to \underbrace{(L/2L)}_{(\ZZ/2)^{\mathrm{rank}(L)}} \to \aut (\hat e^L) \to \aut (e^L)
 \]
 for $\alpha\in L/2L$, we get the map $\hat e^\beta \to (-1)^{(\alpha,\beta)}\hat
 e^\beta$. The map turns out to be onto, and the group $\aut(e^L)$ contains the
 reflection group of the lattice. This extension is usually non-split.

 Now the Lie algebra is $L\oplus \{$1 dimensional spaces spanned by $(\hat e^\alpha,-\hat
 e^\alpha)\}$ for $\alpha^2=2$ with the convention that $-\hat e^\alpha$ ($-1$ in the
 vector space) is $-\hat e^\alpha$ (-1 in the group $\hat e^L$). Now define a Lie
 bracket by the ``obvious rules'' $[\alpha,\beta]=0$ for $\alpha,\beta \in L$ (the
 Cartan subalgebra is abelian), $[\alpha,\hat e^\beta] = (\alpha,\beta)\hat e^\beta$ ($\hat
 e^\beta$ is in the root space of $\beta$), and $[\hat e^\alpha,\hat e^\beta]=0$ if
 $(\alpha,\beta)\ge 0$ (since $(\alpha+\beta)^2>2$), $[\hat e^\alpha,\hat e^\beta]
 = \hat e^\alpha \hat e^\beta$ if $(\alpha,\beta)<0$ (product in the group $\hat e^L$), and $[\hat
 e^\alpha,(\hat e^\alpha)^{-1}]=\alpha$.

 \begin{theorem}
   Assume $L$ is positive definite. Then this Lie bracket forms a Lie algebra (so it
   is skew and satisfies Jacobi).
 \end{theorem}
 \begin{proof}
   Easy but tiresome, because there are a lot of cases; let's do them (or most of
   them).

   We check the Jacobi identity: We want $[[a,b],c]+[[b,c],a]+[[c,a],b]=0$
   \begin{enumerate}
     \item all of $a,b,c$ in $L$. Trivial because all brackets are zero.
     \item two of $a,b,c$ in $L$. Say $\alpha,\beta,e^\gamma$
     \[
        \underbrace{[[\alpha,\beta],e^\gamma]}_0+\underbrace{[[\beta,e^\gamma],\alpha]}_{(\beta,\alpha)(-\alpha,\beta)e^\gamma}+[[e^\gamma,\alpha],\beta]
     \]
     and similar for the third term, giving a sum of 0.

     \item one of $a,b,c$ in $L$. $\alpha,e^\beta,e^\gamma$. $e^\beta$ has weight
     $\beta$ and $e^\gamma$ has weight $\gamma$ and $e^\beta e^\gamma$ has weight
     $\beta+\gamma$. So check the cases, and you get Jacobi:
\begin{align*}
  [[\alpha,e^\beta],e^\gamma] &= (\alpha,\beta)[e^\beta,e^\gamma] \\
  [ [e^\beta,e^\gamma],\alpha] &= -[\alpha,[e^\beta,e^\gamma]] =
  -(\alpha,\beta+\gamma)[e^\beta,e^\gamma] \\
  [ [e^\gamma,\alpha],e^\beta] &= -[ [\alpha,e^\gamma],e^\beta] =
  (\alpha,\gamma)[e^\beta,e^\gamma],
\end{align*}
  so the sum is zero.

     \item none of $a,b,c$ in $L$. This is the really tiresome one,
     $e^\alpha,e^\beta,e^\gamma$. The main point of going through this is to show that
     it isn't as tiresome as you might think. You can reduce it to two or three cases.
     Let's make our cases depending on $(\alpha,\beta)$, $(\alpha,\gamma)$,
     $(\beta,\gamma)$.
     \begin{enumerate}
       \item if 2 of these are 0, then all the $[[\ast,\ast],\ast]$ are zero.

       \item $\alpha=-\beta$. By case a, $\gamma$ cannot be orthogonal to them, so say
       $(\alpha,\gamma)=1$ $(\gamma,\beta)=-1$; adjust so that $e^\alpha e^\beta=1$,
       then calculate
\begin{align*}
  [ [e^\gamma,e^\beta],e^\alpha] - [ [e^\alpha,e^\beta],e^\gamma] + [
  [e^\alpha,e^\gamma],e^\beta]
  &= e^\alpha e^\beta e^\gamma - (\alpha,\gamma)e^\gamma + 0\\
  &= e^\gamma - e^\gamma = 0.
\end{align*}

       \item $\alpha=-\beta=\gamma$, easy because $[e^\alpha,e^\gamma]=0$ and
       $[[e^\alpha,e^\beta],e^\gamma] = -[ [e^\gamma,e^\beta],e^\alpha] $

       \item We have that each of the inner products is 1, 0 or $-1$. If some
        $(\alpha,\beta)=1$, all brackets are 0.
     \end{enumerate}
     This leaves two cases, which we'll do next time
   \end{enumerate}

 \end{proof}
