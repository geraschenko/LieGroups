 \stepcounter{lecture}
 \setcounter{lecture}{6}
 \sektion{Lecture 6 - Hopf Algebras} \index{Hopf Algebras|idxbf}

 Last time: We showed that a finite dimensional Lie algebra $\g$ uniquely determines a
 connected simply connected Lie group. We also have a ``map'' in the other direction
 (taking tangent spaces). So we have a nice correspondence between Lie algebras and
 connected simply connected Lie groups.

 There is another nice kind of structure: Associative algebras. How do these relate to
 Lie algebras and groups?

 Let $\Gamma$ be a finite group and let $\CC[\Gamma]:=\{\sum_{g} c_gg|g\in \Gamma,
 c_g\in\CC\}$ be the $\CC$ vector space with basis $\Gamma$. We can make $\CC[\Gamma]$
 into an associative algebra by taking multiplication to be the multiplication in
 $\Gamma$ for basis elements and linearly extending this to the rest of
 $\CC[\Gamma]$.\footnote{``If somebody speaks Danish\index{Danish}, I would be happy
 to take lessons.''}

 \begin{remark}Recall that the tensor product V and W is the linear span of elements
 of the form $v\otimes w$, modulo some linearity relations. If $V$ and $W$ are
 infinite dimensional, we will look at the \emph{algebraic tensor product} of $V$ and
 $W$, i.e. we only allow finite sums of the form $\sum a_i\otimes b_i$.
 \end{remark}

 We have the following maps
 \begin{itemize}
 \item[] Comultiplication\index{comultiplication}: $\Delta:\CC[\Gamma]\to
  \CC[\Gamma]\otimes \CC[\Gamma],$ given by $\Delta(\sum x_g
  g)=\sum x_gg\otimes g$
 \item[] Counit\index{counit}:  $\varepsilon:\CC[\Gamma]\to \CC$, given by $\varepsilon(\sum x_g
  g)=\sum x_g$.
 \item[] Antipode\index{antipode}: $S:\CC[\Gamma]\to \CC[\Gamma]$ given by $S(\sum x_g
  g)=\sum x_gg^{-1}$.
 \end{itemize}

 You can check that
 \begin{itemize}
 \item $\Delta(xy)=\Delta(x)\Delta(y)$ (i.e.\ $\Delta$ is an algebra homomorphism),

 \item $(\Delta\otimes\id)\circ \Delta = (\id\otimes\Delta)\circ \Delta$. (follows
 from the associativity of $\otimes$),

 \item $\varepsilon(xy)=\varepsilon(x)\varepsilon(y)$ (i.e.\ $\varepsilon$ is an
 algebra homomorphism),

 \item $S(xy)=S(y)S(x)$ (i.e. $S$ is an algebra antihomomorphism).
 \end{itemize}

 Consider
 \[
    \CC[\Gamma]\xrightarrow{\Delta} \CC[\Gamma]\otimes \CC[\Gamma]
    \xrightarrow{S\otimes\id,\id\otimes S} \CC[\Gamma]\otimes
    \CC[\Gamma]\xrightarrow{m} \CC[\Gamma].
 \]
 You get
 \[
   m(S\otimes \id)\Delta(g) = m(g^{-1}\otimes g) = e
 \]
 so the composition sends $\sum x_g g$ to $(\sum_g x_g) e = \varepsilon(x)1_A$.

 So we have
 \begin{enumerate}
 \item $A=\CC[\Gamma]$ an associative algebra with $1_A$
 \item $\Delta:A\to A\otimes A$ which is coassociative and is a homomorphism of
 algebras
 \item $\varepsilon: A\to \CC$ an algebra homomorphism, with $(\varepsilon\otimes
 \id)\Delta = (\id\otimes \varepsilon)\Delta = \id$.
 \end{enumerate}

 \begin{definition}Such an $A$ is called a \emph{bialgebra}\index{bialgebra|idxbf}, with
  comultiplication $\Delta$ and counit $\varepsilon$.
 \end{definition}

 We also have $S$, the antipode, which is an algebra anti-automorphism, so it is a
 linear isomorphism with $S(ab)=S(b)S(a)$, such that
 \[\xymatrix{
  A\otimes A \ar[rr]^{S\otimes\id}_{\id\otimes S} & & A\otimes A \ar[d]^m \\
  A\ar[u]^{\Delta}\ar[r]^{\varepsilon} & \CC\ar[r]^{1_A} & A
 }\]
 \begin{definition}
   A bialgebra with an antipode is a \emph{Hopf algebra}.
 \end{definition}

  If $A$ is finite dimensional, let $A^*$ be the dual vector space. Define the
  multiplication, $\Delta_*$, $S_*$, $\varepsilon_*, 1_{A^*}$ on $A^*$ in the following
  way:
 \begin{itemize}
 \item $lm(a):=(l\otimes m)(\Delta a)$ for all $l,m\in A^*$
 \item $\Delta_*(l)(a\otimes b):= l(ab)$
 \item $S_*(l)(a) := l(S(a))$
 \item $\varepsilon_*(l) := l(1_A)$
 \item $1_{A^*}(a) := \varepsilon(a)$
 \end{itemize}

 \begin{theorem}
   $A^*$ is a Hopf algebra with this structure, and we say it is \emph{dual to} $A$.
   If $A$ is finite dimensional, then $A^{**}=A$.
 \end{theorem}
 \begin{exercise}
   Prove it.
 \end{exercise}

 We have an example of a Hopf algebra ($\CC[\Gamma]$), what is the dual Hopf
 algebra?\footnote{ If you want to read more, look at S.~Montgomery's \textsl{Hopf
 algebras}, AMS, early 1990s. \cite{Montgomery}} Let's compute $A^*=\CC[\Gamma]^*$.

 Well, $\CC[\Gamma]$ has a basis
 $\{g\in \Gamma\}$. Let $\{\delta_g\}$ be the dual basis, so $\delta_g(h)=0$ if $g\neq
 h$ and 1 if $g=h$. Let's look at how we multiply such things
 \begin{itemize}
 \item $\delta_{g_1}\delta_{g_2}(h) = (\delta_{g_1}\otimes \delta_{g_2})(h\otimes
 h) = \delta_{g_1}(h)\delta_{g_2}(h)$.
 \item $\Delta_*(\delta_g)(h_1\otimes h_2) = \delta_g(h_1h_2)$
 \item $S_*(\delta_g)(h) = \delta_g(h^{-1})$
 \item $\varepsilon_*(\delta_g) = \delta_g(e) = \delta_{g,e}$
 \item $1_{A^*}(h) = 1$.
 \end{itemize}

 It is natural to think of $A^*$ as the set of functions $\Gamma\to \CC$, where
 $(\sum x_g\delta_g)(h) = \sum x_g\delta_g(h)$. Then we can think about functions
 \begin{itemize}
 \item $(f_1 f_2)(h) = f_1(h)f_2(h)$
 \item $\Delta_*(f)(h_1\times h_2) = f(h_1h_2)$
 \item $S_*(f)(h) = f(h^{-1})$
 \item $\varepsilon_*(f) = f(e)$
 \item $1_{A^*} = 1$ constant.
 \end{itemize}
 So this is the Hopf algebra $C(\Gamma)$, the space of functions on $\Gamma$. If
 $\Gamma$ is any affine algebraic group, then $C(\Gamma)$ is the space of polynomial
 functions on $\Gamma$, and all this works. The only concern is that we need
 $C(\Gamma\times \Gamma)\cong C(\Gamma)\otimes C(\Gamma)$, which we only have in the
 finite dimensional case; you have to take completions of tensor products
 otherwise.

 So we have the notion of a bialgebra (and duals), and the notion of a Hopf algebra
 (and duals). We have two examples: $A=\CC[\Gamma]$ and $A^*=C(\Gamma)$. A natural
 question is, ``what if $\Gamma$ is an infinite group or a Lie group?'' and ``what are
 some other examples of Hopf algebras?''

 Let's look at some infinite dimensional examples. If $A$ is an infinite dimensional
 Hopf algebra, and $A\otimes A$ is the algebraic tensor product (finite linear
 combinations of formal $a\otimes b\ $ s). Then the comultiplication should be $\Delta:
 A\to A\otimes A$. You can consider cases where you have to take some completion of
 the tensor product with respect to some topology, but we won't deal with this kind of
 stuff. In this case, $A^*$ is too big, so instead of the notion of the dual Hopf
 algebra, we have dual pairs.
 \begin{definition}
   A \emph{dual pairing}\index{dual pairing} of Hopf algebras $A$ and $H$ is a pair with a
   bilinear map $\langle\ ,\,\rangle:A\otimes H \to \CC$ which is nondegenerate such
   that
   \begin{itemize}
   \item[(1)] $\langle \Delta a, l\otimes m\rangle =\langle a,lm\rangle$
   \item[(2)] $\langle ab,l \rangle = \langle a\otimes b, \Delta_* l \rangle$
   \item[(3)] $\langle S a, l \rangle = \langle a,S_* l \rangle$
   \item[(4)] $\varepsilon(a) = \langle a,1_{H} \rangle, \varepsilon_{H}(l) = \langle 1_A,l
   \rangle$
   \end{itemize}
 \end{definition}

 Exmaple: $A=\CC[x]$, then what is $A^*$? You can evaluate a polynomial at 0, or you
 can differentiate some number of times before you evaluate at 0. $A^* = $ span of
 linear functionals on polynomial functions of $\CC$ of the form
 \[
    l_n(f) = \left( \der{}{x}\right)^n f(x){\big |}_{x=0}.
 \]
 A basis for $\CC[x]$ is $1,x^n$ with $n\ge 1$, and we have
 \[
    l_n(x^m) = \left\{
    \begin{array}{cc}
      m! & ,n=m\\
      0 & ,n\neq m
    \end{array}\right.
 \]
 What is the Hopf algebra structure on $A$? We already have an algebra with identity.
 Define $\Delta(x) = x\otimes 1+1\otimes x$ and extend it to an algebra homomorphism,
 then it is clearly coassociative. Define $\varepsilon(1)=1$ and $\varepsilon(x^n)=0$
 for all $n\ge 1$. Define $S(x)=-x$, and extend to an algebra homomorphism. It is easy
 to check that this is a Hopf algebra.

 Let's compute the Hopf algebra structure on $A^*$. We have
 \begin{align*}
   l_nl_m(x^N) &= (l_n\otimes l_m)(\Delta(x^N)) \\
            &= (l_n\otimes l_m) (\sum \binom{N}{k} x^{N-k}\otimes x^k)
 \end{align*}
 \begin{exercise}
   Compute this out. The answer is that $A^*=\CC[y=\der{}{x}]$, and the Hopf algebra
   structure is the same as $A$.
 \end{exercise}
 This is an example of a dual pair: $A=\CC[x], H=\CC[y]$, with $\langle x^n,y^m\rangle
 = \delta_{n,m} m!$.

 Summary: If $A$ is finite dimensional, you get a dual, but in the infinite
 dimensional case, you have to use dual pairs.

 \subsektion{The universal enveloping algebra} \index{universal enveloping algebra|(}

 The idea is to construct a map from Lie algebras to associative algebras so that the
 representation theory of the associative algebra is equivalent to the representation
 theory of the Lie algebra.

  1) let $V$ be a vector space, then we can form the free associative algebra (or
 tensor algebra) of $V$: $T(V) = \CC\oplus (\oplus_{n\ge 1} V^{\otimes n})$. The
 multiplication is given by concatenation: $(v_1\otimes\cdots\otimes v_n)\cdot
 (w_1\otimes\cdots\otimes w_m) = v_1\otimes\cdots\otimes v_n\otimes w_1\otimes\cdots
 w_m$. It is graded: $T_n(V)T_m(V)\subseteq T_{n+m}(V)$. It is also a Hopf algebra,
 with $\Delta(x)=x\otimes 1+1\otimes x$, $S(x)=-x$, $\varepsilon(1)=1$ and
 $\varepsilon(x)=0$. If you choose a basis $e_1,\dots, e_n$ of $V$, then $T(V)$ is the
 free associative algebra $\langle e_1,\dots, e_n\rangle$. This algebra is
 $\ZZ_+$-graded: $T(V) = \oplus_{n\ge 0} T_n(V)$, where the degree of 1 is zero and the
 degree of each $e_i$ is 1. It is also a $\ZZ$-graded bialgebra:
 $\Delta(T_n(V))\subseteq \oplus (T_i\oplus T_{n-i}), S(T_n(V))\subset T_n(V),
 \varepsilon : T(V)\rightarrow \CC$ is a mapping of graded spaces $((\CC)_n=\{0\})$.
 \begin{definition}
   Let $A$ be a Hopf algebra. Then a two-sided ideal $I\subseteq A$ is a \emph{Hopf
   ideal}\index{Hopf ideal} if $\Delta(I)\subseteq A\otimes I + I\otimes A$, $S(I)=I$,
   and $\varepsilon(I)=0$.
 \end{definition}
 You can check that the quotient of a Hopf algebra by a Hopf ideal is a Hopf algebra
 (and that the kernel of a map of Hopf algebras is always a Hopf ideal).

 \begin{exercise}
   Show that $I_0 = \langle v\otimes w-w\otimes v| v,w\in V=T_1(V)\subseteq T(V)\rangle$
   is a homogeneous Hopf ideal.
 \end{exercise}
 \begin{corollary}
   $S(V) = T(V)/I_0$ is a graded Hopf algebra.
 \end{corollary}
  Choose a basis $e_1,\dots, e_n$ in $V$, so that $T(V)=\langle e_1,\dots, e_n\rangle$
  and $S(V) = \langle e_1,\dots, e_n\rangle/\langle e_ie_j-e_je_i\rangle$
  \begin{exercise}
    Prove that the Hopf algebra $S(V)$ is isomorphic to $\CC[e_1]\otimes\cdots\otimes
    \CC[e_n]$.
  \end{exercise}
  \begin{remark}
    From the discussion of $\CC[x]$, we know that $S(V)$ and $S(V^*)$ are dual.
  \end{remark}
  \begin{exercise}
    Describe the Hopf algebra structure on $T(V^*)$ that is determined by the
    pairing $\langle v_1\otimes \cdots\otimes v_n, l_1\otimes \cdots\otimes l_m\rangle
    = \delta_{m,n} l_1(v_1)\cdots l_n(v_n)$. (free coalgebra of $V^*$)
  \end{exercise}

  Now assume that $\g$ is a Lie algebra.
  \begin{definition}
    The universal enveloping algebra of $\g$ is $U(\g) = T(\g)/\langle x\otimes
    y-y\otimes x - [x,y]\rangle$.
  \end{definition}
    Exercise: prove that $\langle x\otimes y-y\otimes x - [x,y]\rangle$ is a Hopf
    ideal.
  \begin{corollary}
    $U\g$ is a Hopf algebra.
  \end{corollary}

  If $e_1,\dots, e_n$ is a basis for $V$. $U\g = \langle e_1,\dots, e_n| e_ie_j-e_je_i
  = \sum_k c_{ij}^k e_k\rangle$, where $c_{ij}^k$ are the structure constants of $[\ ,\,]$.
  \begin{remark}
    The ideal $\langle e_ie_j-e_je_i\rangle$ is homogeneous, but $\langle x\otimes
 y-y\otimes x -
    [x,y]\rangle$ is not, so $U\g$ isn't graded, but it is \emph{filtered}.
  \end{remark}
