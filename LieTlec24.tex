 \stepcounter{lecture}
 \setcounter{lecture}{24}
 \sektion{Lecture 24}

 Last time we constructed the sequences
 \[
    1\to K^\times \to \Gamma_V(K)\to O_V(K)\to 1
 \]
 \[
    1\to \pm 1\to \pin_V(K)\to O_V(K)\xrightarrow{N} K^\times/(K^\times )^2
 \]
 \[
    1\to \pm 1 \to \spin_V(K) \to SO_V(K) \xrightarrow{N} K^\times/(K^\times)^2
 \]

 \subsektion{Spin representations of Spin and Pin groups} Notice that
 $\pin_V(K)\subseteq C_V(K)^\times$, so any module over $C_V(K)$ gives a representation of
 $\pin_V(K)$. We already figured out that $C_V(K)$ are direct sums of matrix algebras
 over $\RR,\CC$, and $\HH$.

 What are the representations (modules) of complex Clifford algebras? Recall that
 $C_{2n}(\CC)\cong \MM_{2^n}(\CC)$, which has a representations of dimension $2^n$,
 which is called the spin representation of $\pin_V(K)$ and $C_{2n+1}(\CC)\cong
 \MM_{2^n}(\CC)\times \MM_{2^n}(\CC)$, which has 2 representations, called the spin
 representations of $\pin_{2n+1}(K)$.

 What happens if we restrict these to $\spin_V(\CC)\subseteq \pin_V(\CC)$? To do that,
 we have to recall that $C^0_{2n}(\CC)\cong \MM_{2^{n-1}}(\CC)\times
 \MM_{2^{n-1}}(\CC)$ and $C^0_{2n+1}(\CC)\cong \MM_{2^n}(\CC)$. So in EVEN dimensions
 $\pin_{2n}(\CC)$ has 1 spin representation of dimension $2^n$ splitting into 2 HALF
 SPIN representations of dimension $2^{n-1}$ and in ODD dimensions, $\pin_{2n+1}(\CC)$
 has 2 spin representations of dimension $2^n$ which become the same on restriction to
 $\spin_V(\CC)$.

 Now we'll give a second description of spin representations. We'll just do the
 even dimensional case (odd is similar). Say $\dim V=2n$, and say we're over $\CC$.
 Choose an orthonormal basis $\gamma_1,\dots, \gamma_{2n}$ for $V$, so that
 $\gamma_i^2=1$ and $\gamma_i\gamma_j = -\gamma_j\gamma_i$. Now look at the group $G$
 generated by $\gamma_1,\dots, \gamma_{2n}$, which is finite, with order $2^{1+2n}$
 (you can write all its elements explicitly). You can see that representations of
 $C_V(\CC)$ correspond to representations of $G$, with $-1$ acting as $-1$ (as opposed
 to acting as 1). So another way to look at representations of the Clifford algebra,
 you can look at representations of $G$.

 Let's look at the structure of $G$:
 \begin{itemize}
 \item[(1)] The center is $\pm 1$. This uses the fact that we are in even dimensions,
 lest $\gamma_1\cdots \gamma_{2n}$ also be central.

 \item[(2)] The conjugacy classes: 2 of size 1 (1 and $-1$), $2^{2n}-1$ of size 2
 ($\pm \gamma_{i_1}\cdots \gamma_{i_{n}}$), so we have a total of $2^{2n}+1$ conjugacy
 classes, so we should have that many representations. $G/$center is abelian,
 isomorphic to $(\ZZ/2\ZZ)^{2n}$, which gives us $2^{2n}$ representations of dimension 1, so there is
 only one more left to find! We can figure out its dimension by recalling that the sum
 of the squares of the dimensions of irreducible representations gives us the order of
 $G$, which is $2^{2n+1}$. So $2^{2n}\times 1^1+1\times d^2=2^{2n+1}$, where $d$ is
 the dimension of the mystery representation. Thus, $d=\pm 2^n$, so $d=2^n$. Thus,
 $G$, and therefore $C_V(\CC)$, has an irreducible representation of dimension $2^n$
 (as we found earlier in another way).
 \end{itemize}

 \begin{example}
   Consider $O_{2,1}(\RR)$. As before, $O_{2,1}(\RR)\cong SO_{2,1}(\RR)\times (\pm
   1)$, and $SO_{2,1}(\RR)$ is not connected: it has two components, separated by the
   spinor norm $N$. We have maps
   \[
    1\to \pm 1 \to \spin_{2,1}(\RR)\to SO_{2,1}(\RR) \xrightarrow{N} \pm 1.
    \]
    $\spin_{2,1}(\RR)\subseteq C^*_{2,1}(\RR) \cong \MM_2(\RR)$, so $\spin_{2,1}(\RR)$
   has one 2 dimensional spin representation. So there is a map $\spin_{2,1}(\RR)\to
   SL_2(\RR)$; by counting dimensions and such, you can show it is an isomorphism. So
   $\spin_{2,1}(\RR)\cong SL_2(\RR)$.
 \end{example}
 Now let's look at some 4 dimensional orthogonal groups
 \begin{example}
   Look at $SO_4(\RR)$, which is compact. It has a complex spin representation of
   dimension $2^{4/2}=4$, which splits into two half spin representations of dimension
   2. We have the sequence
   \[
    1\to \pm 1\to \spin_4(\RR)\to SO_4(\RR)\to 1 \qquad (N=1)
   \]
   $\spin_4(\RR)$ is also compact, so the image in any complex representation is
   contained in some unitary group. So we get two maps $\spin_4(\RR)\to SU(2)\times
   SU(2)$, and both sides have dimension 6 and centers of order 4. Thus, we find that
   $\spin_4(\RR)\cong SU(2)\times SU(2) \cong S^3\times S^3$, which give you the two
   half spin representations.
 \end{example}
 So now we've done the positive definite case.
 \begin{example}
   Look at $SO_{3,1}(\RR)$. Notice that $O_{3,1}(\RR)$ has four components
   distinguished by the maps $\det,N\to \pm 1$. So we get
   \[
    1\to \pm 1 \to \spin_{3,1}(\RR)\to SO_{3,1}(\RR)\xrightarrow{N} \pm 1 \to 1
   \]
   We expect 2 half spin representations, which give us two homomorphisms
   $\spin_{3,1}(\RR)\to SL_2(\CC)$. This time, each of these homomorphisms is an
   isomorphism (I can't think of why right now). The $SL_2(\CC)$s are double covers of
   simple groups. Here, we don't get the splitting into a product as in the positive
   definite case. This isomorphism is heavily used in quantum field theory because
   $\spin_{3,1}(\RR)$ is a double cover of the connected component of the Lorentz
   group (and $SL_2(\CC)$ is easy to work with). Note also that the center of
   $\spin_{3,1}(\RR)$ has order 2, not 4, as for $\spin_{4,0}(\RR)$. Also note that
   the group $PSL_2(\CC)$ acts on the compactified $\CC\cup \{\infty\}$ by $\matrix
   abcd (\tau) = \frac{a\tau + b}{c\tau + d}$. Subgroups of this group are called
   KLEINIAN groups. On the other hand, the group $SO_{3,1}(\RR)^+$ (identity
   component) acts on $\HH^3$ (three dimensional hyperbolic space). To see this, look
   at
 \[\begin{xy}
   (1,1);(.29,.29) **@{-}; (-.75,-.75) **@{.}; (-1,-1) **@{-},
   (-1,1);(-.29,.29) **@{-}; (.75,-.75) **@{.}; (1,-1) **@{-},
   %%%%%%%%%%%%%%%%%%
   (1,1.5);(-1,1.5) **\crv{(1.2,1.9)&(-1.2,1.9)},
   (1,1.5);(-1,1.5) **\crv{(.7,1.2)&(-.7,1.2)},
   (1,1.5);(-1,1.5) **\crv{(.5,.7)&(-.5,.7)},
   %%%%%%%%%%%%%%%%%%
   (1,1);(.78,1.22) **\crv{(1.06,1.12)},
   (-1,1);(-.78,1.22) **\crv{(-1.06,1.12)},
   (1,1);(-1,1) **\crv{(.7,.7)&(-.7,.7)},
   (1,-1);(-1,-1) **\crv{(1.2,-1.4)&(-1.2,-1.4)},
   %%%%%%%%%%%%%%%%%%
   (1,-1.5);(-1,-1.5) **\crv{(1.2,-1.9)&(-1.2,-1.9)},
   (1,-1.5);(.78,-1.22) **\crv{(.9,-1.325)};
   (-.78,-1.22) **\crv{~*=<2pt>@{.} (.34,-.8)&(-.34,-.8)};
   (-1,-1.5) **\crv{(-.9,-1.325)},
   %%%%%%%%%%%%%%%%%%
   (1.1,.5);(.75,.74) **\crv{(1.18,.66)},
   (-1.1,.5);(-.75,.74) **\crv{(-1.18,.66)},
   (1.1,.5);(-1.1,.5) **\crv{(.8,.2)&(-.8,.2)},
   (1.1,-.5);(-1.1,-.5) **\crv{(1.3,-.9)&(-1.3,-.9)},
   (1.1,-.5);(1.1,.5) **\crv{(.5,0)},
   (-1.1,-.5);(-1.1,.5) **\crv{(-.5,0)},
   %%%%%%%%%%%%%%%%%%
   \ar (2.5,-1.5) *{{}_{\text{ norm=}-1}}; (1,-1.5) *+{\,},
   \ar@/_1mm/ (2.5,1.2) *+{{}_{\text{norm}=0}}; (1,1) *+{\,},
   \ar@/_1mm/ (2.5,1.7) *+{{}_{\text{norm}=-1}}; (1,1.5) *+{\,},
   \ar@/_1mm/ (2.5,.7) *+{{}_{\text{norm}=1}}; (1.1,.5) *+{\,},
 \end{xy}\]
   One sheet of norm $-1$ hyperboloid is isomorphic to $\HH^3$ under the induced
   metric. In fact, we'll define hyperbolic space that way. If you're a topologist,
   you're very interested in hyperbolic 3-manifolds, which are $\HH^3/$(discrete
   subgroup of $SO_{3,1}(\RR)$). If you use the fact that $SO_{3,1}(\RR)\cong
   PSL_2(\RR)$, then you see that these discrete subgroups are in fact Klienian
   groups.
 \end{example}

 There are lots of exceptional isomorphisms in small dimension, all of which are very
 interesting, and almost all of them can be explained by spin groups.

 \begin{example}
   $O_{2,2}(\RR)$ has 4 components (given by $\det, N$); $C^0_{2,2}(\RR)\cong
   \MM_2(\RR)\times \MM_2(\RR)$, which induces an isomorphism $\spin_{2,2}(\RR)\to
   SL_2(\RR)\times SL_2(\RR)$, which give you the two half spin representations. Both
   sides have dimension 6 with centers of order 4. So this time we get two non-compact
   groups. Let's look at the fundamental group of $SL_2(\RR)$, which is $\ZZ$, so the
   fundamental group of $\spin_{2,2}(\RR)$ is $\ZZ\oplus \ZZ$. As we recall,
   $\spin_{4,0}(\RR)$ and $\spin_{3,1}(\RR)$ were both simply connected. This shows
   that SPIN GROUPS NEED NOT BE SIMPLY CONNECTED. So we can take covers of it. What do
   the corresponding covers (e.g.\ the universal cover) of $\spin_{2,2}(\RR)$ look
   like? This is hard to describe because for FINITE dimensional complex
   representations, you get finite dimensional representations of the Lie algebra $L$,
   which correspond to the finite dimensional representations of $L\otimes \CC$, which
   correspond to the finite dimensional representations of $L'=$ Lie algebra of
   $\spin_{4,0}(\RR)$, which correspond to the finite dimensional representations of
   $\spin_{4,0}(\RR)$, which has no covers because it is simply connected. This means
   that any finite dimensional representation of a cover of $\spin_{2,2}(\RR)$
   actually factors through $\spin_{2,2}(\RR)$. So there is no way you can talk about
   these things with finite matrices, and infinite dimensional representations are
   hard.

   To summarize, the ALGEBRAIC GROUP $\spin_{2,2}$ is simply connected (as an
   algebraic group) (think of an algebraic group as a functor from rings to groups),
   which means that it has no algebraic central extensions. However, the LIE GROUP
   $\spin_{2,2}(\RR)$ is NOT simply connected; it has fundamental group $\ZZ\oplus
   \ZZ$. This problem does not happen for COMPACT Lie groups (where every finite cover
   is algebraic).
 \end{example}

 We've done $O_{4,0}, O_{3,1},$ and $O_{2,2}$, from which we can obviously get
 $O_{1,3}$ and $O_{0,4}$. Note that $O_{4,0}(\RR)\cong O_{0,4}(\RR)$,
 $SO_{4,0}(\RR)\cong SO_{0,4}(\RR)$, $\spin_{4,0}(\RR)\cong \spin_{0,4}(\RR)$.
 However, $\pin_{4,0}(\RR)\not\cong \pin_{0,4}(\RR)$. These two are hard to
 distinguish. We have
 \[\xymatrix{
    \pin_{4,0}(\RR)\ar[d] & \pin_{0,4}(\RR)\ar[d]\\
    O_{4,0}(\RR) \ar@{}[r]|{=} & O_{0,4}(\RR)
 }\]
 Take a reflection (of order 2) in $O_{4,0}(\RR)$, and lift it to the $\pin$ groups.
 What is the order of the lift? The reflection vector $v$, with $v^2=\pm 1$ lifts to
 the element $v\in \Gamma_V(\RR)\subseteq C^*_V(\RR)$. Notice that $v^2=1$ in the case of  $\RR^{4,0}$
 and $v^2=-1$ in the case of $\RR^{0,4}$, so in $\pin_{4,0}(\RR)$, the reflection lifts to
 something of order 2, but in $\pin_{0,4}(\RR)$, you get an element of order 4!. So
 these two groups are different.

 Two groups are \emph{isoclinic} if they are confusingly similar. A similar
 phenomenon is common for groups of the form $2\cdot G\cdot 2$, which means it has a
 center of order 2, then some group $G$, and the abelianization has order $2$. Watch
 out.

 \begin{exercise}
   $\spin_{3,3}(\RR)\cong SL_4(\RR)$.
 \end{exercise}

 \subsektion{Triality} This is a special property of 8 dimensional orthogonal groups.
 Recall that $O_8(\CC)$ has the Dynkin diagram $D_4$, which has a symmetry of order
 three:

    \[\begin{xy}
     (0,0) *\cir<2pt>{};
     a(60)="1" *\cir<2pt>{} **@{-},
     a(180)="2" *\cir<2pt>{} **@{-},
     a(-60)="3" *\cir<2pt>{} **@{-},
     \ar@/_2ex/ "1" *+{\ };"2" *+{\ }
     \ar@/_2ex/ "2" *+{\ };"3" *+{\ }
     \ar@/_2ex/ "3" *+{\ };"1" *+{\ }
   \end{xy}\]

 But $O_8(\CC)$ and $SO_8(\CC)$ do NOT have corresponding symmetries of order three.
 The thing that does have the symmetry of order three is the spin group! The group
 $\spin_8(\RR)$ DOES have ``extra'' order three symmetry. You can see it as follows.
 Look at the half spin representations of $\spin_8(\RR)$. Since this is a spin group
 in even dimension, there are two. $C_{8,0}(\RR)\cong \MM_{2^{8/2 -1}}(\RR)\times \MM_{2^{8/2
 -1}}(\RR) \cong \MM_8(\RR)\times \MM_8(\RR)$. So $\spin_8(\RR)$ has two 8 dimensional
 real half spin representations. But the spin group is compact, so it preserves some
 quadratic form, so you get 2 homomorphisms $\spin_8(\RR)\to SO_8(\RR)$. So
 $\spin_8(\RR)$ has THREE 8 dimensional representations: the half spins, and the one
 from the map to $SO_8(\RR)$. These maps $\spin_8(\RR)\to SO_8(\RR)$ lift to Triality automorphisms $\spin_8(\RR)\to \spin_8(\RR)$.  The center of
 $\spin_8(\RR)$ is $(\ZZ/2)+ (\ZZ/2)$ because the center of the Clifford group is $\pm
 1, \pm \gamma_1\cdots\gamma_8$. There are 3 non-trivial elements of the center, and
 quotienting by any of these gives you something isomorphic to $SO_8(\RR)$. This is
 special to 8 dimensions.

 \subsektion{More about Orthogonal groups} Is $O_V(K)$ a simple group? NO, for the
 following reasons:
 \begin{itemize}
 \item[(1)] There is a determinant map $O_V(K)\to \pm 1$, which is usually onto, so it
 can't be simple.

 \item[(2)] There is a spinor norm map $O_V(K)\to K^\times/(K^\times)^2$

 \item[(3)] $-1\in $ center of $O_V(K)$.

 \item[(4)] $SO_V(K)$ tends to split if $\dim V=4$, abelian if $\dim V=2$, and trivial
 if $\dim V=1$.
 \end{itemize}
 It turns out that they are usually simple apart from these four reasons why they're
 not. Let's mod out by the determinant, to get to $SO$, then look at $\spin_V(K)$,
 then quotient by the center, and assume that $\dim V\ge 5$. Then this is usually
 simple. The center tends to have order 1,2, or 4. If $K$ is a FINITE field, then this
 gives many finite simple groups.

 Note that $SO_V(K)$ is NOT a subgroup of $O_V(K)$, elements of determinant 1 in
 general, it is the image of $\Gamma^0_V(K)\subseteq \Gamma_V(K)\to O_V(K)$, which is
 the correct definition. Let's look at why this is right and the definition you know
 is wrong. There is a homomorphism $\Gamma_V(K)\to \ZZ/2\ZZ$, which takes
 $\Gamma^0_V(K)$ to 0 and $\Gamma^1_V(K)$ to 1 (called the DICKSON INVARIANT). It is
 easy to check that $\det(v) = (-1)^{\text{dickson invariant}(v)}$. So if the
 characteristic of $K$ is not 2, $\det=1$ is equivalent to dickson $=0$, but in
 characteristic 2, determinant is the wrong invariant (because determinant is always
 1).

 Special properties of $O_{1,n}(\RR)$ and $O_{2,n}(\RR)$. $O_{1,n}(\RR)$ acts on
 hyperbolic space $\HH^n$, which is a component of norm $-1$ vectors in $\RR^{n,1}$.
 $O_{2,n}(\RR)$ acts on the ``Hermitian symmetric space'' (Hermitian means it has a
 complex structure, and symmetric means really nice).
There are three ways to construct this space:
 \begin{itemize}
 \item[(1)] It is the set of positive definite 2 dimensional subspaces of $\RR^{2,n}$

 \item[(2)] It is the norm 0 vectors $\w$ of $\mathbb{P}\CC^{2,n}$ with $(\w,\bar \w)=0$.

 \item[(3)] It is the vectors $x+iy\in \RR^{1,n-1}$ with $y\in C$, where the cone $C$
 is the interior of the norm 0 cone.
 \end{itemize}

 \begin{exercise}
   Show that these are the same.
 \end{exercise}

 Next week, we'll mess around with $E_8$.
