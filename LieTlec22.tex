 \stepcounter{lecture}
 \setcounter{lecture}{22}
 \sektion{Lecture 22 - Clifford algebras}\index{Clifford algebra|(idxbf}

 With Lie algebras of small dimensions, there are accidental isomorphisms. Almost all
 of these can be explained with Clifford algebras and Spin groups.

 Motivational examples that we'd like to explain:
 \begin{enumerate}
   \item $SO_2(\RR) = S^1$: $S^1$ can double cover $S^1$ itself.
   \item $SO_3(\RR)$: has a simply connected double cover $S^3$.
   \item $SO_4(\RR)$: has a simply connected double cover $S^3\times S^3$.

   \item $SO_5(\CC)$: Look at $Sp_4(\CC)$, which acts on $\CC^4$ and on
   $\Lambda^2(\CC^4)$, which is 6 dimensional, and decomposes as $5\oplus 1$.
   $\Lambda^2(\CC^4)$ has a symmetric bilinear form given by $\Lambda^2(\CC^4)\otimes
   \Lambda^2(\CC^4)\to \Lambda^4(\CC^4)\simeq \CC$, and $Sp_4(\CC)$ preserves this form.
    You get that $Sp_4(\CC)$ acts on
   $\CC^5$, preserving a symmetric bilinear form, so it maps to $SO_5(\CC)$. You can
   check that the kernel is $\pm 1$. So $Sp_4(\CC)$ is a double cover of $SO_5(\CC)$.

   \item $SO_5(\CC)$: $SL_4(\CC)$ acts on $\CC^4$, and we still have our 6 dimensional
   $\Lambda^2(\CC^4)$, with a symmetric bilinear form. So you get a homomorphism
   $SL_4(\CC) \to SO_6(\CC)$, which you can check is surjective, with kernel $\pm 1$.
 \end{enumerate}
 So we have double covers $S^1$, $S^3$, $S^3\times S^3$, $Sp_4(\CC)$, $SL_4(\CC)$ of
 the orthogonal groups in dimensions 2,3,4,5, and 6, respectively. All of these look
 completely unrelated. By the end of the next lecture, we will have an understanding
 of these groups, which will be called $\spin_2(\RR)$, $\spin_3(\RR)$, $\spin_4(\RR)$,
 $\spin_5(\CC)$, and $\spin_6(\CC)$, respectively.

 \begin{example}
   We have not yet defined Clifford algebras, but here are some examples of Clifford
   algebras over $\RR$.
  \begin{itemize}
   \item $\CC$ is generated by $\RR$, together with $i$, with $i^2=-1$
   \item $\HH$ is generated by $\RR$, together with $i,j$, each squaring to $-1$, with
   $ij+ji=0$.

   \item Dirac wanted a square root for the operator $\nabla =
   \frac{\partial^2}{\partial x^2} + \frac{\partial^2}{\partial y^2}+
   \frac{\partial^2}{\partial z^2} - \frac{\partial^2}{\partial t^2}$ (the wave
   operator in 4 dimensions). He supposed that the square root is of the form $A =
   \gamma_1\pder{}{x}+\gamma_2\pder{}{y}+\gamma_3\pder{}{z}+\gamma_4\pder{}{t}$ and
   compared coefficients in the equation $A^2=\nabla$. Doing this yields
   $\gamma_1^2=\gamma_2^2=\gamma_3^2=1$, $\gamma_4^2=-1$, and
   $\gamma_i\gamma_j+\gamma_j\gamma_i=0$ for $i\neq j$.

   Dirac solved this by taking the $\gamma_i$ to be $4\times 4$ complex matrices. $A$
   operates on vector-valued functions on space-time.
 \end{itemize}
\end{example}
\begin{definition}
   A general \emph{Clifford algebra} over $\RR$ should be generated by elements
   $\gamma_1,\dots, \gamma_n$ such that $\gamma_i^2$ is some given real, and
   $\gamma_i\gamma_j+\gamma_j\gamma_i=0$ for $i\neq j$.
 \end{definition}
 \begin{definition}[better definition]
   Suppose $V$ is a vector space over a field $K$, with some quadratic
   form\footnote{$N$ is a \emph{quadratic form}\index{quadratic form} if it is a
   homogeneous polynomial of degree 2 in the coefficients with respect to some basis.}
   $N:V\to K$. Then the \emph{Clifford algebra} $C_V(K)$ is generated by the vector
   space $V$, with relations $v^2 = N(v)$.
 \end{definition}
 We know that $N(\lambda v) = \lambda^2N(v)$ and that the expression $(a,b) :=
 N(a+b)-N(a)-N(b)$ is bilinear. If the characteristic of $K$ is not 2, we have
 $N(a)=\frac{(a,a)}{2}$. Thus, you can work with symmetric bilinear forms instead of
 quadratic forms so long as the characteristic of $K$ is not 2. We'll use quadratic
 forms so that everything works in characteristic 2.
 \begin{warning}
   A few authors (mainly in index theory) use the relations $v^2=-N(v)$.
%
%   This is a terrible convention introduced by Atiyah and Bott.
%
   Some people add a factor of 2, which usually doesn't matter, but is wrong in
   characteristic 2.
 \end{warning}
 \begin{example}
   Take $V=\RR^2$ with basis $i,j$, and with $N(xi+yj)=-x^2-y^2$. Then the relations
   are $(xi+yj)^2=-x^2-y^2$ are exactly the relations for the quaternions:
   $i^2=j^2=-1$ and $(i+j)^2 = i^2+ij+ji+j^2=-2$, so $ij+ji=0$.
 \end{example}

 \begin{remark}\label{lec22Rmk:QuadForm}
   If the characteristic of $K$ is not 2,  a ``completing the square'' argument shows
   that any quadratic form is isomorphic to $c_1x_1^2+\cdots +
   c_nx_n^2$, and if one can be obtained from another other by permuting the $c_i$ and
   multiplying each $c_i$ by a non-zero square, the two forms are isomorphic.

   It follows that every quadratic form on a vector space over $\CC$
   is isomorphic to $x_1^2+\cdots +x_n^2$, and that every quadratic form on a
   vector space over $\RR$ is isomorphic to $x_1^2+\cdots + x_m^2 - x_{m+1}^2 - \cdots
   - x_{m+n}^2$ ($m$ pluses and $n$ minuses) for some $m$ and $n$. One can check that
   these forms over $\RR$ are non-isomorphic.

   We will always assume that $N$ is non-degenerate (i.e.\ that the associated bilinear
   form is non-degenerate), but one could study Clifford algebras arising from degenerate
   forms.
 \end{remark}
 \begin{warning}
   The criterion in the remark is not sufficient for classifying quadratic forms. For
   example, over the field $\FF_3$, the forms $x^2+y^2$ and $-x^2-y^2$ are isomorphic
   via the isomorphism $\matrix 111{-1}:\FF_3^2\to \FF_3^2$, but $-1$ is not a square
   in $\FF_3$.  Also, completing the square doesn't work in characteristic 2.
 \end{warning}
 \begin{remark}
   The tensor algebra $TV$ has a natural $\ZZ$-grading, and to form the Clifford
   algebra $C_V(K)$, we quotient by the ideal generated by the even elements
   $v^2-N(v)$. Thus, the algebra $C_V(K)=C_V^0(K)\oplus C_V^1(K)$ is
   $\ZZ/2\ZZ$-graded. A $\ZZ/2\ZZ$-graded algebra is called a
   \emph{superalgebra}\index{superalgebra}.
%   In light of Remark
%   \ref{lec22Rmk:QuadForm}, the quadratic form can be ``diagonalized'' when the
%   characteristic of $K$ is not 2. That is, there is a basis $v_1,\dots, v_n$ so that
%   $v_iv_j=-v_jv_i$ for $i\neq j$, so for homogeneous elements $x,y\in C_V(K)$,
%   $xy=(-1)^{|x|\, |y|}yx$, where $|x|$ and $|y|$ are the degrees of $x$ and $y$. A
%   superalgebra satisfying this relation is called \emph{supercommutative}.
%
%   odd elements do not supercommute with themselves unless $N=0$
 \end{remark}

 \underline{Problem}: Find the structure of $C_{m,n}(\RR)$, the Clifford algebra over
 $\RR^{n+m}$ with the form $x_1^2+\cdots + x_m^2 - x_{m+1}^2 - \cdots - x_{m+n}^2$.

 \begin{example}
   \begin{itemize}
   \item[]
   \item $C_{0,0}(\RR)$ is $\RR$.

   \item $C_{1,0}(\RR)$ is $\RR[\e]/(\e^2-1) = \RR(1+\e)\oplus \RR(1-\e) = \RR\oplus
   \RR$. Note that the given basis, this is a direct sum of \emph{algebras} over $\RR$.

   \item $C_{0,1}(\RR)$ is $\RR[i]/(i^2+1)=\CC$, with $i$ odd.\\

   \item $C_{2,0}(\RR)$ is
   $\RR[\alpha,\beta]/(\alpha^2-1,\beta^2-1,\alpha\beta+\beta\alpha)$. We get a
   homomorphism $C_{2,0}(\RR)\to \MM_2(\RR)$, given by $\alpha \mapsto \matrix 100{-1}$
   and $\beta \mapsto \matrix 0110$. The homomorphism is onto because the two given
   matrices generate $\MM_2(\RR)$ as an algebra. The dimension of $\MM_2(\RR)$ is 4, and
   the dimension of $C_{2,0}(\RR)$ is at most 4 because it is spanned by $1$, $\alpha$,
   $\beta$, and $\alpha\beta$. So we have that $C_{2,0}(\RR)\simeq \MM_2(\RR)$.

   \item $C_{1,1}(\RR)$ is
   $\RR[\alpha,\beta]/(\alpha^2-1,\beta^2+1,\alpha\beta+\beta\alpha)$. Again, we get an
   isomorphism with $\MM_2(\RR)$, given by $\alpha\mapsto \matrix 100{-1}$ and
   $\beta\mapsto\matrix 01{-1}0$
   \end{itemize}
   Thus, we've computed the Clifford algebras
   \[\begin{array}{cccccc}
    m\backslash n  & 0 & 1 & 2 \\
   0 & \RR & \CC & \HH \\
   1 & \RR\oplus \RR & \MM_2(\RR) \\
   2 & \MM_2(\RR)
   \end{array}\]
 \end{example}
 \begin{remark}
   If $\{v_1,\dots, v_n\}$ is a basis for $V$, then $\{v_{i_1}\cdots v_{i_k}|i_1<\cdots
   <i_k,\ k\le n\}$ spans $C_V(K)$, so the dimension of $C_V(K)$ is less than or equal to
   $2^{\dim V}$. The tough part of Clifford algebras is showing that it cannot be smaller.
 \end{remark}

 Now let's try to analyze larger Clifford algebras more systematically. What is
 $C_{U\oplus V}$ in terms of $C_U$ and $C_V$? One might guess $C_{U\oplus V} \cong
 C_U\otimes C_V$. For the usual definition of tensor product, this is false (e.g.\
 $C_{1,1}(\RR) \neq C_{1,0}(\RR)\otimes C_{0,1}(\RR)$). However, for the
 \emph{superalgebra} definition of tensor product, this is correct. The superalgebra
 tensor product is the regular tensor product of vector spaces, with the product given by
 $(a\otimes b)(c\otimes d) = (-1)^{\deg b\cdot \deg c} ac\otimes bd$ for homogeneous
 elements $a$, $b$, $c$, and $d$.

 \smallskip
 Let's specialize to the case $K=\RR$ and try to compute $C_{U\oplus V}(K)$. Assume for
 the moment that $\dim U=m$ is even. Take $\alpha_1$, $\dots$, $\alpha_m$ to be
 an orthogonal basis for $U$ and let $\beta_1,\dots, \beta_n$ to be an orthogonal basis
 for $V$. Then set $\gamma_i = \alpha_1\alpha_2\cdots \alpha_m \beta_i$. What are the
 relations between the $\alpha_i$ and the $\gamma_j$? We have
 \[
   \alpha_i\gamma_j = \alpha_i \alpha_1\alpha_2\cdots \alpha_m \beta_j
   = \alpha_1\alpha_2\cdots \alpha_m \beta_i \alpha_i = \gamma_j\alpha_i
 \]
 since $\dim U$ is even, and $\alpha_i$ anti-commutes with everything except itself.
 \begin{align*}
    \gamma_i\gamma_j &= \gamma_i\alpha_1\cdots \alpha_m \beta_j
    = \alpha_1\cdots \alpha_m \gamma_i \beta_j\\
    &= \alpha_1\cdots \alpha_m \alpha_1\cdots \alpha_m \underbrace{\beta_i \beta_j}_{-\beta_j\beta_i}
    = -\gamma_j\gamma_i\\
  \gamma_i^2 &= \alpha_1\cdots \alpha_m\alpha_1\cdots \alpha_m \beta_i \beta_i
    = (-1)^{\frac{m(m-1)}{2}} \alpha_1^2\cdots \alpha_m^2 \beta_i^2 \\
    &= (-1)^{m/2} \alpha_1^2\cdots \alpha_m^2 \beta_i^2 & \text{($m$ even)}
 \end{align*}
 So the $\gamma_i$'s commute with the $\alpha_i$ and satisfy the relations of some
 Clifford algebra. Thus, we've shown that $C_{U\oplus V} (K) \cong C_U(K)\otimes C_W(K)$,
 where $W$ is $V$ with the quadratic form multiplied by $(-1)^{\half\dim U}\alpha_1^2
 \cdots \alpha_m^2 = (-1)^{\half\dim U}\cdot\,$discriminant$(U)$, and this is the usual
 tensor product of algebras over $\RR$.

 Taking $\dim U=2$, we find that
 \begin{align*}
   C_{m+2,n}(\RR) &\cong \MM_2(\RR)\otimes C_{n,m}(\RR)\\
   C_{m+1,n+1}(\RR) &\cong \MM_2(\RR) \otimes C_{m,n}(\RR)\\
   C_{m,n+2}(\RR) &\cong \HH\otimes C_{n,m}(\RR)
 \end{align*}
 where the indices switch whenever the discriminant is positive. Using these formulas, we
 can reduce any Clifford algebra to tensor products of things like $\RR$, $\CC$, $\HH$,
 and $\MM_2(\RR)$.

 Recall the rules for taking tensor products of matrix algebras (all tensor products are
 over $\RR$).
 \begin{itemize}
   \item $\RR\otimes X \cong X$.

   \item $\CC\otimes \HH \cong \MM_2(\CC)$.

   This follows from the isomorphism $\CC\otimes C_{m,n}(\RR)\cong C_{m+n}(\CC)$.

   \item $\CC\otimes \CC\cong \CC\oplus \CC$.

   \item $\HH\otimes \HH \cong \MM_4(\RR)$.

   You can see by thinking of the action on
   $\HH\cong \RR^4$ given by $(x\otimes y)\cdot z = xzy^{-1}$.

   \item $\MM_m\bigl(\MM_n(X)\bigr) \cong \MM_{mn}(X)$.
   \item $\MM_m(X)\otimes \MM_n(Y) \cong \MM_{mn}(X\otimes Y)$.
 \end{itemize}

 Filling in the middle of the table is easy because you can move diagonally by tensoring
 with $\MM_2(\RR)$. It is easy to see that $C_{8+m,n}(\RR)\cong C_{m,n+8}(\RR) \cong
 C_{m,n}\otimes \MM_{16}(\RR)$, which gives the table a kind of mod 8 periodicity. There
 is a more precise way to state this: $C_{m,n}(\RR)$ and $C_{m',n'}(\RR)$ are \emph{super
 Morita equivalent}\index{super Morita equivalence} if and only if $m-n\equiv m'-n'\mod
 8$.

 \newpage
 \[
 \mbox{\footnotesize
 \rotatebox{-90}{
 \iflilbook
 \xymatrix @!0 @R=3em @C=5em{
  \ar@{}[d]|(.4){\mbox{\large $m$}} \ar@{}[r]|(.4){\mbox{\large $n$}}
        & 0 & 1 & 2 & 3 & 4 & 5 & 6 & 7 & 8 \\
 0 & \RR \ar@(l,l)[dddddddd]_(.82){\otimes\, \MM_{16}} \ar@(ur,ul)[rrrrrrrr]^(.85){\otimes\, \MM_{16}} &
     \CC\ar@(d,r)[dddl]^(.7){\otimes\, \MM_2} |(.47)\hole &
     \HH \ar@(d,r)[lldddd] |(.213)\hole & \HH\oplus \HH &
     \MM_2(\HH) & \MM_4(\CC) & **[l] \MM_8(\RR) & \MM_8(\RR)\oplus \MM_8(\RR) & **[r] \MM_{16}(\RR) \\
 1 & \RR\oplus \RR \ar@(r,d)[urrr]_(.7){\quad\otimes\, \HH} |(.287)\hole  \\
 2 & \MM_2(\RR) \ar@(r,d)[uurrrr] |(.378)\hole \\
 3 & \MM_2(\CC) \\
 4 & \MM_2(\HH) \\
 5 & \qquad\quad\MM_2(\HH)\oplus \MM_2(\HH) \ar[dr]^(.65){\otimes\, \MM_2} \\
 6 & \MM_4(\HH) & \qquad \MM_4(\HH)\oplus \MM_4(\HH) \ar[dr]^(.65){\otimes\, \MM_2} \\
 7 & \MM_8(\CC) & & \text{etc.}\\
 8 & \MM_{16}(\RR) \\
 9 & \MM_{16}(\RR)\oplus \MM_{16}(\RR) \\
 }
 \else
 \xymatrix @!0 @R=3em @C=5em{
  \ar@{}[d]|(.4){\mbox{\large $m$}} \ar@{}[r]|(.4){\mbox{\large $n$}}
        & 0 & 1 & 2 & 3 & 4 & 5 & 6 & 7 & 8 \\
 0 & \RR \ar@(l,l)[dddddddd]_(.82){\otimes\, \MM_{16}} \ar@(ur,ul)[rrrrrrrr]^(.85){\otimes\, \MM_{16}} &
     \CC\ar@(d,r)[dddl]^(.7){\otimes\, \MM_2} |(.485)\hole &
     \HH \ar@(d,r)[lldddd] |(.213)\hole & \HH\oplus \HH &
     \MM_2(\HH) & \MM_4(\CC) & **[l] \MM_8(\RR) & \MM_8(\RR)\oplus \MM_8(\RR) & **[r] \MM_{16}(\RR) \\
 1 & \RR\oplus \RR \ar@(r,d)[urrr]_(.7){\quad\otimes\, \HH} |(.293)\hole  \\
 2 & \MM_2(\RR) \ar@(r,d)[uurrrr] |(.379)\hole \\
 3 & \MM_2(\CC) \\
 4 & \MM_2(\HH) \\
 5 & \qquad\quad\MM_2(\HH)\oplus \MM_2(\HH) \ar[dr]^(.65){\otimes\, \MM_2} \\
 6 & \MM_4(\HH) & \qquad \MM_4(\HH)\oplus \MM_4(\HH) \ar[dr]^(.65){\otimes\, \MM_2} \\
 7 & \MM_8(\CC) & & \text{etc.}\\
 8 & \MM_{16}(\RR) \\
 9 & \MM_{16}(\RR)\oplus \MM_{16}(\RR) \\
 }
 \fi
 }}
 \]
