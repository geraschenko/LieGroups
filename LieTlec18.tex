 \stepcounter{lecture}
 \setcounter{lecture}{18}
 \sektion{Lecture 18 - Representations of Lie algebras}

 Let $\g$ be a semisimple Lie algebra over an algebraically closed field $k$ of
 characteristic 0. Then we have the root decomposition $\g = \h\oplus \bigoplus_\alpha
 \g_\alpha$. Let $V$ be a finite dimensional representation of $\g$. Because all
 elements of $\h$ are semisimple, and because Jordan decomposition is preserved, the
 elements of $\h$ can be simultaneously diagonalized. That is, we have a \emph{weight
 decomposition}\index{weight decomposition|idxbf} $V = \bigoplus_{\mu\in \h^*} V_\mu$,
 where $V_\mu = \{v\in V| hv=\mu(h)v \text{ for all } h\in \h\}$. We call $V_\mu$ a
 \emph{weight space}\index{weight space|idxbf}, and $\mu$ a
 \emph{weight}\index{weight|idxbf}. Define $P(V) = \{\mu \in \h^*| V_\mu\neq 0\}$. The
 multiplicity of a weight $\mu\in P(V)$ is $\dim V_\mu$, and is denoted $m_\mu$.
 \begin{example}
   You can take $V=k$ (the trivial representation). Then $P(V)=\{0\}$ and $m_0=1$.
 \end{example}
 \begin{example}
   If $V=\g$ and we take the adjoint representation\index{adjoint representation|idxit},
   then we have that $P(V)=\Delta\cup \{0\}$, with $m_\alpha=1$ for $\alpha\in
   \Delta$, and $m_0$ is equal to the rank of $\g$.
 \end{example}
 \begin{example}\index{sl(3)@$\sl(3)$|idxit} Let $\g=\sl(3)$. The weights of the
 adjoint representation are shown by the solid arrows (together with zero, which has
 multiplicity two).
   \[\begin{xy}
   (0,0)="c",
   \ar@{.>} a(30)     ="1" *+!LD{\varepsilon_1},
   \ar@{.>} a(150)   ="2" *+!RD{\varepsilon_2},
   \ar@{.>} a(-90)  ="3" *+!U{\varepsilon_3},
   \ar@{-->} a(-150) *+!RU{-\varepsilon_1},
   \ar@{-->} a(-30)  *+!LU{-\varepsilon_2},
   \ar@{-->} a(90) *+!D{-\varepsilon_3},
   \ar@{->} "1"-"2" *+!L{\varepsilon_1 - \varepsilon_2},
   \ar@{->} "1"-"3" *+!DL{\varepsilon_1 - \varepsilon_3},
   \ar@{->} "2"-"1" *+!R{\varepsilon_2 - \varepsilon_1},
   \ar@{->} "2"-"3" *+!DR{\varepsilon_2 - \varepsilon_3},
   \ar@{->} "3"-"1" *+!UR{\varepsilon_3 - \varepsilon_1},
   \ar@{->} "3"-"2" *+!UL{\varepsilon_3 - \varepsilon_2},
  \end{xy}\]
   The weights of the standard 3-dimensional representation are
   $\{\varepsilon_1,\varepsilon_2,\varepsilon_3\}$, shown in dotted lines.

   In general, the weights of the dual of a representation are the negatives of the
   original representation because $\langle h\phi, v\rangle$ is defined as $-\langle
   \phi,hv\rangle$. Thus, the dashed lines show the weights of the dual of the
   standard representation.
 \end{example}
%   \anton{Note that the sum of these is a $G_2$. (We noted earlier that $G_2$ is the sum of
%   an $\sl(3)$, a standard representation of $\sl(3)$, and a dual of the standard
%   representation).}
 If $V$ is a finite dimensional representation, then its weight decomposition has the
 following are properties.\index{weight decomposition!properties of|(idxbf}
 \begin{enumerate}
  \item \label{lec18weightp1} For any root $\alpha$ and $\mu\in P(V)$, $\mu(H_\alpha)\in \ZZ$.

    To see this, consider $V$ as a representation of the $\sl(2)$ spanned by
    $X_\alpha$, $H_\alpha$, and $Y_\alpha$. Our characterization of finite dimensional
    representations of $\sl(2)$ implies the result.

 \item \label{lec18weightp2} For $\alpha\in \Delta$ and $\mu\in P(V)$,  $\g_\alpha V_\mu\subseteq V_{\mu+\alpha}$

    This follows from the standard calculation:
    \begin{align*}
      h(x_\alpha v) &= x_\alpha hv + [h,x_\alpha]v\\
            &= x_\alpha \mu(h)v + \alpha(h) x_\alpha v\\
            &= (\mu+\alpha)(h)x_\alpha v.
    \end{align*}

   \item \label{lec18weightp3} If $\mu\in P(V)$ and $w\in \weyl$, then $w(\mu)\in P(V)$ and
   $m_\mu=m_{w(\mu)}$.

    It is sufficient to check this when $w$ is a simple reflection $r_i$. Consider $V$
    as a representation of the copy of $\sl(2)$ spanned by $X_i$, $Y_i$, and $H_i$. If
    $v\in V_\mu$, then we have that $h\cdot v = \mu(h) v$ for all $h\in \h$. By
    property \ref{lec18weightp1}, we know that $\mu(H_i)=l$ is a non-negative integer.
    From the characterization of finite dimensional representations of $\sl(2)$, we
    know that there is a corresponding vector with $H_i$-eigenvalue $-l$, namely $u =
    Y_i^l v$. By property \ref{lec18weightp2}, $u\in V_{\mu-l\alpha_i}$. But
    $\mu-l\alpha_i = \mu-\mu(H_i)\alpha_i = \mu -
    \frac{2(\mu,\alpha_i)}{(\alpha_i,\alpha_i)}\alpha_i = r_i\mu$. Putting it all
    together, if we consider the $\sl(2)$ subrepresentation of $V$ generated by
    $V_\mu$ and $V_{r_i\mu}$, the symmetry of finite dimensional representations of
    $\sl(2)$ tells us that $\dim V_{r_i\mu}=\dim V_\mu$, as desired.

%   To see this, it is sufficient to check for simple reflections $w=r_i$. Recall that
%   for each reflection, there is an element $S_i = \exp X_i \exp (-Y_i) \exp
%   X_i$ of the simply connected group $G$. Exercise \ref{lec14Ex:Weyl} proved that
%   $Ad_{S_i}(\h)=\h$.
%   \begin{exercise}
%     Show that $S_i^{-1}\cdot V_\mu = V_{r_i \mu}$. (Hint: look at the solution to
%     Exercise \ref{lec14Ex:Weyl})
%     \begin{solution}
%        Let $\rho$ be the name of the representation, and let $\tilde \rho$ denote the
%        corresponding representation of $G$. If $a\in G$ and $\gamma(t)$ is a path in $G$ with
%        $\gamma'(0)=x\in \g$, then we have that
%        \begin{align*}
%          \rho_{Ad_a x} &= \der{}{t}\tilde\rho_{a\gamma(t)a^{-1}} \Bigr|_{t=0} \\
%          &= \tilde\rho_a \Bigl(\der{}{t}\tilde\rho_{\gamma(t)}\Bigr|_{t=0}\Bigr)
%                \tilde\rho_{a^{-1}}\\
%          &= \tilde\rho_a \rho_x \tilde\rho_{a^{-1}}.
%        \end{align*}
%        This equips us to mirror the calculation in Exercise \ref{lec14Ex:Weyl}. Say
%        $x\in V_\mu$ and $h\in \h$, then
%        \begin{align*}
%          \rho_h \tilde\rho_{S_i^{-1}} x &= \tilde\rho_{S_i^{-1}} \rho_{Ad_{S_i}h} x\\
%                &= \tilde\rho_{S_i^{-1}}\bigl( \mu(Ad_{S_i}h)x \bigr)\\
%                &= \mu(Ad_{S_i}h)\,\tilde\rho_{S_i^{-1}}x\\
%                &= (r_i\mu)(h)\,\tilde\rho_{S_i^{-1}}x\\
%        \end{align*}
%        so $S_i^{-1} \cdot V_\mu = \tilde\rho_{S_i^{-1}}V_\mu = V_{r_i\mu}$.
%     \end{solution}
%   \end{exercise}
 \end{enumerate}
 \begin{remark}\label{lec18Rmk:findim}
   Note that the proof of property \ref{lec18weightp2} did not require that $V$ be
   finite dimensional. Properties \ref{lec18weightp1} and \ref{lec18weightp3} used
   finite dimensionality, but in a weak way. Consider the $\sl(2)$ spanned by $X_i$,
   $Y_i$, and $H_i$. It is enough for each vector $v$ in a weight space of $V$ to be
   contained in a finite dimensional $\sl(2)$ subrepresentation. In particular, if
   each $X_i$ and $Y_i$ act locally nilpotently,\footnote{We say that a linear
   operator $A$ is \emph{locally nilpotent} if for each vector $v$ there is an integer
   $n(v)$ such that $A^{n(v)}v=0$.} then all three properties hold.
 \end{remark}\index{weight decomposition!properties of|)idxbf}
 \begin{example}
   If $\g=\sl(3)$, then we get $\weyl=D_{2\cdot 3}=S_3$. The orbit of a point can have a
   couple of different forms. If the point is on a hyperplane orthogonal to a root,
   then you get a triangle. For a generic point, you get a hexagon (which is not
   regular, but still symmetric).
   \[
    \begin{xy}<1.75em,0em>:
      a(180)+a(180); a(0)+a(0) **@{-},
      a(120)+a(120); a(-60)+a(-60) **@{-},
      a(60)+a(60); a(-120)+a(-120) **@{-},
      (0,0);<2.5em,0em>:
      a(60);
      a(180) **@{--};
      a(-60) **@{--};
      a(60) **@{--},
    \end{xy}\qquad
    \begin{xy}<1.75em,0em>:
      a(180)+a(180); a(0)+a(0) **@{-},
      a(120)+a(120); a(-60)+a(-60) **@{-},
      a(60)+a(60); a(-120)+a(-120) **@{-},
      (0,0);<2.5em,0em>:
      a(40);
      a(80) **@{--};
      a(160) **@{--};
      a(-160) **@{--};
      a(-80) **@{--};
      a(-40) **@{--};
      a(40) **@{--};
    \end{xy}
   \]
 \end{example}
 It is pretty clear that knowing the weights and multiplicities gives us a lot of
 information about the representation, so we'd better find a good way to exploit this
 information.

 Let $V$ be a representation of $\g$. Then $V$ is also a representation of the
 associated simply connected group $G$, and we get the commutative square
 \[\xymatrix{
   G\ar[r] & GL(V)\\
   \g \ar[r] \ar[u]^{\exp} & \gl(V)\ar[u]_{\exp}
 }\]
 If $h\in \h$, then $\exp h\in G$, and we can evaluate the group character of the
 representation $V$ on $\exp h$ as
 \[
    \chi_V(\exp h) = tr(\exp h) = \sum_{\mu\in P(V)} m_\mu e^{\mu(h)}
 \]
 where the second equality is because every eigenvalue $\mu(h)$ of $h$ yields an
 eigenvalue $e^{\mu(h)}$ of $\exp h$. Since characters tell us a lot about finite
 dimensional representations, it makes sense to consider the following definition.
 \begin{definition}
   The \emph{character}\index{character|idxbf}\index{chV@$ch\, V$|see{character}} of
   the representation $V$ is the formal sum
   \[
      ch\, V = \sum_{\mu\in P(V)} m_\mu e^\mu.
   \]
 \end{definition}
 You can and multiply these (formal) expressions; $ch$ is additive with respect to
 direct sum and multiplicative with respect to tensor products:
 \begin{align*}
   ch(V\oplus W) &= ch\, V+ch\, W\\
   ch(V\otimes W) &= (ch\, V)(ch\, W)
 \end{align*}
 This is because $V_\mu\otimes W_\nu \subseteq (V\otimes W)_{\mu+\nu}$ (or you can use
 the relationship with group characters). You can also check that the $ch\, V^*$
 is $\sum m_\mu e^{-\mu}$.
% If you don't care about multiplication, then you can do
% infinite-dimensional stuff.
 \begin{remark}
   We only evaluated $\chi_V$ on the image of the Cartan subalgebra. Is it possible
   that we've lost some information about the behavior of $\chi_V$ on the rest of $G$?
   The answer is no. Since $\chi_V$ is constant on conjugacy classes, and any Cartan
   subalgebra is conjugate to any other Cartan subalgebra (Theorem \ref{lec14T:CSA}),
   we know how $\chi_V$ behaves on the union of all Cartan subalgebras. Since the
   union of all Cartan subalgebras is dense in $\g$, $\exp \g$ is dense in
   $G$\anton{this can be proven with Bruhat decomposition ... is there another way?},
   and $\chi_V$ is continuous, the behavior of $\chi_V$ on the image of a single
   Cartan determines it completely.
 \end{remark}
 \begin{exercise}
   Show that the union of all Cartan subalgebras is dense in $\g$.
   \begin{solution}
     Every regular semisimple element is in some Cartan subalgebra; namely, the Cartan
     subalgebra of elements that commute with it. We will show that regular semisimple
     elements are dense in $\g$.

     Choose a basis for $\g$, which gives you a corresponding basis for $\gl(\g)$. Say
     $\g$ has rank $r$. Let $I$ be an indexing set so that for a matrix $A\in
     \gl(\g)$, the set $\{M_\gamma(A)\}_{\gamma\in I}$ is the set of all $(n-r)\times
     (n-r)$ minors of $A$. Define $f_\gamma:\g\to k$ by $ f_\gamma(x) =
     \det\bigl(M_\gamma(ad_x)\bigr)$. Since $ad$ is linear, $f_\gamma$ is a polynomial
     map for each $\gamma$. Now consider union of all of the zero sets of all of the
     $f_\gamma$. This is a Zariski closed set, so its complement in $\g$ is a Zariski
     open set\index{Zariski open set}. Since $\g$ has a regular element (a semisimple
     element $h$, where $ad_h$ is rank $n-r$), that open set is non-empty, and since
     $\g\cong \mathbb{A}^{\dim \g}$ is irreducible, this set is dense. \anton{To
     complete the proof, we need to know that the regular semisimple elements are
     exactly the elements $x$ so that $ad_x$ has rank $n-r$, which is not true. How
     can this be fixed?}
   \end{solution}
 \end{exercise}

 \subsektion{Highest weights} Fix a set of simple roots $\Pi =
 \{\alpha_1,\dots,\alpha_n\}$. A \emph{highest weight}\index{weight!highest|idxbf} of
 a representation $V$ is a $\lambda\in P(V)$ such that $\alpha_i+\lambda \not\in
 P(V)$ for all $\alpha_i\in\Pi$. A \emph{highest weight vector} is a vector in
 $V_\lambda$.

 Let $V$ be irreducible, let $\lambda$ be a highest weight, and let $v\in V_\lambda$
 be a highest weight vector. Since $V$ is irreducible, $v$ generates: $V = (U\g)v$.
 We know that $\n^+ v=0$ and that $\h$ acts on $v$ by scalars. By PBW\index{PBW|idxit},
 $U\g = U\n^-\otimes U\h\otimes U\n^+$, so $V=U\n^- v$. Thus, $V$ is generated from
 $v$ by applying various $Y_\alpha$, where $\alpha\in \Delta^+$. In particular,
 the multiplicity $m_\lambda$ is one. This also tells us that any other weight $\mu$
 is ``less than'' $\lambda$ in the sense that $\lambda -\mu = \sum_{\alpha\in
 \Delta^+} l_\alpha \alpha$, where the $l_\alpha$ are non-negative.
%
% Thus, all weights of the irreducible representation are given by
% applying negative roots. There are some relations among the different ways of
% applying roots: $[Y_{\alpha_1},Y_{\alpha_2}] = Y_{\alpha_1+\alpha_2}$.  Note that in
% particular, we get that $m_\lambda =1$.
% \[\begin{xy}
%   (0,0)="c" *+!DL{\lambda},
%   \ar_{\alpha_1} "c";a(180)="1"
%   \ar^{\alpha_2} "c";a(-60)="2"
%   \ar^{\!\alpha_3} "c";"1"+"2" *+{\,}
%   \ar "1" *+{\,};"1"+"1"
%   \ar "1" *+{\,};"1"+"2" *+{\,}
%   \ar "2" *+{\,};"2"+"1" *+{\,}
%   \ar "2" *+{\,};"2"+"2"
%   \ar "1"+"2";"1"+"2"+"1"
%   \ar "1"+"2";"1"+"2"+"2"
%   \ar "1"+"2";"1"+"2"+"1"+"2"
% \end{xy}\]

 It follows that in an irreducible representation, the highest weight is unique. If
 $\mu$ is another highest weight, then  we get $\lambda \le \mu$ and $\mu\le \lambda$,
 which implies $\mu = \lambda$.

 \begin{remark}\label{lec18Rmk:finirrep}
   If $V$ is an irreducible \emph{finite dimensional} representation with highest
   weight $\lambda$, then for any $w\in \weyl$, property \ref{lec18weightp3} tells us
   that $w(\lambda)$ is a highest weight with respect to the set of simple roots
   $\{w\alpha_1,\dots, w\alpha_n\}$. So $P(V)$ is contained in the convex hull of the
   set $\{w\lambda\}_{w\in \weyl}$.

   We also know that $\lambda$ is a highest weight for each $\sl(2)$ spanned by
   $X_\alpha$, $Y_\alpha$, and $H_\alpha$, with $\alpha\in \Delta^+$ (from the
   definition of highest weight). So
   $\lambda(H_i)=(\lambda,\check\alpha_i)\in \ZZ_{\ge 0}$ for each $i$.
 \end{remark}

 \begin{definition}
   The lattice generated by the roots, $Q= \ZZ\alpha_1\oplus \cdots\oplus \ZZ
   \alpha_n$, is called the \emph{root lattice}\index{root!lattice|idxbf}.
 \end{definition}
 \begin{definition}
   The lattice $P=\{\mu\in \h^*|(\mu,\check\alpha_i)\in \ZZ\text{ for } 1\le i\le n\}$
   is called the \emph{weight lattice}\index{weight!lattice|idxbf}.
 \end{definition}
 \begin{definition}
   The set $\{\mu\in \h^*|(\mu,\check\alpha_i)\ge 0\text{ for }1\le i\le n\}$ is
   called the \emph{Weyl chamber}\index{Weyl chamber|idxbf}, and the intersection of
   the Weyl chamber with the weight lattice is called the set of \emph{dominant
   integral weights}\index{weight!dominant integral|idxbf}, and is denoted $P^+$.
 \end{definition}
 $P$ and $Q$ have the same rank. It is clear that $Q$ is contained in $P$, and in
 general this containment is strict.

 $P/Q$ is isomorphic to the center of the simply connected group corresponding to
 $\g$.\anton{prove this or give a ref please}

% If $\lambda$ is a highest weight of an irreducible representation with respect to
% some choice of simple roots $\alpha_1,\dots, \alpha_n$, then for any $w\in \weyl$,
% $w(\lambda)$ will be the highest weight with respect to $w(\alpha_1),\dots, w(\alpha_n)$.
% Thus, we get that $P(V)$ is a subset in the convex hull of the orbit of the highest
% weight under the Weyl group, $\weyl\lambda$.
%
% If $\lambda$ is a highest weight of an irreducible $V$, then we know that
% $\lambda(H_i)\in \ZZ_{\ge 0}$. Sometimes this is written $(\lambda,\check
% \alpha_i)=\lambda(H_i)\ge 0$. To see this, note that we are restricting to our little
% $\sl(2)$ again.
%
% There are a few lattices in $\h^*$. If you look at the lattice generated by the
% roots, then we call that the \emph{root lattice}, and it is denoted $Q$:
% \[
%    Q = \ZZ\alpha_1\oplus \cdots\oplus \ZZ\alpha_n.
% \]
% There is another one, called the \emph{weight lattice}, and is called $P$:
% \[
%    P = \{\mu\in \h^*| (\mu,\check \alpha_i)\in \ZZ\}.
% \]
% Let $P^+ = \{\lambda\in P| (\lambda,\check \alpha_i)\ge 0\}$. These are called the
% integral dominant elements. Both $P$ and $Q$ have the same rank, but $P$ is usually
% bigger (and never smaller, obviously).
 \begin{example}
   For $\g=\sl(2)$, the root lattice is $2\ZZ$ (because $[H,X]=2X$),
   and the weight lattice is $\ZZ$.
 \end{example}

 \begin{example}
   In the three rank two cases, the weight lattices and Weyl chambers are
  \[\begin{array}{ccc}
   \sl(3) & \so(5)\cong \sp(4) & G_2\\
   \renewcommand\latticebody{\ifnum\latticeA=3 \ifnum\latticeB=2
                             \else \drop{\bullet} \fi
                             \else \drop{\bullet} \fi}
   \quad\begin{xy}a(120):: %sets coordinate system to that of simple roots
     {\ar_(.8){\alpha_1} (0,0);(1,0) *+{\,}
      \ar^(.8){\alpha_2} (0,0);(0,1) *+{\,}},
    (.66667,.33333):(.5,\halfrootthree):: %sets coordinate system to that of fundamental weights (reversed)
       (0,0) *!LD{\begin{pspicture}(0,0)(2,2.2)
                  \pswedge[fillstyle=solid,fillcolor=lightgray,linewidth=0pt]{2.2}{30}{90}
                  \end{pspicture}},
       (-2.5,0);(4.2,0) **@{--},
       (0,-2.5);(0,4.2) **@{--},
       (3,2) *{P^+},
       (0,0);(0,0) \croplattice{-2}3{-3}5 {-2.1}{3.1}{-2.1}{3.6}
   \end{xy}&
   \renewcommand\latticebody{\ifnum\latticeB=1 \ifnum\latticeA=-1
                             \else \ifnum\latticeA=4 \else \drop{\bullet} \fi\fi
                             \else \drop{\bullet} \fi}
   \quad\begin{xy}<1.75em,0em>:(-1,1):: %sets coordinate system to that of simple roots
     {\ar_(.8){\alpha_1} (0,0);(1,0) *+{\,}
      \ar (0,0);(0,1) *++!U{\mbox{\scriptsize $\alpha_2$}} *+{\,}},
    (1,.5):(1,1):: %sets coordinate system to that of fundamental weights (reversed)
       (0,0) *!LD{\begin{pspicture}(0,0)(2,2.2)
                  \pswedge[fillstyle=solid,fillcolor=lightgray,linewidth=0pt]{2.2}{45}{90}
                  \end{pspicture}},
       (-3.5,0);(5,0) **@{--},
       (0,-2);(0,3.5) **@{--},
       (4,1) *{P^+},
       (0,0);(0,0) \croplattice{-3}4{-3}4 {-4}{4.1}{-1.6}{3.1}
   \end{xy}&
   \renewcommand\latticebody{\ifnum\latticeB=0 \ifnum\latticeA=4
                             \else \drop{\bullet} \fi
                             \else \drop{\bullet} \fi}
   \quad\begin{xy}<1.5em,0em>: a(120)+a(180):: %sets coordinate system to that of simple roots
     {\ar_(.8){\alpha_1} (0,0);(1,0) *+{\,}
      \ar^(.8){\alpha_2} (0,0);(0,1) *+{\,}},
    (2,1):(1,0)+a(60):: %sets coordinate system to that of fundamental weights (reversed)
       (0,0) *!LD{\begin{pspicture}(0,0)(2,2.2)
                  \pswedge[fillstyle=solid,fillcolor=lightgray,linewidth=0pt]{2.2}{60}{90}
                  \end{pspicture}},
       (-2.4,0);(4,0) **@{--},
       (0,-1.3);(0,2.4) **@{--},
       (3,.5) *!DL{P^+},
       (0,0);(0,0) \croplattice{-4}4{-3}4 {-4.1}{4.1}{-1.1}{2.1}
   \end{xy}
 \end{array}\]
 \end{example}
 \begin{exercise}
   Show that $P^+$ is the fundamental domain of the action of $\weyl$ on $P$. That is,
   show that for every $\mu\in P$, the $\weyl$-orbit of $\mu$ intersects $P^+$ in exactly
   one point. (Hint: use Proposition \ref{lec14P:simply})
   \begin{solution}
     \anton{add the solution}
   \end{solution}
 \end{exercise}

 We have already shown that the highest weight of an irreducible finite dimensional
 representation is an element of $P^+$ (this is exactly the second part of Remark
 \ref{lec18Rmk:finirrep}). The rest of the lecture will be devoted to proving the
 following remarkable theorem.
 \begin{theorem}\label{lec18Thm:hiweight}
   There is a bijection between $P^+$ and the set of (isomorphism classes of)
   finite dimensional irreducible representations of $\g$, in which an irreducible
   representation corresponds to its highest weight.
 \end{theorem}
 It remains to show that two non-isomorphic finite dimensional irreducible
 representations cannot have the same highest weight, and that any element of $P^+$
 appears as the highest weight of some finite dimensional representation. To prove
 these things, we will use Verma modules.

 Let $V$ be an irreducible representation with highest weight $\lambda$. Then
 $V_\lambda$ is a 1-dimensional representation of the subalgebra $\b^+ :=\h\oplus
 \n^+\subseteq \g$. There is an induced representation $U\g\otimes_{U\b^+}V_\lambda$
 of $\g$, and an induced homomorphism $U\g\otimes_{U\b^+}V_\lambda\to V$ given by
 $x\otimes v\mapsto x\cdot v$.
 \begin{definition} \index{Verma module|(}
   A \emph{Verma module} is $M(\lambda) = U\g\otimes_{U\b^+} V_\lambda$.
 \end{definition}
 The Verma module is universal in the sense that for any representation $V$ with highest
 weight vector $v$ of weight $\lambda$, there is a unique homomorphism of
 representations $M(\lambda)\to V$ sending the highest vector of $M(\lambda)$ to $v$.
 However, there is a problem: $M(\lambda)$ is infinite dimensional.

 To understand $M(\lambda)$ as a vector space, we use PBW\index{PBW|idxit} to get that
 $U\g = U\n^-\otimes_k U\h \otimes_k U\n^+ = U(\n^-)\otimes_k U\b^+$. Since $U\b^+$
 acts on $V_\lambda$ by scalars, we get
 \[
   M(\lambda) = U\n^-\otimes_k U\b^+\otimes_{U\b^+} V_\lambda = U\n^- \otimes_k V_\lambda.
 \]
 If $\Delta^+ = \{\alpha_1,\dots, \alpha_N\}$, with $\Pi=\{\alpha_1,\dots,
 \alpha_n\}$, then by PBW\index{PBW|idxit}, $\{Y_{\alpha_1}^{k_1}\cdots
 Y_{\alpha_N}^{k_N}\}$ is a basis for $U\n^-$, so $\{Y_{\alpha_1}^{k_1}\cdots
 Y_{\alpha_N}^{k_N}v\}$ is a basis for $M(\lambda)=U(\n^-)\otimes_k V_{\lambda}$.
 Thus, even though the Verma module is infinite dimensional, it still has a weight
 decomposition with finite dimensional weight spaces:
 \[
    h(Y_{\alpha_1}^{k_1}\cdots Y_{\alpha_N}^{k_N}v) = (\lambda - k_1\alpha_1
    -\cdots - k_N\alpha_N)(h)(Y_{\alpha_1}^{k_1}\cdots Y_{\alpha_N}^{k_N}v).
 \]
 In particular, we get a nice formula for the multiplicity of a weight. The
 multiplicity of $\mu$ is given by the number of different ways $\lambda-\mu$ can be
 written as a non-negative sum of positive roots, corresponding to the number of basis
 vectors $Y_{\alpha_1}^{k_1}\cdots Y_{\alpha_N}^{k_N}v$ lying in $V_\mu$.
 \[
    m_\mu = \#\Bigl\{\lambda - \mu = \sum_{\alpha_i \in \Delta^+}k_i \alpha_i\Bigm|
    k_i\in \ZZ_{\ge 0}\Bigr\}.
 \]
 This is called the Kostant partition function.\index{Kostant partition function}

 \begin{example}\label{lec18Eg:Verma}
   We are now in a position to calculate the characters of Verma modules. In the rank
   two cases, we get the characters below. For example, since $2\alpha_3 =
   \alpha_3+\alpha_2+\alpha_1 = 2\alpha_1+2\alpha_2$ can be written in these three
   ways as a sum of positive roots, the circled multiplicity (in the $\sl(3)$ case) is
   3.
   \[\begin{array}{cc}
    \sl(3) & \so(5)\cong \sp(4)\\
    \quad \begin{xy}<1.75em,0em>: (-1,0):a(120):: %set the coordinates to negative simple roots
      {\ar (0,0) *++{\,};(-1,0) *+!L{\alpha_1}
       \ar (0,0) *++{\,};(0,-1) *+!R{\alpha_2}
       \ar (0,0) *++{\,};(-1,-1) *+!L{\alpha_3}},
       (4,0) *{1},(3,0) *{1},(2,0) *{1},(1,0) *{1},(0,0) *{1},
       (4,1) *{2},(3,1) *{2},(2,1) *{2},(1,1) *{2},(0,1) *{1},
       (5,2) *{3},(4,2) *{3},(3,2) *{3},(2,2) *\cir<6pt>{} *{3},(1,2) *{2},(0,2) *{1},
       (5,3) *{4},(4,3) *{4},(3,3) *{4},(2,3) *{3},(1,3) *{2},(0,3) *{1},
       (6,4) *{5},(5,4) *{5},(4,4) *{5},(3,4) *{4},(2,4) *{3},(1,4) *{2},(0,4) *{1},
    \end{xy}
    \quad & \quad
    \begin{xy}<1.75em,0em>: (-1,0):(-1,1):: %set the coordinates to negative simple roots
      {\ar (0,0) *++{\,};(-1,0)  *+!L{\alpha_1}
       \ar (0,0) *++{\,};(0,-1) *+!R{\alpha_2}
       \ar (0,0) *++{\,};(-1,-1) *+!D{\alpha_3}
       \ar (0,0) *++{\,};(-2,-1) *+!L{\alpha_4}},
       (4,0) *{1},(3,0) *{1},(2,0) *{1},(1,0) *{1},(0,0) *{1},
       (5,1) *{3},(4,1) *{3},(3,1) *{3},(2,1) *{3},(1,1) *{2},(0,1) *{1},
       (6,2) *{6},(5,2) *{6},(4,2) *{6},(3,2) *{5},(2,2) *{4},(1,2) *{2},(0,2) *{1},
       (7,3) *{10},(6,3) *{10},(5,3) *{9},(4,3) *{8},(3,3) *{6},(2,3) *{4},(1,3) *{2},(0,3) *{1},
       (8,4) *{15},(7,4) *{14},(6,4) *{13},(5,4) *{11},(4,4) *{9},(3,4) *{6},(2,4) *{4},(1,4) *{2},(0,4) *{1},
    \end{xy}\quad
    \end{array}\]
    \[\raisebox{-3em}{\mbox{$G_2$}} \qquad \qquad
    \begin{xy}<1.75em,0em>: (-1,0):a(120)+(-1,0):: %set the coordinates to negative simple roots
      {\ar (0,0) *++{\,};(-1,0)  *+!L{\alpha_1}
       \ar (0,0) *++{\,};(0,-1) *+!R{\alpha_2}
       \ar (0,0) *++{\,};(-1,-1) *+!DR{\alpha_3}
       \ar (0,0) *++{\,};(-3,-2) *+!D{\alpha_4}
       \ar (0,0) *++{\,};(-2,-1) *+!DL{\alpha_5}
       \ar (0,0) *++{\,};(-3,-1) *+!L{\alpha_6}},
       (6,0) *{1},(5,0) *{1},(4,0) *{1},(3,0) *{1},(2,0) *{1},(1,0) *{1},(0,0) *{1},
       (7,1) *{4},(6,1) *{4},(5,1) *{4},(4,1) *{4},(3,1) *{4},(2,1) *{3},(1,1) *{2},(0,1) *{1},
       (9,2) *{11},(8,2) *{11},(7,2) *{11},(6,2) *{11},(5,2) *{10},(4,2) *{9},(3,2) *{7},(2,2) *{4},(1,2) *{2},(0,2) *{1},
       (10,3) *{24},(9,3) *{24},(8,3) *{23},(7,3) *{22},(6,3) *{20},(5,3) *{16},(4,3) *{12},(3,3) *{8},(2,3) *{4},(1,3) *{2},(0,3) *{1},
       (12,4) *{46},(11,4) *{45},(10,4) *{44},(9,4) *{42},(8,4) *{38},(7,4) *{33},(6,4) *{27},(5,4) *{19},(4,4) *{13},(3,4) *{8},(2,4) *{4},(1,4) *{2},(0,4) *{1},
    \end{xy}
   \]
 \end{example}
 \begin{exercise}
   Check that the characters in Example \ref{lec18Eg:Verma} are correct. (Hint: For
   $\so(5)$, at each lattice point, keep track of four numbers: the number of ways to
   write $\lambda-\mu$ as a non-negative sum in the sets $\{\alpha_1,\dots,
   \alpha_4\}$, $\{\alpha_2,\alpha_3,\alpha_4\}$, $\{\alpha_2,\alpha_3\}$, and
   $\{\alpha_2\}$)
   \begin{solution}
     It is not hard to set up a recursive calculation with the numbers in the hint.
     Alternatively, note that the Kostant partition function tells us exactly that
     \begin{align*}
        ch\, M(\lambda) &= e^\lambda\prod_{\alpha\in \Delta^+}(1+e^{-\alpha} +
        e^{-2\alpha}+\cdots)\\
        &= e^\lambda \prod_{\alpha\in\Delta^+}(1-e^{-\alpha})^{-1}.
     \end{align*}
     You can easily (have your computer) compute the coefficients of this power
     series. For example, to compute the character of a Verma module of $G_2$, I think
     of $e^{\alpha_1}$ as \verb!x! and of $e^{\alpha_2}$ as \verb!y!. Then the following
     Mathematica code returns the first 144 multiplicities.\\
     {\raggedright
      \verb!Nmax = 12;! \\
      \verb!mySeries=Series[!
      \hspace*{1em} \verb!((1-x)(1-y)(1- x y)(1- x^2 y)(1- x^3 y)(1- x^3 y^2))^(-1),! \\
      \qquad \verb!{x,0,Nmax},{y,0,Nmax}];! \\
      \verb!TableForm[Table[SeriesCoefficient[! \\
      \hspace{10em}\verb!mySeries,{i,j}],{i,0,Nmax},{j,0,Nmax}]]!
     }
   \end{solution}
 \end{exercise}

 \begin{lemma}\label{lec18Lem:maxlprop}
   A Verma module $M(\lambda)$ has a unique proper maximal submodule $N(\lambda)$.
 \end{lemma}
 \begin{proof}
   $N$ being proper is equivalent to $N\cap V_\lambda = 0$. This property is clearly
   preserved under taking sums, so you get a unique maximal submodule.
 \end{proof}
 \begin{remark}\label{lec18Rmk:injectivity}
   If $V$ and $W$ are irreducible representations with the same highest weight,
   then they are both isomorphic to the unique irreducible quotient
   $M(\lambda)/N(\lambda)$, so they are isomorphic.
 \end{remark}
 \begin{lemma}\label{lec18Lem:Vlfindim}
   If $\lambda\in P^+$, then the quotient $V(\lambda) = M(\lambda)/N(\lambda)$ is
   finite dimensional.
 \end{lemma}
 \begin{proof}
   If $w$ is a weight vector (but not the highest weight vector) in $M(\lambda)$
   such that $X_iw=0$ for $i=1,\dots, n$, then we claim that $w\in N(\lambda)$. To see
   this, you note that
   \[
      (U\g)w = (U\n^-\otimes U\h\otimes U\n^+)w = (U\n^-)w
   \]
   so the submodule generated by $w$ contains only lower weight spaces. In particular,
   the highest weight space $V_\lambda$ cannot be obtained from $w$.

   Fix an $i\le n$. By assumption, $\lambda(H_i)=\langle \lambda, \check
   \alpha_i\rangle = l_i\in \ZZ_{\ge 0}$. Letting $w = Y_i^{l_i+1}v$, we get
   that
   \begin{align*}
     X_iw &= (l_i+1)\bigl(l_i-(l_i+1)+1\bigr) Y_i^{l_i+1}w =0 &
            \text{(by Equation \ref{lec13dag})}\\
     X_jw &= Y_i^{l_i+1} X_j w = 0 & \text{(since $[X_j,Y_i]=0$)}
   \end{align*}
   so $w\in N(\lambda)$. It follows from the Serre relations\index{Serre
   relations|idxit} that in the quotient $V(\lambda)$, the $Y_i$ act locally
   nilpotently. The $X_i$ act locally nilpotently on $M(\lambda)$, so they act locally
   nilpotently on $V(\lambda)$. By Remark \ref{lec18Rmk:findim},
   $P\bigl(V(\lambda)\bigr)$ is invariant under $\weyl$, so it is contained in the convex
   hull of the orbit of $\lambda$. Since each weight space is finite dimensional, it
   follows that $V(\lambda)$ is finite dimensional.
 \end{proof}

 Putting it all together, we can prove the Theorem.
 \begin{proof}[Proof of Theorem \ref{lec18Thm:hiweight}]
   By Remark \ref{lec18Rmk:finirrep}, the highest weight of an irreducible finite
   dimensional representation is an element of $P^+$. By Remark
   \ref{lec18Rmk:injectivity}, non-isomorphic representations have distinct highest
   weights. Finally, by Lemmas \ref{lec18Lem:maxlprop} and \ref{lec18Lem:Vlfindim},
   every element of $P^+$ appears as the highest weight of some finite dimensional
   irreducible representation.
 \end{proof}
 \begin{corollary}
   If $V$ and $W$ are finite dimensional representations, and if $ch\, V = ch\, W$,
   then $V\simeq W$.
 \end{corollary}
 \begin{proof}
   Since their characters are equal, $V$ and $W$ have a common highest weight
   $\lambda$, so they both contain a copy of $V(\lambda)$. By complete reducibility
   (Theorem \ref{lec12Weyl}), $V(\lambda)$ is a direct summand in both $V$ and $W$. It
   is enough to show that the direct complements are isomorphic, but this follows from
   the fact that they have equal characters and fewer irreducible components.
 \end{proof}
 \anton{Do you really need complete reducibility? Does this hold for (some) infinite
 dimensional representations?}

 So it is desirable to be able to compute the character of $V(\lambda)$. This is what
 we will do next lecture.
