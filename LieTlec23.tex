 \stepcounter{lecture}
 \setcounter{lecture}{23}
 \sektion{Lecture 23}

 Last time we defined the Clifford algebra $C_V(K)$, where $V$ is a vector space over $K$
 with a quadratic form $N$. $C_V(K)$ is generated by $V$ with $x^2=N(x)$.
 $C_{m,n}(\RR)$ uses the form $x_1^2+\cdots + x_m^2-x_{m+1}^2-\cdots - x_{m+n}^2$. We
 found that the structure depends heavily on $m-n \mod 8$.
 \begin{remark}
   This mod 8 periodicity turns up in several other places:
 \begin{enumerate}
 \item Real Clifford algebras $C_{m,n}(\RR)$ and $C_{m',n'}(\RR)$ are super Morita
 equivalent if and only if $m-n\equiv m'-n'\mod 8$.

 \item \emph{Bott periodicity},\index{Bott periodicity} which says that stable
 homotopy groups of orthogonal groups are periodic mod 8.

 \item Real $K$-theory is periodic with a period of 8.

 \item Even unimodular lattices (such as the $E_8$ lattice) exist in $\RR^{m,n}$
 if and only if $m-n\equiv 0\mod 8$.

 \item The Super Brauer group\index{super Brauer group} of $\RR$ is $\ZZ/8\ZZ$. The
 Super Brauer group consists of super division algebras over $\RR$ (algebras in which
 every non-zero homogeneous element is invertible) with the operation of tensor product
 modulo super Morita equivalence.\footnote{See
 \url{http://math.ucr.edu/home/baez/trimble/superdivision.html}}
 \[
 \begin{xy}<3.75em,0em>:
   (-1,0) *++!R{\RR} *{\bullet},
   (-\halfroottwo,\halfroottwo) *++!RD{\RR[\e_+]} *{\bullet},
   (0,1) *++!D{\CC[\e_+]} *{\bullet},
   (\halfroottwo,\halfroottwo) *+!LD{\HH[\e_-]} *{\bullet},
   (1,0) *++!L{\HH} *{\bullet},
   (\halfroottwo,-\halfroottwo) *+!LU{\HH[\e_+]} *{\bullet},
   (0,-1) *++!U{\CC[\e_-]} *{\bullet},
   (-\halfroottwo,-\halfroottwo) *+!RU{\RR[\e_-]} *{\bullet},
   (0,0), *\xycircle(1,1){}
 \end{xy}\]
 where $\e_{\pm}$ are odd with $\e_\pm^2 = \pm 1$, and $i\in \CC$ is odd,\footnote{One
 could make $i$ even since $\RR[i,\e_\pm]=\RR[\mp \e_\pm i, \e_\pm]$, and $\RR[\mp\e_\pm
 i]\cong \CC$ is entirely even.} but $i,j,k\in \HH$ are even.
 \end{enumerate}
 \end{remark}
 Recall that $C_V(\RR)=C^0_V(\RR)\oplus C^1_V(\RR)$, where $C^1_V(\RR)$ is the odd part
 and $C^0_V(\RR)$ is the even part. It turns out that we will need to know the structure
 of $C^0_{m,n}(\RR)$. Fortunately, this is easy to compute in terms of smaller Clifford
 algebras. Let $\dim U=1$, with $\gamma$ a basis for $U$ and let $\gamma_1,\dots,
 \gamma_n$ an orthogonal basis for $V$. Then $C^0_{U\oplus V}(K)$ is generated by
 $\gamma\gamma_1,\dots, \gamma\gamma_n$. We compute the relations
 \[
    \gamma\gamma_i\cdot \gamma\gamma_j = -\gamma\gamma_j\cdot \gamma\gamma_i
 \]
 for $i\neq j$, and
 \[
    (\gamma\gamma_i)^2 = (-\gamma^2)\gamma_i^2
 \]
 So $C_{U\oplus V}^0(K)$ is itself the Clifford algebra $C_{W}(K)$, where $W$ is $V$ with
 the quadratic form multiplied by $-\gamma^2 = -\text{disc}(U)$. Over $\RR$, this tells
 us that
 \begin{align*}
   C^0_{m+1,n}(\RR) &\cong C_{n,m}(\RR) & \text{(mind the indices)} \\
   C^0_{m,n+1}(\RR) &\cong C_{m,n}(\RR).
 \end{align*}
% So we've worked out all the real algebras and all the even and odd parts of them.

 \begin{remark}
   For complex Clifford algebras, the situation is similar, but easier. One finds that
   $C_{2m}(\CC)\cong \MM_{2^m}(\CC)$ and $C_{2m+1}(\CC) \cong \MM_{2^m}(\CC)\oplus
   \MM_{2^m}(\CC)$, with $C_n^0(\CC)\cong C_{n-1}(\CC)$. You could figure these out by
   tensoring the real algebras with $\CC$ if you wanted. We see a mod 2 periodicity now.
   Bott periodicity for the unitary group is mod 2.
 \end{remark}

 \subsektion{Clifford groups, Spin groups, and Pin groups}\index{Clifford groups|idxbf}
 In this section, we define Clifford groups, denoted $\Gamma_V(K)$, and find an exact
 sequence
 \[
    1\to K^\times \xrightarrow{\mathrm{central}} \Gamma_V(K) \to O_V(K)\to 1.
 \]
 \begin{definition}
   $\Gamma_V(K) = \{x\in C_V(K) \text{ homogeneous\footnotemark} |
   xV\alpha(x)^{-1}\subseteq V\}$ \footnotetext{We assume that $\Gamma_V(K)$ consists of
   homogeneous elements, but this can actually be proven.\anton{ref for this stuff?}}
   (recall that $V\subseteq C_V(K)$), where $\alpha$ is the automorphism of $C_V(K)$
   induced by $-1$ on $V$ (i.e.\ the automorphism which acts by $-1$ on odd elements and
   1 on even elements).
 \end{definition}
   Note that $\Gamma_V(K)$ acts on $V$ by $x\cdot v = xv\alpha(x)^{-1}$.

   Many books leave out the $\alpha$, which is a mistake, though not a serious one.
   They use $xVx^{-1}$ instead of $xV\alpha(x)^{-1}$.
   Our definition is better for the following reasons:
   \begin{enumerate}
   \item It is the correct superalgebra sign. The superalgebra convention
   says that whenever you exchange two elements of odd degree, you pick up a minus sign, and
   $V$ is odd.

   \item Putting $\alpha$ in makes the theory much cleaner in odd dimensions. For
   example, we will see that the described action gives a map $\Gamma_V(K)\to O_V(K)$
   which is onto if we use $\alpha$, but not if we do not. (You get $SO_V(K)$ without the
   $\alpha$, which isn't too bad, but is still annoying.)
   \end{enumerate}
 \begin{lemma}\label{lec23L} \hspace*{-1ex}\footnote{I promised no Lemmas or Theorems, but I was
   lying to you.} The elements of $\Gamma_V(K)$ which act trivially on
   $V$ are the elements of $K^\times \subseteq \Gamma_V(K)\subseteq C_V(K)$.
 \end{lemma}
 \begin{proof}
   Suppose $a_0+a_1\in \Gamma_V(K)$ acts trivially on $V$, with $a_0$ even and $a_1$ odd.
   Then $(a_0+a_1)v=v\alpha(a_0+a_1)=v(a_0-a_1)$. Matching up even and odd parts, we get
   $a_0v=va_0$ and $a_1 v=-va_1$. Choose an orthogonal basis $\gamma_1,\dots, \gamma_n$
   for $V$.\footnote{All these results are true in characteristic 2, but you have to work
   harder ... you can't go around choosing orthogonal bases because they may not exist.}
   We may write
   \[
      a_0 = x + \gamma_1 y
   \]
    where $x \in C_V^0(K)$ and $y \in C_V^1(K)$ and neither $x$ nor $y$ contain a factor
    of $\gamma_1$, so $\gamma_1 x=x\gamma_1$ and $\gamma_1y=y\gamma_1$. Applying the
    relation $a_0v = va_0$ with $v=\gamma_1$, we see that $y=0$, so $a_0$
    contains no monomials with a factor $\gamma_1$.

    Repeat this procedure with $v$ equal to the other basis elements to show that $a_0\in
    K^\times$ (since it cannot have any $\gamma$'s in it). Similarly, write
    $a_1=y+\gamma_1 x$, with $x$ and $y$ not containing a factor of $\gamma_1$. Then the
    relation $a_1 \gamma_1=-\gamma_1a_1$ implies that $x=0$. Repeating with the other
    basis vectors, we conclude that $a_1=0$.

    So $a_0 + a_1 = a_0 \in K \cap \Gamma_V(K) = K^\times$.
 \end{proof}
   Now we define ${-}^T$ to be the identity on $V$, and extend it to an anti-automorphism
   of $C_V(K)$ (``anti'' means that $(ab)^T=b^Ta^T$). Do not confuse $a\mapsto \alpha(a)$
   (automorphism), $a\mapsto a^T$ (anti-automorphism), and $a\mapsto \alpha(a^T)$
   (anti-automorphism).

   Notice that on $V$, $N$ coincides with the quadratic form $N$. Many authors seem not
   to have noticed this, and use different letters. Sometimes they use a sign
   convention which makes them different.

   Now we define the \emph{spinor norm}\index{spinor norm|idxbf} of $a\in C_V(K)$ by
   $N(a)=aa^T$. We also define a twisted version: $N^\alpha(a)=a\alpha(a)^T$.
   \begin{proposition}
   \begin{enumerate}\item[]
     \item The restriction of $N$ to $\Gamma_V(K)$ is a homomorphism whose image lies in
     $K^\times$. $N$ is a mess on the rest of $C_V(K)$.

     \item The action of $\Gamma_V(K)$ on $V$ is orthogonal. That is, we have a
     homomorphism $\Gamma_V(K)\to O_V(K)$.
   \end{enumerate}
   \end{proposition}
   \begin{proof}
    First we show that if $a\in \Gamma_V(K)$, then $N^\alpha(a)$ acts trivially on $V$.
    \begin{align*}
        N^\alpha(a)\, v\, \alpha\bigl(&N^\alpha(a)\bigr)^{-1} =
            a\alpha(a)^T v \Bigl( \alpha(a)\underbrace{\alpha\bigl(\alpha(a)^T\bigr)}_{\smash{=a^T}}
            \Bigr)^{-1}\\
        &= a\underbrace{\alpha(a)^T v (a^{-1})^T}_{
            \smash{=(a^{-1}v^T \alpha(a))^T}} \alpha(a)^{-1} \\
        &= a a^{-1} v \alpha(a)\alpha(a)^{-1} & \hspace*{6em}
            \llap{($T|_V=\id_V$ and $a^{-1}v\alpha(a)\in V$)}\\
        &= v
    \end{align*}
    So by Lemma \ref{lec23L}, $N^\alpha(a)\in K^\times$. This implies that $N^\alpha$ is
    a homomorphism on $\Gamma_V(K)$ because
    \begin{align*}
      N^\alpha(a)N^\alpha(b) &= a\alpha(a)^T N^\alpha(b) \\
       & = aN^\alpha(b) \alpha(a)^T & (N^\alpha(b) \text{ is central})\\
       & = ab\alpha(b)^T\alpha(a)^T\\
       &=(ab)\alpha(ab)^T=N^\alpha(ab).
    \end{align*}
    After all this work with $N^\alpha$, what we're really interested is $N$. On the even
    elements of $\Gamma_V(K)$, $N$ agrees with $N^\alpha$, and on the odd elements,
    $N=-N^\alpha$. Since $\Gamma_V(K)$ consists of homogeneous elements, $N$ is also a
    homomorphism from $\Gamma_V(K)$ to $K^\times$. This proves the first statement of the
    Proposition.

    Finally, since $N$ is a homomorphism on $\Gamma_V(K)$, the action on $V$ preserves
    the quadratic form $N|_V$. Thus, we have a homomorphism $\Gamma_V(K)\to O_V(K)$.
 \end{proof}
 Now let's analyze the homomorphism $\Gamma_V(K)\to O_V(K)$. Lemma \ref{lec23L} says
 exactly that the kernel is $K^\times$. Next we will show that the image is all of
 $O_V(K)$. Say $r\in V$ and $N(r)\neq 0$.
 \begin{align}
   r v\alpha(r)^{-1} &= -rv\frac{r}{N(r)} = v - \frac{vr^2+rvr}{N(r)} \notag\\
    &= v - \frac{(v,r)}{N(r)}r \label{lec23star}\\
    &= \begin{cases}
       -r &\text{if }v=r\\
        v &\text{if }(v,r)=0
    \end{cases}
 \end{align}
 Thus, $r$ is in $\Gamma_V(K)$, and it acts on $V$ by reflection through the hyperplane
 $r^\perp$. One might deduce that the homomorphism $\Gamma_V(K)\to O_V(K)$ is surjective
 because $O_V(K)$ is generated by reflections. This is wrong; $O_V(K)$ is \emph{not}
 always generated  by reflections!\index{orthogonal group!not generated by reflections}
 \begin{exercise}
   Let $H=\FF_2^2$, with the quadratic form $x^2+y^2+xy$, and let $V=H\oplus H$. Prove
   that $O_V(\FF_2)$ is not generated by reflections.
   \begin{solution}
     In $H$, the norm of any non-zero vector is 1. It is immediate to check that the
     reflection of a non-zero vector $v$ through another non-zero vector $u$ is
     \[
        r_u(v) = \begin{cases}
          u &\text{if }u=v\\
          v+u & \text{if }u\ne v
        \end{cases}
     \]
     so reflection through a non-zero vector fixes that vector and swaps the two other
     non-zero vectors. Thus, the reflection in $H$ generate the symmetric group on three
     elements $S_3$, acting on the three non-zero vectors.

     If $u$ and $v$ are non-zero vectors, then $(u,v)\in H\oplus H$ has norm $1+1=0$, so
     one cannot reflect through it. Thus, every reflection in $V$ is ``in one of the
     $H$'s,'' so the group generated by reflections is $S_3\times S_3$. However,
     swapping the two $H$'s is clearly an  orthogonal transformation, so reflections do
     not generate $O_V(\FF_2)$.
   \end{solution}
 \end{exercise}
 \begin{remark}\label{lec23Rmk:allOK}
   It turns out that this is the \emph{only} counterexample. For any other vector space
   and/or any other non-degenerate quadratic form, $O_V(K)$ is generated by reflections.
   The map $\Gamma_V(K)\to O_V(K)$ is surjective even in the example above. Also, in
   every case except the example above, $\Gamma_V(K)$ is generated as a group by non-zero
   elements of $V$ (i.e.\ every element of $\Gamma_V(K)$ is a monomial).\anton{ref for
   this stuff?}
 \end{remark}
 \begin{remark}
   Equation \ref{lec23star} is the definition of the reflection of $v$ through $r$. It is
   only possible to reflect through vectors of non-zero norm. Reflections in
   characteristic 2 are strange; strange enough that people don't call them reflections,
   they call them \emph{transvections}\index{transvections}.
 \end{remark}

 Thus, we have the diagram
 \begin{equation}
  \xymatrix{
    &1 \ar[r] & K^\times \ar[r] \ar@{}[d]|{\|} & \Gamma_V(K) \ar[d]^N
        \ar[r] & O_V(K) \ar[r] \ar@{.>}[d]^N & 1\\
    1 \ar[r] & \pm 1 \ar[r] & K^\times \ar[r]^{x\mapsto x^2} & K^\times \ar[r] &
        K^\times/(K^\times)^2 \ar[r] & 1}
  \label{lec23dag}
 \end{equation}
 where the rows are exact, $K^\times$ is in
 the center of $\Gamma_V(K)$ (this is obvious, since $K^\times$ is in the center of
 $C_V(K)$), and $N:O_V(K)\to K^\times/(K^\times)^2$ is the unique homomorphism sending
 reflection through $r^\perp$ to $N(r)$ modulo $(K^\times)^2$.


% Summarizing: We have the exact sequence
% \[
%    1\to K^\times \to \Gamma_V(K) \to O_V(K)\to 1
% \]
% where $K^\times$ lands in the center of $\Gamma_V(K)$.
% \[\xymatrix{
% 1 \ar[r] & K^\times \ar[r] \ar@{}[d]|{\|} & \Gamma_V(K) \ar[d]^N \ar[r] & O_V(K)
% \ar[r] \ar@{.>}[d]^N & 1\\
% 1 \ar[r] & K^\times \ar[r]^{x\mapsto x^2} & K^\times \ar[r] & K^\times/(K^\times)^2
% \ar[r] & 1
% }\]

 \begin{definition}
   $\pin_V(K) = \{x\in \Gamma_V(K)| N(x)=1\}$, and $\spin_V(K) =\pin_V^0(K)$, the even
   elements of $\pin_V(K)$.
 \end{definition}
 On $K^\times$, the spinor norm is given by $x\mapsto x^2$, so the elements of spinor
 norm 1 are $=\pm 1$. By restricting the top row of (\ref{lec23dag}) to elements of norm
 1 and even elements of norm 1, respectively, we get exact sequences
 \[
 \xymatrix @R=1em{
    1\ar[r] &\pm 1 \ar[r] & \pin_V(K) \ar[r] & O_V(K) \ar@{.>}[r]^(.45)N & K^\times/(K^\times)^2 \\
    1\ar[r] &\pm 1 \ar[r] & \spin_V(K) \ar[r] & SO_V(K) \ar@{.>}[r]^(.45)N & K^\times/(K^\times)^2
 }
 \]
 To see exactness of the top sequence, note that the kernel of $\phi$ is $K^\times\cap
 \pin_V(K)=\pm 1$, and that the image of $\pin_V(K)$ in $O_V(K)$ is exactly the elements
 of norm 1. The bottom sequence is similar, except that the image of $\spin_V(K)$ is not
 all of $O_V(K)$, it is only $SO_V(K)$; by Remark \ref{lec23Rmk:allOK}, every element of
 $\Gamma_V(K)$ is a product of elements of $V$, so every element of $\spin_V(K)$ is a
 product of an even number of elements of $V$. Thus, its image is a product of an even
 number of reflections, so it is in $SO_V(K)$.

 ?????????????????????????????????????????????????????????????

 These maps are NOT always onto, but there are many important cases when they are,
 like when $V$ has a positive definite quadratic form. The image is the set of
 elements of $O_V(K)$ or $SO_V(K)$ which have spinor norm 1 in
 $K^\times/(K^\times)^2$.

 What is $N:O_V(K) \to K^\times/(K^\times)^2$? It is the UNIQUE homomorphism such that
 $N(a)=N(r)$ if $a$ is reflection in $r^\perp$, and $r$ is a vector of norm $N(r)$.

 \begin{example}
   Take $V$ to be a positive definite vector space over $\RR$. Then $N$ maps to 1 in
   $\RR^\times/(\RR^\times)^2=\pm 1$ (because $N$ is positive definite). So the spinor
   norm on $O_V(\RR)$ is TRIVIAL.
 \end{example}
 So if $V$ is positive definite, we get double covers
 \[
    1\to \pm 1 \to \pin_V(\RR) \to O_V(\RR)\to 1
 \]
 \[
    1\to \pm 1 \to \spin_V(\RR) \to SO_V(\RR)\to 1
 \]
 This will account for the weird double covers we saw before.

 What if $V$ is negative definite. Every reflection now has image $-1$ in
 $\RR^\times/(\RR^\times)^2$, so the spinor norm $N$ is the same as the determinant
 map $O_V(\RR)\to \pm 1$.

 So in order to find interesting examples of the spinor norm, you have to look at
 cases that are neither positive definite nor negative definite.

 Let's look at Losrentz space: $\RR^{1,3}$.

 \[\begin{xy}
   (-1,-1);(1,1) **@{-},
   (1,-1);(-1,1) **@{-},
   (0,1.08) *{{}_{\text{norm}>0}},
   (1.5,0) *{{}_{\text{norm}<0}},
   (1,1);(-1,1) **\crv{(1.2,1.4)&(-1.2,1.4)},
   (1,1);(-1,1) **\crv{(.7,.7)&(-.7,.7)},
   (1,-1);(-1,-1) **\crv{(1.2,-1.4)&(-1.2,-1.4)},
   (1,-1);(-1,-1) **\crv{~*=<2pt>@{.} (.7,-.7)&(-.7,-.7)},
   \ar@/_1mm/ (2.5,1.2) *+{{}_{\text{norm}=0}}; (1,1) *+{\,},
 \end{xy}\]

 Reflection through a vector of norm $<0$ (spacelike vector, $P$: parity reversal) has
 spinor norm $-1$, det $-1$ and reflection through a vector of norm $>0$ (timelike
 vector, $T$: time reversal) has spinor norm $+1$, det $-1$. So $O_{1,3}(\RR)$ has 4
 components (it is not hard to check that these are all the components), usually
 called $1$, $P$, $T$, and $PT$.

 \begin{remark}
   For those who know Galois cohomology. We get an exact sequence of algebraic groups
   \[
   1\to GL_1\to \Gamma_V\to O_V \to 1
   \]
   (algebraic group means you don't put a field). You do not necessarily get an exact
   sequence when you put in a field.

   If
   \[
    1\to A\to B \to C\to 1
    \]
    is exact,
   \[
    1\to A(K) \to B(K)\to C(K)
   \]
   is exact. What you really get is
   \begin{align*}
    1&\to H^0(\mathrm{Gal}(\bar K/K),A) \to H^0(\mathrm{Gal}(\bar K/K), B) \to
    H^0(\mathrm{Gal}(\bar K/K), C)\to\\
    &\to H^1(\mathrm{Gal}(\bar K/K), A)\to \cdots
   \end{align*}
   It turns out that $H^1(Gal(\bar K/K),GL_1)=1$. However, $H^1(Gal(\bar K/K),\pm
   1)=K^\times / (K^\times)^2$.

   So from
   \[
   1\to GL_1\to \Gamma_V\to O_V \to 1
   \]
   you get
   \[
    1\to K^\times \to \Gamma_V(K) \to O_V(K) \to 1= H^1(Gal(\bar K/K),GL_1)
   \]
   However, taking
   \[
    1\to \mu_2 \to \spin_V \to SO_V \to 1
   \]
   you get
   \[
    1\to \pm 1\to \spin_V(K) \to SO_V(K) \xrightarrow{N}
    K^\times/(K^\times)^2=H^1(\bar K/K,\mu_2)
   \]
   so the non-surjectivity of $N$ is some kind of higher Galois cohomology.
   \begin{warning}
     $\spin_V \to SO_V$ is onto as a map of ALGEBRAIC GROUPS, but $\spin_V(K)\to
     SO_V(K)$ need NOT be onto.
   \end{warning}
 \end{remark}

 \begin{example}
   Take $O_3(\RR)\cong SO_3(\RR)\times \{\pm 1\}$ as 3 is odd (in general
   $O_{2n+1}(\RR)\cong SO_{2n+1}(\RR)\times \{\pm 1\}$). So we have a sequence
   \[
    1\to \pm 1\to \spin_3(\RR) \to SO_3(\RR) \to 1.
   \]
   Notice that $\spin_3(\RR)\subseteq C_3^0(\RR)\cong \HH$, so $\spin_3(\RR)\subseteq
   \HH^\times$, and in fact we saw that it is $S^3$.
 \end{example}
