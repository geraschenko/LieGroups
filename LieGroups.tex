%%%%%%%%%%%%%%%%%%%%%%%%%%%%%%%%%%%%%%%%%%%%%%%%%%%%%%%%%%%%%%%%%%%%%%%%%
%%    This is the main document for the Math 261A notes. To compile    %%
%%    these notes, you should put the following files in the same      %%
%%    directory:                                                       %%
%%    LieGroups.tex (this file),                                       %%
%%    LieTpre.tex,                                                     %%
%%    LieTlec01.tex, LieTlec02.tex,..., LieTlec30.tex, LieTlec31.tex   %%
%%    LieGroupsBib.bib                                                 %%
%%                                                                     %%
%%    If you are using MiKTeX, you can TeXify:                         %%
%%    (1)-(6) texify LieGroups.tex (Shift+Ctrl+X in WinEdt)            %%
%%    Otherwise, you should run the following commands:                %%
%%    (1) latex LieGroups                                              %%
%%    (2) makeindex LieGroups                                          %%
%%       (Note to Cygwin users: makeindex will complain if you use     %%
%%       the Cygwin shell for this; use cmd instead.)                  %%
%%    (3) bibtex LieGroups                                             %%
%%    (4) latex LieGroups                                              %%
%%    (5) makeindex LieGroups                                          %%
%%    (6) latex LieGroups                                              %%
%%                                                                     %%
%%    Next, if you want to make a pdf, you should convert to ps        %%
%%    first, because converting directly to pdf kills the              %%
%%    hyperlinks and pstricks stuff.                                   %%
%%    (7) dvips LieGroups                                              %%
%%    (8) ps2pdf LieGroups.ps                                          %%
%%                                                                     %%
%%    Anton                                                            %%
%%%%%%%%%%%%%%%%%%%%%%%%%%%%%%%%%%%%%%%%%%%%%%%%%%%%%%%%%%%%%%%%%%%%%%%%%

 \newif\ifproofmode     % True if for proofreading
 \newif\iflilbook       % True if for making a little book
 \proofmodetrue
 \lilbookfalse
  \proofmodefalse
%  \lilbooktrue

 \iflilbook
   \documentclass[twoside]{article}
 \else
   \documentclass[12pt]{article}
 \fi

 %\usepackage{ascii}
 %\usepackage{graphicx}     % I don't like including graphics ... I'd rather draw them
 %\usepackage{latexsym}     % I don't think I use any of these symbols
 %\usepackage{titlesec}     % maybe one day I'll learn this package
 %\usepackage{xcolor}       % If I ever want to use color ... I think this loads with pstricks

 \usepackage{mathdots}      % For \iddots, which are backwords \ddots. Load before amsmath
 \usepackage{amsmath}       % I think this gives me some symbols
 \usepackage{amsthm}        % Does theorem stuff
 \usepackage{amssymb}       % more symbols and fonts
 \usepackage{empheq}        % Some more extensible arrows, like \xmapsto
% package hyperref is loaded last
 \usepackage{ifthen}        % For conditional stuff
 \usepackage{makeidx}       % For making the index
 \usepackage{manfnt}        % This package has the cool ``warning sign'' \dbend
 \usepackage{mathrsfs}      % Sheafy font \mathscr{}
 \usepackage{multicol}      % Supports muliple columns
 \usepackage{picinpar}      % for pictures in paragraphs
 \usepackage{pstricks}      % PStricks ... only used for shading here
 \usepackage{relsize}       % contextually size fonts
 \usepackage{graphicx}      % for the rotating
 \usepackage{titletoc}      % Table of Contents stuff
     \titlecontents{section}[0em]{\addvspace{1em}}{\thecontentslabel}{}{\titlerule*{} \textbf{\thecontentspage}}
     \titlecontents{subsection}[2em]{}{\thecontentslabel}{}{\titlerule*{}\thecontentspage}
 \usepackage[all]{xy}       % Include XY-pic
    \SelectTips{cm}{10}     % Use the nicer arrowheads
    \everyxy={<2.5em,0em>:} % Sets the scale I like
    \xyoption{web}          % Include the lattice feature, I don't know why%
                            % it isn't loaded already

 \ifthenelse{\boolean{proofmode}}{%
      \usepackage[colorlinks, linkcolor=red!40!blue!40!black, pagebackref]{hyperref}
                            % This allows hyperlinks, and no longer needs pdflatex%
                            % because dvips, ps2pdf remembers hyperlinks now.
      \usepackage{showidx}    % For showing the index entries in the corner of the page
                            % must go after hyperref
    }{
     \usepackage[colorlinks, pagebackref]{hyperref}
                            % This allows hyperlinks, and no longer needs pdflatex%
                            % because dvips, ps2pdf remembers hyperlinks now.
    }

 \makeatletter
    \ifproofmode
      \renewcommand\@showidx[1]{%
      \insert\indexbox{\scriptsize
        \hsize1.5\marginparwidth
        \hangindent\marginparsep \parindent\z@
        \everypar{}\let\par\@@par \parfillskip\@flushglue
        \lineskip\normallineskip
        \baselineskip .8\normalbaselineskip\sloppy
        \raggedright \leavevmode
        \vrule \@height .7\normalbaselineskip \@width \z@\relax
            \hfuzz=100pt #1\relax
        \vrule \@height \z@ \@depth .3\normalbaselineskip \@width \z@}}
    \fi
   \renewenvironment{theindex}{
     \@restonecoltrue\if@twocolumn\@restonecolfalse\fi
     \columnseprule \z@  \columnsep 30\p@
     \parindent\z@ \parskip\z@ \@plus .3\p@\relax
     \twocolumn[\sektion{\indexname}\indexpreamble]
     \let\item\@idxitem
   }{\if@restonecol\onecolumn\else\clearpage\fi}
 \makeatother

%%%%%%% Pagestyle stuff %%%%%%%%%%%%%%%%%%%%%%%%%%%%%%%%%%%%%%%%%%%%%%%%
 \usepackage{fancyhdr}
     %%%%%%%% Stuff to get marginpars like I like 'em (assumes single-sided pages) %%%%%
     \usepackage{calc}
     \reversemarginpar
     \newlength\fullwidth
     \setlength\fullwidth{\textwidth+2\marginparsep}
     \newcommand\mpar[2][\ ]{\marginpar{\parbox{\marginparwidth}{\raggedleft\scriptsize #1}%
                     \rlap{\hspace*{\fullwidth}{\parbox{\marginparwidth}{\raggedright\scriptsize #2}}}}}
     %%%%%%% End marginpar stuff %%%%%%%%%%%%%%%%%%%%%%%%%%%%%%%%%%%%%%%%

     %%%%%%% Stuff for keeping track of sections %%%%%%%%%%%%%%%%%%%%%%%%
      \newcommand{\sektion}[1]{\newpage \section*{#1} \gdef\sectitle{#1}%
                              \addcontentsline{toc}{section}{#1}%
                              \setcounter{footnote}{0}}
      \newcommand{\subsektion}[1]{\subsection*{#1}%
                              \addcontentsline{toc}{subsection}{#1}}
      \newcommand\sectitle{} %This is the empty section title, before any section title is set
      \newcounter{lecture}
      \setcounter{lecture}{0}
     %%%%%%% End stuff for keeping track of sections %%%%%%%%%%%%%%%%%%%%
   \pagestyle{fancy}
   \fancyhf{} %delete the current section for header and footer
 \ifthenelse{\boolean{lilbook}}{%
   \usepackage[bindingoffset=.25in,%
             paperheight=8.5in,%
             paperwidth=5.5in,%
             outer=.5in,%
             inner=.5in,%
             bottom=.25in,%
             top=.25in,%
             includeheadfoot]{geometry}
   \addtolength{\headwidth}{-.5in}
   \fancyhead[LE,RO]{\thepage}
   \fancyhead[LO]{\sectitle}
   \fancyhead[RE]{Notes for Math 261A - Lie groups and Lie algebras}
%   \setlength{\headheight}{14.5pt}
 }{
   \usepackage[paperheight=11in,%
             paperwidth=8.5in,%
             outer=1.7in,%
             inner=1.7in,%
             bottom=.7in,%
             top=.7in,%
             includeheadfoot]{geometry}
   \addtolength{\headwidth}{-.25in}
   \fancyhead[R]{\thepage}
   \fancyhead[L]{\sectitle}
   \setlength{\headheight}{14.5pt}
 }
     % \renewcommand{\headrulewidth}{0in}
  \raggedbottom
%%%%%%% End Pagestyle stuff %%%%%%%%%%%%%%%%%%%%%%%%%%%%%%%%%%%%%%%%%%%%

%%%%%%%%%%%%%%%%%%%%% Theorem Styles %%%%%%%%%%%%%%%%%%%%%%%%%%%%

 %% Make the equation counter reset each lecture %%
 \makeatletter
    \@addtoreset{equation}{lecture}
 \makeatother

 \theoremstyle{plain}
 \newtheorem{theorem}[equation]{Theorem}
 \newtheorem*{claim}{Claim}
 \newtheorem*{lemma*}{Lemma}
 \newtheorem{lemma}[equation]{Lemma}
 \newtheorem{corollary}[equation]{Corollary}
 \newtheorem{proposition}[equation]{Proposition}

 \theoremstyle{definition}
 \newtheorem{definition}[equation]{Definition}
 \newtheorem{example}[equation]{Example}
 \newtheorem{exercise}{$\blacktriangleright$
                       \hypertarget{Ex\theexercise}{Exercise}}[lecture]

 \theoremstyle{remark}
 \newtheorem{remark}[equation]{Remark}

    %%%%%%%% Solutions %%%%%%%%%%
 \ifthenelse{\boolean{proofmode}}{%
    \usepackage[nosolutionfiles]{answers}}{%
    \usepackage{answers}}
 \Newassociation{solution}{Solution}{exSolutions}
 \ifthenelse{\boolean{proofmode}}{%
    \renewenvironment{Solution}[1]{%\small%
        \begin{trivlist} \item \textit{Solution}~#1.}{%
            \hspace*{\fill} $\blacksquare$\end{trivlist}}%
 }{% else
    \renewenvironment{Solution}[1]{%
        \begin{trivlist} \item \textbf{Solution \hyperlink{Ex#1}{#1}.}}{\end{trivlist}}%
 }

  %%%%%%%% Warnings %%%%%%%%%%
 \newenvironment{warning}{%
    \begin{trivlist} \item[] \noindent%
    \begingroup\hangindent=2pc\hangafter=-2
    \clubpenalty=10000%
    \hbox to0pt{\hskip-\hangindent\dbend\hfill}\ignorespaces
    \refstepcounter{equation}\textit{Warning}~\theequation.}%
    {\endgraf\endgroup\end{trivlist}}
%%%%%%%%%%%%%%%% End theorem styles %%%%%%%%%%%%%%%%%%%%%%%%%%%%%

%%%%%%%%%%%%%%%% Anton's Shortcuts %%%%%%%%%%%%%%%%%%%%%%%%%%%%%%
 \ifthenelse{\boolean{proofmode}}{%
    \newcommand{\anton}[1]{[[\index{"!@notes and corrections}\ensuremath{\bigstar\bigstar\bigstar} #1]]}}{%
    \newcommand{\anton}[1]{}}
 \renewcommand{\a}{\ensuremath{\mathfrak{a}}}
 \DeclareMathOperator{\aut}{Aut}
   % \newcommand{\aut}{\mathrm{Aut}\,}
 \renewcommand{\b}{\ensuremath{\mathfrak{b}}}
 \newcommand{\CC}{\ensuremath{\mathbb{C}}}
 \newcommand{\D}{\ensuremath{\mathcal{D}}}
 \newcommand{\der}[2]{\ensuremath{\frac{d #1}{d #2}}}
 \newcommand{\e}{\ensuremath{\varepsilon}}
 \DeclareMathOperator{\End}{End}
   % \newcommand{\End}{\mathrm{End}}
 \newcommand{\FF}{\ensuremath{\mathbb{F}}}
 \newcommand{\g}{\ensuremath{\mathfrak{g}}}
 \newcommand{\gl}{\ensuremath{\mathfrak{gl}}}
 \newcommand{\HH}{\ensuremath{\mathbb{H}}}
 \newcommand{\HyperPageForJoke}[1]{\index{joke|idxit}\hyperpage{#1}}
 \newcommand{\h}{\ensuremath{\mathfrak{h}}}
 \newcommand{\half}{\ensuremath{\frac{1}{2}}}
 \let\hom\relax % kills the old hom
 \DeclareMathOperator{\hom}{Hom}
    % \renewcommand{\hom}{\mathrm{Hom}}                      % old \hom has small h
 \renewcommand{\labelitemi}{--}                    % changes the default bullet in itemize
 \DeclareMathOperator{\lie}{Lie}
    % \newcommand{\lie}{\mathrm{Lie}}
 \newcommand{\id}{\mathrm{Id}}
 \newcommand{\idxbf}[1]{\textbf{\hyperpage{#1}}}
 \newcommand{\idxit}[1]{\textit{\hyperpage{#1}}}
 \newcommand{\idxbfit}[1]{\textbf{\textit{\hyperpage{#1}}}}
 \DeclareMathOperator{\im}{im}
    % \newcommand{\im}{\mathrm{im\,}}
 \newcommand{\m}{\ensuremath{\mathfrak{m}}}             % only in lec 13
% \newcommand{\M}{\ensuremath{\mathcal{M}}}              % only in lec 01
 \renewcommand{\matrix}[4]{\ensuremath{\left( %
            \begin{smallmatrix} #1 & #2 \\ #3 & #4 \end{smallmatrix} \right)}}
 \newcommand{\mat}[1]{\ensuremath{\left( %
            \begin{array}{cccccccccccccccccccccccccccc} #1 \end{array}\right)}}
 \newcommand{\MM}{\ensuremath{\mathbb{M}}}
 \newcommand{\n}{\ensuremath{\mathfrak{n}}}
 \newcommand{\pder}[2]{\ensuremath{\frac{\partial #1}{\partial #2}}}
 \DeclareMathOperator{\pin}{Pin}
    % \newcommand{\pin}{\mathrm{Pin}}
 \newcommand{\QQ}{\ensuremath{\mathbb{Q}}}
 \newcommand{\RR}{\ensuremath{\mathbb{R}}}
 \renewcommand{\sl}{\ensuremath{\mathfrak{sl}}}
 \newcommand{\so}{\ensuremath{\mathfrak{so}}}
 \renewcommand{\sp}{\ensuremath{\mathfrak{sp}}}         % old \sp is equivalent to ^
 \DeclareMathOperator{\spin}{Spin}
    % \newcommand{\spin}{\mathrm{Spin}}
 \renewcommand{\thelecture}{\arabic{lecture}}
 \renewcommand{\theequation}{\thelecture.\arabic{equation}}
 \newcommand{\udot}{\ensuremath{{\lower .083333em \hbox{\LARGE \kern -.083333em$\cdot$}}}}
 \newcommand{\W}{\ensuremath{\Omega}}
 \newcommand{\w}{\ensuremath{\omega}}
 \newcommand{\weyl}{\ensuremath{\mathfrak{W}}}
 \newcommand{\ZZ}{\ensuremath{\mathbb{Z}}}
%%%%%%%%%%%%% End Anton's shortcuts %%%%%%%%%%%%%%%%%%%%%%%%%%%%%%

\makeindex
\begin{document}
 \title{\vspace*{-2cm} Notes for Math 261A \\ Lie groups and Lie algebras \vspace*{-12mm}}
 \author{}
% \date{}
 \maketitle
 \phantomsection    % This makes the hyperref package happier for some reason
 \ifthenelse{\boolean{proofmode}}{}{\Opensolutionfile{exSolutions}[ExerciseSolutions]}
 {\thispagestyle{empty}
  \vspace*{-2em}
  \addcontentsline{toc}{section}{Contents}
  \tableofcontents
 }
 {
   \sektion{How these notes came to be} Among the Berkeley professors, there was once
  Allen Knutson\index{Knutson, Allen}, who would teach Math 261. But it happened that
  professor Knutson was on sabbatical at UCLA, and eventually went there for good. During
  this turbulent time, Maths 261AB were cancelled two years in a row. The last of these
  four semesters (Spring 2006), some graduate students gathered together and asked
  Nicolai Reshetikhin to teach them Lie theory in a giant reading course. When the dust
  settled, there were two other professors willing to help in the instruction of Math
  261A, Vera Serganova and Richard Borcherds. Thus Tag Team 261A was born.

  After a few lectures, professor Reshetikhin suggested that the students write up the
  lecture notes for the benefit of future generations. The first four lectures were
  produced entirely by the ``editors''. The remaining lectures were \LaTeX ed
  \index{LaTeX@\LaTeX} by Anton Geraschenko in class and then edited by the people in
  the following table. The columns are sorted by lecturer.

  \bigskip

  \hbox{\ifthenelse{\boolean{lilbook}}{
         \footnotesize \hskip -25pt % This is about half the ``overflow'' of the table
         }{
         \small \hskip -15pt }
   \index{Reshetikhin, Nicolai}
   \index{Serganova, Vera}
   \index{Borcherds, Richard E.}
 %  \index{Geraschenko, Anton}
 %  \index{George, Nathan}
 %  \index{Christianson, Hans}
 %  \index{Peters, Emily}
 %  \index{Mkrtchyan, Sevak}
 %  \index{Cimasoni, David}
 %  \index{Chen, Qingtau}
 %  \index{Blasiak, Jonah}
 %  \index{Thiel, Hannes}
 %  \index{Martirosyan, Lilit}
 %  \index{Canez, Santiago}
 %  \index{Liesinger, Katie}
 %  \index{McMillan, Aaron}
 %  \index{Do, Hanh Duc}
 %  \index{Huang, An}
 %  \index{Vito-Cruz, Martin}
   \begin{tabular}{clcclccl}
     \multicolumn{2}{c}{Nicolai Reshetikhin} &&
     \multicolumn{2}{c}{Vera Serganova} &&
     \multicolumn{2}{c}{Richard Borcherds} \\
     \cline{1-2} \cline{4-5} \cline{7-8} \vspace{-3mm}\\
     1 & Anton Geraschenko   && 11 & Sevak Mkrtchyan      && 21 & Hanh Duc Do       \\
     2 & Anton Geraschenko   && 12 & Jonah Blasiak        && 22 & An Huang          \\
     3 & Nathan George       && 13 & Hannes Thiel         && 23 & Santiago Canez    \\
     4 & Hans Christianson   && 14 & Anton Geraschenko    && 24 & Lilit Martirosyan \\
     5 & Emily Peters        && 15 & Lilit Martirosyan    && 25 & Emily Peters      \\
     6 & Sevak Mkrtchyan     && 16 & Santiago Canez       && 26 & Santiago Canez   \\
     7 & Lilit Martirosyan   && 17 & Katie Liesinger      && 27 & Martin Vito-Cruz \\
     8 & David Cimasoni      && 18 & Aaron McMillan       && 28 & Martin Vito-Cruz \\
     9 & Emily Peters        && 19 & Anton Geraschenko    && 29 & Anton Geraschenko  \\
     10 & Qingtau Chen       && 20 & Hanh Duc Do          && 30 & Lilit Martirosyan  \\
       &                     &&    &                      && 31 & Sevak Mkrtchyan \\
   \end{tabular}
  }
  \smallskip
  Richard Borcherds then edited the last third of the notes. The notes were further
  edited (and often expanded or rearranged) by Crystal Hoyt, Sevak Mkrtchyan, and Anton
  Geraschenko.

  \bigskip

  \ifthenelse{\boolean{proofmode}}{%
  If you are reading this, then you have the version of the notes that are for
  proofreading. You will see things written in the upper righthand margins. Those are
  index entries. Please note things that should be indexed, but aren't. You will also
  see comments on things that need to be fixed, which are in double square brackets
  with three big stars, \anton{like this}. Please let Anton know if you have a solution
  for one of these.}{}
  Send corrections and comments to
  \href{mailto:geraschenko@gmail.com}{\path{geraschenko@gmail.com}}.

 \sektion{Dependence of results and other information}
 \anton{make a flow chart of dependence of results}

 Within a lecture, everything uses the same counter, with the exception of exercises.
 Thus, item $a$.$b$ is the $b$-th item in Lecture $a$, whether it is a theorem, lemma,
 example, equation, or anything else that deserves a number and isn't an exercise.
}{
  \stepcounter{lecture}
 \setcounter{lecture}{1}
 \sektion{Lecture 1} \index{Reshetikhin, Nicolai|(}

 \begin{definition}
   A \emph{Lie group}\index{Lie group} is a smooth manifold $G$ with a group structure such that the
   multiplication $\mu:G\times G\to G$ and inverse map $\iota:G\to G$ are smooth maps.
 \end{definition}
 \begin{exercise}
   If we assume only that $\mu$ is smooth, does it follow that $\iota$ is smooth?
   \begin{solution}
     Yes. Consider $\mu^{-1}(e)\subseteq G\times G$. We would like to use the implicit
     function theorem to show that there is a function $f$ (which is as smooth as
     $\mu$) such that $(h,g)\in \mu^{-1}(e)$ if and only if $g=f(h)$. This function
     will be $\iota$. You need to check that for every $g$, the derivative of left
     multiplication by $g$ at $g^{-1}$ is non-singular (i.e.\ that $dl_{g}(g^{-1})$ is
     a non-singular matrix). This is obvious because we have an inverse, namely
     $dl_{g^{-1}}(e)$.
   \end{solution}
 \end{exercise}

 \begin{example}
   The group of invertible endomorphisms of $\CC^n$, $GL_n(\CC)$, is a Lie group. The
   automorphisms of determinant 1, $SL_n(\CC)$, is also a Lie group.
 \end{example}
 \begin{example}
   If $B$ is a bilinear form on $\CC^n$, then we can consider the Lie group
   \[
    \{ A\in GL_n(\CC)| B(Av,Aw)=B(v,w) \text{ for all }v,w\in \CC^n \}.
   \]
   If we take $B$ to be the usual dot product, then we get the group $O_n(\CC)$. If we
   let $n=2m$ be even and set $B(v,w)= v^T \matrix 0{1_m}{-1_m}0 w$, then we get
   $Sp_{2m}(\CC)$.
 \end{example}
% \begin{example}
%  $SL_n, O_n, Sp_{2n}$ (for $O_n$ and $Sp_{2n}$, you can describe them as those
%  linear transformations preserving some quadratic form).
% \end{example}
 \begin{example}
   $SU_n\subseteq SL_n(\CC)$ is a \emph{real form}\index{real form} (look in lectures
   27,28, and 29 for more on real forms).
 \end{example}
 \begin{example}
   We'd like to consider infinite matrices, but the multiplication wouldn't make
 sense, so we can think of $GL_n\subseteq GL_{n+1}$ via $A\mapsto
 \matrix{A}{0}{0}{1}$, then define $GL_{\infty}$ as $\bigcup_n GL_n$. That is,
 invertible infinite matrices which look like the identity almost everywhere.
 \end{example}
 Lie groups are hard objects to work with because they have global characteristics,
 but we'd like to know about representations of them. Fortunately, there are things
 called Lie algebras, which are easier to work with, and representations of Lie
 algebras tell us about representations of Lie groups.

 \begin{definition}
   A \emph{Lie algebra}\index{Lie algebra} is a vector space $V$ equipped with a
   \emph{Lie bracket} $[\ ,\,]:V\times V\to V$, which satisfies
   \begin{enumerate}
     \item Skew symmetry: $[a,a]=0$ for all $a\in V$, and

     \item Jacobi identity\index{Jacobi identity}: $[a,[b,c]]+[b,[c,a]]+[c,[a,b]]=0$
     for all $a,b,c\in V$.
   \end{enumerate}
   A \emph{Lie subalgebra} of a Lie algebra $V$ is a subspace $W\subseteq V$ which is
   closed under the bracket: $[W,W]\subseteq W$.
 \end{definition}

 \begin{example}
   If $A$ is a finite dimensional associative algebra, you can set $[a,b]=ab-ba$. If
   you start with $A=\MM_n$, the algebra of $n\times n$ matrices, then you get the Lie
   algebra $\gl_n$\index{gl(n)@$\gl(n)$}. If you let $A\subseteq \MM_n$ be the algebra
   of matrices preserving a fixed flag $V_0\subset V_1\subset \cdots V_k \subseteq
   \CC^n$, then you get \emph{parabolic}index{parabolic subalgebras} Lie sub-algebras
   of $\gl_n$.
 \end{example}
% \underline{Examples}:
% \begin{itemize}
% \item[(a)] $A$, a finite dimensional associative algebra with $[a,b]=ab-ba$.
% \item[(b)] $A=\MM_n \rightsquigarrow \gl_n$
% \item[(c)] $A\subseteq \MM_n$ matrices of the form ... $\rightsquigarrow$ parabolic Lie sub-algebras of $\gl_n$
% \end{itemize}

 \begin{example}
   Consider the set of vector fields on $\RR^n$, $\mathrm{Vect}(\RR^n) = \{
   \ell=\Sigma e^i(x)\pder{}{x_i} | [\ell_1,\ell_2]=\ell_1\circ
   \ell_2-\ell_2\circ\ell_1\}$.
   \begin{exercise}
     Check that $[\ell_1,\ell_2]$ is a first order differential operator.
     \begin{solution}
       Just do it.
     \end{solution}
   \end{exercise}
 \end{example}
 \begin{example}
   If $A$ is an associative algebra, we say that $\partial:A\to A$ is a derivation if
   $\partial(ab)=(\partial a)b+a\partial b$. Inner derivations are those of the form
   $[d,\cdot ]$ for some $d\in A$; the others are called outer derivations. We denote
   the set of derivations of $A$ by $\D(A)$, and you can verify that it is a Lie algebra.
   Note that $\mathrm{Vect}(\RR^n)$ above is just $\D(C^\infty(\RR^n))$.
 \end{example}
 The first Hochschild cohomology\index{cohomology!Hochschild}, denoted $H^1(A,A)$, is
 the quotient $\D(A)/\{$inner derivations$\}$.

 \begin{definition}
   A \emph{Lie algebra homomorphism} is a linear map $\phi:L\to L'$ that takes the
   bracket in $L$ to the bracket in $L'$, i.e.\
   $\phi([a,b]_L)=[\phi(a),\phi(b)]_{L'}$. A \emph{Lie algebra isomorphism} is a
   morphism of Lie algebras that is a linear isomorphism.\footnote{The reader may
   verify that this implies that the inverse is also a morphism of Lie algebras.}
 \end{definition}

 A very interesting question is to classify Lie algebras (up to isomorphism) of
 dimension $n$ for a given $n$. For $n=2$, there are only two: the trivial bracket
 $[\ ,\,]=0$, and $[e_1,e_2]=e_2$. For $n=3$, it can be done without too much trouble.
 Maybe $n=4$ has been done, but in general, it is a very hard problem.

 If $\{e_i\}$ is a basis for $V$, with $[e_i,e_j]=c_{ij}^k e_k$ (the $c_{ij}^k$ are
 called the \emph{structure constants} of $V$), then the Jacobi identity is some
 quadratic relation on the $c_{ij}^k$, so the variety of Lie algebras is some
 quadratic surface in $\CC^{3n}$.\index{variety of Lie algebras}

 Given a smooth real manifold $M^n$ of dimension $n$, we can construct
 $\mathrm{Vect}(M^n)$, the set of smooth vector fields on $M^n$. For $X\in
 \mathrm{Vect}(M^n)$, we can define the \emph{Lie derivative}\index{Lie derivative}
 $L_X$ by $(L_X\cdot f)(m)=X_m\cdot f$, so $L_X$ acts on $C^\infty(M^n)$ as a
 derivation.
 \begin{exercise}
  Verify that $[L_X,L_Y]=L_X\circ L_Y-L_Y\circ L_X$ is of the form $L_Z$ for a unique
  $Z\in \mathrm{Vect}(M^n)$. Then we put a Lie algebra structure on
  $\mathrm{Vect}(M^n)=\D(C^\infty(M^n))$ by $[X,Y]=Z$.
 \end{exercise}
 There is a theorem (Ado's Theorem\footnote{Notice that if $\g$ has no center, then
 the adjoint representation\index{adjoint representation} $ad:\g\to \gl(\g)$ is
 already faithful. See Example \ref{lec07eg:adjoint} for more on the adjoint
 representation. For a proof of Ado's Theorem, see Appendix E of
 \cite{FulHar}})\index{Ado's Theorem} that any Lie algebra $\g$ is isomorphic to a Lie
 subalgebra of $\gl_n$, so if you understand everything about $\gl_n$, you're in
 pretty good shape.
}{   % Anton, geraschenko@gmail.com
  \stepcounter{lecture}
 \setcounter{lecture}{2}
 \sektion{Lecture 2}

 Last time we talked about Lie groups, Lie algebras, and gave examples. Recall that
 $M\subseteq L$ is a Lie subalgebra if $[M,M]\subseteq M$. We say that $M$ is a
 \emph{Lie ideal}\index{Lie ideal} if $[M,L]\subseteq M$.

 \begin{claim}
   If $M$ is an ideal, then $L/M$ has the structure of a Lie algebra such that the
   canonical projection is a morphism of Lie algebras.
 \end{claim}
 \begin{proof}
   Take $l_1,l_2\in L$, check that $[l_1+M,l_2+M]\subseteq [l_1,l_2]+M$.
 \end{proof}

 \begin{claim} For $\phi:L_1\to L_2$ a Lie algebra homomorphism,
   \begin{enumerate}
   \item $\ker \phi \subseteq L_1$ is an ideal,
   \item $\im \phi\subseteq L_2$ is a Lie subalgebra,
   \item $L_1/\ker\phi \cong \im\phi$ as Lie algebras.
   \end{enumerate}
 \end{claim}
 \begin{exercise}
   Prove this claim.
 \end{exercise}

 \subsektion{Tangent Lie algebras to Lie groups} Let's recall some differential
 geometry. You can look at \cite{Lee:ISM} as a reference. If $f:M\to N$ is a
 differentiable map, then $df:TM\to TN$ is the derivative. If $G$ is a group, then we
 have the maps $l_g:x\mapsto gx$ and $r_g:x\mapsto xg$. Recall that a smooth vector
 field is a smooth section of the tangent bundle $TM\to M$.
 \begin{definition}
   A vector field $X$ is \emph{left invariant} if $(dl_g)\circ X=X\circ l_g$ for all
   $g\in G$. The set of left invariant vector fields is called $\mathrm{Vect}_L(G)$.
   \[\xymatrix{
    TG \ar[r]^{dl_g} & TG\\
    G\ar[u]^X \ar[r]^{l_g} & G\ar[u]_X
   }\]
 \end{definition}

 \begin{proposition}
   $\mathrm{Vect}_L(G)\subseteq \mathrm{Vect}(G)$ is a Lie subalgebra.
 \end{proposition}
 \begin{proof}
   We get an induced map $l_g^*:C^\infty(G)\to C^\infty(G)$, and $X$ is left invariant
   if and only if $L_X$ commutes with $l_G^*$. Then\\
    $X,Y$ left invariant $\Longleftrightarrow [L_X,L_Y]$ invariant $\Longleftrightarrow [X,Y]$
    left invariant.
 \end{proof}
 All the same stuff works for right invariant vector fields $\mathrm{Vect}_R(G)$.

 \begin{definition}
   $\g = \mathrm{Vect}_L(G)$ is the tangent Lie algebra of $G$. \index{Lie algebra!of
   a Lie group}
 \end{definition}

 \begin{proposition}
   There are vector space isomorphisms $\mathrm{Vect}_L(G)\simeq T_eG$ and
   $\mathrm{Vect}_R(G)\simeq T_eG$. Moreover, the Lie algebra structures on $T_eG$
   induced by these isomorphisms agree.
 \end{proposition}
 Note that it follows that $\dim \g=\dim G$.
 \begin{proof}
   \anton{I think this proof could be improved ... it is incomplete as is}
   Recall fibre bundles. $dl_g:T_eG\xrightarrow{\sim} T_gG$, so $TG\simeq T_e\times
   G$. $X$ is a section of $TG$, so it can be thought of as
   $X:G\to T_eG$, in which case the left invariant fields are exactly those which are
   constant maps, but the set of constants maps to $T_eG$ is isomorphic to $T_eG$.
 \end{proof}

 If $G$ is an $n$ dimensional $C^\w$ Lie group, then $\g$ is an $n$ dimensional Lie algebra.
 If we take local coordinates near $e\in G$ to be $x^1,\dots, x^n:U_e\to \RR^n$ with
 $m:\RR^n\times \RR^n\to \RR^n$ the multiplication (defined near 0). We have a power
 series for $m$ near 0,
 \[
    m(x,y) = Ax+By+\alpha_2(x,y)+\alpha_3(x,y)+\cdots
 \]
 where $A,B:\RR^n\to \RR^n$ are linear, $\alpha_i$ is degree $i$. Then we can consider
 the condition that $m$ be associative (only to degree 3): $m(x,m(y,z))=m(m(x,y),z)$.
 \begin{align*}
   m(x,m(y,z)) &= Ax+Bm(y,z)+\alpha_2(x,m(y,z))+\alpha_3(x,m(y,z))+\cdots\\
        &=
        Ax+B(Ay+Bz+\alpha_2(y,z)+\alpha_3(y,z)))\\
        &\qquad +\alpha_2(x,Ay+Bz+\alpha_2(y,z))+\alpha_3(x,Ay+Bz)\\
   m(m(x,y),z) &= \anton{\text{compute this stuff}}
 \end{align*}
 Comparing first order terms (remember that $A,B$ must be non-singular), we can get that
 $A=B=I_n$. From the second order term, we can get that $\alpha_2$ is bilinear!
 Changing coordinates ($\phi(x)=x+\phi_2(x)+\phi_3(x)+\cdots$, with
 $\phi^{-1}(x)=x-\phi_2(x)+\tilde\phi_3(x)+\cdots$), we use the fact that
 $m_\phi(x,y)=\phi^{-1}m(\phi x,\phi y)$ is the new multiplication, we have
 \[
    m_\phi(x,y) = x+y + (\underbrace{\phi_2 (x) +\phi_2 (y)+\phi_2 (x+y)}_{\text{can
    be any symm form}}) + \alpha_2(x,y)+ \cdots
 \]
 so we can tweak the coordinates to make $\alpha_2$ skew-symmetric. Looking at order 3, we
 have
 \begin{equation}\label{lec02Eq:1}
  \alpha_2(x,\alpha_2(y,z))+\alpha_3(x,y+z) =
  \alpha_2(\alpha_2(x,y),z)+\alpha_3(x+y,z)
 \end{equation}
 \begin{exercise}
   Prove that this implies the Jacobi identity for $\alpha_2$. (hint: skew-symmetrize
   equation \ref{lec02Eq:1})
 \end{exercise}
 Remarkably, the Jacobi identity is the only obstruction to associativity; all
 other coefficients can be eliminated by coordinate changes.

 \begin{example}
   Let $G$ be the set of matrices of the form $\matrix{a}{b}{0}{a^{-1}}$ for $a,b$
   real, $a>0$. Use coordinates $x,y$ where $e^x=a$, $y=b$, then
   \begin{align*}
      m((x,y),(x',y')) &= (x+x',e^xy'+ye^{-x'})\\
              &= (x+x',y+y'+ (\underbrace{xy'-x'y}_{skew})+ \cdots).
   \end{align*}
   The second order term is skew symmetric, so these are good coordinates. There are
   $H,E\in T_eG$ corresponding to $x$ and $y$ respectively so that
   $[H,E]=E$\footnote{what does this part mean?}.
 \end{example}

 \begin{exercise}
    Think about this. If $a,b$ commute, then $e^ae^b=e^{a+b}$. If
    they do not commute, then $e^ae^b=e^{f(a,b)}$. Compute $f(a,b)$ to order 3.
 \end{exercise}
}{   % Anton, geraschenko@gmail.com
   \stepcounter{lecture}
  \setcounter{lecture}{3}
  \sektion{Lecture 3}

 Last time we saw how to get a Lie algebra $\lie(G)$ from a Lie group $G$.

 \begin{itemize}
 \item[] $\mathrm{Lie}(G)=\mathrm{Vect}_L(G)\simeq \mathrm{Vect}_R(G)$.

 \item[] Let $x^1,...,x^n$ be local coordinates near $e\in G$, and let  $m(x,y)^i$
 be the $i^{th}$ coordinate of $(x,y) \mapsto m(x,y)$. In this local coordinate
 system, $m(x,y)^i = x^i + y^i + \frac{1}{2} \sum c^i_{jk} x^j y^k + \cdots$. If
 $e_1,...,e_n \in T_eG$ is the basis induced by $x^1,...,x^n$, $(e_i \sim
 \partial_i)$, then
 \[
 [e_i,e_j] = \sum_k c_{ij}^k e_k.
 \]
 \end{itemize}

 \begin{example} Let $G$ be $GL_n$, and let $(g_{ij})$ be coordinates.
 Let $X:GL_n\rightarrow TGL_n$ be a vector field.
 \begin{align*}
 L_X(f)(g) &=
 \sum_{i,j} X_{ij}(g)\frac{\partial f(g)}{\partial g_{ij}} \\
    &\qquad \qquad \text{, where } L_X(l_h^*(f))(g) = \left\{
            \begin{array}{l} l_h:g\mapsto hg \\
                        l_h^*(f)(g) = f(h^{-1}g)
            \end{array} \right\} \\
 &= \sum_{i,j} X_{ij}(g) \frac{\partial f(h^{-1}g)}{\partial g_{ij}}
    = \sum_{i,j} X_{ij} (g) \frac{\partial (h^{-1}g)_{kl}}{\partial
    g_{ij}} \frac{\partial f(x)}{\partial x_kl} |_{x=h^{-1}g} \\
 &= \langle \frac{\partial(h^{-1}g)_{kl}}{\partial g_{ij}} = \sum_m
    (h^{-1})_{km}\underbrace{\frac{\partial g_{ml}}{\partial
    g_{ij}}}_{=\delta_{im}\delta_{lj}} = (h^{-1})_{ki}\delta_{jl} \rangle \\
 &= \sum_{i,j,k} X_{ij}(g)(h^{-1})_{ki} \frac{\partial f}{\partial
    x_{kj}}|_{x=h^{-1}g} \\
 &= \sum_{j,k} \left( \sum_i
    (h^{-1})_{ki}X_{ij}(g)\right) \frac{\partial f}{\partial x_{kj}}|_{x=h^{-1}g}
 \end{align*}
 \end{example}

 If we want $X$ to be left invariant,
 $\sum_i(h^{-1})_{ki}X_{ij}(g)=X_{kj}(h^{-1}g)$, then $L_X(l_h^*(f))
 = l_h^*(L_X(f))$, (left invariance of $X$).

 \begin{example}
 All solutions are $X_{ij}(g) = (g\cdot M)_{ij}$, $M$-constant $n\times n$ matrix.
 gives that left invariant vector fields on $GL_n \approx n\times n$ matrices
 $=\mathfrak{gl}_n$. The ``Natural Basis'' is $e_{ij}=(E_{ij})$, $L_{ij}=\sum_m
 (g)_{mj}\frac{\partial}{\partial g_{mi}}$. \mpar[\anton{what is this?}]{}
 \end{example}
 \mpar[\anton{What is up with these examples?}]{}
 \begin{example}
 Commutation relations between $L_{ij}$ are the same as commutation relations between
 $e_{ij}$.
 \end{example}

 Take $\tau \in T_eG$.  Define the vector field: $v_{\tau}: G\rightarrow TG$ by
 $v_{\tau}(g)=dl_g(\tau)$, where $l_g : G\rightarrow G$ is left multiplication.
 $v_{\tau}$ is a left invariant vector field by construction.

 Consider $\phi: I \rightarrow G$, $\frac{d\phi(t)}{dt} =
 v_{\tau}(\phi(t))$, $\phi(0)=e$.

 \begin{proposition}
 \begin{enumerate}\item[]
 \item $\phi(t+s) = \phi(t)\phi(s)$
 \item $\phi$ extends to a smooth map $\phi: \mathbb{R} \rightarrow G$.
 \end{enumerate}
 \end{proposition}

 \begin{proof}
 \begin{enumerate}
 \item Fix $s$ and $\alpha(t)=\phi(s)\phi(t), \beta(t)=\phi(s+t)$.
    \begin{itemize}
    \item $\alpha(0)=\phi(s)=\beta(0)$
    \item $\frac{d\beta(a)}{dt}=\frac{d\phi(s+t)}{dt} = v_{\tau}(\beta(t))$
    \item $\frac{d\alpha(t)}{dt}=\frac{d}{dt}(\phi(s)\phi(t))
    = dl_{\phi(s)}\cdot v_{\tau}(\phi(t)) = v_{\tau}(\phi(s)\phi(t)) =
    v_{\tau}(\alpha(t))$, where the second equality is because $v_{\tau}$ is linear.
    \end{itemize}
 $\implies \alpha$ satisfies same
 equation as $\beta$ and same initial conditions, so by uniqueness,
 they coincide for $|t|<\epsilon$.
 \item Now we have (1) for $|t+s|<\epsilon$, $|t|<\epsilon$,
 $|s|<\epsilon$.  Then extend $\phi$ to $|t|<2\epsilon$.  Continue
 this to cover all of $\mathbb{R}$.
 \end{enumerate}
 \end{proof}

 This shows that for all $\tau \in T_eG$, we have a mapping
 $\RR \rightarrow G$ and it's image is a 1-parameter
 (1 dimensional) Lie subgroup\index{one-parameter subgroup} in $G$.
 \begin{eqnarray*}
 \exp : \g=T_eG & \rightarrow & G \\
        \tau    & \mapsto     & \phi_{\tau}(1)=\exp(\tau)
 \end{eqnarray*}\index{exponential map}
 Notice that $\lambda\tau \mapsto
 \exp(\lambda\tau)=\phi_{\lambda\tau}(1)=\phi_{\tau}(\lambda)$

 \begin{example} $GL_n$, $\tau \in \mathfrak{gl}_n=T_eGL_n$,
 $\frac{d\phi(t)}{dt}=v_{\tau}(\phi(t))\in T_{\phi(t)}GL_n \simeq
 \mathfrak{gl}_n$. \[ v_{\tau}(\phi(t))=\phi(t)\cdot \tau,\;
 \frac{d\phi(t)}{dt}=\phi(t)\cdot \tau,\; \phi(0)=I,\;\]
 \[\phi(t)=\exp(tI)=\sum_{n=0}^{\infty} \frac{t^n\tau^n}{n!}\] $\exp:
 \mathfrak{gl}_n \rightarrow GL_n$
 \[
 \left[ L_{\gamma(0)=g}(f)(g)=\frac{d}{dt}f(\gamma(t))|_{t=0}\right]
 \]
 \end{example}

  {\bf Baker-Campbell-Hausdorff formula:}\index{Baker-Campbell-Hausdorff|idxbf}
  \[ e^X\cdot e^Y = e^{H(X,Y)} \]
  \[ H(X,Y) = \underbrace{X+Y}_{\rm sym} + \underbrace{\frac{1}{2}[X,Y]}_{\rm skew} +
  \underbrace{\frac{1}{12}([X[X,Y]]+[Y[Y,X]])}_{\rm symmetric} + \cdots
  \]

 \begin{proposition}
  \begin{enumerate}
  \item Let $f:G\rightarrow H$ be a Lie group
  homomorphism, then the diagram
  $\xymatrix{
    G \ar[r]^f \ar@{<-}[d]_\exp & H \ar@{<-}[d]^\exp\\
    Lie(G)\ar[r]^{df_e} & Lie(H)
  }$
  is commutative.
  \item If $G$ is \emph{connected}, then $(df)_e$ defines the Lie group
  homomorphism $f$ uniquely.
  \end{enumerate}
 \end{proposition}
 \begin{proof}
 Next time.
 \end{proof}

 \begin{proposition} $G,H$ Lie groups, $G$ \emph{simply connected},
 then $\alpha:\mathrm{Lie}(G) \rightarrow \mathrm{Lie}(H)$ is a Lie
 algebra homomorphism if and only if there is a Lie group homomorphism
 $A:G\rightarrow H$ lifting $\alpha$.
 \end{proposition}
 \begin{proof}
 Next time.
 \end{proof}

 $\{$Lie algebras $\}\xrightarrow{\exp}\{$ Lie
 groups(connected, simply connected)$\}$ is an equivalence of
 categories.
}{   % Nathan George, nathandgeorge@gmail
  \stepcounter{lecture}
 \setcounter{lecture}{4}
 \sektion{Lecture 4}

\begin{theorem}
Suppose $G$ is a topological group.  Let $U \subset G$ be an open neighbourhood of $e
\in G$.  If $G$ is connected, then
\begin{eqnarray*}
G = \bigcup_{n \geq 1} U^n.
\end{eqnarray*}
\end{theorem}
\begin{proof}
Choose a non-empty open set $V \subset U$ such that $V = V^{-1}$, for example $V = U
\cap U^{-1}$. Define $H = \cup_{n \geq 1} V^n$, and observe $H$ is an abstract
subgroup, since $V^n V^m \subseteq V^{n+m}$.  $H$ is open since it is the union of
open sets.  If $\sigma \notin H$, then $\sigma H \not\subset H$, since otherwise if
$h_1, h_2 \in H$ satisfy $\sigma h_1 = h_2$, then $\sigma = h_2 h_1^{-1} \in H$.  Thus
$H$ is a complement of the union of all cosets not containing $H$.  Hence $H$ is
closed. Since $G$ is connected, $H = G$.
\end{proof}
\begin{theorem}
Let $f: G \to H$ be a Lie group homomorphism.  Then the following diagram commutes:
 \[\xymatrix{
 T_e \ar[r]^{(df)_e} \ar[d]_{\exp} & T_e H\ar[d]^{\exp}\\
 G \ar[r]^f & H
  }\]
Further, if $G$ is connected, $(df)_e$ determines $f$ uniquely.
\end{theorem}

\begin{proof}
1) Commutative diagram.  Fix $\tau \in T_eG$ and set $\eta = df_e \tau \in T_eH$.
Recall we defined the vector field $V_\tau(g) = (dl_g)(\tau)$, then if $\phi(t)$
solves
\begin{eqnarray*}
\frac{ d \phi}{dt} = V_\tau ( \phi(t)) \in T_{\phi(t)}G,
\end{eqnarray*}
we have $\exp(\tau) = \phi(1)$.  Let $\psi$ solve
\begin{eqnarray*}
\frac{d \psi}{dt} = V_\eta( \psi(t)),
\end{eqnarray*}
so that $\exp ( \eta) = \psi(1)$.  Observe $\tilde{\psi}(t) = f( \phi(t))$ satisfies
\begin{eqnarray*}
\frac{d \tilde{\psi}}{dt} = (df)\left( \frac{d\phi}{dt} \right) =
V_\eta(\tilde{\psi}),
\end{eqnarray*}
so by uniqueness of solutions to ordinary differential equations, $\psi =
\tilde{\psi}$.

2) Uniqueness of $f$.  The exponential map is an isomorphism of a neighborhood of $0
\in \g$ and a neighborhood of $e \in G$.  But if $G$ is connected, $G = \cup_{n \geq
1} (\text{nbd } e)^n$.
\end{proof}

\begin{theorem}
Suppose $G$ is a topological group, with $G^0 \subset G$ the connected component of
$e$.  Then 1) $G^0$ is normal and 2) $G / G^0$ is discrete.
\end{theorem}

\begin{proof}
2) $G^0 \subset G$ is open implies $\text{pr}^{-1}( [e]) = e G^0$ is open in $G$,
which in turn implies $\text{pr}^{-1}([g]) \in G/G^0$ is open for every $g \in G$.
Thus each coset is both open and closed, hence $G/G^0$ is discrete.

1) Fix $g \in G$ and consider the map $G \to G$ defined by $x \mapsto g x g^{-1}$.
This map fixes $e$ and is continuous, which implies it maps $G^0$ into $G^0$.  In
other words, $g G^0 g^{-1} \subset G^0$, or $G^0$ is normal.
\end{proof}

We recall some basic notions of algebraic topology.  Suppose $M$ is a connected
topological space.  Let $x, y \in M$, and suppose $\gamma(t): [0,1] \to M$ is a path
from $x$ to $y$ in $M$.  We say $\tilde{\gamma}(t)$ is {\it homotopic} to $\gamma$ if
there is a continuous map $h(s,t): [0,1]^2 \to M$ satisfying
\begin{eqnarray*}
\bullet h(s,0) = x, \,\, h(s,1) = y \\
\bullet h(0,t) = \gamma(t), \,\, h(1,t) = \tilde{\gamma}(t).
\end{eqnarray*}
We call $h$ the {\it homotopy}.  On a smooth manifold, we may replace $h$ with a
smooth homotopy.  Now fix $x_0 \in M$.  We define the first fundamental group of $M$
\begin{eqnarray*}
\pi_1(M, x_0) = \left\{ \text{ homotopy classes of loops based at }x_0 \right\}.
\end{eqnarray*}
It is clear that this is a group with group multiplication composition of paths.  It
is also a fact that the definition does not depend on the base point $x_0$:
\begin{eqnarray*}
\pi_1(M, x_0) \simeq \pi_1(M, x_0').
\end{eqnarray*}
By $\pi_1(M)$ we denote the isomorphism class of $\pi_1(M, \cdot)$. Lastly, we say $M$
is {\it simply connected} if $\pi_1(M) = \{e\}$, that is if all closed paths can be
deformed to the trivial one.

\begin{theorem}
\label{lec04T:4} Suppose $G$ and $H$ are Lie groups with Lie algebras $\g, \h$
respectively. If $G$ is simply connected, then any Lie algebra homomorphism $\rho: \g
\to \h$ lifts to a Lie group homomorphism $R : G \to H$.
\end{theorem}

In order to prove this theorem, we will need the following lemma.
\begin{lemma}
Let $\xi : \mathbb{R} \to \g$ be a smooth mapping.  Then
\begin{eqnarray*}
\frac{dg}{dt} = (dl_g)(\xi(t))
\end{eqnarray*}
has a unique solution on all of $\mathbb{R}$ with $g(t_0) = g_0$.
\end{lemma}
For convenience, we will write $g \xi:= (dl_g)(\xi)$.
\begin{proof}
Since $\g$ is a vector space, we identify it with $\mathbb{R}^n$ and for sufficiently
small $r>0$, we identify $B_r(0) \subset \g$ with a small neighbourhood of $e$,
$U_e(r) \subset G$, under the exponential map.  Here $B_r(0)$ is measured with the
usual Euclidean norm $\| \cdot \|$.  Note for any $g \in U_e(r)$ and $|t-t_0|$
sufficiently small, we have $\|g \xi(t) \| \leq C$.  Now according to Exercise
\ref{ex-l4-1}, the solution with $g(t_0) = e$ exists for sufficiently small $|t-t_0|$
and
\begin{eqnarray*}
g(t) \in U_e(r) \,\, \forall |t-t_0| < \frac{r}{C'}.
\end{eqnarray*}
Now define $h(t) = g(t) g_0$ so that $h(t) \in U_{g_0}(r)$ for $|t-t_0| < r/C'$.  That
is, $r$ and $C'$ do not depend on the choice of initial conditions, and we can cover
$\mathbb{R}$ by intervals of length, say $r/C'$.
\end{proof}
\begin{exercise}
\label{ex-l4-1} Verify that there is a constant $C'$ such that if $|t-t_0|$ is
sufficiently small, we have
\begin{eqnarray*}
\|g(t)\| \leq C' |t-t_0|.
\end{eqnarray*}
  \begin{solution}
  We calculate:
  \begin{eqnarray*}
  \frac{d}{dt} \| g(t) \|^2 & = & 2 \langle \frac{d}{dt} g, g \rangle \\
  & \leq & 2\left\| \frac{d}{dt} g \right\|  \| g \|\\
  & \leq & 2\| \xi\| \| g \|^2.
  \end{eqnarray*}
  That is, $\eta(t) := \|g(t)\|^2$ satisfies the differential inequality:
  \begin{eqnarray*}
  \frac{d}{dt} \eta(t) \leq \|\xi\| \eta(t),
  \end{eqnarray*}
  which in turn implies (Gronwall's inequality) that
  \begin{eqnarray*}
  \eta(t)  \leq  e^{2\int_{t_0}^t \| \xi(s) \| ds}
  \end{eqnarray*}
  so that
  \begin{eqnarray*}
  \|g\| &  \leq & e^{\int_{t_0}^t \| \xi(s) \| ds} \\
  & \leq & C'|t - t_0|
  \end{eqnarray*}
  since for $|t-t_0|$ sufficiently small, exponentiation is Lipschitz.
  \end{solution}
\end{exercise}


\begin{proof}[Proof of Theorem \ref{lec04T:4}]
We will construct $R: G \to H$.  Beginning with $g(t): [0,1] \to G$ satisfying $g(0) =
e$, $g(1) = g$, define $\xi(t) \in \g$ for each $t$ by
\begin{eqnarray*}
g(t) \xi(t) = \frac{d}{dt} g(t).
\end{eqnarray*}
Let $\eta(t) = \rho(\xi(t))$, and let $h(t):[0,1] \to H$ satisfy
\begin{eqnarray*}
\frac{d}{dt} h(t) = h(t) \eta(t), \,\,\,\, h(0) = e.
\end{eqnarray*}
Define $R(g) = h(1)$.

{\bf Claim:} $h(1)$ does not depend on the path $g(t)$, only on $g$.

\noindent {\it Proof of Claim}.  Suppose $g_1(t)$ and $g_2(t)$ are two different paths
connecting $e$ to $g$.  Then there is a smooth homotopy $g(t,s)$ satisfying $g(t,0) =
g_1(t)$, $g(t,1) = g_2(t)$. Define $\xi(t,s)$ and $\eta(t,s)$ by
\begin{eqnarray*}
\frac{\partial g}{ \partial t} & = & g(t,s) \xi(t,s); \\
\frac{\partial g}{ \partial s} & = & g(t,s) \eta(t,s).
\end{eqnarray*}
Observe
\begin{eqnarray}
\frac{\partial^2 g}{ \partial s \partial t} & = & g \eta \circ \xi + g
\frac{\partial \xi }{\partial t} \,\,\, \text{and } \label{l4-eq-3}\\
\frac{\partial^2 g}{ \partial t \partial s} & = & g \xi \circ \eta + g \frac{\partial
\eta}{\partial s} \label{l4-eq-4},
\end{eqnarray}
and \eqref{l4-eq-3} is equal to \eqref{l4-eq-4} since $g$ is smooth. Consequently
\begin{eqnarray*}
\frac{\partial \eta}{\partial t} - \frac{\partial \xi}{ \partial s} =
     [ \eta, \xi ].
\end{eqnarray*}
Now define an $s$ dependent family of solutions $h(\cdot, s)$ to the equations
\begin{eqnarray*}
\frac{\partial h}{\partial t} (t,s) = h(t,s) \rho(\xi(t,s)), \,\,\, h(0,s) = e.
\end{eqnarray*}
Define $\theta(t,s)$ by
\begin{eqnarray}
\label{l4-eq-5} \left\{ \begin{array}{c} \displaystyle{\frac{\partial \theta}{\partial
t}} - \displaystyle{\frac{
    \partial \rho(\xi)}{\partial s}} = \left[ \rho(\xi), \theta
    \right], \\
  \theta(0,s) = 0. \end{array} \right.
\end{eqnarray}
Observe $\tilde{\theta}(t,s) = \rho(\eta(t,s))$ also satisfies equation
\eqref{l4-eq-5}, so that $\theta = \tilde{\theta}$ by uniqueness of solutions to ODEs.
Finally,
\begin{eqnarray*}
g \eta(1,s) = \frac{\partial g}{\partial s} (1,s) = 0 \implies \theta(1,s) = 0
\implies \frac{\partial h }{\partial s}(1,s) = 0,
\end{eqnarray*}
justifying the claim.

We need only show $R:G \to H$ is a homomorphism.  Let $g_1, g_2 \in G$ and set $g =
g_1g_2$.  Let $\tilde{g}_i(t)$ be a path from $e$ to $g_i$ in $G$ for each $i = 1,2$.
Then the path $\tilde{g}(t)$ defined by
\begin{eqnarray*}
\tilde{g}(t) = \left\{ \begin{array}{l} \tilde{g}_1(2t), \,\, 0 \leq t
  \leq \frac{1}{2}, \\ g_1 \tilde{g}_2(2t-1), \,\, \frac{1}{2} \leq t
  \leq 1 \end{array} \right.
\end{eqnarray*}
goes from $e$ to $g$.  Let $\tilde{h}_i$ for $i = 1,2$ and $\tilde{h}$ be the paths in
$H$ corresponding to $\tilde{g}_1$, $\tilde{g}_2$, and $\tilde{g}$ respectively and
calculate
\begin{eqnarray*}
R(g_1 g_2) = R(g) = \tilde{h}(1) = \tilde{h}_1(1) \tilde{h}_2(1) = R(g_1) R(g_2).
\end{eqnarray*}
\end{proof}
}{   % Hans Christianson, hans@math
  \stepcounter{lecture}
 \setcounter{lecture}{5}
 \sektion{Lecture 5}

 Last time we talked about connectedness, and proved the following things:
 \begin{itemize}
 \item[-] Any connected topological group $G$ has the
 property that $G=\bigcup_n V^n$, where $V$ is any neighborhood of $e\in G$.

 \item[-] If $G$ is a connected Lie group, with $\alpha:\mathrm{Lie}(G)\to
 \mathrm{Lie}(H)$ a Lie algebra homomorphism, then if there exists $f:G\to H$ with
 $df_e=\alpha$, it is unique.

 \item[-] If $G$ is connected, simply connected, with $\alpha:\mathrm{Lie}(G)\to
 \mathrm{Lie}(H)$ a Lie algebra homomorphism, then there is a unique $f:G\to H$ such
 that $df_e=\alpha$.
 \end{itemize}

 \subsektion{Simply Connected Lie Groups}
 The map $p$ in $Z\to X\xrightarrow{p} Y$ is a \emph{covering
 map}\index{covering map} if it is a locally trivial fiber bundle with discrete fiber
 $Z$.  Locally trivial means that for any $y \in Y$ there is a neighborhood $U$ such
 that if $f:U \times Z \rightarrow Z$ is the map defined by $f(u,z)=u$, then the
 following diagram commutes:
 \[\xymatrix{
 p^{-1}U \ar[d]  \ar@{}[r]|{\simeq} \ & U\times Z \ar[dl]^(.5)f\\
 **[l] Y \supseteq U }\]

 The exact sequence defined below is an important tool.  Suppose we have a locally
 trivial fiber bundle with fiber $Z$ (not necessarily discrete), with $X,Y$ connected.
 Choose $x_0\in X, z_0\in Z, y_0\in Y$ such that $p(x_0)=y_0$, and $i:Z \rightarrow
 p^{-1}(y_0)$ is an isomorphism such that $i(z_0)=x_0$:
 \[\xymatrix{
   *!<1em,0em>{ x_0 \in \pi^{-1}(y_0)} \ar[d] & Z\ar[l]_(.3)i\\
   y_0 }\]
   We can define $p_*:\pi_1(X,x_0)\to \pi_1(Y,y_0)$ in the obvious way ($\pi_1$ is a functor).
   Also define $i_*:\pi_1(Z,z_0)\to \pi_1(X,x_0)$. Then we can
 define $\partial:\pi_1(Y,y_0)\to \pi_0(Z)=\{$connected components of $Z\}$ by taking
 a loop $\gamma$ based at $y_0$ and lifting it to some path $\tilde{\gamma}$.  This path is not unique, but up to fiber-preserving homotopy it is. The new path $\tilde{\gamma}$ starts at
 $x_0$ and ends at $x_0'$.  Then we define $\partial$ to be the map associating the connected component of $x_0'$ to the homotopy class of $\gamma$.

 \begin{claim} The following sequence is exact:
 \[\xymatrix{
  \pi_1(Z,z_0)\ar[r]^{i_*} & \pi_1(X,x_0)\ar[r]^{p_*} & \pi_1(Y,y_0)\ar[r]^\partial & \pi_0(Z)\ar[r] &
  \{0\}
 }\]
   \begin{enumerate}
   \item $\im i_* = \ker p_*$
   \item $\{$fibers of $\partial \} \simeq \pi_1(Y,y_0)/\im p_*$
   \item $\partial $ is surjective.
   \end{enumerate}
 \end{claim}
 \begin{proof}
   \begin{enumerate}\item[]
   \item $\ker p_*$ is the set of all loops which map to contractible loops in
   $Y$, which are loops which are homotopic to a loop in $\pi^{-1}(y_0)$ based at
   $x_0$.  These are exactly the loops of $\im i_*$.
   \item The fiber of $\partial$ over the connected component $Z_z\subseteq Z$ is the
   set of all (homotopy classes of) loops in $Y$ based at $y_0$ which lift to a path connecting $x_0$ to a
   point in the connected component of $\pi^{-1}(y_0)$ containing $i(Z_z)$.
   If two loops $\beta , \, \gamma$ based at $y_0$ are in the same fiber, homotope them so
   that they have the same endpoint.  Then
   $\tilde{\gamma} \tilde \beta^{-1}$ is a loop based at $x_0$.  So fibers of $\partial$ are
   in one to one correspondence with loops in $Y$ based at $y_0$, modulo images of
   loops in $X$ based at $x_0$, which is just $\pi_1(Y, y_0)/ \im p_*$.
   \item This is obvious, since X is connected.
   \end{enumerate}
 \end{proof}

 Now assume we have a covering space with discrete fiber, i.e.  maps
 \[\xymatrix{
  X \ar[d]^{p} & Z \ar[l] \\
  Y
 }\]
 such that $\pi_1(Z, z_0) = \{ e \}$ and $\pi_0(Z)=Z$.
 Then we get the sequence
 \[\xymatrix{
  \{e\} \ar[r]^{i_*} & \pi_1(X,x_0)\ar[r]^{p_*} & \pi_1(Y,y_0)\ar[r]^\partial & Z \ar[r] &
  \{0\}
 }\]
and since $p_*$ is injective, $Z=\pi_1(Y)/\pi_1(X)$.

 Classifying all covering spaces of $Y$ is therefore the same as describing all
 subgroups of $\pi_1(Y)$. The \emph{universal cover}\index{universal cover} of $Y$ is
 the space $\tilde Y$ such that $\pi_1(\tilde Y)=\{e\}$, and for any other covering
 $X$, we get a factorization of covering maps $\tilde Y \xrightarrow{f}
 X\xrightarrow{p} Y$.

 We construct $\tilde X$, the universal cover, in the following way: fix
 $x_0\in X$, and define $\tilde X_{x_0}$ to be the set of basepoint-fixing homotopy
 classes of paths connecting $x_0$ to some $x \in X$.  We have a natural projection
 $[\gamma_{x_0,x}]\mapsto x$, and the fiber of this projection (over $x_0$) can be
 identified with $\pi_1(X,x_0)$. It is clear for any two basepoints $x_0$ and $x_0'$,
 $\tilde X_{x_0} \simeq \tilde X_{x_0'}$ via any path $\gamma_{x_0, x_0'}$ . So we
 have
 \[\xymatrix{
  \tilde X_{x_0} \ar[d]^{p} & \pi_1(X) \ar[l] \\
  X
 }\]
 \begin{claim}
   $\tilde X_{x_0}$ is simply connected.
 \end{claim}
 \begin{proof}
   We need to prove that $\pi_1(\tilde X_{x_0})$ is trivial, but we know that the
   fibers of $p$ can be identified with both $\pi_1(X)$ and $\pi_1(X)/\pi_1(\tilde
   X_{x_0})$, so we're done.
 %\renewcommand{\qedsymbol}{$\square_\mathrm{Claim}$}
 \end{proof}

 Let $G$ be a connected Lie group.  We would like to produce a simply connected Lie
 group which also has the Lie algebra $\lie(G)$.  It turns out that the obvious
 candidate, $\tilde G_e$, is just what we are looking for. It is not hard to see that
 $\tilde G_e$ is a smooth manifold (typist's note:  it is not that easy either.  See
 \cite{Hatcher}, pp. 64-65, for a description of the topology on $\tilde G_e$.  Once we have
 a topology and a covering space map, the smooth manifold structure of $G$ lifts to
 $\tilde G_e$. -- Emily).  We show it is a group as follows.

 Write $\gamma_g$ for $\gamma:[0,1]\to
 G$ with endpoints $e$ and $g$.  Define multiplication by
 $[\gamma_g][\gamma'_h]:=[\{\gamma_g(t)\gamma'_h(t)\}_{t\in [0,1]}]$.
 The unit element is the homotopy class of a contractible loop, and the inverse is
 given by $[\{\gamma(t)^{-1}\}_{t\in [0,1]}]$.

 \begin{claim}
   \begin{enumerate}\item[]
   \item $\tilde G = \tilde G_e$ is a group.
   \item $p:\tilde G\to G$ is a group homomorphism.
   \item $\pi_1(G)\subseteq \tilde G$ is a normal subgroup.
   \item $\lie(\tilde G)=\lie(G)$.
   \item $\tilde G\to G$ is the universal cover (i.e.\ $\pi_1(G)$ is discrete).
   \end{enumerate}
 \end{claim}
\begin{proof}
 \begin{enumerate}
 \item Associativity is inherited from associativity in $G$, composition with the identity does not change the homotopy class of a path, and the product of an element and its inverse is the identity.
 \item This is clear, since $p([\gamma_g][\tilde\gamma_h])=gh$.
 \item We know $\pi_1(G)=\ker p$, and kernels of homomorphisms are normal.
 \item \label{item4} The topology on $\tilde G$ is induced by the topology
 of $G$ in the following way:  If $\mathcal{U}$ is a basis for the topology on $G$ then
 fix a path $\gamma_{e,g}$ for all $g\in G$.  Then $\tilde{\mathcal{U}} = \{ \tilde U_{\gamma_{e,g}} \} $
 is a basis for the topology on $\tilde G$ with $\tilde U_{\gamma_{e,g}} $ defined to be the set of
 paths of the form $\gamma_{e,g}^{-1} \beta \gamma_{e,g}$ with $\beta$ a loop based at $g$
 contained entirely in $U$.

 Now take $U$ a connected, simply connected neighborhood of $e \in G$.  Since all
 paths in $U$ from $e$ to a fixed $g\in G$ are homotopic, we have that $U$ and $\tilde
 U$ are diffeomorphic and isomorphic, hence Lie isomorphic.  Thus $Lie(\tilde
 G)=Lie(G)$.

 \item As established in (\ref{item4}), $G$ and $\tilde G$ are diffeomorphic
 in a neighborhood of the identity.  Thus all points $x \in p^{-1}(e)$ have a
 neighborhood which does not contain any other inverse images of $e$, so $p^{-1}(e)$
 is discrete; and $p^{-1}(e)$ and $\pi_1(G)$ are isomorphic.
 \end{enumerate}
 \end{proof}

 We have that for any Lie group $G$ with a given Lie algebra $\lie(G)=\g$,
 there exists a simply connected Lie group $\tilde G$ with the same Lie algebra, and
 $\tilde G$ is the universal cover of $G$.

 \begin{lemma}\label{lec05L:discCentral}
   A discrete normal subgroup $H\subseteq G$ of a connected topological group $G$ is
   always central.
 \end{lemma}
 \begin{proof}
   For any fixed $h\in H$, consider the map $\phi_h:G\to H, g\mapsto ghg^{-1}h^{-1}$,
   which is continuous. Since $G$ is connected, the image is also connected, but $H$
   is discrete, so the image must be a point. In fact, it must be $e$ because
   $\phi_h(h)=e$. So $H$ is central.
 \end{proof}

 \begin{corollary}
   $\pi_1(G)$ is central, because it is normal and discrete. In particular, $\pi_1(G)$
   is commutative.
 \end{corollary}

 \begin{corollary}
   $G\simeq \tilde G/\pi_1(G)$, with $\pi_1(G)$ discrete central.
 \end{corollary}
 The following corollary describes all (connected) Lie groups with a given Lie algebra.

 \begin{corollary}
  Given a Lie algebra $\g$, take $\tilde
 G$ with Lie algebra $\g$. Then any other connected $G$ with Lie algebra $\g$ is a quotient of
 $\tilde G$ by a discrete central subgroup of $\tilde G$.
\end{corollary}

 Suppose $G$ is a topological group and $G^0$ is a connected component of $e$.
 \begin{claim}
   $G^0\subseteq G$ is a normal subgroup, and $G/G^0$ is a discrete group.
 \end{claim}
 If we look at $\{$Lie groups$\}\to \{$Lie algebras$\}$, we have an ``inverse'' given
 by exponential: $\exp(\g)\subseteq G$. Then $G^0=\bigcup_n (\exp \g)^n$. So for a
 given Lie algebra, we can construct a well-defined isomorphism class of connected,
 simply connected Lie groups. When we say ``take a Lie group with this Lie algebra'',
 we mean to take the connected, simply connected one.

\textbf{Coming Attractions:}
 We will talk about $U\g$, the universal enveloping algebra, $C(G)$, the Hopf algebra,
 and then we'll do classification of Lie algebras.
}{   % Emily Peters, eep@math
  \stepcounter{lecture}
 \setcounter{lecture}{6}
 \sektion{Lecture 6 - Hopf Algebras} \index{Hopf Algebras|idxbf}

 Last time: We showed that a finite dimensional Lie algebra $\g$ uniquely determines a
 connected simply connected Lie group. We also have a ``map'' in the other direction
 (taking tangent spaces). So we have a nice correspondence between Lie algebras and
 connected simply connected Lie groups.

 There is another nice kind of structure: Associative algebras. How do these relate to
 Lie algebras and groups?

 Let $\Gamma$ be a finite group and let $\CC[\Gamma]:=\{\sum_{g} c_gg|g\in \Gamma,
 c_g\in\CC\}$ be the $\CC$ vector space with basis $\Gamma$. We can make $\CC[\Gamma]$
 into an associative algebra by taking multiplication to be the multiplication in
 $\Gamma$ for basis elements and linearly extending this to the rest of
 $\CC[\Gamma]$.\footnote{``If somebody speaks Danish\index{Danish}, I would be happy
 to take lessons.''}

 \begin{remark}Recall that the tensor product V and W is the linear span of elements
 of the form $v\otimes w$, modulo some linearity relations. If $V$ and $W$ are
 infinite dimensional, we will look at the \emph{algebraic tensor product} of $V$ and
 $W$, i.e. we only allow finite sums of the form $\sum a_i\otimes b_i$.
 \end{remark}

 We have the following maps
 \begin{itemize}
 \item[] Comultiplication\index{comultiplication}: $\Delta:\CC[\Gamma]\to
  \CC[\Gamma]\otimes \CC[\Gamma],$ given by $\Delta(\sum x_g
  g)=\sum x_gg\otimes g$
 \item[] Counit\index{counit}:  $\varepsilon:\CC[\Gamma]\to \CC$, given by $\varepsilon(\sum x_g
  g)=\sum x_g$.
 \item[] Antipode\index{antipode}: $S:\CC[\Gamma]\to \CC[\Gamma]$ given by $S(\sum x_g
  g)=\sum x_gg^{-1}$.
 \end{itemize}

 You can check that
 \begin{itemize}
 \item $\Delta(xy)=\Delta(x)\Delta(y)$ (i.e.\ $\Delta$ is an algebra homomorphism),

 \item $(\Delta\otimes\id)\circ \Delta = (\id\otimes\Delta)\circ \Delta$. (follows
 from the associativity of $\otimes$),

 \item $\varepsilon(xy)=\varepsilon(x)\varepsilon(y)$ (i.e.\ $\varepsilon$ is an
 algebra homomorphism),

 \item $S(xy)=S(y)S(x)$ (i.e. $S$ is an algebra antihomomorphism).
 \end{itemize}

 Consider
 \[
    \CC[\Gamma]\xrightarrow{\Delta} \CC[\Gamma]\otimes \CC[\Gamma]
    \xrightarrow{S\otimes\id,\id\otimes S} \CC[\Gamma]\otimes
    \CC[\Gamma]\xrightarrow{m} \CC[\Gamma].
 \]
 You get
 \[
   m(S\otimes \id)\Delta(g) = m(g^{-1}\otimes g) = e
 \]
 so the composition sends $\sum x_g g$ to $(\sum_g x_g) e = \varepsilon(x)1_A$.

 So we have
 \begin{enumerate}
 \item $A=\CC[\Gamma]$ an associative algebra with $1_A$
 \item $\Delta:A\to A\otimes A$ which is coassociative and is a homomorphism of
 algebras
 \item $\varepsilon: A\to \CC$ an algebra homomorphism, with $(\varepsilon\otimes
 \id)\Delta = (\id\otimes \varepsilon)\Delta = \id$.
 \end{enumerate}

 \begin{definition}Such an $A$ is called a \emph{bialgebra}\index{bialgebra|idxbf}, with
  comultiplication $\Delta$ and counit $\varepsilon$.
 \end{definition}

 We also have $S$, the antipode, which is an algebra anti-automorphism, so it is a
 linear isomorphism with $S(ab)=S(b)S(a)$, such that
 \[\xymatrix{
  A\otimes A \ar[rr]^{S\otimes\id}_{\id\otimes S} & & A\otimes A \ar[d]^m \\
  A\ar[u]^{\Delta}\ar[r]^{\varepsilon} & \CC\ar[r]^{1_A} & A
 }\]
 \begin{definition}
   A bialgebra with an antipode is a \emph{Hopf algebra}.
 \end{definition}

  If $A$ is finite dimensional, let $A^*$ be the dual vector space. Define the
  multiplication, $\Delta_*$, $S_*$, $\varepsilon_*, 1_{A^*}$ on $A^*$ in the following
  way:
 \begin{itemize}
 \item $lm(a):=(l\otimes m)(\Delta a)$ for all $l,m\in A^*$
 \item $\Delta_*(l)(a\otimes b):= l(ab)$
 \item $S_*(l)(a) := l(S(a))$
 \item $\varepsilon_*(l) := l(1_A)$
 \item $1_{A^*}(a) := \varepsilon(a)$
 \end{itemize}

 \begin{theorem}
   $A^*$ is a Hopf algebra with this structure, and we say it is \emph{dual to} $A$.
   If $A$ is finite dimensional, then $A^{**}=A$.
 \end{theorem}
 \begin{exercise}
   Prove it.
 \end{exercise}

 We have an example of a Hopf algebra ($\CC[\Gamma]$), what is the dual Hopf
 algebra?\footnote{ If you want to read more, look at S.~Montgomery's \textsl{Hopf
 algebras}, AMS, early 1990s. \cite{Montgomery}} Let's compute $A^*=\CC[\Gamma]^*$.

 Well, $\CC[\Gamma]$ has a basis
 $\{g\in \Gamma\}$. Let $\{\delta_g\}$ be the dual basis, so $\delta_g(h)=0$ if $g\neq
 h$ and 1 if $g=h$. Let's look at how we multiply such things
 \begin{itemize}
 \item $\delta_{g_1}\delta_{g_2}(h) = (\delta_{g_1}\otimes \delta_{g_2})(h\otimes
 h) = \delta_{g_1}(h)\delta_{g_2}(h)$.
 \item $\Delta_*(\delta_g)(h_1\otimes h_2) = \delta_g(h_1h_2)$
 \item $S_*(\delta_g)(h) = \delta_g(h^{-1})$
 \item $\varepsilon_*(\delta_g) = \delta_g(e) = \delta_{g,e}$
 \item $1_{A^*}(h) = 1$.
 \end{itemize}

 It is natural to think of $A^*$ as the set of functions $\Gamma\to \CC$, where
 $(\sum x_g\delta_g)(h) = \sum x_g\delta_g(h)$. Then we can think about functions
 \begin{itemize}
 \item $(f_1 f_2)(h) = f_1(h)f_2(h)$
 \item $\Delta_*(f)(h_1\times h_2) = f(h_1h_2)$
 \item $S_*(f)(h) = f(h^{-1})$
 \item $\varepsilon_*(f) = f(e)$
 \item $1_{A^*} = 1$ constant.
 \end{itemize}
 So this is the Hopf algebra $C(\Gamma)$, the space of functions on $\Gamma$. If
 $\Gamma$ is any affine algebraic group, then $C(\Gamma)$ is the space of polynomial
 functions on $\Gamma$, and all this works. The only concern is that we need
 $C(\Gamma\times \Gamma)\cong C(\Gamma)\otimes C(\Gamma)$, which we only have in the
 finite dimensional case; you have to take completions of tensor products
 otherwise.

 So we have the notion of a bialgebra (and duals), and the notion of a Hopf algebra
 (and duals). We have two examples: $A=\CC[\Gamma]$ and $A^*=C(\Gamma)$. A natural
 question is, ``what if $\Gamma$ is an infinite group or a Lie group?'' and ``what are
 some other examples of Hopf algebras?''

 Let's look at some infinite dimensional examples. If $A$ is an infinite dimensional
 Hopf algebra, and $A\otimes A$ is the algebraic tensor product (finite linear
 combinations of formal $a\otimes b\ $ s). Then the comultiplication should be $\Delta:
 A\to A\otimes A$. You can consider cases where you have to take some completion of
 the tensor product with respect to some topology, but we won't deal with this kind of
 stuff. In this case, $A^*$ is too big, so instead of the notion of the dual Hopf
 algebra, we have dual pairs.
 \begin{definition}
   A \emph{dual pairing}\index{dual pairing} of Hopf algebras $A$ and $H$ is a pair with a
   bilinear map $\langle\ ,\,\rangle:A\otimes H \to \CC$ which is nondegenerate such
   that
   \begin{itemize}
   \item[(1)] $\langle \Delta a, l\otimes m\rangle =\langle a,lm\rangle$
   \item[(2)] $\langle ab,l \rangle = \langle a\otimes b, \Delta_* l \rangle$
   \item[(3)] $\langle S a, l \rangle = \langle a,S_* l \rangle$
   \item[(4)] $\varepsilon(a) = \langle a,1_{H} \rangle, \varepsilon_{H}(l) = \langle 1_A,l
   \rangle$
   \end{itemize}
 \end{definition}

 Exmaple: $A=\CC[x]$, then what is $A^*$? You can evaluate a polynomial at 0, or you
 can differentiate some number of times before you evaluate at 0. $A^* = $ span of
 linear functionals on polynomial functions of $\CC$ of the form
 \[
    l_n(f) = \left( \der{}{x}\right)^n f(x){\big |}_{x=0}.
 \]
 A basis for $\CC[x]$ is $1,x^n$ with $n\ge 1$, and we have
 \[
    l_n(x^m) = \left\{
    \begin{array}{cc}
      m! & ,n=m\\
      0 & ,n\neq m
    \end{array}\right.
 \]
 What is the Hopf algebra structure on $A$? We already have an algebra with identity.
 Define $\Delta(x) = x\otimes 1+1\otimes x$ and extend it to an algebra homomorphism,
 then it is clearly coassociative. Define $\varepsilon(1)=1$ and $\varepsilon(x^n)=0$
 for all $n\ge 1$. Define $S(x)=-x$, and extend to an algebra homomorphism. It is easy
 to check that this is a Hopf algebra.

 Let's compute the Hopf algebra structure on $A^*$. We have
 \begin{align*}
   l_nl_m(x^N) &= (l_n\otimes l_m)(\Delta(x^N)) \\
            &= (l_n\otimes l_m) (\sum \binom{N}{k} x^{N-k}\otimes x^k)
 \end{align*}
 \begin{exercise}
   Compute this out. The answer is that $A^*=\CC[y=\der{}{x}]$, and the Hopf algebra
   structure is the same as $A$.
 \end{exercise}
 This is an example of a dual pair: $A=\CC[x], H=\CC[y]$, with $\langle x^n,y^m\rangle
 = \delta_{n,m} m!$.

 Summary: If $A$ is finite dimensional, you get a dual, but in the infinite
 dimensional case, you have to use dual pairs.

 \subsektion{The universal enveloping algebra} \index{universal enveloping algebra|(}

 The idea is to construct a map from Lie algebras to associative algebras so that the
 representation theory of the associative algebra is equivalent to the representation
 theory of the Lie algebra.

  1) let $V$ be a vector space, then we can form the free associative algebra (or
 tensor algebra) of $V$: $T(V) = \CC\oplus (\oplus_{n\ge 1} V^{\otimes n})$. The
 multiplication is given by concatenation: $(v_1\otimes\cdots\otimes v_n)\cdot
 (w_1\otimes\cdots\otimes w_m) = v_1\otimes\cdots\otimes v_n\otimes w_1\otimes\cdots
 w_m$. It is graded: $T_n(V)T_m(V)\subseteq T_{n+m}(V)$. It is also a Hopf algebra,
 with $\Delta(x)=x\otimes 1+1\otimes x$, $S(x)=-x$, $\varepsilon(1)=1$ and
 $\varepsilon(x)=0$. If you choose a basis $e_1,\dots, e_n$ of $V$, then $T(V)$ is the
 free associative algebra $\langle e_1,\dots, e_n\rangle$. This algebra is
 $\ZZ_+$-graded: $T(V) = \oplus_{n\ge 0} T_n(V)$, where the degree of 1 is zero and the
 degree of each $e_i$ is 1. It is also a $\ZZ$-graded bialgebra:
 $\Delta(T_n(V))\subseteq \oplus (T_i\oplus T_{n-i}), S(T_n(V))\subset T_n(V),
 \varepsilon : T(V)\rightarrow \CC$ is a mapping of graded spaces $((\CC)_n=\{0\})$.
 \begin{definition}
   Let $A$ be a Hopf algebra. Then a two-sided ideal $I\subseteq A$ is a \emph{Hopf
   ideal}\index{Hopf ideal} if $\Delta(I)\subseteq A\otimes I + I\otimes A$, $S(I)=I$,
   and $\varepsilon(I)=0$.
 \end{definition}
 You can check that the quotient of a Hopf algebra by a Hopf ideal is a Hopf algebra
 (and that the kernel of a map of Hopf algebras is always a Hopf ideal).

 \begin{exercise}
   Show that $I_0 = \langle v\otimes w-w\otimes v| v,w\in V=T_1(V)\subseteq T(V)\rangle$
   is a homogeneous Hopf ideal.
 \end{exercise}
 \begin{corollary}
   $S(V) = T(V)/I_0$ is a graded Hopf algebra.
 \end{corollary}
  Choose a basis $e_1,\dots, e_n$ in $V$, so that $T(V)=\langle e_1,\dots, e_n\rangle$
  and $S(V) = \langle e_1,\dots, e_n\rangle/\langle e_ie_j-e_je_i\rangle$
  \begin{exercise}
    Prove that the Hopf algebra $S(V)$ is isomorphic to $\CC[e_1]\otimes\cdots\otimes
    \CC[e_n]$.
  \end{exercise}
  \begin{remark}
    From the discussion of $\CC[x]$, we know that $S(V)$ and $S(V^*)$ are dual.
  \end{remark}
  \begin{exercise}
    Describe the Hopf algebra structure on $T(V^*)$ that is determined by the
    pairing $\langle v_1\otimes \cdots\otimes v_n, l_1\otimes \cdots\otimes l_m\rangle
    = \delta_{m,n} l_1(v_1)\cdots l_n(v_n)$. (free coalgebra of $V^*$)
  \end{exercise}

  Now assume that $\g$ is a Lie algebra.
  \begin{definition}
    The universal enveloping algebra of $\g$ is $U(\g) = T(\g)/\langle x\otimes
    y-y\otimes x - [x,y]\rangle$.
  \end{definition}
    Exercise: prove that $\langle x\otimes y-y\otimes x - [x,y]\rangle$ is a Hopf
    ideal.
  \begin{corollary}
    $U\g$ is a Hopf algebra.
  \end{corollary}

  If $e_1,\dots, e_n$ is a basis for $V$. $U\g = \langle e_1,\dots, e_n| e_ie_j-e_je_i
  = \sum_k c_{ij}^k e_k\rangle$, where $c_{ij}^k$ are the structure constants of $[\ ,\,]$.
  \begin{remark}
    The ideal $\langle e_ie_j-e_je_i\rangle$ is homogeneous, but $\langle x\otimes
 y-y\otimes x -
    [x,y]\rangle$ is not, so $U\g$ isn't graded, but it is \emph{filtered}.
  \end{remark}
}{   % Sevak Mkrtchyan, sevak@math
  \stepcounter{lecture}
 \setcounter{lecture}{7}
 \sektion{Lecture 7}

 Last time we talked about Hopf algebras. Our basic examples were $\CC[\Gamma]$ and
 $C(\Gamma)=\CC[\Gamma]^*$. Also, for a vector space $V$, $T(V)$ is a Hopf algebra.
 Then $S(V)=T(V)/\langle x\otimes y-y\otimes x|x,y\in V\rangle$. And we also have $U\g
 = T\g/\langle x\otimes y-y\otimes x-[x,y]|x,y\in \g\rangle$.

 Today we'll talk about the universal enveloping algebra. Later, we'll talk about
 deformations of associative algebras because that is where recent progress in
 representation theory has been.

 \subsektion{Universality of \texorpdfstring{$U\g$}{Ug}} We have that
 $\g\hookrightarrow T\g\to U\g$. And $\sigma :\g\hookrightarrow U\g$ canonical
 embedding (of vector spaces and Lie algebras). Let $A$ be an associative algebra with
 $\tau:\g\to L(A)=\{A|[a,b]=ab-ba\}$ a Lie algebra homomorphism such that
 $\tau([x,y])=\tau(x)\tau(y)-\tau(y)\tau(x)$.
 \begin{proposition}\label{lec07P:Ug}
   For any such $\tau$, there is a unique $\tau':U\g\to A$ homomorphism of associative
   algebras which extends $\tau$:
   \[\xymatrix{
    U\g \ar[r]^{\tau'} & A\\
    \g\ar[u]_\sigma \ar[ur]_\tau
   }\]
 \end{proposition}
 \begin{proof}
   Because $T(V)$ is generated (freely) by $1$ and $V$, $U\g$ is generated by $1$ and
   the elements of $\g$. Choose a basis $e_1,\dots,e_n$ of $\g$. Then we have that
   $\tau(e_i)\tau(e_j)-\tau(e_j)\tau(e_i)=\sum_k c^k_{ij} \tau(e_k)$. The   elements
   $e_{i_1}\cdots e_{i_k}$ (this is a product) span $U\g$ for indices $i_j$. From the
   commutativity of the diagram, $\tau'(e_i)=\tau(e_i)$. Since $\tau'$ is a
   homomorphism of associative algebras, we have that $\tau'(e_{i_1}\cdots
   e_{i_k})=\tau'(e_{i_1})\cdots \tau'(e_{i_k})$, so $\tau'$ is determined by $\tau$
   uniquely: $\tau'(e_{i_1}\cdots e_{i_k})=\tau(e_{i_1})\cdots \tau(e_{i_k})$. We have
   to check that the ideal we mod out by is in the kernel. But that ideal is in the
   kernel because $\tau$ is a mapping of Lie algebras.
 \end{proof}
 \begin{definition}\index{representations} A linear representation
 of $\g$ in $V$ is a pair $(V, \phi: \g\to \End(V))$, where $\phi$ is a Lie algebra
 homomorphism. If $A$ is an associative algebra, then  $(V, \phi:A\to \End(V))$ a
 linear representation of $A$ in $V$.
 \end{definition}
 \begin{corollary} There is a bijection between representations of $\g$ (as a Lie
    algebra) and representations of $U\g$ (as an associative algebra).
 \end{corollary}
 \begin{proof} $(\Rightarrow)$ By the universality, $A=End(V)$, $\tau=\phi$.
$(\Leftarrow) \g\subset L(U\g)$ is a Lie subalgebra.
 \end{proof}
 \begin{example}[Adjoint representation]\label{lec07eg:adjoint}
  \index{adjoint representation|idxbfit}
  $ad:\g\to \End \g$ given by $x:y\mapsto~[x,y]$. This is also a representation of
  $U\g$. Let $e_1,\dots, e_n$ be a basis in $\g$. Then we have that $ad_{e_i}(e_j) =
  [e_i,e_j] = \sum_k c^k_{ij} e_k$, so the matrix representing the adjoint action of
  the element $e_i$ is the matrix $(ad_{e_i})_{jk}=(c^k_{ij})$ of structural
  constants. You can check that $ad_{[e_i,e_j]} = (ad_{e_i})(ad_{e_j})
  -(ad_{e_j})(ad_{e_i})$ is same as the Jacobi identity for the $c^k_{ij}$. We get
  $ad:U\g\to \End(\g)$ by defining it on the monomials $e_{i_1}\cdots e_{i_k}$ as
  $ad_{e_{i_1}\cdots e_{i_k}}= (ad_{e_{i_1}})\cdots (ad_{e_{i_k}})$ (the product of
  matrices).
 \end{example}

 Let's look at some other properties of $U\g$.
\subsektion{Gradation in \texorpdfstring{$U\g$}{Ug}}
 Recall that $V$ is a \emph{$\ZZ_+$-graded vector space} if $V=\oplus_{n=0}^\infty
 V_n$. A linear map $f:V\to W$ between graded vector spaces is
 \emph{grading-preserving} if $f(V_n)\subseteq W_n$. If we have a tensor product
 $V\otimes W$ of graded vector spaces, it has a natural grading given by $(V\otimes
 W)_{n} = \oplus_{i=0}^n V_i\otimes W_{n-i}$. The ``geometric meaning'' of this is
 that there is a linear action of $\CC$ on $V$ such that $V_n=\{x|t(x)=t^n\cdot
 x\text{ for all } t\in \CC\}$. A graded morphism is a linear map respecting this
 action, and the tensor product has the diagonal action of $\CC$, given by $t(x\otimes
 y) = t(x)\otimes t(y)$.
\begin{example} If $V=\CC[x]$, $\frac{d}{dx}$ is not grading preserving, $x\frac{d}{dx}$ is.
\end{example}
We say that $(V,[\ ,\,])$ is a \emph{$\ZZ_+$-graded Lie algebra} if $[\ ,\,]:V\otimes
V\to V$ is grading-preserving.
\begin{example}
  Let $V$ be the space of polynomial vector fields on
  $\CC=Span(z^n\frac{d}{dz})_{n\geq 0}$. Then $V_n=\CC z^n\frac{d}{dz}$.
\end{example}
 An associative algebra $(V,m:V\otimes V\to V)$ is \emph{$\ZZ_+$-graded} if $m$ is grading-preserving.
\begin{example}
   \begin{itemize}
   \item[]

   \item[(1)] $V=\CC[x]$, where the action of $\CC$ is given by $x\mapsto tx$.

   \item[(2)] $V=\CC[x_1,\dots, x_n]$ where the degree of each variable is 1 ... this
   is the $n$-th tensor power of the previous example.

   \item[(3)] Lie algebra:
   $\mathrm{Vect}(\CC) = \{\sum_{n\ge 0} a_n x^{n+1}\der{}{x}\}$ with
   $\mathrm{Vect}_n(\CC) = \CC x^{n+1}\der{}{x}$, $deg(x)=1$.  You can embed
   $\mathrm{Vect}(\CC)$ into polynomial vector fields on $S^1$ (Virasoro algebra).

   \item[(4)] T(V) is a $\ZZ_+$-graded associative algebra, as is $S(V)$. However,
   $U\g$ is not because we have modded out by a non-homogeneous ideal. But the ideal
   is not so bad. $U\g$ is a \emph{$\ZZ_+$-filtered algebra}:
   \end{itemize}
 \end{example}

 \subsektion{Filtered spaces and algebras}\
 \begin{definition}
   $V$ is a \emph{filtered space}\index{filtered space} if it has an increasing
   filtration
   \[
    V_0\subset V_1\subset V_2 \subset \cdots \subset V
   \]
   such that $V=\bigcup V_i$, and $V_n=$ is a subspace of dimension less than or equal
   to $n$. $f:V\to W$ is a \emph{morphism of filtered vector spaces} if
   $f(V_n)\subseteq W_n$.
 \end{definition}
 We can define filtered Lie algebras and associative algebras as such that the
 bracket/multiplication are filtered maps.

 There is a functor from filtered vector spaces to graded associative algebras $Gr:V\to Gr(V)$, where $Gr(V)= V_0\oplus V_1/V_0\oplus V_2/V_1\cdots$. If $f:V\to W$ is filtration preserving, it induces a map $Gr(f):Gr(V)\to Gr(W)$ functorially such that this diagram commutes:
 \[\xymatrix{
 V\ar[r]^f \ar[d]_{Gr} & W\ar[d]^{Gr}\\
 Gr(V)\ar[r]^{Gr(f)} & Gr(W)
 }\]

 Let $A$ be an associative filtered algebra (i.e.\ $A_iA_j\subseteq A_{i+j}$) such
that for all $a\in A_i, b\in A_j$,  $ab-ba \in A_{i+j-1}$.
\begin{proposition} For such an $A$,
  \begin{itemize}
  \item[(1)] $Gr(A)$ has a natural structure of an associative, commutative algebra
  (that is, the multiplication in $A$ defines an associative, commutative
  multiplication in $Gr(A)$).
  \item[(2)] For $a\in A_{i+1}, b\in A_{j+1}$, the operation $\{aA_i,bA_j\}=
  aA_ibA_j-bA_jaA_i \mod A_{i+j}$ is a lie bracket on $Gr(A)$.
  \item[(3)] $\{x,yz\} = \{x,y\}z+y\{x,z\}$.
  \end{itemize}
\end{proposition}
  \begin{proof}
    Exercise$_1$. You need to show that the given bracket is well defined, and then do a little dance, keeping track of which graded component you are in.
  \end{proof}

  \begin{definition}
    A commutative associative algebra $B$ is called a \emph{Poisson algebra} if $B$ is also a Lie algebra with lie bracket $\{\ ,\,\}$ (called a Poisson bracket) such that $\{x,yz\} = \{x,y\}z+y\{x,z\}$ (the bracket is a derivation).
  \end{definition}
  \begin{example}
    Let $(M,\w)$ be a symplectic manifold (i.e.\ $\w$ is a closed non-degenerate
    2-form on $M$), then functions on $M$ form a Poisson algebra. We could have
    $M=\RR^{2n}$ with coordinates $p_1,\dots, p_n,q_1,\dots, q_n$, and $\w = \sum_i
    dp_i\wedge dq_i$. Then the multiplication and addition on $C^\infty(M)$ is the
    usual one, and we can define $\{f,g\} = \sum_{ij} p^{ij}
    \pder{f}{x^i}\pder{g}{x^j}$, where $\w = \sum \w_{ij}dx^i\wedge dx^j$ and
    $(p^{ij})$ is the inverse matrix to $(\w_{ij})$. You can check that this is a
    Poisson bracket.
  \end{example}

  Let's look at $U\g = \langle 1,e_i|e_ie_j-e_je_i = \sum_k c^k_{ij} e_k\rangle$. Then
  $U\g$ is filtered, with $(U\g)_n = Span\{e_{i_1}\cdots e_{i_k}|k\le n\}$. We have
  the obvious inclusion $(U\g)_n\subseteq (U\g)_{n+1}$ and $(U\g)_0=\CC\cdot 1$.
  \begin{proposition}
    \begin{itemize}
    \item[]
    \item[(1)] $U\g$ is a filtered algebra (i.e.\ $(U\g)_r(U\g)_s\subseteq
    (U\g)_{r+s}$)
    \item[(2)] $[(U\g)_r,(U\g)_s]\subseteq (U\g)_{r+s-1}$.
    \end{itemize}
  \end{proposition}
  \begin{proof}
    1) obvious. 2) Exercise$_2$ (almost obvious).
  \end{proof}

  Now we can consider $Gr(U\g) = \CC\cdot 1\oplus (\bigoplus_{r\ge 1}
  (U\g)_r/(U\g)_{r-1})$
  \begin{claim}
    $(U\g)_r/(U\g)_{r-1} \simeq S^r(\g) = $ symmetric elements of $(\CC[e_1,\dots, e_n])_r$.
  \end{claim}
  \begin{proof}
    Exercise$_3$.
  \end{proof}
  So $Gr(U\g)\simeq S(\g)$ as a commutative algebra.

 $S(\g) \cong $ Polynomial functions on $\g^* = \hom_\CC(\g,\CC)$.

 Consider $C^\infty(M)$. How can we construct a bracket $\{\ ,\,\}$ which satisfies
 Liebniz (i.e.\ $\{f,g_1g_2\}=\{f,g_1\}g_2+\{f,g_2\}g_1$). We expect that $\{f,g\}(x)=
 p^{ij}(x)\pder{f}{x^i}\pder{g}{x^j} = \langle p(x),df(x)\wedge dg(x)\rangle$. Such a
 $p$ is called a bivector field (it is a section of the bundle $TM\wedge TM\to M$). So
 a Poisson structure on $C^\infty(M)$ is the same as a bivector field $p$ on $M$
 satisfying the Jacobi identity. You can check that the Jacobi identity is some
 bilinear identity on $p^{ij}$ which follows from the Jacobi identity on $\{\ ,\,\}$.
 This is equivalent to the Schouten identity, which says that the Schouten bracket of
 some things vanishes [There should be a reference here]. This is more general than
 the symplectic case because $p^{ij}$ can be degenerate.

 Let $\g$ have the basis $e_1,\dots, e_n$ and corresponding coordinate functions
 $x^1,\dots, x^n$. On $\g^*$, we have that dual basis $e^1,\dots, e^n$ (you can
 identify these with the coordinates $x^1,\dots, x^n$), and coordinates $x_1,\dots,
 x_n$ (which you can identify with the $e_i$). The bracket on polynomial functions on
 $\g^*$ is given by
 \[
    \{p,q\} = \sum c_{ij}^k x_k \pder{p}{x_i}\pder{q}{x_j}.
 \]
 This is a Lie bracket and clearly acts by derivations.

 Next we will study the following. If you have polynomials $p,q$ on $\g^*$, you can
 try to construct an associative product $p\ast_t q = pq+tm_1(p,q)+\cdots$. We will
 discuss deformations of commutative algebras. The main example will be the universal
 enveloping algebra as a deformation of polynomial functions on $\g^*$.
}{   % Lilit Martirosyan, lilit@math
  \stepcounter{lecture}
 \setcounter{lecture}{8}
 \sektion{Lecture 8 - The PBW Theorem and Deformations}

 Last time, we introduced the universal enveloping algebra $U\g$ of a Lie algebra
 $\g$, with its universality property. We discussed graded and filtered spaces and
 algebras. We showed that under some condition on a filtered algebra $A$, the graded
 algebra $Gr(A)$ is a Poisson algebra. We also checked that $U\g$ satisfies this
 condition, and that $Gr(U\g)\simeq S(\g)$ as graded commutative algebras. The latter
 space can be understood as the space $Pol(\g^*)$ of polynomial functions on $\g^*$.
 It turns out that the Poisson bracket on $Gr(U\g)$, expressed in $Pol(\g^*)$, is
 given by
 \[
    \{f,g\}(x) = x([df_x,dg_x])
 \]
 for $f,g\in Pol(\g^*)$ and $x\in \g^*$. Note that $f$ is a function on $\g^*$ and $x$
 an element of $\g^*$, so $df_x$ is a linear form on $T_x\g^*=\g^*$, that is,
 $df_x\in\g$.

 Suppose that $V$ admits a filtration $V_0\subset V_1\subset V_2\subset\cdots $. Then,
 the associated graded space $Gr(V)=V_0\oplus \bigoplus_{n\ge 1} (V_n/V_{n+1})$ is also
 filtered. (Indeed, every graded space $W=\bigoplus_{n\ge 0}W_n$ admits the filtration
 $W_0\subset W_0\oplus W_1\subset W_0\oplus W_1\oplus W_2\subset\cdots$) A natural
 question is:
 When do we have $V\simeq Gr(V)$ as filtered spaces ?

 For the filtered space $U\g$, the answer is a consequence of the following theorem.
 \begin{theorem}[Poincar\'e-Birkhoff-Witt] \index{Poincar\'e-Birkhoff-Witt|see{PBW}}
 \index{PBW|idxbf}
   Let $e_1,\dots, e_n$ be any linear basis for $\g$. Let us also denote by
   $e_1,\dots, e_n$ the image of this basis in the universal enveloping algebra $U\g$.
   Then the monomials $e_1^{m_1}\cdots e_n^{m_n}$ form a basis for $U\g$.
 \end{theorem}
 \begin{corollary}
 There is an isomorphism of filtered spaces $U\g\simeq Gr(U\g)$.
 \end{corollary}
 \begin{proof}[Proof of the corollary]

In $S(\g)$, $e_1^{m_1}\cdots e_n^{m_n}$ also forms a basis, so we get an isomorphism
$U\g\simeq S(\g)$ of filtered vector spaces by simple identification of the bases.
Since $Gr(U\g)\simeq S(\g)$ as graded algebras, the corollary is proved.
 \end{proof}
 \begin{remark}
 The point is that these spaces are isomorphic as {\em filtered} vector spaces. Saying
 that two infinite dimensional vector spaces are isomorphic is totally useless.
 \end{remark}
 \begin{proof}[Proof of the theorem]
  By definition, the unordered monomials $e_{i_1}\cdots e_{i_k}$ for $k\le p$ span the
  subspace $T_0\oplus \cdots \oplus T_p$ of $T(\g)$, where $T_i=\g^{\otimes i}$. Hence,
  they also span the quotient $(U\g)_p := T_0\oplus \cdots \oplus T_p/\langle x\otimes
  y-y\otimes x -[x,y]\rangle$. The goal is now to show that the {\em ordered} monomials
  $e^{m_1}_1\cdots e^{m_n}_n$ for $m_1+\dots+m_n\le p$ still span $(U\g)_p$. Let's prove
  this by induction on $p\ge 0$. The case $p=0$ being trivial, consider $e_{i_1}\cdots
  e_{i_a}\cdots e_{i_k}$, with $k\le p$, and assume that $i_a$ has the smallest value
  among the indices $i_1,\dots, i_k$. We can move $e_{i_a}$ to the left as follows
  \begin{align*}
  e_{i_1}\cdots e_{i_a}\cdots e_{i_k}=e_{i_a}e_{i_1}\cdots \hat e_{i_a}\cdots e_{i_k}
  + \sum_{b=1}^{a-1} e_{i_1}\cdots e_{i_{b-1}} [e_{i_b},e_{i_a}]\cdots\hat
  e_{i_a}\cdots e_{i_k}.
  \end{align*}
  Using the commutation relations $[e_{i_b},e_{i_a}]=\sum_\ell c_{i_bi_a}^\ell
  e_\ell$, we see that the term to the right belongs to $(U\g)_{k-1}$. Iterating this
  procedure leads to an equation of the form
  \[
  e_{i_1}\cdots e_{i_a}\cdots e_{i_k}=e_1^{m_1}\cdots e_n^{m_n} + \hbox{terms in
    $(U\g)_{k-1}$,}
  \]
  with $m_1+\dots+m_n=k\le p$. We are done by induction. The proof of the theorem is
  completed by the following homework.[This should really be done here]
 \end{proof}
 \begin{exercise}
   Prove that these ordered monomials are linearly independant.
 \end{exercise}

 Let's ``generalize'' the situation. We have $U\g$ and $S(\g)$, both of which are
 quotients of $T(\g)$, with kernels $\langle x\otimes y-y\otimes x -[x,y]\rangle$ and
 $\langle x\otimes y-y\otimes x\rangle$. For any $\varepsilon \in \CC$, consider the
 associative algebra $S_\varepsilon(\g) = T(\g)/\langle x\otimes y-y\otimes x
 -\varepsilon[x,y]\rangle$. By construction, $S_0(\g)=S(g)$ and $S_1(\g)=U\g$. Recall
 that they are isomorphic as filtered vector spaces.
 \begin{remark}
   If $\varepsilon\neq 0$, the linear map $\phi_\varepsilon: S_\varepsilon(\g) \to
   U\g$ given by $\phi_\varepsilon(x)=\varepsilon x$ for all $x\in \g$ is an
   isomorphism of filtered algebras. So, we have nothing new here.
 \end{remark}
 We can think of $S_\varepsilon (\g)$ as a non-commutative deformation of the
 associative commutative algebra $S(\g)$. (Note that commutative deformations of the
 algebra of functions on a variety correspond to deformations of the variety.)

 \index{universal enveloping algebra|)}
 \subsektion{Deformations of associative algebras}\index{deformations!of associative
 algebras|idxbf}
 Let $(A,m:A\otimes A\to A)$ be an associative algebra, that is, the linear map $m$
 satisfies the quadratic equation
 \begin{equation}\label{lec08Eq:1}
 m(m(a,b),c)=m(a,m(b,c)). %\tag{$\dag$}
 \end{equation}
 Note that if $\varphi:A\to A$ is a linear automorphism, the multiplication
 $m_\varphi$ given by $m_\varphi(a,b)=\varphi^{-1}(m(\varphi(a),\varphi(b)))$ is also
 associative. We like to think of $m$ and $m_\varphi$ as equivalent associative
 algebra structures on $A$. The ``moduli space'' of associative algebras on the vector
 space $A$ is the set of solutions to equation \ref{lec08Eq:1} modulo this equivalence
 relation.

 One can come up with a notion of deformation for almost any kind of object. In these
 deformation theories, we are interested in some cohomology theories because they
 parameterize obstructions to deformations. The knowledge of the cohomology of a
 given Lie algebra $\g$, enables us say a lot about the deformations of $\g$. We'll
 come back to this question in the next lecture.

  Let us turn to our original example: the family of associative algebras
  $S_\varepsilon(\g)$. Recall that by the PBW theorem, we have an isomorphism of
  filtered vector spaces $S_\varepsilon(\g)\stackrel{\psi}{\to}S(\g)= Pol(\g^*)$, but
  this is not an isomorphisms of associative algebras. Therefore, the multiplication
  defined by $f\ast g:=\psi(\psi^{-1}(f)\cdot \psi^{-1}(g))$ is not the normal
  multiplication on $S(\g)$. We claim that the result is of the form
  \[
     f\ast g = fg + \sum_{n\ge 1}\varepsilon^n m_n(f,g) ,
  \]
  where $m_n$ is a bidifferential operator of order $n$, that is, it is of the form
  \[
  m_n(f,g)=\sum_{I,J}p_n^{I,J} \partial^If\partial^Jg,
  \]
  where $I$ and $J$ are multi-indices of length $n$, and $p_n^{I,J}\in Pol(\g^*)$. The
  idea of the proof is to check this for $f=\psi(e_1^{r_1}\cdots e_n^{r_n}) $ and
  $g=\psi(e_1^{l_1}\cdots e_n^{l_n})$ by writing
  \[
  e_1^{r_1}\cdots e_n^{r_n} \cdot e_1^{l_1}\cdots e_n^{l_n}=e_1^{l_1+r_1}\cdots
  e_n^{l_n+r_n} + \sum_{k\ge 1} \varepsilon^k m_k(e_1^{r_1}\cdots
  e_n^{r_n},e_1^{l_1}\cdots e_n^{l_n})
  \]
  in $S_\varepsilon(\g)$ using the commuting relations.

  \begin{exercise}
   Compute the $p_n^{I,J}$ for the Lie algebra $\g$ generated by $X$, $Y$, and $H$
   with bracket $[X,Y]=H, [H,X]=[H,Y]=0$. This is called the Heisenberg Lie
   algebra.\index{Heisenberg algebra}
   \begin{solution}
     We would like to compute the coefficients of the product $(X^aH^bY^r)(X^sH^cY^d)$
     once it is rewritten in the PBW basis by repeatedly applying the relations
     $XY-YX=\e H$, $HX=XH$, and $HY=YH$. Check by induction that
     \[
        Y^r X^s = \sum_{n=0}^\infty \e^n (-1)^n n! \binom{r}{n}\binom{s}{n}
        X^{r-n}H^lY^{s-n}.
     \]
     It follows that $p_n^{I,J}$ is zero unless $I=(Y,\dots, Y)$ and $J=(X,\dots,X)$,
     in which case $p_n^{I,J}=\frac{(-1)^n}{n!} H^n $.
   \end{solution}
  \end{exercise}

  So we have a family of products on $Pol(\g^*)$ which depend on $\varepsilon$ in the
  following way:
  \[
   f\ast g = fg + \sum_{n\ge 1}\varepsilon^n m_n(f,g)
  \]
  Since $f,g$ are polynomials and $m_n$ is a bidifferential operator of order $n$,
  this series terminates, so it is a polynomial in $\varepsilon$. If we try to extend
  this product to $C^\infty(\g^*)$, then there are questions about the convergence of
  the product $\ast$. There are two ways to deal with this problem. The first one is
  to take these matters of convergence seriously, consider some topology on
  $C^\infty(\g^*)$ and demand that the series converges. The other solution is to
  forget about convergence and just think in terms of formal power series in
  $\varepsilon$. This is the so-called ``formal deformation'' approach. As we shall
  see, there are interesting things to say with this seemingly rudimentary point
  of view.

  \subsektion{Formal deformations of associative algebras} Let $(A,m_0)$ be an
  associative algebra over $\CC$. Then, a formal deformation of $(A,m_0)$ is a
  $\CC[[h]]$-linear map $m:A[[h]]\otimes_{\CC[[h]]} A[[h]] \to A[[h]]$ such that
  \[
   m(a,b) = m_0(a,b) + \sum_{n\ge 1} h^n m_n(a,b)
  \]
  for all $a,b\in A$, and such that $(A[[h]],m)$ is an associative algebra. We say
  that two formal deformations $m$ and $\tilde m$ are equivalent if there is a
  $\CC[[h]]$-automorphism $A[[h]]\xrightarrow{\varphi} A[[h]]$ such that $\tilde m =
  m_\varphi$, with $\varphi(x) = x + \sum_{n\ge 1} h^n \varphi_n(x)$ for all $x\in
  A$, where $\varphi_n$ is an endomorphism of $A$.

  \smallskip
  \noindent \underline{Question}: Describe the equivalence classes of formal
  deformations of a given associative algebra.

   When $(A,m_0)$ is a commutative algebra, the answer is known. Philosophically and
   historically, this case is relevant to quantum mechanics. In classical mechanics,
   observables are smooth functions on a phase space $M$, i.e\ they form a commutative
   associative algebra $C^\infty(M)$. But when you quantize this system (which is
   needed to describe something on the order of the Planck scale), you cannot think of
   observables as functions on phase space anymore. You need to deform the commutative
   algebra $C^\infty(M)$ to a noncommutative algebra. And it works...

   From now on, let $(A,m_0)$ be a commutative associative algebra. Let's write
   $m_0(a,b)=ab$, and $m(a,b)=a\ast b$. (This is called a star product\index{star
   product}, and the terminology goes back to the sixties and the work of J.~Vey).
   Then we have
   \[
    a\ast b = ab + \sum_{n\ge 1} h^n m_n(a,b).
   \]
   Demanding the associativity of $\ast$ imposes an infinite number of equations for
   the $m_n$'s, one for each order:
   \begin{itemize}
   \item[$h^0$:] $a(bc)=(ab)c$
   \item[$h^1$:] $am_1(b,c) + m_1(a,bc) = m_1(a,b)c+m_1(ab,c)$
   \item[$h^2$:] $\dots$ \anton{we should compute this one}
   \item[$\vdots$]
   \end{itemize}

   \begin{exercise}
   Show that the bracket $\{a,b\} = m_1(a,b)-m_1(b,a)$ defines a Poisson structure on
   $A$. This means that we can think of a Poisson structure on an algebra as the
   remnants of a deformed product where $a\ast b-b\ast a = h\{a,b\} + O(h)$.
   \end{exercise}

   One easily checks that if two formal deformations $m$ and $\tilde m$ are equivalent
   via $\varphi$ (i.e: $\tilde m=m_\varphi$), then the associated $m_1,\tilde m_1$ are
   related by $m_1(a,b) = \tilde m_1(a,b)+\varphi_1(ab)-\varphi_1(a)b-a\varphi_1(b)$.
   In particular, two equivalent formal deformations induce the same Poisson
   structure. Also, it is possible to choose a representative in an equivalence class
   such that $m_1$ is skew-symmetric (and then, $m_1(a,b)=\frac{1}{2}\{a,b\}$). This
   leads to the following program for the classification problem:
   \begin{enumerate}
   \item Classify all Poisson structures on $A$.

   \item Given a Poisson algebra $(A,\{\ ,\,\})$, classify all equivalence
   classes of star products on $A$ such that $m_1(a,b)=\frac{1}{2}\{a,b\}$.
   \end{enumerate}

   Under some mild assumption, it can be assumed that a star product is symmetric,
   i.e.\ that it satisfies the equation $m_n(a,b)=(-1)^n m_n(b,a)$ for all $n$. The
   program given above was completed by Maxim Kontsevitch \index{Kontsevitch, Maxim}
   for the algebra of smooth functions on a manifold $M$. Recall that Poisson
   structures on $C^\infty(M)$ are given by bivector fields on $M$ that satisfy the
   Jacobi identity.

  \begin{theorem}[Kontsevich, 1994]
  Let $A$ be the commutative associative algebra $C^\infty(M)$, and let us fix a
  Poisson bracket $\{\ ,\,\}$ on $A$. Equivalence classes of symmetric star products
  on $A$ with $m_1(a,b)=\frac{1}{2}\{a,b\}$ are in bijection with formal deformations
  of $\{\ ,\,\}$ modulo formal diffeomorphisms of $M$.
  \end{theorem}
  A \emph{formal deformation} of $\{\ ,\,\}$ is a Poisson bracket $\{\ ,\,\}_h$ on $A[[h]]$
  such that
  \[
    \{a,b\}_h = \{a,b\} + \sum_{n\ge 1} h^n \mu_n(a,b)
  \]
  for all $a,b$ in $A$. A \emph{formal diffeomorphism} of $M$ is an automorphism
  $\varphi$ of $A[[h]]$ such that $\varphi(f) = f+\sum_{n\ge 1} h^n \varphi_n(f)$
  and $\varphi(fg)=\varphi(f)\varphi(g)$ for all $f,g$ in $A$.

  We won't prove the theorem (it would take about a month) \anton{find a reference}.
  As Poisson algebras are Lie algebras, it relates deformations of associative
  algebras to deformations of Lie algebras.

  \subsektion{Formal deformations of Lie algebras}
  Given a Lie algebra $(\g, [\ ,\,])$, you want to know how many formal deformations
  of $\g$ there are. Sometimes, there are none (like in the case of $\sl_n$, as we
  will see later). Sometimes, there are plenty (as for triangular matrices). The
  goal is now to construct some invariants of Lie algebras which will tell you whether
  there are deformations, and how many of them there are. In order to do this, we
  should consider cohomology theories for Lie algebras. We will focus first on the
  standard complex $C^\udot (\g,\g) = \bigoplus_{n\ge 0} C^n(\g,\g)$, where
  $C^n(\g,\g) = \hom(\Lambda^n \g,\g)$.
}{   % David Cimasoni, cimasoni@math
  \stepcounter{lecture}
 \setcounter{lecture}{9}
 \sektion{Lecture 9}

 \newcommand{\diagramsize}{1.25em}

 Let's summarize what has happened in the last couple of lectures.
 \begin{enumerate}
 \item We talked about $T(\g)$, and then constructed three algebras:
 \begin{itemize}
   \item $U\g =  T(\g)/\langle x\otimes y-y\otimes x-[x,y]\rangle$, with $U\g=S_1(\g)
     \simeq S_\varepsilon (\g)$ as filtered associative algebras, for all non-zero
     $\varepsilon \in \CC$.

   \item $S_\varepsilon (\g) = T(\g)/\langle x\otimes y-y\otimes
      x-\varepsilon[x,y]\rangle$ is a family of associative algebras, with $S_\varepsilon
      (\g) \simeq S_0(\g)$ as filtered vector spaces.

   \item $S_0(\g)\cong Pol(\g^*) = T(\g)/\langle x\otimes y-y\otimes x\rangle = S_0(\g)$
      is an associative, commutative algebra with a Poisson structure defined by the Lie
      bracket.
 \end{itemize}

 \item We have two ``pictures'' of deformations of an algebra
    \begin{enumerate}
    \item There is a simple ``big'' algebra $B$ (such as $B=T(\g)$) and a family of
    ideals $I_\varepsilon$. Then we get a family $B/I_\varepsilon=A_\varepsilon$. This
    becomes a deformation family of the associative algebra $A_0$ if we identify
    $A_\varepsilon \simeq A_0$ as vector spaces (these are called \emph{torsion free}
    deformations). Fixing this isomorphism gives a family of associative products on
    $A_0$.

    We can think of this geometrically as a family of (embedded) varieties.
    \item Alternatively, we can talk about deformations intrinsically (i.e., without referring to some bigger $B$).
    Suppose we have $A_0$ and a family of associative products $a\ast_\varepsilon  b$
    on $A_0$.
    \begin{example}
      Let $Pol(\g^*) \xrightarrow{\phi} S_\varepsilon (\g)$ be the isomorphism of the PBW theorem. Then
      define \( f\ast g = \phi^{-1} (\phi(f)\cdot \phi(g)) = fg + \sum_{n\ge 1}
      \varepsilon^n m_n(f,g)\).
    \end{example}
    \end{enumerate}
    Understanding deformations makes a connection between representation theory and
    Poisson geometry. A second course on Lie theory should discuss symplectic leaves
    of $Pol(\g^*)$, which happen to be coadjoint orbits and correspond to
    representations.  This is why deformations are relevant to representation theory.
 \end{enumerate}

 Let $A$ be a Poisson algebra with bracket $\{\ ,\,\}$, so it is a commutative
 algebra, and a Lie algebra, with the bracket acting by derivations. Typically,
 $A=C^\infty(M)$. Equivalence classes of formal (i.e., formal power series) symmetric
 (i.e.,$m_n(f,g)=(-1)^n m_n(g,f)$ ) star products on $C^\infty(M)$ are in bijection with
 equivalence classes of formal deformations of $\{\ ,\,\}$ on $C^\infty(M)[[h]]$.

 Apply this to the case $A=C^\infty(\g^*)$. The associative product on
 $S_\varepsilon(\g)$ comes from the product on $T(\g)$. The question is, ``how many
 equivalence classes of star products are there on $A$?'' Any formal deformation of
 the Poisson structure on $(A,\{\ ,\,\}_\g)$ is a PBW deformation\mpar[\anton{What is a PBW
 deformation? What's going one here?}]{} of some formal deformation of the Lie algebra
 $C^\infty(\g^*)$ (with Lie bracket $\{f,g\}(x)=x(df\wedge dg)$). Such a deformation
 is equivalent to a formal deformation of the Lie algebra structure on $\g$. This is
 one of the reasons that deformations of Lie algebras are important --- they describe
 deformations of certain associative algebras. When one asks such questions, some
 cohomology theory always shows up.

 \subsektion{Lie algebra cohomology}\index{Lie algebra
 cohomology|idxbf}\index{cohomology!of Lie algebras|idxbf} Recall that $(M,\phi)$ is a
 $\g$-module if $\phi:\g\to \End(M)$ is a Lie algebra homomorphism. We will write $xm$
 for $\phi(x)m$. Define $C^\udot (\g,M) = \bigoplus_{q\ge 0} C^q(\g,M)$ where
 $C^q(\g,M) = \hom(\Lambda^q \g, M)$ (linear maps). We define $d:C^q\to C^{q+1}$ by
 \begin{align*}
  \iflilbook
    &dc(x_1\wedge\cdots\wedge x_{q+1}) =\\
  \else
    dc(x_1&\wedge\cdots\wedge x_{q+1}) =\\
  \fi
     &= \sum_{1\le s< t\le q+1} (-1)^{s+t-1} c([x_s,x_t]\wedge x_1 \wedge\cdots \wedge
     \hat x_s \wedge \cdots \wedge \hat x_t \wedge \cdots \wedge x_{q+1}) \\
  &\qquad + \sum_{s=1}^{q+1} (-1)^s x_s c(x_1\wedge \cdots \wedge \hat x_s \wedge
  \cdots \wedge x_{q+1})
 \end{align*}
 \begin{exercise}
   Show that $d^2=0$.
 \end{exercise}

 \underline{Motivation}: If $\g=\mathrm{Vect}(\mathcal{M})$,
 $M=C^\infty(\mathcal{M})$, then $C^q(\g,M) = \W^q(\mathcal{M})$, with the Cartan
 formula
 \begin{align*}
  \iflilbook
    &(d\w)(\xi_1 \wedge\cdots \wedge \xi_{q+1}) = \\
  \else
    (d\w)(\xi_1 &\wedge\cdots \wedge \xi_{q+1}) = \\
  \fi
     &= \sum_{1\le s< t\le q+1} (-1)^{s+t-1} \w ([\xi_s,\xi_t]\wedge \xi_1 \wedge\cdots
     \wedge \hat \xi_s \wedge \cdots \wedge \hat \xi_t \wedge \cdots \wedge\xi_{q+1}) \\
  &\qquad + \sum_{s=1}^{q+1} (-1)^s \xi_s \w (\xi_1\wedge \cdots \wedge \hat \xi_s \wedge
  \cdots \wedge \xi_{q+1})
\end{align*}
 for vector fields $\xi_i$.

 Another motivation comes from the following proposition.
 \begin{proposition}
   $C^\udot (\g,\CC) \simeq \W^\udot_R(G) \subseteq \W^\udot(G)$ where $\CC$ is the
   1 dimensional trivial module over $\g$ (so $xm=0$).
 \end{proposition}
 \begin{exercise}
   Prove it.
 \end{exercise}
 \begin{remark}
   This was Cartan's original motivation for Lie algebra cohomology. It turns out that
   the inclusion $\W^\udot_R(G)\hookrightarrow \W^\udot(G)$ is a homotopy equivalence
   of complexes (i.e.\ the two complexes have the same homology), and the proposition
   above tells us that $C^\udot(\g,\CC)$ is homotopy equivalent to $\W_R(G)$. Thus, by
   computing the Lie algebra cohomology of $\g$ (the homology of the complex
   $C^\udot(\g,\CC)$), one obtains the De Rham cohomology of $G$ (the homology of the
   complex $\W^\udot(G)$).
 \end{remark}

 Define $H^q(\g,M) = \ker(d:C^q\to C^{q+1})/\im(d:C^{q-1}\to C^q)$ as always. Let's
 focus on the case $M=\g$, the adjoint representation\index{adjoint representation|idxit}:
 $x\cdot m = [x,m]$.
 \begin{itemize}
 \item[$H^0(\g,\g)$] We have that $C^0 = \hom(\CC,\g)\cong \g$, and
 \[
    dc(y) = y\cdot c = [y,c].
 \]
  so $\ker (d:C^0\to C^1)$ is the set of $c\in \g$ such that $[y,c]=0$ for all $y\in \g$.
  That is, the kernel is the center of $\g$, $Z(\g)$. So $H^0(\g,\g)=Z(\g)$.

 \mpar[\anton{these can be more general ... $H^i(\g,V)$. See \cite{HiltonStammbach}}]{}
 \item[$H^1(\g,\g)$]\label{lec09H1(g,g)} The kernel of $d:C^1(\g,\g)\to C^2(\g,\g)$ is
 \[\!\!
 \{\mu:\g\to \g| d\mu(x,y)=\mu([x,y])-[x,\mu(y)]-[\mu(x),y]=0 \text{ for all }x,y\in
 \g\},
 \]
 which is exactly the set of derivations of $\g$.
%
% : those $\mu$ such that
% \[
%  \mu([x,y]) = [\mu(x),y]+[x,\mu(y)]
% \]
%
 The image of $d:C^0(\g,\g)\to C^1(\g,\g)$ is the set of \emph{inner derivations},
 $\{dc:\g\to \g|dc(y)=[y,c]\}$.  The Liebniz rule is satisfied because of the Jacobi
 identity. So
 \[
    H^1(\g,\g) = \{\text{derivations}\}/\{\text{inner derivations}\} =: \text{\em
    outer derivations}.
 \]

 \item[$H^2(\g,\g)$] Let's compute $H^2(\g,\g)$. Suppose $\mu\in C^2$, so
 $\mu:\g\wedge\g\to \g$ is a linear map. What does $d\mu=0$ mean?
 \begin{align*}
    d\mu(x_1,x_2,x_3) &= \mu([x_1,x_2],x_3) - \mu([x_1,x_3],x_2) + \mu([x_2,x_3],x_1)
    \\ &\qquad - [x_1,\mu(x_2,x_3)]
    + [x_2,\mu(x_1,x_3)] - [x_3,\mu(x_1,x_2)]\\
     &= -\mu(x_1,[x_2,x_3]) - [x_1,\mu(x_2,x_3)] + \text{cyclic permutations}
 \end{align*}
 Where does this kind of thing show up naturally?

 Consider deformations of Lie algebras:
 \[
    [x,y]_h = [x,y] + \sum_{n\ge 1} h^n m_n(x,y)
 \]
 where the $m_n:\g\times \g\to \g$ are bilinear. The deformed bracket $[\ ,\,]_h$ must
 satisfy the Jacobi identity,
 \[
    [a,[b,c]_h]_h + [b,[c,a]_h]_h + [c,[a,b]_h]_h = 0
 \]
 which gives us relations on the $m_n$. In degree $h^N$, we get
 \begin{align}
    [a,m_N(b,c)] + m_N(a,[b,c]) + \sum_{k=1}^{N-1} m_k(a,m_{N-k}(b,c)) +& \nonumber \\
    [b,m_N(c,a)] + m_N(b,[c,a]) + \sum_{k=1}^{N-1} m_k(b,m_{N-k}(c,a)) +& \nonumber \\
    [c,m_N(a,b)] + m_N(c,[a,b]) + \sum_{k=1}^{N-1} m_k(c,m_{N-k}(a,b)) &=0
    \label{lec09Eq:h^N}
 \end{align}
 \begin{exercise} \label{lec09Ex2}
   Derive equation \ref{lec09Eq:h^N}.
   \begin{solution}
     We have $[a,b]_h := \sum_{n=0}^\infty h^n m_n(a,b)$, where $m_0(a,b)=[a,b]$.
     Now we compute
     \begin{align*}
       [a,[b,c]_h]_h &= [a,\sum_{l\ge 0} h^l m_l(b,c)]_h\\
                &= \sum_{l\ge 0} h^l \sum_{k\ge 0} h^k m_k(a,m_l(b,c))\\
                &= \sum_{N\ge 0} h^N m_k(a,m_{N-k}(b,c)) & (N=k+l)
     \end{align*}
     Adding the cyclic permutations and looking at the coefficient of $h^N$, we get
     the desired result.
   \end{solution}
 \end{exercise}
 Define $[m_K,m_{N-K}](a,b,c)$ as \mpar[\anton{Gerstenhaber}]{}
 \[
  m_K\bigl(a,m_{N-K}(b,c)\bigr) + m_K\bigl(b,m_{N-K}(c,a)\bigr) + m_K\bigl(c,m_{N-K}(a,b)\bigr).\]
  Then equation \ref{lec09Eq:h^N} can be written as
 \begin{equation}\label{lec09Eq:2}
    dm_N = \sum_{k=1}^{N-1} [m_k,m_{N-k}] %\tag{$\ddag$}
 \end{equation}

 \begin{theorem}
  Assume that for all $n \leq N-1$, we have the relation $dm_n = \sum_{k=1}^{n-1}
  [m_k,m_{n-k}]$.  Then $d(\sum_{k=1}^{N-1} [m_k,m_{N-k}])=0$.
 \end{theorem}
 \begin{exercise} \label{lec09Ex3}
    Prove it.
 \end{exercise}
 The theorem tells us that if we have a ``partial deformation'' (i.e.\ we have found
 $m_1,\dots, m_{N-1}$), then the expression $\sum_{k=1}^{N-1} [m_k,m_{N-k}]$ is a
 3-cocycle. Furthermore, equation \ref{lec09Eq:2} tells us that if we are to extend
 our deformation to one higher order, $\sum_{k=1}^{N-1} [m_k,m_{N-k}]$ must represent
 zero in $H^3(\g,\g)$.

  Taking $N=1$, we get $dm_1=0$, so   $\ker(d:C^2\to C^3) = $ space of first
  coefficients of formal deformations of $[\ ,\,]$. It will turn out that $H^2$ is the
  space of equivalence classes of $m_1$.\mpar[\anton{State clearly what $H^2$ is,
  please, or ref}]{}
 \end{itemize}

It is worth noting that the following ``pictorial calculus" may make some of the above
computations easier.  In the following pictures, arrows are considered to be oriented
downwards, and trivalent vertices with two lines coming in and one going out represent
the Lie bracket.  So, for example, the antisymmetry of the Lie bracket is expressed as

  \[\begin{xy}<\diagramsize,0em>:
   (0,.5);(.5,-.5) **\crv{(1,0)},
   (1,.5);(.5,-.5) **\crv{(0,0)},
   (.5,-.5);(.5,-1) **@{-},
 \end{xy}
 = -
 \begin{xy}<\diagramsize,0em>:
   (0,.5);(.5,-.5) **@{-},
   (1,.5);(.5,-.5) **@{-},
   (.5,-.5);(.5,-1) **@{-},
 \end{xy}
 \]
and the Jacobi identity is
 % \[\begin{xy}<\diagramsize,0em>:
 %  (0,1);(.5,0) **@{-} ?*@{>},
 %  (1,1);(.5,0) **@{-} ?*@{>},
 %  (.5,0);(1,-1) **@{-} ?*@{>},
 %  (2,1);(1,-1) **@{-} ?*@{>},
 %  (1,-1);(1,-1.5) **@{-} ?*@{>},
 %\end{xy}\]

 \begin{align*}
   {\begin{xy}<\diagramsize,0em>:
     (0,1);(.5,0) **@{-},
     (1,1);(.5,0) **@{-},
     (.5,0);(1,-1) **@{-},
     (2,1);(1,-1) **@{-},
     (1,-1);(1,-1.5) **@{-},
   \end{xy}} +
   {\begin{xy}<\diagramsize,0em>:
     (0,1);(1,-1) **\crv{(0,-1)&(2,0)},
     (1,1);(1,0) **@{-},
     (2,1);(1,0) **@{-},
     (1,0);(1,-1) **\crv{(0,-.5)},
     (1,-1);(1,-1.5) **@{-},
   \end{xy}} +
   {\begin{xy}<\diagramsize,0em>:
     (0,1);(.5,0) **@{-},
     (1,1);(1.3,-1) **@{-},
     (2,1);(.5,0) **@{-},
     (.5,0);(1.3,-1) **@{-},
     (1.3,-1);(1.3,-1.5) **@{-},
   \end{xy}} &= \\
   {\begin{xy}<\diagramsize,0em>:
     (0,1);(.5,0) **@{-},
     (1,1);(.5,0) **@{-},
     (.5,0);(1,-1) **@{-},
     (2,1);(1,-1) **@{-},
     (1,-1);(1,-1.5) **@{-},
   \end{xy}} -
   {\begin{xy}<\diagramsize,0em>:
     (0,1);(1,-1) **@{-},
     (1,1);(1.5,0) **@{-},
     (2,1);(1.5,0) **@{-},
     (1.5,0);(1,-1) **@{-},
     (1,-1);(1,-1.5) **@{-},
   \end{xy}} +
   {\begin{xy}<\diagramsize,0em>:
     (0,1);(.5,0) **@{-},
     (1,1);(1.3,-1) **@{-},
     (2,1);(.5,0) **@{-},
     (.5,0);(1.3,-1) **@{-},
     (1.3,-1);(1.3,-1.5) **@{-},
   \end{xy}} &= 0
 \end{align*}
We can also use pictures to represent cocycles.  Take $\mu \in H^n(\g, \g)$.  Then we
draw $\mu$ as
\[{\begin{xy}<\diagramsize,0em>:
   (1,0) *+{\mu} *\cir{};
   (0,1) **@{-}, (2,1) **@{-}, (1,-1) **@{-},
   (1,.8) *{\dots}
 \end{xy}}\]
 with $n$ lines going in.  Then, the Cartan formula\index{Cartan!formula} for the differential says that

 \[ d \left(
 {\begin{xy}<\diagramsize,0em>:
   (1,0) *+{\mu} *\cir{};
   (0,1) **@{-}, (2,1) **@{-}, (1,-1) **@{-},
   (1,.8) *{\dots}
 \end{xy}} \right)
 =
 \sum_{1\le i\le j\le n+1} (-1)^{i+j+1}
 {\begin{xy}<\diagramsize,0em>:
   (.7,1) *+!D{i};(0,0) **@{-},
   (1.6,1) *+!D{j};(0,0) **@{-};
   (1,-1) *+{\mu} *\cir{} **@{-};
   (0,1) **@{-}, (2,1) **@{-}, (1,-2) **@{-},
 \end{xy}}
 + \sum_{1\le i\le n+1}
 {\begin{xy}<\diagramsize,0em>:
   (1,0) *+{\mu} *\cir{};
   (0,1) **@{-}, (2,1) **@{-}, (.5,-1) **@{-},
   (1,1) *+!D{i}; (.5,-1) **\crv{(0,0)};
   (.5,-1.5) **@{-}
 \end{xy}}
 \]
and the bracket of two cocycles $\mu \in H^m$ and $\nu \in H^n$ is
 \[ [\mu, \nu] = \sum_{1\le i\le n}
 {\begin{xy}<\diagramsize,0em>:
   (1,.5) *+{\mu} *\cir{};
   (.3,1.5) *+!D{i} **@{-}, (1.7,1.5) **@{-},
   (1,-1) *+{\nu} *\cir{} **@{-};
   (-1,1.5) **@{-}, (-.3,1.5) **@{-}, (2.3,1.5) **@{-}, (3,1.5) **@{-}, (1,-2) **@{-},
 \end{xy}}
 - \sum_{1\le i\le m}
 {\begin{xy}<\diagramsize,0em>:
   (1,.5) *+{\nu} *\cir{};
   (.3,1.5) *+!D{i} **@{-}, (1.7,1.5) **@{-},
   (1,-1) *+{\mu} *\cir{} **@{-};
   (-1,1.5) **@{-}, (-.3,1.5) **@{-}, (2.3,1.5) **@{-}, (3,1.5) **@{-}, (1,-2) **@{-},
 \end{xy}}
 \]\mpar{\anton{This is not consistent with the Gerstenhaber bracket on the previous page}}
 \begin{exercise}
   Use pictures to show that $d[\mu,\nu]=\pm [d\mu, \nu] \pm [\mu, d\nu]$.
 \end{exercise}
 Also, these pictures can be used to do the calculations in Exercises \ref{lec09Ex2}
 and \ref{lec09Ex3}.
}{   % Emily Peters, eep@math
  \stepcounter{lecture}
 \setcounter{lecture}{10}
 \sektion{Lecture 10}

 Here is the take-home exam, it's due on Tuesday:

 \begin{itemize}
 \item[(1)] $B\subset SL_{2}(\CC)$ are upper triangular matrices, then

 \begin{itemize}
 \item Describe $X=SL_{2}(\CC)/B$

 \item $SL_{2}(\CC)$ acts on itself via left multiplication implies that it acts on
 $X$. Describe the action.
 \end{itemize}

 \item[(2)] Find $\exp\left(
 \begin{array}
 [c]{cccc}%
 0 & x_{1} &  & 0\\
 & \ddots & \ddots & \\
 &  & 0 & x_{n-1}\\
 0 &  &  & 0
 \end{array}
 \right)  $

 \item[(3)] Prove that if $V,W$ are filtered vector spaces (with increasing filtration)
 and $\phi:V\to W$ satisfies $\phi(V_{i})\subseteq W_{i}$, and
 $Gr(\phi):Gr(V)\xrightarrow{\sim} Gr(W)$ an isomorphism, then $\phi$ is a linear
 isomorphism of filtered spaces.
 \end{itemize}

 \subsektion{Lie algebra cohomology}

 Recall $C^{\udot}(\g,M)$ from the previous lecture, for $M$ a finite dimensional
 representation of $\g$ (and $\g$ finite dimensional). There is a book by D.~Fuchs,
 \textsl{Cohomology of $\infty$ dimensional Lie algebras} \cite{Fuchs}.

 We computed that $H^{0}(\g,\g)=Z(\g)\simeq\g/[\g,\g]$ and that $H^{1}(\g,\g)$ is the
 space of exterior derivations of $\g$. Say $c\in Z^{1}(\g,\g)$,\footnote{$Z^n(\g,M)$
 is the space of $n$-cocycles, i.e.\ the kernel of $d:C^n(\g,M)\to C^{n+1}(\g,M)$.} so
 $[c]\in H^{1}(\g,\g)$. Define $\tilde\g_{c}=\g\oplus\CC\partial_{c}$ with the bracket
 $[(x,t),(y,s)]=([x,y]-tc(y)+sc(x),0)$. So if $e_{1},\dots,e_{n}$ is a basis in $\g$
 with the usual relations $[e_{i},e_{j}]=c_{ij}^{k}e_{k}$, then we get one more
 generator $\partial_{c}$ such that $[\partial _{c},x]=c(x)$. Then $H^{1}(\g,\g)$ is
 the space of equivalence classes of extensions
 \[
 0\rightarrow\g\rightarrow\tilde{\g}\rightarrow\CC\rightarrow0
 \]
 up to the equivalences $f$ such that the diagram commutes:
 \[
 \xymatrix{
 0 \ar[r] &\g \ar[r]\ar[d]^\id & \tilde \g \ar[r] \ar[d]^f & \CC \ar[d]^\id \ar[r] & 0\\
 0 \ar[r] &\g \ar[r] & \tilde\g' \ar[r] & \CC \ar[r] & 0\\
 }
 \]
 This is the same as the space of exterior derivations.

 \subsektion{\texorpdfstring{$H^{2}(\g,\g)$}{H2(g,g)} and Deformations of Lie algebras}

 A \emph{deformation}\index{deformation!of a Lie algebra} of $\g$ is the vector space
 $\g[[h]]$ with a bracket $[a,b]_{h} =[a,b]+\sum_{n\geq1}h^{n}m_{n}(a,b)$ such that
 $m_{n}(a,b)=-m_{n}(b,a)$ and
 \[
 [a,[b,c]_{h}]_{h}+[b,[c,a]_{h}]_{h}+[c,[a,b]_{h}]_{h}=0.
 \]
 The $h^{N}$ order term of the Jacobi identity yields equation \ref{lec09Eq:h^N},
 which was
 \[
 [a,m_{N}(b,c)]+m_{N}(a,[b,c])+\sum_{k=1}^{N-1}m_{k}(a,m_{N-k}(b,c))+\text{cycle}=0
 \]
 where ``cycle'' is the same thing, with $a$, $b$, and $c$ permuted cyclically.
 For $\mu\in C^{2}(\g,\g)$, we compute
 \[
 d\mu(a,b,c)=-[a,\mu(b,c)]-\mu(a,[b,c])+\text{cycle}.
 \]
 Define
 \[
 \{m_{k},m_{N-k}\}(a,b,c)\overset{def}{=}m_{k}(a,m_{N-k}(b,c))+\text{cycle}%
 \]\mpar[\anton{More Gerstenhaber, inconsistent notation}]{}
 This is called the Gerstenhaber bracket ... do a Google search for it if you like ...
 it is a tiny definition from a great big theory.

 Then we can rewrite equation \ref{lec09Eq:h^N} as equation \ref{lec09Eq:2}, which was
 \[
 dm_{N} = \sum_{k=1}^{N-1} \{m_{k},m_{N-k}\}. %\tag{$\ddag$}
 \]
 In partiular, $dm_{1}=0$, so $m_{1}$ is in $Z^{2}(\g,\g)$.

 Equivalences: $[a,b]'_{h}\simeq[a,b]_{h}$ if $[a,b]'_{h}=
 \phi^{-1}([\phi(a),\phi(b)]_{h})$ for some $\phi(a) = a+\sum_{n\ge1} h^{n}
 \phi_{n}(a)$. then
 \[
  m'_{1}(a,b) = m_{1}(a,b) - \phi_{1}([a,b]) + [a,\phi_{1}(b)] + [\phi _{1}(a),b].
 \]
 which we can write as $m'_{1} = m_{1}+d\phi_{1}$. From this we can conclude

 \begin{claim}
 The space of equivalence classes of possible $m_{1}$ is exactly
 $H^{2}(\g,\g)$.\mpar[\anton{not justified}]{}
 \end{claim}

 \begin{claim}
 [was HW]If $m_{1}$ is a 2-cocycle, and $m_{N-1},\dots, m_{2}$ satisfy the equations we
 want, then
 \[
 d\left(  \sum_{k=1}^{N-1} \{m_{k},m_{N-k}\} \right)  = 0.
 \]
 \end{claim}

 This is not enough; we know that $\sum_{k=1}^{N+1}\{m_{k},m_{N-k}\}$ is in
 $Z^{3}(\g,\g)$, but to find $m_{N}$, we need it to be trivial in $H^{3}(\g,\g)$
 because of equation \ref{lec09Eq:2}. If the cohomology class of
 $\sum_{k=1}^{N+1}\{m_{k},m_{N-k}\}$ is non-zero, it's class in $H^3(\g,\g)$ is called
 an \emph{obstruction} to $n$-th order deformation. If $H^{3}(\g,\g)$ is zero, then
 any first order deformation (element of $H^{2}(\g,\g)$) extends to a deformation, but
 if $H^{3}(\g,\g)$ is non-zero, then we don't know that we can always extend. Thus,
 $H^3(\g,\g)$ is the space of all possible obstructions to extending a deformation.

 Let's keep looking at cohomology spaces. Consider $C^{\udot}(\g,\CC)$, where $\CC$ is
 a one dimensional trivial representation of $\g$ given by $x\mapsto 0$ for any
 $x\in\g$.
 %Note that it is neat that such a representation exists.

 First question: take $U\g$, with the corresponding 1 dimensional representation
 $\varepsilon:U\g\rightarrow\CC$ given by $\varepsilon(x)=0$ for $x\in\g$.

 \begin{exercise}
   Show that $(U\g,\varepsilon,\Delta,S)$ is a Hopf algebra with the $\e$ above,
   $\Delta(x)=1\otimes x+x\otimes 1$, and $S(x)=-x$  for $x\in\g$. Remember that
   $\Delta$ and $\e$ are algebra homomorphisms, and that $S$ is an anti-homomorphism.
 \end{exercise}

 Let's compute $H^{1}(\g,\CC)$ ($H^{0}$ is boring, just a point). This is
 $\ker(C^{1}\xrightarrow{d} C^{2})$. Well, $C^{1}(\g,\CC)=\hom(\g,\CC)$,
 $C^{2}(\g,\CC)=\hom (\Lambda^{2}\g,\CC)$, and
 \[
 dc(x,y) = c([x,y]).
 \]
 So the kernel is the set of $c$ such that $c([x,y])=0$ for all $x,y\in\g$. Thus,
 $\ker(d)\subseteq C^{1}(\g,\CC)$ is the space of $\g$-invariant linear functionals. Recall
 that $\g$ acts on $\g$ by the adjoint action, and on
 $\g^{*}=C^{1}(\g,\g)$ by the coadjoint action ($x:l\mapsto l_{x}$ where $l_{x}%
 (y)=l([x,y])$). Under the coadjoint action, $l\in \g^*$ is $\g$-invariant if
 $l_{x}=0$. Note that $C^{0}$ is just one point, so its image doesn't have anything in
 it.

 Now let's compute $H^{2}(\g,\CC)=\ker(d:C^{2}\rightarrow C^{3})/\im(d:C^{1}%
 \rightarrow C^{2})$. Let $c\in Z^{2}$, then
 \[
 dc(x,y,z)=c([x,y],z)-c([x,z],y)+c([y,z],x)=0
 \]
 for all $x,y,z\in\g$. Now let's find the image of $d:C^{1}\rightarrow C^{2}$: it is
 the set of functions of the form $dl(x,y)=l([x,y])$ where $l\in\g^{\ast }$. It is
 clear that $l([x,y])$ are (trivial) 2-cocycles because of the Jacobi identity. Let's
 see what can we cook with this $H^{2}$.

 \begin{definition}
 A \emph{central extension}\index{central extension|idxbf} of $\g$ is a short exact sequence
 \[
 0\to \CC \to \tilde{\g}\to \g\to 0.
 \]
 Two such extensions are equivalent if there is a Lie algebra isomorphism
 $f:\tilde{\g}\to \tilde \g'$ such that the diagram commutes:
 \[
 \xymatrix{
 0 \ar[r] &\CC \ar[r]\ar[d]^\id & \tilde \g \ar[r] \ar[d]^f & \g \ar[d]^\id \ar[r] & 0\\
 0 \ar[r] &\CC \ar[r] & \tilde\g' \ar[r] & \g \ar[r] & 0\\
 }
 \]
 \end{definition}

 \begin{theorem} \label{lec10ThmCextns}
 $H^{2}(\g,\CC)$ is isomorphic to the space of equivalence classes of central
 extensions of $\g$.
 \end{theorem}

 \begin{proof}
 Let's describe the map in one direction. If $c\in Z^{2}$, then consider $\tilde\g =
 \g\oplus\CC$ with the bracket $[(x,t),(y,s)] = ([x,y],c(x,y))$. Equivalences of
 extensions boil down to $c(x,y)\mapsto c(x,y)+l([x,y])$.
 \begin{exercise}
   Finish this proof. \anton{This shouldn't be an exercise}
 \end{exercise}
 \end{proof}
 Let's do some (infinite dimensional) examples of central extensions.
 \begin{example}\index{central extension|idxit}
 [Affine Kac-Moody algebras]\index{Kac-Moody algebra} If $\g\subseteq \gl_n$, then we
 define the \emph{loop space}\index{loop space|idxit} or \emph{loop
 algebra}\index{loop algebra} $\mathcal{L}\g$ to be the set of maps $S^1\to \g$. To
 make the space more manageable, we only consider Laurent polynomials, $z\mapsto
 \sum_{m\in \ZZ}a_m z^m$ for $a_m\in \g$ with all but finitely many of the $a_m$ equal
 to zero. The bracket is given by $[f,g]_{\mathcal{L}\g}(z)=[f(z),g(z)]_\g$.

 Since $\g\subseteq \gl_n$, there is an induced trace $tr:\g\to \CC$. This gives a
 non-degenerate inner product on $\mathcal{L}\g$:
 \[
    (f,g):= \oint_{|z|=1} tr\bigl(f(z^{-1})g(z)\bigr) \frac{dz}{z}.
 \]
 There is a natural 2-cocylce on $\mathcal{L}\g$, given by
 \[
    c(f,g) = \frac{1}{2\pi i} \oint_{|z|=1} tr\bigl(f(z)g'(z)\bigr) \frac{dz}{z} =
           \underset{z=0}{Res}\biggl(tr\bigl(f(z)g^{\prime}(z)\bigr)\biggr),
 \]
 and a natural outer derivation $\partial: \mathcal{L}\g\to \mathcal{L}\g$ given by
 $\partial x(z) = \pder{x(z)}{z}$.

 The Kac-Moody algebra \anton{corresponding to the inner product?} is
 $\mathcal{L}\g\oplus \CC \partial\oplus \CC c$. A second course on Lie theory should
 have some discussion of the representation theory of this algebra.
 \end{example}

 \begin{example}
 Let $\gl_{\infty}$\index{gl(infinity)@$\gl_\infty$} be the algebra of matrices with
 finitely many non-zero entries. It is not very interesting. Let $\gl_{\infty}^{1}$ be
 the algebra of matrices with finitely many non-zero diagonals. $\gl_\infty^1$ is
 ``more infinite dimensional'' than $\gl_\infty$, and it is more interesting.
 \begin{exercise} Define
  \[
  J=\left(
  \begin{array}
  [c]{c|c}%
  I & 0\\\hline 0 & -I
  \end{array}
  \right).
  \]
  \mpar[\anton{Who calls that $J$?}]{}
   For $x,y\in\gl_{\infty}$, show that
   \[
   c(x,y)=tr(x[J,y])
   \]
   is well defined (i.e.\ is a finite sum).
 \end{exercise}
 This $c$ is a non-trivial 1-cocycle, i.e.\ $[c]\in H^{2}(\gl_{\infty}^{1},\CC)$ is
 non-zero. By the way, instead of just using linear maps, we require that the maps
 $\Lambda^{2}\gl_{\infty} ^{1}\rightarrow\CC$ are graded linear maps. This is
 $H_{\text{graded}}^{2}$.

 Notice that in $\gl_{n}$, $tr(x[J,y]) = tr(J[x,y])$ is a trivial cocycle (it is $d$ of
 $l(x)=tr(Jx)$. So we have that $H^{2}(\gl_{n},\CC)=\{0\}$.

 We can define $a_{\infty}= \gl_{\infty}\oplus\CC c$. This is some non-trivial central
 extension.
 \mpar[\small \anton{how do we show $H^2(\gl_n,\CC)=0$?, what is this $a_\infty$, and
 what are the relations to Conformal Field Theory and $\mathrm{Vect}(S^{1})\to$
 Virasoro.}]{\anton{what is the grading on $\CC$? It is all in degree 0, but
 $\Lambda \gl_{\infty}$ has positive and negative graded components}}
 \end{example}

 To summarize the last lectures:
 \begin{enumerate}
 \item We related Lie algebras and Lie Groups. If you're interested in
 representations of Lie Groups, looking at Lie algebras is easier.

 \item From a Lie algebra $\g$, we constructed $U\g$, the universal enveloping algebra.
 This got us thinking about associative algebras and Hopf algebras.

 \item We learned about dual pairings of Hopf algebras. For example, $\CC[\Gamma]$ and
 $C(\Gamma)$ are dual, and $U\g$ and $C(G)$ are dual (if $G$ is affine algebraic and
 we are looking at polynomial functions). This pairing is a starting point for many
 geometric realizations of representations of $G$. Conceptually, the notion of the
 universal enveloping algebra is closely related to the notion of the group algebra
 $\CC[\Gamma]$.

 \item Finally, we talked about deformations.
 \end{enumerate}

 \index{Reshetikhin, Nicolai|)}
}{   % Qingtau Chen, chenqtau@math
  \stepcounter{lecture}
 \setcounter{lecture}{11}
 \sektion{Lecture 11 - Engel's Theorem and Lie's Theorem} \index{Serganova, Vera|(}

 In the next ten lectures, we will cover
 \begin{enumerate}
 \item Classification of semisimple Lie algebras. This will include root systems and
       Dynkin diagrams.
%    When you study groups, there are solvable and nilpotent groups ... there is a
%    similar classification for Lie algebras.

 \item Representation theory of semisimple Lie algebras and the Weyl character formula.

 \item Compact connected Lie Groups.
 \end{enumerate}
 A reference for this material is Fulton and Harris \cite{FulHar}.

 The first part is purely algebraic: we will study Lie algebras. $\g$ will be a Lie
 algebra, usually finite dimensional, over a field $k$ (usually of characteristic 0).

 Any Lie algebra $\g$ contains the ideal $\D(\g) = [\g,\g]$, the vector subspace
 generated by elements of the form $[X,Y]$ for $X,Y\in \g$.
 \begin{exercise}
   Show that $\D\g$ is an ideal in $\g$.
   \begin{solution}
     $[\g,\D\g]\subseteq [\g,\g] = \D\g$, so $\D\g$ is an ideal.
   \end{solution}
 \end{exercise}
 \begin{exercise} \label{lec11hardEx}
   Let $G$ be a simply connected Lie group with Lie algebra $\g$. Then $[G,G]$ is the
   subgroup of $G$ generated by elements of the form $ghg^{-1}h^{-1}$ for $g,h\in G$.
   Show that $[G,G]$ is a connected closed normal Lie subgroup of $G$, with Lie
   algebra $\D\g$.
  \begin{solution}
   $[G,G]$ is normal because $r[g,h]r^{-1}=[rgr^{-1},rhr^{-1}]$. To see
   that $[G,G]$ is connected, let $\gamma_{gh}:[0,1]\to G$ be a path from $g$ to $h$.
   Then $t\mapsto g\gamma(t)g^{-1}\gamma(t)^{-1}$ is a path in $[G,G]$ from
   the identity to $[g,h]$. Since all the generators of $[G,G]$ are connected
   to $e\in G$ by paths, all of $[G,G]$ is connected to $e$.

   Now we show that the Lie algebra of $[G,G]$ is $\D\g$. Consider the Lie algebra
   homomorphism $\pi:\g\to \g/\D\g$. Since $G$ is simply connected, Theorem
   \ref{lec04T:4} says there is a Lie group homomorphism $p:G\to H$ lifting $\pi$.
   \[\xymatrix{
    \D\g \ar[r] &\g \ar[r]^\pi \ar[d]_\exp & \g/\D\g\ar[d]^\exp\\
    [G,G]\ar[r] & G\ar@{.>}[r]^p & **[r] H\cong \RR^n
   }\]
   where $H$ is the simply connected Lie group with Lie algebra $\g/\D\g$. Note that
   the Lie algebra of the kernel of $p$ must be contained in $\ker \pi = \D\g$. Also,
   $\g/\D\g$ is abelian, so $H$ is abelian, so $[G,G]$ is in the kernel of $p$. This
   shows that $\lie([G,G])\subseteq \D\g$.

   To see that $\D\g\subseteq \lie([G,G])$, assume that $\g\subseteq \gl(V)$. Then for
   $X,Y\in \g$ consider the path $\gamma(t) =
   \exp(X\sqrt{t})\exp(Y\sqrt{t})\exp(-X\sqrt{t})\exp(-Y\sqrt{t})$ in $[G,G]$:
   \begin{align*}
     \gamma(t) &= \left(1+X\sqrt{t}+\frac{1}{2}X^2t+\cdots\right)
                 \left(1+Y\sqrt{t}+\frac{1}{2}Y^2t+\cdots\right) \times \\
               & \hspace{5em}
                 \left(1-X\sqrt{t}+\frac{1}{2}X^2t+\cdots\right)
                 \left(1-Y\sqrt{t}+\frac{1}{2}Y^2t+\cdots\right) \\
               &= 1+\sqrt{t}(X+Y-X-Y) + \\
               &\hspace{2.5em} t(XY-X^2-XY-YX -Y^2 +X^2+Y^2) +\cdots\\
               &=1+ t[X,Y] + O(t^{3/2})
   \end{align*}
   So $\gamma'(0)=[X,Y]$. This shows that $[G,G]$ is a connected component of the
   kernel of $p$.

   Since $[G,G]$ is a connected component of $p^{-1}(0)$, it is closed in $G$.
  \end{solution}
 \end{exercise}

   \begin{warning} Exercise \ref{lec11hardEx} is a tricky problem. Here are some
   potential pitfalls:
     \begin{enumerate}
       \item For $G$ connected, we do not necessarily know that the exponential map is
       surjective, because $G$ may not be complete. For example, $\exp: \sl_2(\CC)\to
       SL_2(\CC)$ is not surjective.\footnote{Assume $\matrix {-1}10{-1}$ is in the
       image, then its pre-image must have eigenvalues $(2n+1)i\pi$ and $-(2n+1)i\pi$
       for some integer $n$. So the pre-image has distinct eigenvalues, so it is
       diagonalizable. But that implies that $\matrix {-1}10{-1}$ is diagonalizable,
       contradiction.}

       \item If $H\subseteq G$ is a subgroup with Lie algebra $\h$, then $\h\subseteq
       \g$ closed is not enough to know that $H$ is closed in $G$. For example, take
       $G$ to be a torus, and $H$ to be a line with irrational slope.

       \item The statement is false if we relax the condition that $G$ is simply
       connected. Let
       \begin{align*}
          H &:= \left\{ \mbox{\scriptsize $\mat{1 & x & y\\0 & 1 & z\\0 & 0 &1}$}\right\}\times S^1\\
          K &:= \left\{ \Biggl(
            \mbox{\scriptsize $\mat{
              1 & 0 & n\\
              0 & 1 & 0\\
              0 & 0 & 1}$},c^n\Biggr)
          \biggm|n\in \ZZ \right\}
          \subseteq H
       \end{align*}
       where $c$ is an element of $S^1$ of infinite order. Then $K$ is normal in $H$
       and $G=H/K$ is a counterexample.
     \end{enumerate}
   \end{warning}

% If you believe these exercises, then the next definitions should make sense.

 \begin{definition}
   Define $\D^0\g=\g$, and $\D^n\g = [\D^{n-1}\g,\D^{n-1}\g]$. This is called the
   \emph{derived series}\index{derived series} of $\g$. We say $\g$ is
   \emph{solvable}\index{solvable} if $\D^n\g=0$ for some $n$ sufficiently large.
 \end{definition}
 \begin{definition}
   We can also define $\D_0\g=\g$, and $\D_n\g = [\g,\D_{n-1}\g]$. This is called the
   \emph{lower central series}\index{lower central series} of $\g$. We say that $\g$
   is \emph{nilpotent}\index{nilpotent} if $\D_n\g=0$ for some $n$ sufficiently large.
 \end{definition}

 Note that $\D_1\g=\D^1\g$ by $\D\g$. Solvable and nilpotent Lie algebras are
 hard to classify. Instead, we will do the classification of semisimple Lie algebras
 (see Definition \ref{lec11def:semisimple}).

 The following example is in some sense universal (see corollaries \ref{lec11Engelcor}
 and \ref{lec11Liecor}):
 \begin{example}
   Let $\gl(n)$\index{gl(n)@$\gl(n)$} be the Lie algebra of all $n\times n$ matrices,
   and let $\b$\index{b@$\b$}\index{upper triangular|see{$\b$}} be the subalgebra of
   upper triangular matrices. I claim that $\b$ is solvable. To see this, note that
   $\D\b$ is the algebra of \emph{strictly} upper triangular matrices, and in general,
   $\D^k\b$ has zeros on the main diagonal and the $2^{k-2}$ diagonals above the main
   diagonal (for $k\ge 2$). Let $\mathfrak{n}=\D\b$. You can check that $\mathfrak{n}$
   is in fact nilpotent.
 \end{example}

\noindent Useful facts about solvable/nilpotent Lie algebras:
 \index{useful facts about solvable and nilpotent Lie algebras|(idxbf}
 \begin{enumerate}
 \item \label{lec11Fexactsolv} If you have an exact sequence of Lie algebras
 \[
    0\to \a\to \g\to \g/\a \to 0
 \]
 then $\g$ is solvable if and only if $\a$ and $\g/\a$ are solvable.

 \item \label{lec11Fexactnil} If you have an exact sequence of Lie algebras
 \[
    0\to \a\to \g\to \g/\a \to 0
 \]
 then if $\g$ is nilpotent, so are $\a$ and $\g/\a$.

 \begin{warning}
   The converse is not true. Diagonal matrices $\mathfrak{d}$\index{d@$\mathfrak{d}$} is
   nilpotent, and we have
 \[
    0\to \mathfrak{n}\to \b\to \mathfrak{d}\to 0.
 \]
 Note that $\b$ is not nilpotent, because $\D\b=\D_n\b=\matrix{0}{\ast}{0}{0}$.
 \end{warning}

 \item \label{lec11Fsumsolv} If $\a,\b\subset \g$ are solvable ideals, then the sum
 $\a+\b$ is solvable. To see this, note that we have
 \[
    0\to \a\to \a+\b \to \underbrace{(\a+\b)/\a}_{\simeq \b/(\a\cap\b)} \to 0
 \]
 $\a$ is solvable by assumption, and $\b/(\a\cap\b)$ is a quotient of a solvable
 algebra, so it is solvable by (1). Applying (1) again, $\a+\b$ is solvable.

% \item[(1)] A subalgebra of a solvable (nilpotent) algebra is again solvable
% (nilpotent).
% \item[(2)] A quotient of a solvable (nilpotent) algebra is again solvable
% (nilpotent).
% \item[(3)] If you have a short exact sequence
% \[
%    0\to \a\to \g \to \mathfrak{f}\to 0
% \]
% with $\a$ and $\mathfrak{f}$ solvable, then $\g$ is also solvable. This doesn't work
% for nilpotent. Diagonal matrices is nilpotent, and we have
% \[
%    0\to \mathfrak{n}\to \b\to \mathfrak{d}\to 0
% \]
% To see that $\b$ is not nilpotent, notice that $\D\b=\matrix{0}{\ast}{0}{0}$, but
% then $\D_2\b = \matrix{0}{\ast}{0}{0}$.

 \item \label{lec11Fscalers} If $k\subseteq F$ is a field extension, with a Lie algebra
 $\g$ over $k$, we can make a Lie algebra $\g\otimes_k F$ over $F$. Note that
 $\g\otimes_k F$ is solvable (nilpotent) if and only if $\g$ is.
 \end{enumerate}
 \index{useful facts about solvable and nilpotent Lie algebras|)\idxbf}


 We will now prove Engel's theorem and Lie's theorem.

% They are as ubiquitous as Schur's lemma.

 For any Lie algebra $\g$, we have  the adjoint representation: $X\mapsto ad_X\in
 \gl(\g)$ given by $ad_X(Y)=[X,Y]$. If $\g$ is nilpotent, then $ad_X$ is a nilpotent
 operator for any $X\in \g$. The converse is also true as we will see shortly (Cor.\
 \ref{lec11Engelcor2}).

 \begin{theorem}[Engel's Theorem] \label{lec11Engel} \index{Engel's Theorem|idxbf}
   Let $\g\subseteq \gl(V)$, and assume that $X$ is nilpotent for any $X\in \g$. Then
   there is a vector $v\in V$ such that $\g \cdot v =0$.
 \end{theorem}
 Note that the theorem holds for any representation $\rho$ of $\g$ in which every
 element acts nilpotently; just replace $\g$ in the statement of the theorem by
 $\rho(\g)$.
 \begin{corollary}\label{lec11Engelcor}
   If $V$ is a representation of $\g$ in which every element acts nilpotently, then
   one can find $\{0\}=V_0\subsetneq V_1\subsetneq \cdots \subsetneq V_n=V$ a complete
   flag such that $\g(V_i)\subseteq V_{i-1}$. That is, there is a basis in which all
   of the elements of $\g$ are strictly upper triangular.
 \end{corollary}
 \begin{warning}
 Note that the theorem isn't true if you say ``suppose $\g$ is nilpotent'' instead of
 the right thing. For example, the set of diagonal matrices $\mathfrak{d}\subset
 \gl(V)$ is nilpotent.
 \end{warning}
 \begin{proof}
  Let's prove the theorem by induction on $\dim \g$.We first show that $\g$ has an
  ideal $\a$ of codimension 1. To see this, take a maximal proper subalgebra
  $\a\subset \g$. Look at the representation of $\a$ on the quotient space $\g/\a$.
  This representation, $\a\to \gl(\g/\a)$, satisfies the condition of the
  theorem,\footnote{For any $X\in \a$, since $X$ is nilpotent, $ad_X$ is also
  nilpotent.} so by induction, there is some $X\in \g$ such that $ad_\a(X)=0$ modulo
  $\a$. So $[\a,X]\subseteq \a$, so $\mathfrak{b} = kX\oplus \a$ is a new subalgebra of $\g$
  which is larger, so it must be all of $\g$. Thus, $\a$ must have had codimension 1.
  Therefore, $\a\subseteq \g$ is actually an ideal (because $[X,\a]=0$).

  Next, we prove the theorem. Let $V_0 = \{v\in V|\a
  v=0\}$, which is non-zero by the inductive hypothesis. We claim that $\g
  V_0\subseteq V_0$. To see this, take $x\in \g$, $v\in V_0$, and $y\in \a$. We have
  to check that $y(xv)=0$. But
  \[yxv = x\underbrace{yv}_0+\underbrace{[y,x]}_{\in \a}v=0.\]
  Now, we have that $\g=kX\oplus \a$, and $\a$ kills $V_0$, and that
  $X:V_0\to V_0$ is nilpotent, so it has a kernel. Thus, there is some $v\in
  V_0$ which is killed by $X$, and so $v$ is killed by all of $\g$.
 \end{proof}
 \begin{corollary}\label{lec11Engelcor2}
   If $ad_X$ is nilpotent for every $X\in \g$, then $\g$ is nilpotent as a Lie algebra.
 \end{corollary}
 \begin{proof}
   Let $V=\g$, so we have $ad:\g\to \gl(\g)$, which has kernel $Z(\g)$. By Engel's
   theorem, we know that there is an $x\in\g$ such that $(ad\,\g)(x)=0$. This implies that
   $Z(\g)\neq 0$. By induction we can assume $\g/Z(\g)$ is nilpotent. But then $\g$
   itself must be nilpotent as well because $\D_n(\g/Z(\g))=0$  implies
   $\D_{n+1}(\g)=0$.
 \end{proof}

 \begin{warning} If $\g\subseteq \gl(V)$ is a nilpotent subalgebra, it does not
 imply that every $X\in \g$ is nilpotent (take diagonal matrices for example).
 \end{warning}
 \begin{theorem}[Lie's Theorem]\label{lec11Lie} \index{Lie's Theorem|idxbf}
   Let $k$ be algebraically closed and of characteristic 0. If $\g\subseteq \gl(V)$ is a
   solvable subalgebra, then all elements of $\g$ have a common eigenvector in $V$.
 \end{theorem}
 This is a generalization of the statement that two commuting operators have a common
 eigenvector.
 \begin{corollary}\label{lec11Liecor}
   If $\g$ is solvable, then there is a complete flag
 $\{0\}=V_0\varsubsetneq V_1\varsubsetneq \cdots \varsubsetneq V_n=V$
 such that $\g(V_i)\subseteq V_i$. That is, there is a basis in which all elements of
 $\g$ are upper triangular.
 \end{corollary}
 \begin{proof}
   If $\g$ is solvable, take any subspace $\a\subset \g$ of codimension 1 containing
   $\D\g$, then $\a$ is an ideal. We're going to try to do the same kind of induction
   as in Engel's theorem.

   For a linear functional $\lambda: \a\to k$, let
   \[
    V_\lambda=\{v\in V| Xv = \lambda(X)v \text{ for all } X\in \a\}.
   \]
   $V_\lambda\neq 0$ for some $\lambda$ by induction hypothesis.
 \begin{claim}
   $\g(V_\lambda)\subseteq V_\lambda$.
 \end{claim}
 \begin{proof}[Proof of Claim]
  Choose $v\in V_\lambda$ and $X\in \a$, $Y\in \g$. Then
  \begin{align*}
    X(Yv) &= Y(\underbrace{Xv}_{\lambda(X)v}) +
    \underbrace{[X,Y]v}_{\lambda([X,Y])v}
  \end{align*}
  We want to show that $\lambda[X,Y]=0$. There is a trick. Let $r$ be the largest
  integer such that $v, Yv, Y^2v, \dots, Y^r v$ is a linearly independent set. We know
  that $Xv=\lambda(X)v$ for any $X\in \a$. We claim that $XY^jv \equiv \lambda(X)Y^jv
  \mod (span\{v,Yv,\ldots,Y^{j-1}v\})$. This is clear for $j=0$, and by induction, we
  have
  \begin{align*}
    X Y^j v & = Y\hspace{-1em}\underbrace{XY^{j-1}v}_{
                \substack{\equiv \lambda(X)Y^{j-1}v \\
                \hspace{-3.25em} \mod span\{v,\dots, Y^{j-2}v\}}}
                +\hspace{-2em} \underbrace{[X,Y]Y^{j-1}v}_{%
                    \substack{\equiv \lambda([X,Y])Y^{j-1}v \\ \hspace{1.75em}
                    \mod span\{v,\dots, Y^{j-2}v\}}}\\
      &\equiv \lambda(X) Y^j v \mod span\{v,\dots, Y^{j-1}v\}
  \end{align*}
%  \footnote[1]{Notes editor's comment: Use
%  $XY^jv=YXY^{j-1}v+[X,Y]Y^{j-1}v$ and induction on $j$. Warning: You can't argue by
%  saying $XY^jv=Y^jXv+[X,Y^j]v=\lambda(X)Y^jv+\lambda([X,Y^j])v$, because $Y^j$ might
%  not be in $\g$ and therefor $[X,Y^j]$ might not be in $\a$.}.

  So the matrix for $X$ can be written as $\lambda(X)$ on the diagonal and stuff above
  the diagonal (in this basis). So the trace of $X$ is $(r+1)\lambda(X)$. Then we have
  that $tr([X,Y])=(r+1)\lambda([X,Y])$, since the above statement was proved for any
  $X\in \a$ and $[X,Y]\in\a$. But the trace of a commutator is always 0. Since the
  characteristic of $k$ is 0, we can conclude that $\lambda[X,Y]=0$.
 \renewcommand{\qedsymbol}{$\square_{\text{Claim}}$}
 \end{proof}
 To finish the proof, write $\g=kT\oplus \a$, with $T:V_{\lambda}\to V_\lambda$ (we
 can do this because of the claim). Since $k$ is algebraically closed, $T$ has a
 non-zero eigenvector $w$ in $V_\lambda$. This $w$ is the desired common eigenvector.
 \end{proof}
 \begin{remark}
   If $k$ is not algebraically closed, the theorem doesn't hold. For example, consider
   the (one dimensional) Lie algebra generated by a rotation of $\RR^2$.

   The theorem also fails if $k$ is not characteristic 0. Say $k$ is characteristic
   $p$, then let
%   \[
%    x = \mat{0 & 1 &  & & 0\\
%               & 0 & 1 & & \\
%               0&   &  & \ddots & 0 \\
%               1 & 0 & & & 0 } \quad , \quad
%    y = \mat{0 & & & 0\\
%               &1& & \\
%               & &2 &\\
%               & & &\ddots\\
%               & & & & p-1}
%   \]
   $x$ be the permutation matrix of the $p$-cycle $(p\ p-1\ \cdots \ 2\ 1)$ (i.e.\ the
   matrix $\matrix 0{I_{p-1}}10$), and let $y$ be the diagonal matrix
   $diag(0,1,2,\dots, p-1)$. Then $[x,y]=x$, so the Lie algebra generated by $x$ and
   $y$ is solvable. However, $y$ is diagonal, so we know all of its eigenvectors, and
   none of them is an eigenvector of $x$.
 \end{remark}

 \begin{corollary}
   Let $k$ be of characteristic 0. Then $\g$ is solvable if and only if $\D\g$ is
   nilpotent.
 \end{corollary}
 If $\D\g$ is nilpotent, then $\g$ is solvable from the definitions. If $\g$ is
 solvable, then look at everything over the algebraic closure of $k$, where $\g$ looks
 like upper triangular matrices, so $\D\g$ is nilpotent. All this is independent of
 coefficients (by useful fact (\ref{lec11Fscalers})).

 \subsektion{The radical}

 There is a unique maximal solvable ideal in $\g$ (by useful fact
(\ref{lec11Fsumsolv}): sum of solvable ideals is solvable), which is called the
radical of $\g$.
 \begin{definition}\label{lec11def:semisimple}
   We call $\g$ \emph{semisimple}\index{semisimple} if $\mathrm{rad}\,\g=0$.
 \end{definition}
 \begin{exercise}
   Show that $\g/\mathrm{rad}\,\g$ is always semisimple.
   \begin{solution}
     Let $\pi:\g\to \g/\mathrm{rad}\,\g$ be the canonical projection, and assume
     $\a\in \g/\mathrm{rad}\,\g$ is solvable. Then $\D^k \a=0$ for some $k$, so $\D^k
     \pi^{-1}(\a)\subseteq \mathrm{rad}\,\g$. Since $\mathrm{rad}\,\g$ is solvable, we
     have that $\D^N \pi^{-1}(\a)=0$ for some $N$. By definition of
     $\mathrm{rad}\,\g$, we get that $\pi^{-1}(\a)\subseteq \mathrm{rad}\,\g$, so
     $\a=0\subseteq \g/\mathrm{rad}\,\g$. Thus, $\g/\mathrm{rad}\,\g$ is semisimple.
   \end{solution}
 \end{exercise}
 If $\g$ is one dimensional, generated by $X$, then we have that $[\g,\g]=0$, so $\g$
 cannot be semisimple.

 If $\g$ is two dimensional, generated by $X$ and $Y$, then we have that $[\g,\g]$ is
 one dimensional, spanned by $[X,Y]$. Thus, $\g$ cannot be semisimple because $\D\g$
 is a solvable ideal.

 There is a semisimple Lie algebra of dimension $3$, namely $\sl_2$.

 Semisimple algebras have really nice properties. Cartan's criterion (Theorem
 \ref{lec12Cartan}) says that $\g$ is semisimple if and only if the Killing form (see
 next lecture) is non-degenerate. Whitehead's theorem (Theorem \ref{lec12Whitehead})
 says that if $V$ is a non-trivial irreducible representation of a semisimple Lie
 algebra $\g$, then $H^i(\g,V)=0$ for all $i$. Weyl's theorem (Theorem
 \ref{lec12Weyl}) says that every finite dimensional representation of a semisimple
 Lie algebra is the direct sum of irreducible representations. If $G$ is simply
 connected and compact, then $\g$ is semisimple (See Lecture 20).
}{   % Sevak Mkrtchyan, sevak@math
  \stepcounter{lecture}
 \setcounter{lecture}{12}
 \sektion{Lecture 12 - Cartan Criterion, Whitehead and Weyl Theorems}

% We will prove the Cartan criterion:
% \begin{theorem}[Cartan criterion]\label{lec12CarCrit}
%   $\g$ is semisimple if and only if the Killing form is non-degenerate.
% \end{theorem}
% This will give us information about Cohomology of semisimple Lie algebras and some
% other stuff.

 \subsektion{Invariant forms and the Killing form}\index{Killing
 form|idxbf}\index{invariant form}

 Let $\rho:\g\to \gl(V)$ be a representation. To make the notation cleaner, we will
 write $\bar X$ for $\rho(X)$. We can define a bilinear form on $\g$ by $B_V(X,Y):=
 tr(\bar X \bar Y)$. This form is symmetric because $tr(AB)=tr(BA)$ for any linear
 operators $A$ and $B$.

 We also have that
 \begin{align*}
 B_V([X,Y],Z) &= tr(\bar X\bar Y\bar Z-\bar Y\bar X\bar Z) = tr(\bar X\bar Y\bar Z) - tr(\bar X\bar Z\bar Y)\\
            &=tr(\bar X\bar Y\bar Z-\bar X\bar Z\bar Y)=B_V(X,[Y,Z]),
 \end{align*}
 so $B$ satisfies
 \[
    B([X,Y],Z) = B(X,[Y,Z]).
 \]
 Such a form is called an \emph{invariant form}. It is called invariant because it is
 implied by $B$ being $Ad$-invariant.\footnote{If $G$ is connected, the two versions of
 invariance are equivalent.\anton{why?}} Assume that for any $g\in G$ and $X,Z\in \g$, we
 $B(Ad_g X,Ad_g Z)=B(X,Z)$. Let $\gamma$ be a path in $G$ with $\gamma'(0)=Y$. We get
 that
 \[
    B([Y,X],Z)+B(X,[Y,Z]) = \der{}{t}\bigg|_{t=0}
        B\bigl(Ad_{\gamma(t)}(X),Ad_{\gamma(t)}(Z)\bigr) = 0.
 \]



% If $\g$ is any Lie algebra,
% you can define a bilinear symmetric form by the formula $B(X,Y)=tr(ad_X\circ ad_Y)$.
% One important property $B$ satisfies is that $B([Y,X],Z) + B(X,[Y,Z])=0$, or
% $B([X,Y],Z)=B(X,[Y,Z])$. Such a form is called invariant. Why is it called invariant?
% If you look at the Lie group $G$ acting on $\g$ via the adjoint representation. The
% given condition is equivalent to the condition that $B(Ad_g X, Ad_g Y)=B(X,Y)$: if
% $\gamma'(0)=Y$, then
% \[
%    \left.\der{}{t}\right|_{t=0} B(Ad_\gamma X, Ad_\gamma Z) = B([Y,X],Z)+B(X,[Y,Z]).
% \]
%
% Let's talk about invariant forms a bit.
%
% If we are given a representation $\g\subseteq
% \gl(V)$, then we can always construct a form $B_V(X,Y):= tr_V(XY)$. This form is
% always invariant because
% \begin{align*}
% B_V([X,Y],Z) &= tr(XYZ-YXZ) = tr(XYZ) - tr(XZY)\\
%            &=tr(XYZ-XZY)=B_V(X,[Y,Z])
% \end{align*}

 \begin{definition}
   The \emph{Killing form}, denoted by $B$, is the special case where $\rho$ is the
   adjoint representation\index{adjoint representation}. That is, $B(X,Y):=
   tr(ad_X\circ ad_Y)$.
 \end{definition}
 \begin{exercise}
   Let $\g$ be a simple Lie algebra over an algebraically closed field. Check that two
   invariant forms on $\g$ are proportional.
   \begin{solution}
     An invariant form $B$ induces a homomorphism $\g\to \g^*$. Invariance says that
     this homomorphism is an intertwiner of representations of $\g$ (with the adjoint action on
     $\g$ and the coadjoint action on $\g^*$). Since $\g$ is simple, these are both
     irreducible representations. By Schur's Lemma, any two such homomorphisms must be
     proportional, so any two invariant forms must be proportional.
   \end{solution}
 \end{exercise}
 \begin{exercise}[In class]\label{lec12Ex1}
   If $\g$ is solvable, then $B(\g,\D\g)=0$.
  \ifthenelse{\boolean{proofmode}}{\begin{solution}}{\begin{proof}[Solution]}
   First note that if $Z = [X,Y] \in \D\g$, then $ad_Z = [ad_X,ad_Y] \in \D( ad\,\g)$
   since the adjoint representation is a Lie algebra homomorphism.  Moreover, $\g$
   solvable implies that the image of the adjoint representation, $ad(\g) \simeq \g /
   Z(\g)$, is solvable.  Therefore, in some basis of $V$ of a representation of
   $ad(\g)$, all matrices of $ad(\g)$ are upper triangular (by Lie's Theorem), and
   those of $\D(ad\g)$ are all strictly upper triangular. The product of an upper
   triangular matrix and a strictly upper triangular matrix will be strictly upper
   triangular and therefore have trace 0.
%%%%%%%% For some reason, this doesn't work with the \ifthenelse syntax%%%%
  \ifproofmode \end{solution}\else                                        %
   \renewcommand\qedsymbol{$\blacksquare$}                                %
   \end{proof}                                                            %
   \begin{solution}                                                       %
     Done in class.                                                       %
   \end{solution}                                                         %
  \fi                                                                     %
%%%%%%%%%%%%%%%%%%%%%%%%%%%%%%%%%%%%%%%%%%%%%%%%%%%%%%%%%%%%%%%%%%%%%%%%%%%
 \end{exercise}
 The converse of this exercise is also true.  It will follow as a corollary
 of our next theorem (Corollary \ref{lec12CorT1} below).
 \begin{theorem}\label{lec12T1}
   Suppose $\g\subseteq \gl(V)$, $char\, k=0$, and $B_V(\g,\g)=0$. Then $\g$ is solvable.
 \end{theorem}
 For the proof, we will need the following facts from linear algebra.
 \begin{lemma}\label{lec12L1}\hspace*{-1em}
   \footnote{This is different from what we did in class.  There is an easier way
   to do this if you are willing to assume $k = \CC$ and use complex conjugation.  See
   Fulton and Harris for this method.} Let $X$ be a diagonalizable linear operator in
   $V$, with $k$ algebraically closed. If $X=A\cdot diag(\lambda_1,\dots,
   \lambda_n)\cdot A^{-1}$ and $f:k\to k$ is a function, we define $f(X)$ as $A\cdot
   diag(f(\lambda_1),\dots, f(\lambda_n))\cdot A^{-1}$. Suppose $tr (X \cdot f(X))=0$
   for any $\QQ$-linear map $f:k \rightarrow k$ such that $f$ is the identity on
   $\QQ$, then $X=0$.
 \end{lemma}
 \begin{proof}
   Consider only $f$ such that the image of $f$ is $\QQ$. Let
   $\lambda_1,\dots, \lambda_m$ be the eigenvalues of $X$ with multiplicities
   $n_1,\dots, n_m$.  We obtain $tr(X\cdot\nobreak f(X)) = n_1 \lambda_1 f(\lambda_1)+\ldots
   +n_m \lambda_m f(\lambda_n)=0$. Apply $f$ to this identity to obtain $n_1
   f(\lambda_1)^2+\ldots+n_m f(\lambda_m)^2=0$ which implies $f(\lambda_i)=0$ for all
   $i$. If some $\lambda_i$ is not zero, we can choose $f$ so that $f(\lambda_i) \neq
   0$, so $\lambda_i=0$ for all $i$. Since $X$ is diagonalizable, $X=0$.
 \end{proof}

 \begin{lemma}[Jordan Decomposition]\label{lec12Ljordan}
   \index{Jordan decomposition|HyperPageForJoke}
   Given $X\in \gl(V)$, there are unique $X_s, X_n\in \gl(V)$ such that $X_s$ is
   diagonalizable, $X_n$ is nilpotent, $[X_s,X_n]=0$, and $X=X_n+X_s$.  Furthermore,
   $X_s$ and $X_n$ are polynomials in $X$.
 \end{lemma}
 \begin{proof}
   All but the last statement is standard; see, for example, Corollay 2.5 of Chapter
   XIV of \cite{Lang:Algebra}. To see the last statement, let the characteristic
   polynomial of $X$ be $\prod_i(x-\lambda_i)^{n_i}$. By the chinese remainder
   theorem, we can find a polynomial $f$ such that $f(x)\equiv \lambda_i \mod
   (x-\lambda_i)^{n_i}$. Choose a basis so that $X$ is in Jordan form and compute
   $f(X)$ block by block.  On a block with $\lambda_i$ along the diagonal
   $(X-\lambda_i I)^{n_i}$ is 0, so $f(X)$ is $\lambda_i I$ on this block. Then
   $f(X)=X_s$ is diagonalizable and $X_n=X-f(X)$ is nilpotent.
 \end{proof}

 \begin{lemma}\label{lec12L3}
   Let $\g\subseteq \gl(V)$. The adjoint representation $ad:\g\to\gl(\g)$ preserves
   Jordan decomposition\index{Jordan decomposition!under the adjoint representation}:
   $ad_{X_s}=(ad_X)_s$ and $ad_{X_n} = (ad_X)_n$. In particular, $ad_{X_s}$ is a
   polynomial in $ad_{X}$.
 \end{lemma}
 \begin{proof}
   Suppose the eigenvalues of $X_s$ are $\lambda_1,\dots, \lambda_m$, and we are in a
   basis where $X_s$ is diagonal. Check that
   $ad_{X_s}(E_{ij})=[X_s,E_{ij}]=(\lambda_i-\lambda_j)E_{ij}$. So $X_s$
   diagonalizable implies $ad_{X_s}$ is diagonalizable (because it has a basis of
   eigenvectors). We have that $ad_{X_n}$ is nilpotent because the monomials in the
   expansion of $(ad_{X_n})^k(Y)$ have $X_n$ to at least the $k/2$ power on one side
   of $Y$. So we have that $ad_X = ad_{X_s}+ad_{X_n}$, with $ad_{X_s}$ diagonalizable,
   $ad_{X_n}$ nilpotent, and the two commute, so by uniqueness of the Jordan
   decomposition, $ad_{X_s}=(ad_X)_s$ and $ad_{X_n}=(ad_X)_n$.
 \end{proof}

 \begin{proof}[Proof of Theorem \ref{lec12T1}]
   It is enough to show that $\D\g$ is nilpotent. Let
   $X\in \D\g$, so $X=\sum [Y_i,Z_i]$. It suffices to show that $X_s = 0$.  To do
   this, let $f:k \rightarrow k$ be any $\QQ$-linear map fixing $\QQ$.
   \begin{align*}
     B_V(f(X_s),X_s) &= B_V(f(X_s), X) & \text{($X_n$ doesn't contribute)}\\
     &= B_V{\Big (}f(X_s), \sum_i [Y_i,Z_i]{ \Big )} \\
     &= \sum_i B_V(\underbrace{[f(X_s),Y_i]}_{\in \g ?},Z_i) & \text{($B_V$ invariant)}\\
     &= 0 & (\text{assuming } [f(X_s),Y_i]\in \g)
   \end{align*}
   Then by Lemma \ref{lec12L1}, $X_s=0$.

   To see that $[f(X_s),Y_i]\in \g$, suppose the eigenvalues of $X_s$ are
   $\lambda_1,\dots, \lambda_m$.  Then the eigenvalues of $f(X_s)$ are $f(\lambda_i)$,
   the eigenvalues of $ad_{X_s}$ are of the form $\mu_{ij} := \lambda_i - \lambda_j$,
   and eigenvalues of $ad_{f(X_s)}$ are $\nu_{ij} := f(\lambda_i) - f(\lambda_j) =
   f(\mu_{ij})$.  If we define $g$ to be a polynomial such that $g(\mu_{ij}) =
   \nu_{ij}$, then $ad_{f(X_s)}$ and $g(ad_{X_s})$ are diagonal (in some basis) with
   the same eigenvalues in the same places, so they are equal. So we have
   \begin{align*}
   [f(X_s),Y_i] &= g(ad_{X_s})(Y_i) \\
    &= h(ad_X)(Y_i) \in \g & \text{(using Lemma \ref{lec12L3})}
   \end{align*}
   for some polynomial $h$.

   The above arguments assume $k$ is algebraically closed, so if it's not apply the
   above to $\g\otimes_k \bar k$.  Then $\g\otimes_k \bar k$ solvable implies $\g$
   solvable as mentioned in the previous lecture.
 \end{proof}

 \begin{corollary}\label{lec12CorT1}
   $\g$ is solvable if and only if $B(\D\g,\g)=0$.
 \end{corollary}
 \begin{proof} ($\Leftarrow$)
   We have that $B(\D\g,\g)=0$ implies $B(\D\g,\D\g)=0$ which  implies that $ad
   (\D\g)$ is solvable.
   The adjoint representation of $\D\g$ gives the exact sequence
   \[
    0\to Z(\D\g) \to \D\g \to ad(\D\g) \to 0.
   \]
   Since $Z(\D\g)$ and $ad(\D\g)$ are solvable, $\D\g$ is solvable by useful
   fact (\ref{lec11Fexactsolv}) of Lecture 11, so $\g$ is solvable.

   ($\Rightarrow$) This is exercise \ref{lec12Ex1}.
 \end{proof}

 \begin{theorem}[Cartan's Criterion]\label{lec12Cartan}\index{Cartan!criterion|idxbf}
    The Killing form is non-degenerate if and only if $\g$ is semisimple.
 \end{theorem}
 \begin{proof}
   Say $\g$ is semisimple. Let $\a=\ker B$. Because $B$ is invariant, we get that $\a$
   is an ideal, and $B|_\a=0$. By the previous theorem (\ref{lec12T1}), we have that
   $\a$ is solvable, so $\a=0$ (by definition of semisimple).

%   Now suppose $B$ is non-degenerate on $\g$. $\g$ semisimple equivalent to $\g$
%   having no abelian ideals (the commutator of an ideal is an ideal so the
%   last step of the lower central series of a solvable ideal
%   would be an abelian ideal).

   Suppose that $\g$ is not semisimple, so $\g$ has a non-trivial solvable ideal. Then
   the last non-zero term in its derived series is some abelian ideal $\a\subseteq
   \g$.\footnote{Quick exercise: why is $\a$ an ideal?} For any $X\in \a$, the matrix
   of $ad_X$ is of the form $\matrix{0}{\ast}{0}{0}$ with respect to the (vector
   space) decomposition $\g=\a\oplus \g/\a$, and for $Y\in \g$, $ad_Y$ is of the form
   $\matrix{\ast}{\ast}{0}{\ast}$. Thus, we have that $tr(ad_X\circ ad_Y)=0$ so $X\in
   \ker B$, so $B$ is degenerate.
 \end{proof}

 \begin{theorem}
   Any semisimple Lie algebra is a direct sum of simple algebras.
 \end{theorem}
 \begin{proof}
   If $\g$ is simple, then we are done. Otherwise, let $\a\subseteq \g$ be an ideal.
   By invariance of $B$, $\a^\perp$ is an ideal. On $\a\cap \a^\perp$, $B$ is zero, so
   the intersection is a solvable ideal, so it is zero by semisimplicity of $\g$.
   Thus, we have that $\g=\a\oplus\a^\perp$. The result follows by induction on
   dimension.
 \end{proof}
 \begin{remark}
   In particular, if $\g = \bigoplus \g_i$ is semisimple, with each $\g_i$ simple, we
   have that $\D \g = \bigoplus \D\g_i$. But $\D \g_i$ is either $0$ or $\g_i$, and it
   cannot be $0$ (lest $\g_i$ be a solvable ideal). Thus $\D\g=\g$.
 \end{remark}

 \begin{theorem}[Whitehead] \label{lec12Whitehead} \index{Whitehead's Theorem|idxbf}
   If $\g$ is semisimple and $V$ is an irreducible non-trivial representation of $\g$,
   then $H^i(\g,V)=0$ for all $i\ge 0$.
 \end{theorem}
 \begin{proof}
  The proof uses the Casimir operator\index{Casimir operator|idxbfit}, $C_V \in \gl(V)$.
  Assume for the moment that $\g\subseteq \gl(V)$. Choose a basis $e_1,\dots, e_n$ in
  $\g$, with dual basis $f_1,\dots, f_n$ in $\g$ (dual with respect to $B_V$, so
  $B_V(e_i,f_j)=\delta_{ij}$).  It is necessary that $B_V$ be non-degenerate for such
  a dual basis to exist, and this is where we use that $\g$ is semisimple.
  The\footnote{We will soon see that $C_V$ is independent of the basis $e_1,\dots,
  e_n$, so the article ``the'' is apropriate.} Casimir operator is defined to be $C_V
  = \sum e_i \circ f_i \in \gl(V)$ (where $\circ$ is composition of linear operators
  on $V$). The main claim is that $[C_V,X]=0$ for any $X\in \g$. This can be checked
  directly: put $[X,f_i]=\sum a_{ij} f_j, [X,e_i] = \sum b_{ij} e_j$, then apply $B_V$
  to obtain $a_{ji} = B_V(e_i,[X,f_j]) = B_V([e_i,X],f_j) = -b_{ij}$, where the middle
  equality is by invariance of $B_V$.
   \begin{align*}
   [X, C_V] &= \sum_i Xe_if_i - e_i Xf_i + e_i Xf_i - e_if_iX \\
        &= \sum_i [X, e_i] f_i + e_i [X,f_i] \\
        &= \sum_i \sum_j b_{ij} e_j f_i +  a_{ij} e_i f_j \\
        &= \sum_i\sum_j (a_{ij}+b_{ji})e_if_j = 0.
   \end{align*}

%  $C_V$ is the image of the identity in $\hom(\g, \g)$ under the
%   composition $\hom(\g, \g) = \g\otimes\g^* \simeq_B \g\otimes \g \rightarrow \gl(V)$,
%   where the last map takes $X \otimes Y$ to $X \circ Y$.  This
%   composition preserves the bracket product, and the
%   identity commutes with everything in $\hom(\g, \g)$, so

   Suppose $V$ is irreducible, and $k$ is algebraically closed. Then the condition
   $[C_V,X]=0$ means precisely that $C_V$ is an intertwiner so by Schur's
   lemma, $C_V=\lambda \id$. We can compute
   \begin{align*}
     tr_V C_V &= \sum_{i=1}^{\dim \g} tr(e_i f_i) \\
            &= \sum B_V(e_i,f_i) = \dim \g.
   \end{align*}
   Thus, we have that $\lambda = \frac{\dim \g}{\dim
   V}$, in particular, it is non-zero.

   For any representation $\rho:\g \to \gl(V)$, we can still talk about $C_V$, but we
   define it for the image $\rho(\g)$, so $C_V = \frac{\dim \rho(\g)}{\dim V}\id$. We get
   that $[C_V,\rho(X)]=0$. The point is that if $V$ is non-trivial irreducible, we
   have that $C_V$ is non-zero.

   Now consider the complex calculating the cohomology:
   \[
    \hom(\Lambda^k \g, V) \xrightarrow{d} \hom(\Lambda^{k+1}\g, V)
   \]
   We will construct a chain homotopy\footnote{Don't worry about the term ``chain
   homotopy'' for now. It just means that $\gamma$ satisfies the equation in Exercise
   \ref{lec12Exhomotopy}. See Proposition 2.12 of \cite{Hatcher} if you're
   interested.} $\gamma:C^{k+1}\to C^k$ between the zero map on the complex and the
   map $C_V = \frac{\dim \rho(\g)}{\dim V}\id$:
   \[
    \gamma c(x_1,\dots,x_k) = \sum_i e_i c(f_i,x_1,\dots, x_k)
   \]
   \begin{exercise}\label{lec12Exhomotopy}
     Check directly that $(\gamma d+d\gamma)c = C_V c$.
     \begin{solution}
       yuck.
     \end{solution}
   \end{exercise}
   Thus $\gamma d +d\gamma = C_V = \lambda \id$ (where $\lambda=\frac{\dim
   \rho(\g)}{\dim V}$).
   Now suppose $dc=0$.  Then we have that $d\gamma (c) =\lambda c$, so
   $c=\frac{d(\gamma(c))}{\lambda}$. Thus, $\ker d/\im d = 0$, as desired.
 \end{proof}

 \begin{remark}\label{lec12RmkH1}
   What is $H^1(\g,k)$, where $k$ is the trivial representation of $\g$? Recall that
   the cochain complex is
   \[
      k\to \hom(\g,k) \xrightarrow{d} \hom(\Lambda^2 \g,k) \to \cdots.
   \]
  If $c \in \hom(\g,k)$ and $c \in \ker d$, then $dc(x,y) = c([x,y])=0$, so $c$ is 0 on
  $\D\g=\g$. So we get that $H^1(\g,k) = (\g/\D\g)^* = 0$.

  However, it is not true that $H^i(\g,k)=0$ for $i\ge 2$. Recall from Lecture 10 that
  $H^2(\g,k)$ parameterizes central extensions of $\g$ (Theorem \ref{lec10ThmCextns}).
  \anton{actually, $H^2(\g,V)$ \emph{is} always zero ... this can be used to prove
  Levi decomposition.}
 \end{remark}
 \begin{exercise}
   Compute $H^j(\sl_2,k)$ for all $j$.
   \begin{solution}
     The complex for computing cohomology is
     \begin{tabbing}
           $0\longrightarrow $
        \= $k \xrightarrow{\ d_0\ }$
        \= $\hom(\sl_2,k) \xrightarrow{\ d_1\ }$
        \= $\hom(\Lambda^2\sl_2,k) \xrightarrow{\ d_2\ }$
        \= $\hom(\Lambda^3 \sl_2,k) \longrightarrow 0$ \\
        \> \ $c\longmapsto dc(x)=-x\cdot c=0$ \\
        \> \>\qquad $f\longmapsto df(x,y)=f([x,y])$\\
        \> \> \> $\alpha\mapsto
        d\alpha(x,y,z)=\alpha([x,y],z)-\alpha([x,z],y)$\\ \>\>\>\> $+\alpha([y,z],x)$
     \end{tabbing}
     We have that $\ker d_1=k$, so $H^0(\sl_2,k)=k$. The kernel of $d_1$ is zero, as
     we computed in Remark \ref{lec12RmkH1}. Since $\hom(\sl_2,k)$ and
     $\hom(\Lambda^2\sl_2,k)$ are both three dimensional, it follows that $d_1$ is
     surjective, and since the kernel of $d_2$ must contain the image of $d_1$, we
     know that $d_2$ is the zero map. This tells us that $H^1(\sl_2,k)=0$,
     $H^2(\sl_2,k)=0$, and $H^3(\sl_2,k)=\hom(\Lambda^3\sl_2,k)\cong k$.
   \end{solution}
 \end{exercise}
 \begin{remark}\label{lec12rmkH1}
 Note that for $\g$ semisimple, we have $H^1(\g,M)=0$ for \emph{any} finite
 dimensional representation $M$ (not just irreducibles). We have already seen that
 this holds when $M$ is trivial and Whitehead's Theorem shows this when $M$ is
 non-trivial irreducible. If $M$ is not irreducible, use short exact sequences to long
 exact sequences in cohomology: if
 \[
    0\to W\to M\to V\to 0
 \]
 is an exact sequence of representations of $\g$, then
 \[
    \to H^1(\g,V)\to H^1(\g,M) \to H^1(\g,W)\to
 \]
 is exact.  The outer guys are 0 by induction on dimension,
 so the middle guy is zero.
 \end{remark}
 We need a lemma before we do Weyl's Theorem.

 \begin{lemma}\label{lec12L4}
   Say we have a short exact sequence
   \[
    0\to W\to M\to V\to 0.
   \]
   If $H^1(\g,\underbrace{\hom_k(V,W)}_{V^*\otimes W})=0$, then the sequence splits.
 \end{lemma}
 \begin{proof}
   Let $X\in \g$. Let $X_W$ represent the induced linear operator on $W$. Then we can write
   $X_M = \matrix{X_W}{c(X)}{0}{X_V}$. What is $c(X)$? It is an element of
   $\hom_k(V,W)$. So $c$ is a linear function from $\g$ to $\hom_k(V,W)$. It will be a
   1-cocycle: we have $[X_M,Y_M]=[X,Y]_M$ because these are representations, which
   gives us
   \[
    X_W c(Y) - c(Y)X_V - \bigl(Y_W c(X) - c(X)Y_V\bigr) = c([X,Y]).
   \]
   In general, $dc(X,Y) = c([X,Y]) - X c(Y) + Y c(X)$, where $X c(Y)$ is given by the
   action of $X \in \g$ on $V^*\otimes W$, which is not necessarily composition.
   In our case this action is by commutation, where $c(Y)$ is
   extended to an endomorphism of $V \oplus W$ by writing it as
   $\matrix{0}{c(Y)}{0}{0}$.  The line above says exactly that $dc=0$.

   Put $\Gamma = \matrix{1_W}{K}{0}{1_V}$.  Conjugating by $\Gamma$ gives an equivalent
   representation.  We have
   \[
   \Gamma X_M \Gamma^{-1} = \mat{X_W & c(X) + KX_V - X_W K\\ 0 & X_V}
   \]
   We'd like to kill the upper right part (to show that $X$ acts on $V$ and $W$
   separately). We have $c\in \hom(\g,V^*\otimes W)$, $K\in V^*\otimes W$.
   Since the first cohomology is zero, $dc=0$, so we can find a $K$ such that $c=dK$.
   Since $c(X) = dK(X) = X(K) = X_W K- KX_V$, the upper right part
   is indeed 0.
 \end{proof}
 \begin{theorem}[Weyl] \label{lec12Weyl} \index{Weyl's Theorem|idxbf} \index{Complete reducibility|see{Weyl's Theorem}}
   If $\g$ is semisimple and $V$ is a finite dimensional representation of $\g$, then $V$
   is semisimple\footnote{For any invariant subspace $W\subseteq V$, there is
   an invariant $W'\subseteq V$ so that $V=W\oplus W'$.} (i.e.\ completely
   reducible).
 \end{theorem}
 \begin{proof}
   The theorem follows immediately from Lemma \ref{lec12L4} and Remark \ref{lec12rmkH1}.
 \end{proof}
 Weyl proved this using the unitary trick\index{unitary trick}, which involves knowing
 about compact real forms.

 \begin{remark}
   We know from Lecture 10 that deformations of $\g$ are enumerated by $H^2(\g,\g)$.
   This means that semisimple Lie algebras do not have any deformations! This suggests
   that the variety of semisimple Lie algebras is discrete. Perhaps we can classify
   them.
 \end{remark}

 $\aut \g$ is a closed Lie subgroup of $GL(\g)$. Let $X(t)$ be a path in $\aut \g$ such
 that $X(0)=1$, and let$\der{}{t} X(t){\big |}_{t=0} = \phi$ be an element of the Lie
 algebra of $\aut \g$. We have that
 \begin{align*}
    [X(t)Y,X(t)Z] &= X(t)([Y,Z])\\
    [\phi Y,Z]+[Y,\phi Z] &= \phi [Y,Z] & (\text{differentiating at }t=0)
 \end{align*}
 so $\lie(\aut \g) = \D er(\g)$, the algebra of derivations of $\g$. (We get equality
 because any derivation can be exponentiated to an automorphism.)

 By the Jacobi identity, $ad_X$ is a derivation on $\g$. So $ad(\g)\subseteq \D er(\g)$.
 \begin{exercise}
   Check that $ad(\g)$ is an ideal.
   \begin{solution}
     Let $D\in \D er(\g)$ and let $X,Y\in \g$. Then
     \begin{align*}
       [D,ad_X]_{\D er(\g)}(Y) &= D([X,Y])-[X,D(Y)]\\
            &= [D(X),Y]+[X,D(Y)]-[X,D(Y)]\\
            &= ad_{D(X)}(Y).
     \end{align*}
   \end{solution}
 \end{exercise}

 We have seen in lecture 9 (page \pageref{lec09H1(g,g)}) that $\D er(\g)/ad(\g) \simeq
 H^1(\g,\g)$. The conclusion is that $\D er(\g) = ad(\g)\cong \g$---that is, all
 derivations on a semisimple Lie algebra are inner.

 Now we know that $G$ and $\aut \g$ have the same Lie algebras. If $f\in \aut \g$ is
 central (i.e.\ commutes with all automorphisms), then we have
 \begin{align*}
   \bigl(\exp(t\cdot ad_x)\bigr) y &= f\circ \bigl(\exp (t\cdot ad_x)\bigr)\circ
   f^{-1}y & (f \text{ is central})\\
   &= \exp(t\cdot ad_{f(x)}) y & (f \text{ an automorphism of }\g)
 \end{align*}
 Comparing the $t^1$ coefficients, we see that $ad_{f(x)}=ad_x$ for all $x$. Since $\g$
 has no center, $f(x)=x$ for all $x$. Therefore, $\aut \g$ has trivial center.

 It follows that the connected component of the identity of $\aut \g$ is $Ad G$.
}{   % Jonah Blasiak, jblasiak@math
  \stepcounter{lecture}
 \setcounter{lecture}{13}
 \sektion{Lecture 13 - The root system of a semisimple Lie algebra}

 The goal for today is to start with a semisimple Lie algebra over a field $k$
 (assumed algebraically closed and characteristic zero), and get a root system.

 Recall Jordan decomposition. For $\g\subseteq \gl(V)$,
 any $x\in \g$ can be written (uniquely) as $x=x_s+x_n$, where $x_s$ is semisimple and
 $x_n$ is nilpotent, both of which are polynomials in $x$. In general, $x_s$ and $x_n$
 are in $\gl(V)$, but not necessarily in $\g$.
 \begin{proposition}
   If $\g\subseteq \gl(V)$ is semisimple, then $x_s, x_n \in \g$.
 \end{proposition}
 \begin{proof}
   Notice that $\g$ acts on $\gl(V)$ via commutator, and $\g$ is an invariant
   subspace. By complete reducibility (Theorem \ref{lec12Weyl}), we can write $\gl(V)
   = \g\oplus \m$ where $\m$ is $\g$-invariant, so
   \[
     [\g,\g]\subseteq \g \qquad \text{and} \qquad [\g,\m]\subseteq \m.
   \]
   We have that $ad_{x_s}$ and $ad_{x_n}$ are polynomials in $ad_x$ (by Lemma
   \ref{lec12L3}), so
   \[
    [x_n,\g]\subseteq \g\ ,\ [x_s,\g]\subseteq \g \qquad \text{and} \qquad
    [x_n,\m]\subseteq \m\ ,\ [x_s,\m]\subseteq \m.
   \]
   Take $x_n = a+b\in \g\oplus \m$, where $a\in \g$ and $b\in \m$. We would like to
   show that $b=0$, for then we would have that $x_n\in \g$, from which it would
   follow that $x_s\in \g$.

%    We have $[b,\g]=0$
%   because $[x_n,\g]=\underbrace{[a,\g]}_{\in \g}+\underbrace{[b,\g]}_{\in
%   \m}\subseteq \g$. It is enough to show that $x_n\in \g$, so it suffices to show
%   that $b=0$.

   Decompose $V= V_1\oplus \cdots \oplus V_n$ with the $V_i$ irreducible. Since $x_n$
   is a polynomial in $x$, we have that $x_n(V_i)\subseteq V_i$, and $a(V_i)\subseteq
   V_i$ since $a\in \g$, so $b(V_i)\subseteq V_i$. Moreover, we have that
   \[
     [x_n,\g]=\underbrace{[a,\g]}_{\in \g}+\underbrace{[b,\g]}_{\in \m}\subseteq \g,
   \]
   so $[b,\g]=0$ (i.e.\ $b$ is an intertwiner). By Schur's lemma, $b$ must be a scalar
   operator on $V_i$ (i.e. $b|_{V_i}=\lambda_i\id$). We have $tr_{V_i}(x_n)=0$ because
   $x_n$ is nilpotent. Also $tr_{V_i}(a)= 0$ because $\g$ is semisimple implies
   $\D\g=\g$, so $a=\sum [x_k,y_k]$, and the traces of commutators are $0$. Thus,
   $tr_{V_i}(b)=0$, so $\lambda_i=0$ and $b=0$. Now $x_n=a\in \g$, and so $x_s\in \g$.
 \end{proof}
 Since the image of a semisimple Lie algebra is semisimple, the proposition tells us
 that for any representation $\rho:\g\to \gl(V)$, the semisimple and nilpotent parts
 of $\rho(x)$ are in the image of $\g$. In fact, the following corollary shows that
 there is an \emph{absolute} Jordan decomposition\index{Jordan decomposition!absolute}
 $x=x_s+x_n$ within $\g$.
 \begin{corollary}\label{lec13CorJordan}
   If $\g$ is semisimple, and $x\in \g$, then there are $x_s,x_n\in \g$ such that for
   any representation $\rho:\g\to \gl(V)$, we have $\rho(x_s)=\rho(x)_s$ and
   $\rho(x_n)=\rho(x)_n$.
 \end{corollary}
 \begin{proof}
   Consider the (faithful) representation $ad:\g\to \gl(\g)$. By the proposition,
   there are some $x_s,x_n\in \g$ such that $(ad_x)_s=ad_{x_s}$ and
   $(ad_x)_n=ad_{x_n}$. Since $ad$ is faithful, $ad_x=ad_{x_n}+ad_{x_s}$ and
   $ad_{[x_n,x_s]}=[ad_{x_n},ad_{x_s}]=0$ tell us that $x=x_n+x_s$ and $[x_s,x_n]=0$.
   These are our candidates for the absolute Jordan decomposition.

   Given any surjective Lie algebra homomorphism $\sigma:\g\to \g'$, we have that
   $ad_{\sigma(y)}(\sigma(z))=\sigma(ad_y(z))$, from which it follows that
   $ad_{\sigma(x_s)}$ is diagonalizable and $ad_{\sigma(x_n)}$ is nilpotent (note that
   we've used surjectivity of $\sigma$). Thus, $\sigma(x)_n = \sigma(x_n)$ and
   $\sigma(x)_s=\sigma(x_s)$. That is, our candidates are preserved by surjective
   homomorphisms.

   Now given any representation $\rho:\g\to \gl(V)$, the previous paragraph allows us
   to replace $\g$ by its image, so we may assume $\rho$ is faithful. By the
   proposition, there are some $y,z\in \g$ such that $\rho(x)_s=\rho(y),
   \rho(x)_n=\rho(z)$. Then $[\rho(y),-]_{\gl(\rho(\g))}$ is a diagonalizable operator
   on $\gl\bigl(\rho(\g)\bigr) \cong \gl(\g)$, and $[\rho(z),-]_{\gl(\rho(\g))}$ is
   nilpotent. Uniqueness of the Jordan decomposition implies that $\rho(y)=\rho(x_s)$
   and $\rho(z)=\rho(x_n)$. Since $\rho$ is faithful, it follows that $y=x_s$ and
   $z=x_n$.\index{Jordan decomposition!absolute}
 \end{proof}
 \begin{definition}
   We say $x\in \g$ is \emph{semisimple}\index{semisimple!element} if $ad_x$ is diagonalizable. We say $x$ is
   \emph{nilpotent}\index{nilpotent!element} if $ad_x$ is nilpotent.
 \end{definition}
 Given any representation $\rho:\g\to \gl(V)$ with $\g$ semisimple, the corollary
 tells us that if $x$ is semisimple, then $\rho(x)$ is diagonalizable, and if $x$ is
 nilpotent, then $\rho(x)$ is nilpotent. If $\rho$ is faithful, then $x$ is semisimple
 (resp.\ nilpotent) if and only if $\rho(x)$ is semisimple (resp.\ nilpotent).
 \begin{definition}
 We denote the set of all semisimple elements in $\g$ by $\g_{ss}$. We call an $x\in
 \g_{ss}$ \emph{regular}\index{regular element} if $\dim(\ker ad_x)$ is minimal (i.e.\
 the dimension of the centralizer is minimal).
 \end{definition}
 \begin{example}\index{sl(n)@$\sl(n)$}
   Let $\g=\sl_n$. Semisimple elements of $\sl_n$ are exactly the diagonalizable
   matrices, and nilpotent elements are exactly the nilpotent matrices. If $x\in \g$
   is diagonalizable, then the centralizer is minimal exactly when all the eigenvalues
   are distinct. So the regular elements are the diagonalizable matrices with distinct
   eigenvalues.
 \end{example}


 Let $h\in \g_{ss}$ be regular. We have that $ad_h$ is diagonalizable, so we can write
 $\g = \bigoplus_{\mu\in k} \g_\mu$, where $\g_\mu = \{x\in \g| [h,x]=\mu x\}$ are
 eigenspaces of $ad_h$. We know that $\g_0\neq 0$ because it contains $h$. There are
 some other properties:
 \begin{enumerate}
 \item \label{lec13n1} $[\g_\mu,\g_\nu]\subseteq \g_{\mu+\nu}$.%
 \item \label{lec13n2} $\g_0\subseteq\g$ is a subalgebra.%
 \item \label{lec13n3} $B(\g_\mu,\g_\nu) = 0$ if $\mu\neq -\nu$ (here, $B$ is the Killing form, as usual).%
 \item \label{lec13n4} $B|_{\g_\mu \oplus \g_{-\mu}}$ is non-degenerate, and
    $B|_{\g_0}$ is non-degenerate.%
 \end{enumerate}
 \begin{proof}
 Property \ref{lec13n1} follows from the Jacobi identity: if $x\in \g_\mu$ and $y\in
\g_\nu$, then
 \[
    [h,[x,y]] = [[h,x],y] + [x,[h,y]] = \mu[x,y] + \nu[x,y],
 \]
 so $[x,y]\in \g_{\mu+\nu}$. Property \ref{lec13n2} follows immediately from
 \ref{lec13n1}. Property \ref{lec13n3} follows from \ref{lec13n1} because $ad_x\circ
 ad_y:\g_\gamma\to \g_{\gamma+\mu+\nu}$, so $B(x,y)=tr(ad_x\circ ad_y)=0$ whenever
 $\mu+\nu\neq 0$. Finally, Cartan's criterion says that $B$ must be non-degenerate, so
 property \ref{lec13n4} follows from \ref{lec13n3}.
 \end{proof}
 \begin{proposition} \label{lec13P:g0abelian}
   In the situation above ($\g$ is semisimple and $h\in \g_{ss}$ is regular), $\g_0$
   is abelian.
 \end{proposition}
 \begin{proof}
   Take $x\in \g_0$, and write $x=x_s+x_n$. Since $ad_{x_n}$ is a polynomial of
   $ad_x$, we have $[x_n,h]=0$, so $x_n\in \g_0$, from which we get $x_s\in\g_0$.
   Since $[x_s,h]=0$, we know that $ad_{x_s}$ and $ad_h$ are simultaneously
   diagonalizable (recall that $ad_{x_s}$ is diagonalizable). Thus, for generic $t\in
   k$, we have that $\ker ad_{h+tx_s} \subseteq \ker ad_h$. Since $h$ is regular,
   $\ker ad_{x_s} = \ker ad_h=\g_0$. So $[x_s,\g_0]=0$, which implies that $\g_0$ is
   nilpotent by Corollary \ref{lec11Engelcor2} to Engel's Theorem\index{Engel's
   Theorem}. Now we have that $ad_x:\g_0\to \g_0$ is nilpotent, and we want $ad_x$ to
   be the zero map. Notice that $B(\g_0,\D\g_0)=0$ since $\g_0$ is nilpotent, but
   $B|_{\g_0}$ is non-degenerate by property \ref{lec13n4} above, so $\D\g_0=0$, so
   $\g_0$ is abelian.
 \end{proof}
% So it always behaves like in the case $\g=\sl_n$.

 \begin{definition}
    We call $\h:=\g_0$ the \emph{Cartan subalgebra}\index{Cartan!subalgebra|idxbf} of $\g$
    (associated to $h$).
 \end{definition}
 In Theorem \ref{lec14T:CSA}, we will show that any two Cartan subalgebras of a
 semisimple Lie algebra $\g$ are related by an automorphism of $\g$, but for now we
 just fix one. See \cite[\S 15]{Humphreys:LART} for a more general definition
 of Cartan subalgebras.

 \begin{exercise}\label{lec13Ex:hss}
   Show that if $\g$ is semisimple, $\h$ consists of semisimple elements.
   \begin{solution}
     Let $x\in \h$, so $[x,h]=0$. Since $ad_{x_n}$ is a polynomial in $ad_x$, we get
     that $[x_n,h]=0$, so $x_n\in \h$. Thus, it is enough to show that any nilpotent
     element in $\h$ is zero (then $x=x_s+x_n=x_s$ is semisimple). We do this using
     property \ref{lec13n4}, that the Killing form is non-degenerate on $\h$. If $y\in
     \h$, then $B(x_n,y)=tr(ad_{x_n}\circ ad_y)$. By Proposition
     \ref{lec13P:g0abelian}, $\h$ is abelian, so $[x_n,y]=0$, so $ad_{x_n}$ commutes
     with $ad_y$. Thus, we can simultaneously upper triangularize $ad_{x_n}$ and
     $ad_y$ by Engel's theorem\index{Engel's Theorem|idxit}. Since $ad_{x_n}$ is
     nilpotent, it is \emph{strictly} upper triangular so $tr(ad_{x_n}\circ ad_y)=0$.
     So $x_n=0$ by non-degeneracy of $B$.
   \end{solution}
 \end{exercise}

 All elements of $\h$ are simultaneously diagonalizable because they are all
 diagonalizable (by the above exercise) and they all commute (by the above
 proposition). For $\alpha\in\h^*\smallsetminus \{0\}$ consider
 \[
    \g_\alpha = \{x\in \g| [h,x]=\alpha(h)x \text{ for all } h\in \h\}
 \]
 If this $\g_\alpha$ is non-trivial, it is called a \emph{root space}\index{root
 space} and the $\alpha$ is called a \emph{root}\index{root}. The \emph{root
 decomposition}\index{root decomposition|idxbf} (or \emph{Cartan
 decomposition}\index{Cartan!decomposition|idxbf}) of $\g$ is $\g = \h \oplus
 \bigoplus_{\alpha\in \h^*\smallsetminus \{0\}} \g_\alpha$.
 \begin{example} \index{sl(2)@$\sl(2)$}
   $\g=\sl(2)$. Take $H=\matrix{1}{0}{0}{-1}$, a regular element. The Cartan
   subalgebra is $\h=k\cdot H$, a one dimensional subspace. We have $\g_2 =
   \bigl\{\matrix{0}{t}{0}{0}\bigr\}$ and $\g_{-2} =
   \bigl\{\matrix{0}{0}{t}{0}\bigr\}$, and $\g=\h\oplus \g_2\oplus \g_{-2}$.
%   For the classical $X=\matrix{0}{1}{0}{0},
%   Y=\matrix{0}{0}{1}{0}$ we get $[H,X]=2X$, $[H,Y]=-2Y$.
 \end{example}
 \begin{example}\label{lec13Eg:sl3}\index{sl(3)@$\sl(3)$}
   $\g=\sl(3)$. Take
   \[
    \h = \left\{ \mat{x_1 & & \llap{\smash{\raisebox{-1ex}{\mbox{\LARGE $0$}}}}\\
        & x_2\\ \smash{\mbox{\LARGE $0$}} & & x_3}\bigg| x_1+x_2+x_3=0\right\}.
   \]
   Let $E_{ij}$ be the elementary matrices. We have that
   $[diag(x_1,x_2,x_3),E_{ij}]=(x_i-x_j)E_{ij}$. If we take the basis
   $\varepsilon_i(x_1,x_2,x_3)=x_i$ for $\h^*$, then we have roots $\varepsilon_i -
   \varepsilon_j$. They can be arranged in a diagram:
  \[\begin{xy}
  (0,0)="c",
  \ar@{.>} a(30)   ="1" *+!LD{\varepsilon_1},
  \ar@{.>} a(150)  ="2" *+!RD{\varepsilon_2},
  \ar@{.>} a(-90)  ="3" *+!U{\varepsilon_3},
  \ar@{->} "1"-"2" *+!L{\varepsilon_1 - \varepsilon_2},
  \ar@{->} "1"-"3" *+!DL{\varepsilon_1 - \varepsilon_3},
  \ar@{->} "2"-"1" *+!R{\varepsilon_2 - \varepsilon_1},
  \ar@{->} "2"-"3" *+!DR{\varepsilon_2 - \varepsilon_3},
  \ar@{->} "3"-"1" *+!UR{\varepsilon_3 - \varepsilon_1},
  \ar@{->} "3"-"2" *+!UL{\varepsilon_3 - \varepsilon_2},
 \end{xy}\]
   This generalizes to $\sl(n)$.
 \end{example}
 The \emph{rank}\index{rank} of $\g$ is defined to be $\dim \h$. In particular, the
 rank of $\sl(n)$ is going to be $n-1$.

 Basic properties of the root decomposition are:
 \begin{enumerate}
 \item $[\g_\alpha,\g_\beta] \subseteq \g_{\alpha+\beta}$.
 \item $B(\g_\alpha,\g_\beta) = 0$ if $\alpha +\beta \neq 0$.
 \item \label{lec13N3} $B|_{\g_\alpha\oplus \g_{-\alpha}}$ is non-degenerate.
 \item $B|_\h$ is non-degenerate
 \end{enumerate}
 Note that \ref{lec13N3} implies that $\alpha$ is a root if and only if $-\alpha$ is a
 root.
 \begin{exercise}
   Check these properties.
   \begin{solution}
     Since $\Delta$ is a finite set in $\h^*$, we can find some $h\in \h$ so that
     $\alpha(h)\neq \beta(h)$ for distinct roots $\alpha$ and $\beta$. Then this $h$
     is a regular element which gives the right Cartan subalgebra, and the desired
     properties follow from the properties on page \pageref{lec13n1}.
   \end{solution}
 \end{exercise}
 Now let's try to say as much as we can about this root decomposition. Define
 $\h_\alpha\subseteq \h$ as $[\g_\alpha,\g_{-\alpha}]$. Take $x\in \g_\alpha$ and $y\in
 \g_{-\alpha}$ and $h\in \h$. Then compute
 \begin{align*}
   B(\overbrace{[x,y]}^{\in \h_\alpha},h) &= B(x,\overbrace{[y,h]}^{\in \g_\alpha}) & \text{($B$ is invariant)}\\
    &= \alpha(h) B(x,y) & \text{(since $y\in \g_\alpha$)}
 \end{align*}
 It follows that $\h_\alpha^\perp=\ker(\alpha)$, which is of codimension one. Thus,
 $\h_\alpha$ is one dimensional.

 \begin{proposition}
   If $\g$ is semisimple and $\alpha$ is a root, then $\alpha(\h_\alpha) \neq 0$.
 \end{proposition}
 \begin{proof}
   Assume that $\alpha(\h_\alpha)=0$. Then pick $x\in \g_\alpha$, $y\in \g_{-\alpha}$
   such that $[x,y]=h\neq 0$. If $\alpha(h)=0$, then we have that
   $[h,x]=\alpha(h)x=0, [h,y]=0$. Thus $\langle x,y,h \rangle$ is a copy of the
   Heisenberg algebra\index{Heisenberg algebra}, which is solvable (in fact,
   nilpotent). By Lie's Theorem, $ad_\g(x)$ and $ad_\g(y)$ are simultaneously upper
   triangularizable, so $ad_\g(h)=[ad_\g(x),ad_\g(y)]$ is nilpotent. This is a
   contradiction because $h$ is an element of the Cartan subalgebra, so it is
   semisimple.
 \end{proof}

 For each root $\alpha$, we will take $H_\alpha\in \h_\alpha$ such that
 $\alpha(H_\alpha)=2$ (we can always scale $H_\alpha$ to get this). We can choose
 $X_\alpha\in \g_\alpha$ and $Y_\alpha\in \g_{-\alpha}$ such that
 $[X_\alpha,Y_\alpha]=H_\alpha$. We have that
 $[H_\alpha,X_\alpha]=\alpha(H_\alpha)X_\alpha=2X_\alpha$ and $[H_\alpha,Y_\alpha] =
 -2Y_\alpha$. That means we have a little copy of $\sl(2)\cong\langle
 H_\alpha,X_\alpha,Y_\alpha \rangle$.\index{sl(2)@$\sl(2)$} Note that this makes $\g$ a
 representation of $\sl_2$ via $\sl_2\hookrightarrow \g\stackrel{ad}{\hookrightarrow}
 \gl(\g)$.

 We normalize $\alpha(h_\alpha)$ to 2 so that we get the standard basis of $\sl_2$.
 This way, the representations behave well (namely, that various
 coefficients are integers). Next we study these representations.

 \subsektion{Irreducible finite dimensional representations of
 \texorpdfstring{$\sl(2)$}{sl(2)}} Let $H,X,Y$ be the standard basis of $\sl(2)$, and
 let $V$ be an irreducible representation. By Corollary \ref{lec13CorJordan}, the
 action of $H$ on $V$ is diagonalizable and the actions of $X$ and $Y$ on $V$ are
 nilpotent. By Lie's Theorem (applied to the solvable subalgebra generated by $H$ and
 $X$), $X$ and $H$ have a common eigenvector $v$: $Hv=\lambda v$ and $Xv=0$ (since $X$
 is nilpotent, its only eigenvalues are zero). Verify by induction that
 \begin{align}
   HY^r v &= YHY^{r-1}v + [H,Y]Y^{r-1}v = \bigl(\lambda-2(r-1)\bigr)Y^r v + 2Y^r v \notag\\
          &= (\lambda-2r)Y^rv \\
   XY^r v &= YXY^{r-1}v + [X,Y]Y^{r-1}v \notag\\
          &= (r-1)\bigl(\lambda-(r-1)+1\bigr)Y^{r-1}v + \bigl(\lambda-2(r-1) \bigr)Y^{r-1}v\notag\\
          &= r(\lambda-r+1)Y^{r-1}v \label{lec13dag}
 \end{align}
 Thus, the span of $v,Yv,Y^2v,\dots$ is a subrepresentation, so it must be all of $V$
 (since $V$ is irreducible). Since $Y$ is nilpotent, there is a minimal $n$ such that
 $Y^nv=0$. From (\ref{lec13dag}), we get that $\lambda=n-1$ is a non-negative integer.
 Since $v,Yv,\dots, Y^{n-1}v$ have distinct eigenvalues (under $H$), they are linearly
 independent.

 Conclusion: For every non-negative integer $n$, there is exactly one irreducible
 representation of $\sl_2$ of dimension $n+1$, and the $H$-eigenvalues on that
 representation are $n, n-2, n-4, \dots, 2-n,-n$.

% You can check that if $\lambda =n$, you go down to $-n$. In \cite{FulHar}, this
% representation is called $\Gamma_n$, and we have that $\dim \Gamma_n=n+1$.

 \begin{remark}
   As a consequence, we have that in a general root decomposition, $\g=\h\oplus
   \bigoplus_{\alpha\in \Delta} g_\alpha$, each root space is one dimensional. Assume
   that $\dim\g_{-\alpha} >1$. Consider an $\sl(2)$ in $\g$, generated by $\langle
   X_\alpha, Y_\alpha, H_\alpha=[X_\alpha,Y_\alpha]\rangle$ where
   $Y_\alpha\in\g_{-\alpha}$ and $X_\alpha\in\g_\alpha$. Then there is some $Z\in
   \g_{-\alpha}$ such that $[X_\alpha, Z]=0$ (since $\h_\alpha$ is one dimensional).
   Hence, $Z$ is a highest vector with respect to the adjoint action of this $\sl(2)$.
   But we have that $ad_{H_\alpha}(Z) = -2Z$, and the eigenvalue of a highest vector
   must be positive! This shows that the choice of $X_\alpha$ and $Y_\alpha$ is really
   unique.
 \end{remark}
 \begin{definition}
   Thinking of $\g$ as a representation of $\sl_2=\langle
   X_\alpha,Y_\alpha,H_\alpha\rangle$, the irreducible subrepresentation containing
   $\g_\beta$ is called the \emph{$\alpha$-string through
   $\beta$}\index{alpha-string@$\alpha$-string|idxbf}.
 \end{definition}

 Let $\Delta$ denote the set of roots. Then $\Delta$ is a finite subset of $\h^*$ with the
 following properties:
 \begin{enumerate}
 \item \label{lec13p1} $\Delta$ spans $\h^*$.
 \item \label{lec13p2} If $\alpha, \beta\in \Delta$, then $\beta(H_\alpha)\in \ZZ$, and
 $\beta-\bigl(\beta(H_\alpha)\bigr)\alpha \in \Delta$.
 \item \label{lec13p3} If $\alpha, c\alpha\in \Delta$, then $c=\pm 1$.
 \end{enumerate}
 \begin{exercise}
    Prove these properties.
   \begin{solution}
     If $\Delta$ does not span $\h^*$, then there is some non-zero $h\in \h$ such that
     $\alpha(h)=0$ for all $\alpha\in \Delta$. This means that all of the eigenvalues
     of $ad_h$ are zero. Since $h$ is semisimple, $ad_h=0$. And since $ad$ is
     faithful, we get $h=0$, proving property \ref{lec13p1}.

     To prove \ref{lec13p2}, consider the $\alpha$-string through $\beta$. It must be
     of the form $\g_{\beta+n\alpha}\oplus \g_{\beta+(n-1)\alpha}\oplus \cdots\oplus
     \g_{\beta+m\alpha}$ for some integers $n\ge 0\ge m$. From the characterization of
     irreducible finite dimensional representations of $\sl_2$, we know that each
     eigenvalue of $H_\alpha$ is an integer, so $\beta(H_\alpha)=r\in \ZZ$ (since
     $[H_\alpha,X_\beta]=\beta(H_\alpha)X_\beta$). We also know that the eigenvalues
     of $H_\alpha$ are symmetric around zero, so we must have
     $-r=(\beta+s\alpha)(H_\alpha)$ for some $s$ for which $\g_{\beta+s\alpha}$ is in
     the $\alpha$-string through $\beta$. Then we get
     $\beta(H_\alpha)+s\alpha(H_\alpha)=\beta(\alpha)+2s=-r=-\beta(\alpha)$, from which
     we know that $s=-\beta(H_\alpha)$. Thus, $\g_{\beta-\beta(H_\alpha)\alpha}\neq
     0$, so $\beta-\bigl(\beta(H_\alpha)\bigr)\alpha$ is a root.

     Finally, we prove \ref{lec13p3}. If $\alpha$ and $\beta=c\alpha$ are roots, then
     by property \ref{lec13p2}, we know that $\alpha(H_\beta)=2/c$ and
     $\beta(H_\alpha)=2c$ are integers (note that $H_\beta=H_\alpha/c$). It follows
     that $c=\pm \half, \pm 1$, or $\pm 2$. Therefore, it is enough to show that
     $\alpha$ and $2\alpha$ cannot both be roots. To see this, consider the
     $\alpha$-string through $2\alpha$. We have that $[H_\alpha,
     X_{2\alpha}]=2\alpha(H_\alpha) X_{2\alpha}=4X_{2\alpha}$, so the $\alpha$-string
     must have a non-zero element $[Y_\alpha,X_{2\alpha}]\in \g_\alpha$, which is
     spanned by $X_\alpha$. But then we would have that $X_{2\alpha}$ is a multiple of
     $[X_\alpha,X_\alpha]=0$, which is a contradiction.
   \end{solution}
 \end{exercise}
}{   % Hannes Thiel hannes.thiel@gmx.de, thiel@math
  \stepcounter{lecture}
 \setcounter{lecture}{14}
 \sektion{Lecture 14 - More on Root Systems}


% Vera Serganova's office hours are 5-6:30 TuTh, by the way.

% Today we will keep talking about root systems.

 Assume $\g$ is semisimple. Last time, we started with a regular element $h\in
 \g_{ss}$ and constructed the decomposition $\g = \h\oplus \bigoplus_{\alpha \in
 \Delta} \g_\alpha$, where $\Delta\subseteq \h^*$ is the set of roots. We proved that
 each $\g_\alpha$ is one dimensional (we do not call 0 a root). For each root, we
 associated an $\sl(2)$ subalgebra. Given $X_\alpha \in \g_\alpha, Y_\alpha\in
 \g_{-\alpha}$, we set $H_\alpha = [X_\alpha,Y_\alpha]$, and normalized so that
 $\alpha(H_\alpha)=2$.

 Furthermore, we showed that
 \begin{enumerate}
 \item $\Delta$ spans $\h^*$,
 \item $\alpha (H_\beta)\in \ZZ$, with $\alpha
 - \alpha(H_\beta)\beta \in \Delta$ for all $\alpha, \beta \in \Delta$, and
 \item if $\alpha, k\alpha \in \Delta$, then $k=\pm 1$.
 \end{enumerate}

 How unique is this decomposition? We started with some choice of a regular semisimple
 element. Maybe a different one would have produced a different Cartan subalgebra.

 \begin{theorem}\label{lec14T:CSA}
   Let $\h$ and $\h'$ be two Cartan subalgebras\index{Cartan!subalgebra} of a
   semisimple Lie algebra $\g$ (over an algebraically closed field of characteristic
   zero). Then there is some $\phi\in Ad\, G$ such that $\phi(\h)=\h'$. Here, $G$ is
   the Lie group associated to $\g$.
 \end{theorem}
 \begin{proof}
   Consider the map
   \begin{align*}
     \Phi:\h^{\text{reg}}\times \g_{\alpha_1}\times \cdots \times \g_{\alpha_N} &\to \g \\
     (h,x_1,\dots, x_N)\qquad & \mapsto \exp(ad_{x_1})\cdots \exp(ad_{x_N}) h.
   \end{align*}
   Note that $ad_{x_i}h$ is linear in both $x_i$ and $h$, and each $ad_{\g_{\alpha_i}}$
   is nilpotent, so the power series for $\exp$ is finite. It follows that $\Phi$ is a
   polynomial function. Since $\der{}{t}\exp(ad_{tx_i})h\big|_{t=0}=\alpha_i(h)x_i\in
   \g_{\alpha_i}$, the differential if $\Phi$ at $(h,0,\dots,0)$ is
   \[
   D\Phi|_{(h,0,\dots, 0)} =
     \left(\begin{array}{c|ccc}
     \id_\h & & 0 \\ \hline
      & \alpha_1(h) & & \raisebox{-1ex}{\smash{\llap{\LARGE $0$}}}\\
     0& & \ddots\\
      & \smash{\mbox{\LARGE $0$}} & & \alpha_N(h)
     \end{array}\right)
   \]
   with respect to the decomposition $\g=\h\oplus \g_{\alpha_1}\oplus\cdots\oplus
   \g_{\alpha_N}$. $D\Phi|_{(h,0,\dots, 0)}$ is non-degenerate because $h\in
   \h^{\text{reg}}$ implies that $\alpha_i(h)\neq 0$. So $\im \Phi$ contains a Zariski
   open set\index{Zariski open set}.\footnote{This is a theorem from algebraic geometry.
   \cite{FulHar} claims in \S D.3 that this result is in \cite{Hartshorne}, but I
   cannot find it.} Let $\Phi'$ be the analogous map for $\h'$. Since Zariski open sets
   are dense, we have that $\im \Phi\cap \im \Phi' \neq \varnothing$. So there are
   $\psi,\psi'\in Ad\, G$, and $h\in \h, h'\in \h'$ such that $\psi(h)=\psi'(h')$. Thus,
   we have that $\h=\psi^{-1}\psi'(\h)$.
 \end{proof}

 \subsektion{Abstract Root systems}
 We'd like to forget that any of this came from a Lie algebra. Let's just study an
 abstract set of vectors in $\h^*$ satisfying some properties. We know that $B$ is
 non-degenerate on $\h$, so there is an induced isomorphism $s:\h\to \h^*$. By
 definition, $\langle s(h),h'\rangle = B(h,h')$.

 Let's calculate
 \begin{align*}
   \langle sH_\beta, H_\alpha\rangle &= B(H_\beta,H_\alpha) = B(H_\alpha,H_\beta) & \text{($B$ symmetric)}\\
    &= B(H_\alpha, [X_\beta, Y_\beta]) = B([H_\alpha, X_\beta], Y_\beta) &\text{($B$ invariant)} \\
    &=  B(X_\beta,Y_\beta) \beta(H_\alpha) \\
    &= \frac{1}{2} B([H_\beta,X_\beta],Y_\beta) \beta(H_\alpha) & (2X_\beta = [H_\beta,X_\beta])\\
    &=  \frac{1}{2} B(H_\beta, H_\beta)\beta(H_\alpha) & \text{($B$ invariant)}
 \end{align*}
 Thus, we have that $s(H_\beta) = \frac{B(H_\beta,H_\beta)}{2} \beta$. Also, compute
 \begin{align}
   (\alpha, \beta) &:= \langle \alpha, s^{-1}\beta\rangle\notag\\
   &= \alpha\left( \frac{2H_\beta}{B(H_\beta,H_\beta)}\right)\notag\\
   &= \frac{2\alpha(H_\beta)}{B(H_\beta,H_\beta)}. \label{lec14ddag}
 \end{align}
 In particular, letting $\alpha=\beta$, we get $s(H_\beta) =
 \frac{2\beta}{(\beta,\beta)}$. This is sometimes called the
 \emph{coroot}\index{coroot} of $\beta$, and denoted $\check\beta$. We may use
 (\ref{lec14ddag}) to rewrite fact \ref{lec13p2} from last time as:
 \[
 \text{For $\alpha,\beta\in \Delta$, $\frac{2(\alpha,\beta)}{(\beta,\beta)}\in \ZZ$, and
 $\alpha - \frac{2(\alpha,\beta)}{(\beta,\beta)}\beta\in \Delta$.}\tag{$2'$}
 \]
 Now you can define $r_\beta:\h^*\to \h^*$ by $r_\beta(x) =
 x-\frac{2(x,\beta)}{(\beta,\beta)}\beta$. This is the reflection through the
 hyperplane orthogonal to $\beta$ in $\h^*$. The group generated by the $r_\beta$ for
 $\beta\in \Delta$ is a Coxeter group\index{Coxeter group}. If we want to study
 Coxeter groups, we'd better classify root systems.\footnote{We will not talk about
 Coxeter groups in depth in this class.}

 We want to be working in Euclidean space, but we are now in $\h^*$. Let $\h_r$ be the
 real span\footnote{Assuming we are working over $\CC$. Otherwise, we can use the
 $\QQ$ span.} of the $H_\alpha$'s. We claim that $B$ is positive definite on $\h_r$.
 To see this, note that $X_\alpha, Y_\alpha, H_\alpha$ make a little $\sl(2)$ in $\g$,
 and that $\g$ is therefore a representation of $\sl(2)$ via the adjoint actions
 $ad_{X_\alpha},ad_{Y_\alpha},ad_{H_\alpha}$. But we know that in any representation
 of $\sl(2)$, the eigenvalues of $H_\alpha$ must be integers. so
 $ad_{H_\alpha}\circ ad_{H_\alpha}$ has only positive eigenvalues, so
 $B(H_\alpha,H_\alpha)=tr(ad_{H_\alpha}\circ ad_{H_\alpha})>0$.

 Thus, we may think of our root systems in Euclidean space, where the inner
 product on $\h^*$ is given by $(\mu,\nu)\stackrel{def}{=} B(s^{-1}(\mu),s^{-1}(\nu))
 = \langle \mu,s^{-1}\nu\rangle$.

 \begin{definition}
   An \emph{abstract reduced root system}\index{root system!abstract} is a finite set
   $\Delta\subseteq \RR^n\smallsetminus\{0\}$ which satisfies
   \begin{itemize}
   \item[\hypertarget{RS1}{(RS1)}] $\Delta$ spans $\RR^n$,%
   \item[\hypertarget{RS2}{(RS2)}] if $\alpha,\beta\in \Delta$, then
    $\frac{2(\alpha,\beta)}{(\beta,\beta)}\in \ZZ$, and $r_\beta(\Delta)=\Delta$%
   \item[](i.e.\ $\alpha,\beta\in \Delta \Rightarrow r_\beta(\alpha)\in \Delta$, with
   $\alpha-r_\beta(\alpha) \in \ZZ \beta$ ), and %
   \item[\hypertarget{RS3}{(RS3)}] if $\alpha,k\alpha\in \Delta$, then $k=\pm 1$ (this
        is the ``reduced'' part).%
   \end{itemize}
   The number $n$ is called the \emph{rank}\index{rank} of $\Delta$.
 \end{definition}
 Notice that given root systems $\Delta_1\subset \RR^n$, and $\Delta_2\subset \RR^m$, we
 get that $\Delta_1\coprod \Delta_2 \subset \RR^n\oplus \RR^m$ is a root system.
 \begin{definition}
   A root system is \emph{irreducible}\index{root system!irreducible} if it cannot be
   decomposed into the union of two root systems of smaller rank.
 \end{definition}
 \begin{exercise}
   Let $\g$ be a semisimple Lie algebra and let $\Delta$ be its root system. Show that
   $\Delta$ is irreducible if and only if $\g$ is simple.
   \begin{solution}
     If $\Delta$ is reducible, with $\Delta=\Delta_1\cup \Delta_2$, then set $\h_i^*$ to
     be the span of $\Delta_i$, and set $\g_i=\h_i\oplus \bigoplus_{\alpha\in
     \Delta_i}\g_\alpha$ (for $i=1,2$). Then we have that $\g=\g_1\oplus \g_2$ as a
     vector space. We must check that $\g_1$ is an ideal (the by symmetry, $\g_2$ will
     also be an ideal). From the relation $[\g_\alpha,\g_\beta]\subseteq
     \g_{\alpha+\beta}$, we know that it is enough to check that for $\alpha\in
     \Delta_1$,
     \begin{align*}
     [\g_\alpha,\g_{-\alpha}]&\subseteq \h_1, \text{ and} \tag{1}\\
     [\g_\alpha,\h_2]&=0.\tag{2}
     \end{align*}
     Letting $\beta\in \Delta_2$, we have that
     $\beta([X_\alpha,Y_\alpha])=\beta(H_\alpha) =
     \frac{2(\alpha,\beta)}{(\beta,\beta)}=0$ because $\Delta_1$ and $\Delta_2$ are
     orthogonal; 1 follows because $\Delta_2$ spans the orthogonal complement of
     $\h_1$ in $\h$. Similarly, we have
     $[X_\alpha,H_\beta]=\alpha(H_\beta)X_\alpha=0$; 2 follows because the
     $H_\beta$ span $\h_2$.

     Conversely, if $\g=\g_1\oplus \g_2$ as a Lie algebra, then take root
     decompositions $\g_1=\h_1\oplus \bigoplus_{\alpha\in \Delta_1}\g_\alpha$ and
     $\g_2=\h_2\oplus \bigoplus_{\beta\in \Delta_2} \g_\beta$, with respect to regular
     elements $h_1\in \h_1$ and $h_2\in \h_2$. Then for $x_1\in\g_1$ and $x_2\in
     \g_2$, we have that $[h_1+h_2,x_1+x_2]=[h_1,x_1]+[h_2,x_2]$; it follows that
     $h_1+h_2$ is a regular element in $\g$. The Cartan given by this element is
     clearly $\h_1\oplus \h_2$. If $x\in \g_\alpha\subseteq \g_1$, then we have
     $[h_1+h_2,x]=\alpha(h_1)x+0$, so $\alpha$ is a root. Similarly, each $\beta\in
     \Delta_2$ is a root. Since we have accounted for all the root spaces of $\g_1$
     and of $\g_2$, we have a root decomposition $\g=(\h_1\oplus \h_2)\oplus
     \bigoplus_{\alpha\in \Delta_1}\g_\alpha \oplus \bigoplus_{\beta\in
     \Delta_2}\g_\beta$. This shows that $\Delta=\Delta_1\cup \Delta_2$.
   \end{solution}
 \end{exercise}

 Now we will classify all systems of rank 2. Observe that
 $\frac{2(\alpha,\beta)}{(\alpha,\alpha)}\frac{2(\alpha,\beta)}{(\beta,\beta)} =
 4\cos^2 \theta$, where $\theta$ is the angle between $\alpha$ and
 $\beta$. This thing must be an integer. Thus, there are not many choices for
 $\theta$:
 \[\begin{array}{c|ccccc}
   \cos \theta & 0 & \pm \frac{1}{2} & \pm \frac{1}{\sqrt{2}} & \pm
   \frac{\sqrt{3}}{2}\\ \hline
   \theta & \frac{\pi}{2} & \frac{\pi}{3}, \frac{2\pi}{3} & \frac{\pi}{4}, \frac{3\pi}{4}
   & \frac{\pi}{6},\frac{5\pi}{6}
 \end{array}\]
 Choose two vectors with minimal angle between them. If the minimal angle is $\pi/2$,
 then the system is reducible.
 \[\begin{xy}
   <1.5em,0em>:(0,0);
   \ar (1.4,0) *+!L{\alpha},
   \ar (-1.4,0),
   \ar (0,1) *+!D{\beta},
   \ar (0,-1),
 \end{xy}\]
 Notice that $\alpha$ and $\beta$ can be scaled independently.

 If the minimal angle is smaller than $\pi/2$, then $r_\beta(\alpha)\neq \alpha$, so
 the difference $\alpha - r_\beta(\alpha)$ is a non-zero integer multiple of $\beta$
 (in fact, a positive multiple of $\beta$ since $\theta < \pi/2$). If we assume
 $\|\alpha\|\le \|\beta\|$ (we can always switch them), we get that $\|\alpha -
 r_\beta(\alpha)\|<2\|\alpha\|\le 2\|\beta\|$. It follows that
 $\alpha-r_\beta(\alpha)=\beta$.
 \begin{remark}\label{lec14R:acute}
   Observe that we have shown that for any roots $\alpha$ and $\beta$, if
   $\theta_{\alpha,\beta}<\pi/2$, then $\alpha-\beta$ is a root.
 \end{remark}
 \begin{remark}\label{lec14Rmkrank2}
   We have also shown that once we set the direction of the longer root, $\beta$ (thus
   determining $r_\beta$), its length is determined relative to the length of
   $\alpha$.
 \end{remark}
 Now we can obtain the remaining elements of the root system from the
 condition that $\Delta$ is invariant under $r_\alpha$ and $r_\beta$, observing that
 no additional vectors can be added without violating \hyperlink{RS2}{RS2},
 \hyperlink{RS3}{RS3}, or the prescribed minimal angle. Thus, all the irreducible rank
 two root systems are
 \[\begin{array}{ccccc}
   A_2, \theta = \pi/3 &\quad & B_2, \theta = \pi/4 &\quad & G_2, \theta=\pi/6\\
   \begin{xy}
   <3em,0em>:(0,0)="c";
   a(150);a(-30) **@{.},
   \ar "c";a(0) *+!L{\alpha}
   \ar "c";a(60) *+!L{\beta}
   \ar "c";a(120)
   \ar "c";a(180)
   \ar "c";a(-60) *+!U{r_\beta(\alpha)}
   \ar "c";a(-120)
 \end{xy} & &
 \begin{xy}
   (-1.3,1.3);(-1,1) **@{.},
   (1.3,-1.3);(1,-1) **@{.};
   (0,0);(0,0);
   \ar (1,0) *+!L{\alpha},
   \ar (1,1) *+!L{\beta},
   \ar (0,1),
   \ar (-1,1),
   \ar (-1,0),
   \ar (-1,-1),
   \ar (0,-1) *+!U{r_\beta (\alpha)},
   \ar (1,-1),
 \end{xy} & &
 \begin{xy}
   <2em,0em>:
   a(120)+a(120);a(120) **@{.},
   a(-60)+a(-60);a(-60) **@{.},
   (0,0);(0,0);
   \ar (1,0) ="a" *+!L{\alpha},
   \ar a(120) -"a"="b",
   \ar "b"+"a",
   \ar "b"+"a"+"a",
   \ar "b"+"a"+"a"+"a" *+!L{\beta},
   \ar "b"+"b"+"a"+"a"+"a",
   \ar -"a",
   \ar -"b",
   \ar -"b"-"a",
   \ar -"b"-"a"-"a" *+!U{r_\beta (\alpha)\,},
   \ar -"b"-"a"-"a"-"a",
   \ar -"b"-"b"-"a"-"a"-"a",
 \end{xy}
 \end{array}\]

 \subsektion{The Weyl group}\index{Weyl group|(idxbf} Given a root system $\Delta =
 \{\alpha_1,\dots, \alpha_N\}$, we call the group generated by the $r_{\alpha_i}$s the
 \emph{Weyl group}, denoted $\weyl$.

 \begin{remark}
   If $G$ is a Lie group with Lie algebra $\g$, then for each $r_\alpha\in \weyl$, there
   is a group element $S_\alpha\in G$, such that $Ad_{S_\alpha}$ takes $\h$ to itself,
   and induces $r_\alpha$. Consider the $\sl_2\subseteq \g$ generated
   by $X_\alpha$, $Y_\alpha$, and $H_\alpha$. The embedding $\sl_2\to \g$ induces a
   homomorphism $SL(2)\to G$, and $S_\alpha$ is the image of $\matrix 01{-1}0$ under
   this homomorphism.
 \end{remark}
 \begin{exercise}\label{lec14Ex:Weyl}
   Let $\g$ be a semisimple Lie algebra, and let $\h$ be a Cartan subalgebra. For each
   root $\alpha$ define
   \[
    S_\alpha = \exp(X_\alpha)\exp(-Y_\alpha)\exp(X_\alpha).
   \]
   Prove that $Ad_{S_\alpha}(\h)=\h$ and that
   \[
    \langle \lambda, Ad_{S_\alpha}(h)\rangle = \langle r_\alpha(\lambda),h\rangle
   \]
   for any $h\in \h$ and $\lambda\in \h^*$, where $r_\alpha$ is the reflection in
   $\alpha^\perp$.
   \begin{solution} \label{lec14Soln:Weyl}
     Note that
     $Ad_{S_\alpha}=\exp(ad_{X_\alpha})\exp(-ad_{Y_\alpha})\exp(ad_{X_\alpha})$. If
     $h\in \h$, then $ad_{X_\alpha}h=-\alpha(h)X_\alpha$ and
     $ad_{X_\alpha}ad_{X_\alpha}(h)=\alpha(h)ad_{X_\alpha}(X_\alpha)=0$. Using the
     power series expansion for $\exp$, we get that
     \[
       \exp(ad_{X_\alpha})(h)=h - \alpha(h)X_\alpha.
     \]
     Similarly, we apply $\exp(-ad_{Y_\alpha})$ to the result
     \begin{align*}
       \exp(-ad_{Y_\alpha})\bigl(h-&\alpha(h)X_\alpha\bigr) \\
        &= h-\alpha(h)Y_\alpha -
       \alpha(h)\bigl( X_\alpha -\underbrace{[Y_\alpha,X_\alpha]}_{-H_\alpha} +
       \underbrace{\half[Y_\alpha,[Y_\alpha,X_\alpha]]}_{-\half\alpha(H_\alpha)Y_\alpha=-Y_\alpha}
       + 0 \bigr)\\
       &= h - \alpha(h)(X_\alpha + H_\alpha)
     \end{align*}
     and then apply $\exp(ad_{X_\alpha})$
     \begin{align*}
       \exp(ad_{X_\alpha})\bigl(h-&\alpha(h)(X_\alpha+H_\alpha)\bigr)\\
           &= h-\alpha(h)X_\alpha - \alpha(h)\Bigl(\bigl( X_\alpha + 0
           \bigr)+\bigl( H_\alpha - \alpha(H_\alpha)X_\alpha + 0 \bigr)\Bigr)\\
           &= h - \alpha(h)H_\alpha.
     \end{align*}
     This shows that $Ad_{S_\alpha}(\h)=\h$. For $\lambda\in \h^*$, we get
     \begin{align*}
       \langle r_\alpha(\lambda),h\rangle &=
       \lambda(h)-\frac{2(\lambda,\alpha)}{(\alpha,\alpha)}\alpha(h)\\
       &= \lambda(h) - \frac{2\lambda(H_\alpha)}{\alpha(H_\alpha)}\alpha(h) &
       \text{(using Equation \ref{lec14ddag})}\\
       &= \lambda\bigl( h - \alpha(h)H_\alpha \bigr) & \bigl( \alpha(H_\alpha)=2 \bigr)\\
       &= \langle \lambda, Ad_{S_\alpha}(h)\rangle.
     \end{align*}
   \end{solution}
 \end{exercise}
 If $G$ is a connected group with Lie algebra $\g$, then define the \emph{Cartan
 subgroup}\index{Cartan!subgroup} $H\subseteq G$ to be the subgroup generated by the
 image of $\h$ under the exponential map $\exp:\g\to G$. Let
 \[
    N(H)=\{g\in G|gHg^{-1}=H\}
 \]
 be the normalizer of $H$. Then we get a sequence of homomorphisms
 \begin{align*}
    N(H)\to \aut H \to &\aut \h \to \aut \h^*\\
    g\xmapsto{\qquad\qquad\qquad} &\ Ad_g \longmapsto Ad^*_g.
 \end{align*}
 The first map is given by conjugation, the second by differentiation at the identity,
 and the third by the identification of $\h$ with $\h^*$ via the Killing form. The
 final map is given by $g\mapsto Ad^*_g$, where
 $\big(Ad^*_g(l)\big)(h)=l(Ad_{g^{-1}}h)$.
 \begin{proposition}\label{lec14P:WeylgpFact}
   The kernel of the composition above is exactly $H$, and the image is the Weyl
   group. In particular, $\weyl\cong N(H)/H$.
 \end{proposition}
 Before we prove this proposition, we need a lemma.
 \begin{lemma}\label{lec14L:H=C(H)}
   The centralizer of $H$ is $H$.
 \end{lemma}
 \begin{proof}
   If $g$ centralizes $H$, then $Ad_g$ is the identity on $\h$. Furthermore, for any
   $h\in \h$ and $x\in \g_\alpha$,
   \[
     [h,Ad_g x] = Ad_g([h,x])=Ad_g (\alpha(h) x) = \alpha(\h)\cdot Ad_g x
   \]
   so $Ad_g(\g_\alpha)=\g_\alpha$. Say $Ad_g(X_i)=c_iX_i$, where $X_i$ spans the
   simple root space $\g_{\alpha_i}$. Then $Ad_g(Y_i)=\frac{1}{c}Y_i$. Since the
   simple roots are linearly independent, we can find an $h\in \h$ such that $Ad_{\exp
   h}X_i=c_iX_i$. Now we have that $Ad_{g\cdot (\exp h)^{-1}}$ is the identity on
   $\g$, so $g\cdot (\exp h)^{-1}$ is in the center of $G$, which is in $H$\anton{ugg,
   why is $Z(G)\subseteq H$?}, so $g\in H$, as desired.
 \end{proof}
 \begin{proof}[Proof of Proposition \ref{lec14P:WeylgpFact}]
   It is clear that $H$ is in the kernel of the composition. To see that $H$ is
   exactly the kernel, observe that $Ad^*_g$ can only be the identity map if $Ad_g$ is
   the identity map, which can only happen if conjugation by $g$ is the identity map
   on $H$, i.e.\ if $g$ is in the centralizer of $H$. By Lemma \ref{lec14L:H=C(H)},
   $g\in H$.

   Since the $S_\alpha$ in the previous exercise preserves $\h$ under the $Ad$ action,
   it is in the normalizer of $H$. It is easy to see (given Exercise \ref{lec14Ex:Weyl})
   that the image of $S_\alpha$ in $\aut \h^*$ is exactly $r_\alpha$. Thus, every
   element of the Weyl group is in the image.

   We can show that the map preserves the set of roots. If $\alpha$ is a root, with a
   root vector $x$, then we have $ad_h(x)=\alpha(h)x$ for all $h\in \h$. We would like
   to show that $Ad^*_g \alpha$ is also a root. It is enough to observe that $Ad_{g}x$
   is a root vector:
   \begin{align*}
     ad_h(Ad_{g}x) &= [h,Ad_{g}x] = Ad_{g}\big( [Ad_{g^{-1}} h, x] \big)\\
        &= Ad_{g}\big( \alpha(Ad_{g^{-1}} h) x) \big) = \alpha(Ad_{g^{-1}}h) Ad_g(x)\\
        &= \big(Ad^*_g(\alpha)\big)(h)\, (Ad_g x)
   \end{align*}

   Therefore, we can find some element $w$ in the Weyl group so that $w\circ Ad^*_g$
   preserves the set $\Pi$ of simple roots. Since $w$ is in the image of $Ad^*$, it is
   enough to show that whenever $Ad^*_g$ preserves $\Pi$, it is the identity map on
   $\h^*$.

   \anton{complete this proof ... is there a way that doesn't use representation
   theory. For any dominant integral weight $\lambda$,  $V_\lambda$ is isomorphic to
   $V_{Ad^*_g \lambda}$ via $\rho(g)$, so $\lambda=Ad^*_g\lambda$ by Theorem
   \ref{lec18Thm:hiweight}. It follows that $Ad^*_g$ is the identity on $\h^*$.}
 \end{proof}

 \begin{example}[Also see Example \ref{lec13Eg:sl3}]\label{lec14Eg:A_n}
   The root system of $\sl_{n+1}$ is called $A_n$\index{An@$A_n$!and $\sl_{n+1}$|idxbf}.
   We pick an orthonormal basis $\varepsilon_1,\dots,\varepsilon_{n+1}$ of
   $\RR^{n+1}$, the the root system is the set of all the differences: $\Delta = \{
   \varepsilon_i - \varepsilon_j | i\neq j\}$. We have that
   \[
    r_{\varepsilon_i - \varepsilon_j}(\varepsilon_k) = \left\{
    \begin{array}{cl}
      \varepsilon_k & k\neq i,j\\
      \varepsilon_j & k=i\\
      \varepsilon_i & k=j
    \end{array} \right.
   \]
   is a transposition, so we have that $\weyl\simeq S_{n+1}$.
 \end{example}

 Now back to classification of abstract root systems.

 Draw a hyperplane in general position (so that it doesn't contain any roots). This
 divides $\Delta$ into two parts, $\Delta = \Delta^+\coprod \Delta^-$. The roots in
 $\Delta^+$ are called \emph{positive roots}\index{root!positive}, and the elements of
 $\Delta^-$ are called negative roots. We say that $\alpha\in \Delta^+$ is
 \emph{simple}\index{root!simple} if it cannot be written as the sum of other positive
 roots. Let $\Pi$ be the set of simple roots, sometimes called a base. It has the
 properties\index{root!simple!properties of|(idxbf}
 \begin{enumerate}
 \item \label{lec14n1} Any $\alpha\in \Delta^+$ is a sum of simple roots (perhaps with
 repitition): $\alpha = \sum_{\beta \in \Pi} m_\beta \beta$ where $m_\beta \in
 \ZZ_{\ge 0}$.

 \item \label{lec14n2} If $\alpha, \beta\in \Pi$, then $(\alpha,\beta)\le 0$.

 This follows from the fact that if $(\alpha,\beta)>0$, then $\alpha-\beta$ and
 $\beta-\alpha$ are again roots (as we showed when we classified rank 2 root systems),
 and one of them is positive, say $\alpha-\beta$. Then $\alpha = \beta +
 (\alpha-\beta)$, contradicting simplicity of $\alpha$.

 \item \label{lec14n3} $\Pi$ is a linearly independent set.

 If they were linearly dependent, the relation $\sum_{\alpha_i\in \Pi} a_i\alpha_i =
 0$ must have some negative coefficients (because all of $\Pi$ is in one half space),
 so we can always write
 \[
    0\neq a_1\alpha_1+\cdots+ a_r\alpha_r = a_{r+1}\alpha_{r+1}+\cdots+ a_n\alpha_n
 \]
 with all the $a_i\ge 0$. Taking the inner product with the left hand side, we get
 \begin{align*}
    \|a_1\alpha_1+&\cdots+ a_r\alpha_r\|^2 \\
                &= (a_1\alpha_1+\cdots+ a_r\alpha_r, a_{r+1}\alpha_{r+1}+\cdots+ a_n\alpha_n) \le 0
 \end{align*}
 by \ref{lec14n2}, which is absurd.
 \end{enumerate}\index{root!simple!properties of|)idxbf}
 \begin{remark}
   Notice that the hyperplane is $t^\perp$ for some $t$, and the positive roots are the
   $\alpha \in \Delta$ for which $(t,\alpha)>0$. This gives an order on the roots. You
   can inductively prove \ref{lec14n1} using this order.
 \end{remark}
 \begin{remark}
   Notice that when you talk about two roots, they always generate one of the rank 2
   systems, and we know what all the rank 2 systems are.
 \end{remark}

 \begin{lemma}[Key Lemma]\label{lec14L:key}
 Suppose we have chosen a set of positive roots $\Delta^+$, with simple roots $\Pi$.
 Then for $\alpha\in \Pi$, we have that $r_\alpha(\Delta^+)= \Delta^+ \cup \{-\alpha\}
 \smallsetminus \{\alpha\}$.
 \end{lemma}
 \begin{proof}
 For a simple root $\beta\neq \alpha$, we have $r_\alpha(\beta) = \beta + k\alpha$ for
 some non-negative $k$; this must be on the positive side of the hyperplane, so it is
 a positive root. Now assume you have a positive root of the form $\gamma = m\alpha +
 \sum_{\alpha_i\neq \alpha} m_i \alpha_i$, with $m,m_i\ge 0$. Then we have that
 $r_\alpha (\gamma) = -m\alpha + \sum_{\alpha_i\neq \alpha} m_i(\alpha_i-k_i\alpha)
 \in \Delta$. If any of the $m_i$ are strictly positive, then the coefficient of
 $\alpha_i$ in $r_i(\gamma)$ is positive, so $r_\alpha(\gamma)$ must be positive
 because every root can be (uniquely) written as either a non-negative or a
 non-positive combination of the simple roots.
 \end{proof}

 \begin{proposition}\label{lec14P:tansitive}
   The group generated by simple reflections (with respect to some fixed
   $\Pi=\{\alpha_1,\dots, \alpha_n\}$) acts transitively on the set of sets of positive
   roots.
 \end{proposition}
 \begin{proof}
   It is enough to show that we can get from $\Delta^+$ to any other set of simple
   roots $\bar\Delta^+$.

   If $\bar\Delta^+$ contains $\Pi$, then $\bar\Delta^+=\Delta^+$ and we are done.
   Otherwise, there is some $\alpha_i\not\in \bar\Delta^+$ (equivalently,
   $-\alpha_i\in \bar\Delta^+$). Applying $r_i$, Lemma \ref{lec14L:key} tells us that
   \[
     \bigl|r_i(\Delta^+)\smallsetminus \bar\Delta^+\bigr|<
     \bigl|\Delta^+\smallsetminus \bar\Delta^+\bigr|.
   \]
   If we can show for any root $\alpha$ which is simple
   \emph{with respect to $r_i(\Delta)$}, that $r_\alpha$ is a product of simple
   reflections, then we are done by induction. But we have that $\alpha =
   r_i(\alpha_j)$ for some $j$, from which we get that $r_\alpha=r_ir_jr_i$.
 \end{proof}
 \begin{corollary}
   $\weyl$ is generated by simple reflections.
 \end{corollary}
 \begin{proof}
   Any root $\alpha$ is a simple root for some choice of positive roots. To see this,
   draw the hyperplane really close to the given root. Then we know that $\alpha$ is
   obtained from our initial set $\Pi$ by simple reflections. We get that if $\alpha =
   r_{i_1}\cdots r_{i_k}(\alpha_j)$, then $r_\alpha = (r_{i_1}\cdots
   r_{i_k})r_j(r_{i_1}\cdots r_{i_k})^{-1}$.
 \end{proof}
 We define the \emph{length}\index{length} of an element $w\in \weyl$ to be the
 smallest number $k$ so that $w=r_{i_1}\cdots r_{i_k}$, for some simple reflections
 $r_{i_j}$.

 Next, we'd like to prove that $\weyl$ acts \emph{simply transitively} on the set of
 sets of simple roots. To do this, we need the following lemma, which essentially says
 that if you have a string of simple reflections so that some positive root becomes
 negative and then positive again, then you can get the same element of $\weyl$ with
 fewer simple reflections.
 \begin{lemma}\label{lec14L:length}
   Let $\beta_1,\beta_2,\dots, \beta_t$ be a sequence in $\Pi$ (possibly with
   repetition) with $t\ge 2$. Let $r_i=r_{\beta_i}$. If $r_1r_2\cdots r_t(\beta_t)\in
   \Delta^+$, then there is some $s<t$ such that
   \[
        r_1\cdots r_t = r_1\cdots r_{s-1}r_{s+1}\cdots r_{t-1}.
   \]
   (Note that the right hand side omits $r_s$ and $r_t$.)
 \end{lemma}
 \begin{proof}
   Note that $\beta_t$ is positive and $r_1\cdots r_{t-1}(\beta_t)$ is negative, so
   there is a smallest number $s$ for which $r_{s+1}\cdots
   r_{t-1}(\beta_t)=\gamma$ is positive. Then $r_s(\gamma)$ is negative, so by Lemma
   \ref{lec14L:key}, we get $\gamma = \beta_s$. This gives us
   \begin{align*}
     r_s &= (r_{s+1}\cdots r_{t-1})r_t(r_{s+1}\cdots r_{t-1})^{-1}\\
     r_sr_{s+1}\cdots r_{t-1} &= r_{s+1}\cdots r_{t-1}r_t.
   \end{align*}
   Multiplying both sides of the second equation on the left by $r_1\cdots r_{s-1}$ to
   get the result.
 \end{proof}
 \begin{proposition}\label{lec14P:simply}
   $\weyl$ acts \emph{simply transitively} on the set of sets of positive roots.
 \end{proposition}
 \begin{proof}
   Proposition \ref{lec14P:tansitive} shows that the action is transitive, so we need
   only show that any $w\in \weyl$ which fixes $\Delta^+$ must be the identity
   element. If $w$ is a simple reflection, then it does not preserve $\Delta^+$. So we
   may assume that the shortest way to express $w$ as a product simple reflections
   uses at least two simple reflections, say $w = r_{i_1}\cdots r_{i_t}$. Then by
   Lemma \ref{lec14L:length}, we can reduce the length of $w$ by two, contradicting
   the minimality of $t$.
 \end{proof}
 \begin{corollary}
   The length of an element $w\in \weyl$ is exactly $\bigl|
   w(\Delta^+)\smallsetminus \Delta^+\bigr|$.
 \end{corollary}
 \begin{proof}
   By Proposition \ref{lec14P:simply}, $w$ is the unique element taking $\Delta^+$ to
   $w(\Delta^+)$. Say we are building a word, as in Proposition
   \ref{lec14P:tansitive}, to get from $\Delta^+$ to $w(\Delta^+)$. Assume we've
   already applied $r_{i_1}\cdots r_{i_k}$, and next we are going to reflect through
   $r_{i_1}\cdots r_{i_k}(\alpha_j)$. Then we will have applied the element
   \[
     (r_{i_1}\cdots r_{i_k})r_j(r_{i_1}\cdots r_{i_k})^{-1} (r_{i_1}\cdots r_{i_k}) = r_{i_1}\cdots
     r_{i_k}r_j.
   \]
   Thus, each time we reduce $\bigr| r_{i_1}\cdots r_{i_k}(\Delta^+)\smallsetminus
   w(\Delta^+)\bigr|$ by one, we add one simple reflection. This shows that we can
   express $w$ in the desired number of simple reflections.

   On the other hand, Lemma \ref{lec14L:key} tells us that for any sequence of simple
   reflections $r_{i_1}$,\dots, $r_{i_k}$, $\bigr| r_{i_1}\cdots
   r_{i_k}(\Delta^+)\smallsetminus \Delta^+\bigr|\le k$, so $w$ cannot be written as a
   product of fewer than $\bigl| w(\Delta^+)\smallsetminus \Delta^+\bigr|$ simple
   reflections.
 \end{proof}
 \index{Weyl group|)idxbf}

% Let's see how it works for $A_n$. Choose a vector $t= t_1\varepsilon_1+\cdots +
% t_{n+1} \varepsilon_{n+1}$ in such a way that $t_1>\cdots > t_{n+1}$. Then we have
% that $(\varepsilon_i - \varepsilon_j,t) = t_i-t_j$, so the positive roots are the
% $\varepsilon_i - \varepsilon_j$ for which $i< j$. Then the simple roots are $\Pi =
% \{\varepsilon_1-\varepsilon_2, \dots, \varepsilon_n - \varepsilon_{n+1}\}$
}{   % Anton, geraschenko@gmail.com
  \stepcounter{lecture}
 \setcounter{lecture}{15}
 \sektion{Lecture 15 - Dynkin diagrams, Classification of root systems}

 Last time, we talked about root systems $\Delta\subset \RR^n$. We constructed the
 Weyl group $\weyl$, the finite group generated by reflections. We considered $\Pi\subset
 \Delta$, the simple roots. We showed that $\Pi$ forms a basis for $\RR^n$, and that every
 root is a non-negative (or non-positive) linear combination of simple roots.

 If $\alpha,\beta\in \Pi$, then define $n_{\alpha\beta} =
 \frac{2(\alpha,\beta)}{(\beta,\beta)}$. We showed that $n_{\alpha\beta}$ is a
 non-positive integer. Since $n_{\alpha\beta}n_{\beta\alpha} = 4\cos^2
 \theta_{\alpha\beta}$, $n_{\alpha,\beta}$ can only be $0,-1,-2,$ or $-3$. If
 $n_{\alpha\beta}=0$, the two are orthogonal. If $n_{\alpha\beta}=-1$, then the angle
 must be $2\pi/3$ and the two are the same length. If $n_{\alpha\beta}=-2$, the angle
 must be $3\pi/4$ and $||\beta||=\sqrt{2}\ ||\alpha||$. If $n_{\alpha\beta}=-3$, the
 angle is $5\pi/6$ and $||\beta||=\sqrt{3}\ ||\alpha||$. Thus we get:
 \[\begin{array}{|c|c|c|c|} \hline
   n_{\beta\alpha} & n_{\alpha\beta} & \text{relationship} & \text{Dynkin picture} \\ \hline
   0 & 0 & \begin{xy}
  <1.75em,0em>:
  (0,0);
  \ar (1,0) *+!L{\alpha},
  \ar (0,0.7) *+!D{\beta},
\end{xy} & \begin{xy}
   (0,0) *+!R{\alpha} *\cir<2pt>{};
   (1,0) *+!L{\beta} *\cir<2pt>{};
 \end{xy}\\ \hline
   -1 & -1 & \begin{xy}
  <1.75em,0em>:
  (0,0);
  \ar (1,0) *+!L{\alpha},
  \ar (-.5,\halfrootthree) *+!DR{\beta},
\end{xy} & \begin{xy}
   (0,0) *+!R{\alpha} *\cir<2pt>{};
   (1,0) *+!L{\beta} *\cir<2pt>{} **@{-};
 \end{xy}\\ \hline
   -2 & -1 & \begin{xy}
  <1.75em,0em>:
  (0,0);
  \ar (1,0) *+!L{\alpha},
  \ar (-1,1) *+!R{\beta},
\end{xy} & \begin{xy}
   (0,0) *+!R{\alpha} *\cir<2pt>{};
   (1,0) *+!L{\beta} *\cir<2pt>{} **@{=} ?*@{<};
 \end{xy}\\ \hline
   -3 & -1 &  \begin{xy}
  <1.75em,0em>:
  (0,0);
  \ar (1,0) *+!L{\alpha},
  \ar (-1,0)+a(120) *+!R{\beta},
\end{xy} & \begin{xy}
   (0,0)="1" *+!R{\alpha} *\cir<2pt>{};
   (1,0)="2" *+!L{\beta} *\cir<2pt>{} **@{-} ?*@{<},
   \ar@{-} "1" *{\hspace{3pt}};"2" *{\hspace{3pt}} <1.5pt>
   \ar@{-} "1" *{\hspace{3pt}};"2" *{\hspace{3pt}} <-1.5pt>
 \end{xy}\\ \hline
 \end{array}\]
 \begin{definition}
   Given a root system, the \emph{Dynkin diagram}\index{Dynkin diagram|(idxbf} of the
   root system is obtained in the following way.  For each simple root, draw a node.
   We join two nodes by $n_{\alpha\beta}n_{\beta\alpha}$ lines. If there are two or
   three lines (i.e. if the roots are not the same length), then we draw an arrow from
   the longer root to the shorter root. (As always, the alligator eats the big one.)
 \end{definition}
 The Dynkin diagram is independent of the choice of simple roots. For any other choice
 of simple roots, there is an element of the Weyl group that translates between the
 two, and the Weyl group preserves inner products.

 We would really like to classify Dynkin diagrams. To aid the classification, we
 define an undirected version of the Dynkin diagram. Define $e_i =
 \frac{\alpha_i}{(\alpha_i,\alpha_i)^{1/2}}$, for $\alpha_i\in \Pi$. Then the number
 of lines between two vertices is $n_{\alpha_i\alpha_j}n_{\alpha_j\alpha_i} = 4\cdot
 \frac{(\alpha_i,\alpha_j)^2}{(\alpha_i,\alpha_i)(\alpha_j,\alpha_j)} = 4(e_i,e_j)^2$.
 \begin{definition}
   Given a set $\{e_1,\dots, e_n\}$ of linearly independent unit vectors in some
   Euclidean space with the property that $(e_i,e_j)\le 0$ and $4(e_i,e_j)^2\in \ZZ$
   for all $i$ and $j$, the \emph{Coxeter diagram}\index{Coxeter diagram|(idxbf}
   associated to the set is obtained in the following way. For each unit vector, draw
   a node. Join the nodes of $e_i$ and $e_j$ by $4(e_i,e_j)^2$ lines.
 \end{definition}
 Since every Dynkin diagram gives a Coxeter diagram, understanding Coxeter diagrams is
 a good start in classifying Dynkin diagrams.

 \begin{example}
   $A_n$ has $n$ simple roots, given by $\varepsilon_i - \varepsilon_{i+1}$. So the
   graphs are
 \[\text{Dynkin}\qquad \begin{xy}
   (0,0) *\cir<2pt>{};
   (1,0)  *\cir<2pt>{} **@{-};
   p+(.5,0) **@{-};
   p+(.6,0) **{\hspace{1pt}.\hspace{1pt}};
   p+(.5,0)  *\cir<2pt>{} **@{-};
   p+(1,0)  *\cir<2pt>{} **@{-};
 \end{xy} \]
\[\text{Coxeter}\qquad \begin{xy}
   (0,0) *=<0pt>{\bullet};
   (1,0)  *=<0pt>{\bullet} **@{-};
   p+(.5,0) **@{-};
   p+(.6,0) **{\hspace{1pt}.\hspace{1pt}};
   p+(.5,0)  *=<0pt>{\bullet} **@{-};
   p+(1,0)  *=<0pt>{\bullet} **@{-};
 \end{xy} \]
 \end{example}

 Let's prove some properties of Coxeter diagrams.
 \begin{itemize}
 \item[\hypertarget{CX1}{(CX1)}] A subgraph of a Coxeter diagram is a Coxeter diagram.
 This is obvious from the definition.

 \item[\hypertarget{CX2}{(CX2)}] A Coxeter diagram is acyclic.
 \begin{proof}
   Let $e_1,\dots, e_k$ be a cycle in the Coxeter diagram. Then
   \[
     \Bigl(\sum e_i, \sum e_i\Bigr) = k + \sum_{\substack{i< j\\ i,j\text{ adjacent}}}
     \underbrace{2(e_i,e_j)}_{\le {-1}} \le 0
   \]
   which contradicts that the inner product is positive definite.
 \end{proof}

 \item[\hypertarget{CX3}{(CX3)}] The degree of each vertex in a Coxeter diagram is
 less than or equal to 3, where double and triple edges count as two and
 three edges, respectively.
 \begin{proof}
   Let $e_0$ have $e_1,\dots, e_k$ adjacent to it. Since there are no cycles,
   $e_1,\dots, e_k$ are orthogonal to each other. So we can compute
   \begin{align*}
     \left(e_0 - \sum (e_0,e_i) e_i, e_0 - \sum (e_0,e_i) e_i \right) & > 0\\
       1- \sum (e_0,e_i)^2 &> 0
   \end{align*}
   but $(e_0,e_i)^2$ is one fourth of the number of edges connecting $e_0$ and $e_i$.
   So $k$ cannot be bigger than 3.
 \end{proof}

 \item[\hypertarget{CX4}{(CX4)}] Suppose a Coxeter diagram has a subgraph of type
 $A_n$, and only the endpoints of this subgraph have additional edges (say $\Gamma_1$
 at one end and $\Gamma_2$ at the other end). Then we can ``contract'' the stuff in
 the middle and just fuse $\Gamma_1$ with $\Gamma_2$, and the result is a Coxeter
 diagram.
 \[
  \ifthenelse{\boolean{lilbook}}{
  \begin{xy}
   (0,0)="G1" *+{\Gamma_1} *\cir{};
   (1,0)="1" *+!D{e_1} *=<0pt>{\bullet};
   p+(.5,0) **@{-};
   p+(.6,0) **{\hspace{1pt}.\hspace{1pt}};
   p+(.5,0)="2" *=<0pt>{\bullet} **@{-} *+!D{e_k};
   p+(1,0)="G2" *+{\Gamma_2} *\cir{};
   "G1"+(.225,.225); "1" **@{-}; "G1"+(.225,-.225) **@{-},
   "G2"+(-.225,-.225); "2" **@{-}; "G2"+(-.225,.225) **@{-}
 \end{xy} \qquad \longrightarrow \qquad
  \begin{xy}
   (0,0)="G1" *+{\Gamma_1} *\cir{};
   (1,0)="1" *=<0pt>{\bullet} *+!D{e_0};
   p+(1,0)="G2" *+{\Gamma_2} *\cir{};
   "G1"+(.225,.225); "1" **@{-}; "G1"+(.225,-.225) **@{-},
   "G2"+(-.225,-.225); "1" **@{-}; "G2"+(-.225,.225) **@{-}
 \end{xy}}{
  \begin{xy}
   (0,0)="G1" *+{\Gamma_1} *\cir{};
   (1,0)="1" *+!D{e_1} *=<0pt>{\bullet};
   p+(.5,0) **@{-};
   p+(.6,0) **{\hspace{1pt}.\hspace{1pt}};
   p+(.5,0)="2" *=<0pt>{\bullet} **@{-} *+!D{e_k};
   p+(1,0)="G2" *+{\Gamma_2} *\cir{};
   "G1"+(.2,.2); "1" **@{-}; "G1"+(.2,-.2) **@{-},
   "G2"+(-.2,-.2); "2" **@{-}; "G2"+(-.2,.2) **@{-}
 \end{xy} \qquad \longrightarrow \qquad
  \begin{xy}
   (0,0)="G1" *+{\Gamma_1} *\cir{};
   (1,0)="1" *=<0pt>{\bullet} *+!D{e_0};
   p+(1,0)="G2" *+{\Gamma_2} *\cir{};
   "G1"+(.2,.2); "1" **@{-}; "G1"+(.2,-.2) **@{-},
   "G2"+(-.2,-.2); "1" **@{-}; "G2"+(-.2,.2) **@{-}
 \end{xy}}
 \]
 \begin{proof}
   Let $e_1,\dots, e_k$ be the vertices in the $A_k$. Let $e_0 = e_1+\cdots + e_k$.
   Then we can compute that $(e_0,e_0)=1$. If $e_s\in \Gamma_1$ and $e_t\in \Gamma_2$,
   we get that $(e_0,e_s)=(e_1,e_s)$ and $(e_0,e_t)=(e_k,e_t)$.
 \end{proof}
 \end{itemize}
 Thus, a connected Coxeter diagram can have at most one fork (two could be glued to
 give valence 4), at most one double edge, and if there is a triple edge, nothing else
 can be connected to it.

 So the only possible connected Coxeter diagrams (an therefore Dynkin diagrams) so far
 are of the form
 \[ \begin{xy}
   (0,0) *=<0pt>{\bullet};
   (1,-.1) *=<0pt>{\bullet} **@{-};
   p+(.5,-.05) **@{-};
   p+(.6,-.06) **{\hspace{1pt}.\hspace{1pt}};
   p+(.5,-.05) *=<0pt>{\bullet} **@{-};
   %%%%%%%%%%%%%%%%%%%%%%%
   {p+(-.5,-.05) **@{-};
   p+(-.6,-.06) **{\hspace{1pt}.\hspace{1pt}};
   p+(-.5,-.05) *=<0pt>{\bullet} **@{-};
   p+(-1,-.1) *=<0pt>{\bullet} **@{-};},
   %%%%%%%%%%%%%%%%%%%%%%%
   p+(.5,0) **@{-};
   p+(.6,0) **{\hspace{1pt}.\hspace{1pt}};
   p+(.5,0)  *=<0pt>{\bullet} **@{-};
   p+(1,0)  *=<0pt>{\bullet} **@{-};
 \end{xy}\]
 \[\begin{xy}
   (0,0) *=<0pt>{\bullet};
   (1,0)  *=<0pt>{\bullet} **@{-};
   p+(.5,0) **@{-};
   p+(.6,0) **{\hspace{1pt}.\hspace{1pt}};
   p+(.5,0)  *=<0pt>{\bullet} **@{-};
   p+(1,0)  *=<0pt>{\bullet} **@2{-};
   p+(.5,0) **@{-};
   p+(.6,0) **{\hspace{1pt}.\hspace{1pt}};
   p+(.5,0)  *=<0pt>{\bullet} **@{-};
   p+(1,0)  *=<0pt>{\bullet} **@{-};
 \end{xy}
 \]
\[ \begin{xy}
   (0,0)="1" *=<0pt>{\bullet};
   (1,0)="2" *=<0pt>{\bullet} **@{-},
   \ar@{-} "1";"2" <1.5pt>
   \ar@{-} "1";"2" <-1.5pt>
 \end{xy}\]
 Now we switch gears back to Dynkin diagrams. Note that a subgraph of a Dynkin diagram
 is a Dynkin diagram. We will calculate that some diagrams are forbidden. We label
 the vertex corresponding to $\alpha_i$ with a number $m_i$, and check that
 \[
    \left(\sum m_i \alpha_i, \sum m_i \alpha_i\right) = 0.
 \]
\[ \begin{xy}
   (0,0) *+!D{1} *\cir<2pt>{};
   (1,0) *+!D{2} *\cir<2pt>{} **@{-};
   p+(1,0) *+!D{3} *\cir<2pt>{} **@{-};
   p+(1,0) *+!D{4} *\cir<2pt>{} **@{=}; {?*@{>}},
   p+(1,0) *+!D{2} *\cir<2pt>{} **@{-};
 \end{xy} \qquad
 \begin{xy}
   (0,0) *+!D{1} *\cir<2pt>{};
   (1,0) *+!D{2} *\cir<2pt>{} **@{-};
   p+(1,0) *+!D{3} *\cir<2pt>{} **@{-};
   p+(1,0) *+!D{2} *\cir<2pt>{} **@{=}; {?*@{<}},
   p+(1,0) *+!D{1} *\cir<2pt>{} **@{-};
 \end{xy}\]
 \[\begin{xy}
   (0,0) *+!D{1} *\cir<2pt>{};
   (1,0) *+!D{2} *\cir<2pt>{} **@{-};
   p+(1,0) *+!D{3} *\cir<2pt>{} **@{-};
       {p+(0,-1) *+!L{2} *\cir<2pt>{} **@{-};
       p+(0,-1) *+!L{1} *\cir<2pt>{} **@{-};},
   p+(1,0) *+!D{2} *\cir<2pt>{} **@{-};
   p+(1,0) *+!D{1} *\cir<2pt>{} **@{-};
 \end{xy}\qquad
 \begin{xy}
   (0,0) *+!D{1} *\cir<2pt>{};
   (1,0) *+!D{2} *\cir<2pt>{} **@{-};
   p+(1,0) *+!D{3} *\cir<2pt>{} **@{-};
   p+(1,0) *+!D{4} *\cir<2pt>{} **@{-};
        p+(0,-1) *+!L{2} *\cir<2pt>{} **@{-},
   p+(1,0) *+!D{3} *\cir<2pt>{} **@{-};
   p+(1,0) *+!D{2} *\cir<2pt>{} **@{-};
   p+(1,0) *+!D{1} *\cir<2pt>{} **@{-};
 \end{xy}\]
 \[\begin{xy}
   (0,0) *+!D{2} *\cir<2pt>{};
   (1,0) *+!D{4} *\cir<2pt>{} **@{-};
   p+(1,0) *+!D{6} *\cir<2pt>{} **@{-};
       p+(0,-1) *+!L{3} *\cir<2pt>{} **@{-},
   p+(1,0) *+!D{5} *\cir<2pt>{} **@{-};
   p+(1,0) *+!D{4} *\cir<2pt>{} **@{-};
   p+(1,0) *+!D{3} *\cir<2pt>{} **@{-};
   p+(1,0) *+!D{2} *\cir<2pt>{} **@{-};
   p+(1,0) *+!D{1} *\cir<2pt>{} **@{-};
 \end{xy}\]

 Thus we have narrowed our list of possible Dynkin diagrams to a short list.

 The ``classical'' connected Dynkin diagrams are shown below ($n$ is the total number
 of vertices).
 \begin{gather*}
  \begin{xy}
    (0,0) *\cir<2pt>{};
    (1,0)  *\cir<2pt>{} **@{-};
    p+(.5,0) **@{-};
    p+(.6,0) **{\hspace{1pt}.\hspace{1pt}};
    p+(.5,0)  *\cir<2pt>{} **@{-};
    p+(1,0)  *\cir<2pt>{} **@{-};
  \end{xy} \tag{$A_n$}\\
  \begin{xy}
    (0,0) *\cir<2pt>{};
    (1,0)  *\cir<2pt>{} **@{-};
    p+(.5,0) **@{-};
    p+(.6,0) **{\hspace{1pt}.\hspace{1pt}};
    p+(.5,0)  *\cir<2pt>{} **@{-};
    p+(1,0)  *\cir<2pt>{} **@{=} ?*@{>};
  \end{xy} \tag{$B_n$}\\
  \begin{xy}
    (0,0) *\cir<2pt>{};
    (1,0)  *\cir<2pt>{} **@{-};
    p+(.5,0) **@{-};
    p+(.6,0) **{\hspace{1pt}.\hspace{1pt}};
    p+(.5,0)  *\cir<2pt>{} **@{-};
    p+(1,0)  *\cir<2pt>{} **@{=} ?*@{<};
  \end{xy} \tag{$C_n$}\\
  \begin{xy}
    (0,0) *\cir<2pt>{};
    (1,0)  *\cir<2pt>{} **@{-};
    p+(.5,0) **@{-};
    p+(.6,0) **{\hspace{1pt}.\hspace{1pt}};
    p+(.5,0)  *\cir<2pt>{} **@{-};
    p+a(25)  *\cir<2pt>{} **@{-},
    p+a(-25)  *\cir<2pt>{} **@{-};
  \end{xy} \tag{$D_n$}\\
 \end{gather*}
 The ``exceptional'' Dynkin diagrams are
 \begin{gather*}
  \begin{xy}
    (0,0)="1" *\cir<2pt>{};
    (1,0)="2" *\cir<2pt>{} **@{-} ?*@{>},
    \ar@{-} "1" *{\hspace{3pt}};"2" *{\hspace{3pt}} <1.5pt>
    \ar@{-} "1" *{\hspace{3pt}};"2" *{\hspace{3pt}} <-1.5pt>
  \end{xy} \tag{$G_2$}\\
  \begin{xy}
    (0,0) *\cir<2pt>{};
    (1,0)  *\cir<2pt>{} **@{-};
    p+(1,0) *\cir<2pt>{} **@{=}; {?*@{>}},
    p+(1,0)  *\cir<2pt>{} **@{-};
  \end{xy} \tag{$F_4$}\\
  \begin{xy}
    (0,0) *\cir<2pt>{};
    (1,0)  *\cir<2pt>{} **@{-};
    p+(1,0) *\cir<2pt>{} **@{-};
    p+(0,-1) *\cir<2pt>{} **@{-},
    p+(1,0) *\cir<2pt>{} **@{-};
    p+(1,0) *\cir<2pt>{} **@{-};
  \end{xy} \tag{$E_6$}\\
  \begin{xy}
    (0,0) *\cir<2pt>{};
    (1,0)  *\cir<2pt>{} **@{-};
    p+(1,0) *\cir<2pt>{} **@{-};
    p+(1,0) *\cir<2pt>{} **@{-};
    p+(0,-1) *\cir<2pt>{} **@{-},
    p+(1,0) *\cir<2pt>{} **@{-};
    p+(1,0) *\cir<2pt>{} **@{-};
  \end{xy} \tag{$E_7$}\\
  \begin{xy}
    (0,0) *\cir<2pt>{};
    (1,0)  *\cir<2pt>{} **@{-};
    p+(1,0) *\cir<2pt>{} **@{-};
    p+(1,0) *\cir<2pt>{} **@{-};
    p+(1,0) *\cir<2pt>{} **@{-};
    p+(0,-1) *\cir<2pt>{} **@{-},
    p+(1,0) *\cir<2pt>{} **@{-};
    p+(1,0) *\cir<2pt>{} **@{-};
  \end{xy} \tag{$E_8$}\\
 \end{gather*}
 It remains to show that each of these is indeed the Dynkin diagram of some root
 system.

 We have already constructed the root system $A_n$ in Example \ref{lec14Eg:A_n}.

 Next we construct $D_n$\index{Dn@$D_n$!construction of|idxbf}. Let
 $\varepsilon_1,\dots, \varepsilon_n$ be an orthonormal basis for $\RR^n$. Then let
 the roots be
 \[
    \Delta = \{\pm (\varepsilon_i \pm \varepsilon_j) | i< j\le n\}.
 \]
 We choose the simple roots to be
 \[\begin{xy}
   (0,0) *{\varepsilon_1 - \varepsilon_2} *+\frm{-}; (1,0) **@{-};
   (1.5,0) **{\hspace{.2mm}.\hspace{.2mm}};
   (3,0) *{\varepsilon_{n-2} - \varepsilon_{n-1}} *+\frm{-} **@{-};
   (3,-1) *{\varepsilon_{n-1} + \varepsilon_{n}} *+\frm{-} **@{-},
   (5.5,0) *{\varepsilon_{n-1} - \varepsilon_{n}} *+\frm{-} **@{-},
 \end{xy}\tag{$D_n$}\]

 To get the root system for $B_n$\index{Bn@$B_n$!construction of|idxbf}, take $D_n$
 and add $\{\pm\varepsilon_i| i\le n\}$, in which case the simple roots are
 \[\begin{xy}
   (0,0) *{\varepsilon_1 - \varepsilon_2} *+\frm{-}; (1,0) **@{-};
   (1.5,0) **{\hspace{.2mm}.\hspace{.2mm}};
   (2.8,0) *{\varepsilon_{n-1} - \varepsilon_{n}} *+\frm{-} **@{-};
   (4.5,0) *{\varepsilon_{n}} *+\frm{-} **@{=} ?<>(.5)*@{>};
 \end{xy}\tag{$B_n$}\]

 To get $C_n$\index{Cn@$C_n$!construction of|idxbf}, take $D_n$ and add $\{\pm
 2\varepsilon_i|i\le n\}$, then the simple roots are
 \[\begin{xy}
   (0,0) *{\varepsilon_1 - \varepsilon_2} *+\frm{-}; (1,0) **@{-};
   (1.5,0) **{\hspace{.2mm}.\hspace{.2mm}};
   (2.8,0) *{\varepsilon_{n-1} - \varepsilon_{n}} *+\frm{-} **@{-};
   (4.6,0) *{2\varepsilon_{n}} *+\frm{-} **@{=} ?<>(.5)*@{<};
 \end{xy}\tag{$C_n$}\]

 \begin{remark}
  Recall that we can define coroots $\check \alpha = \frac{2\alpha}{(\alpha,\alpha)}$.
  Replacing all the roots with their coroots will reverse the arrows in the Dynkin
  diagram. The dual root system\index{root system!dual} is usually the same as the
  original, but is sometimes different. For example, $C_n$ and $B_n$ are dual.
 \end{remark}

 Now let's construct the exceptional root systems.

 We constructed $G_2$ when we classified rank two root systems on page
 \pageref{lec14Rmkrank2}.

 $F_4$\index{F4@$F_4$!construction of|idxbf} comes from some special properties of a
 cube in 4-space\anton{how so?}. Let $\varepsilon_1$, $\varepsilon_2$, $\varepsilon_3$,
 $\varepsilon_4$ be an orthonormal basis for $\RR^4$. Then let the roots be
 \[
 \left\{\pm(\varepsilon_i \pm \varepsilon_j), \pm \varepsilon_i, \frac{\pm
(\varepsilon_1\pm \varepsilon_2 \pm \varepsilon_3 \pm \varepsilon_4)}{2}\right\}
 \]
% These are orthogonal to some faces of the 4-cube.

 The simple roots are
 \[\begin{xy}
   (0,0) *{\e_1-\e_2} *+\frm{-};
   p+(2,0) *{\e_2-\e_3} *+\frm{-} **@{-};
   p+(1.5,0) *{\e_3} *+\frm{-} **@{=}; {?<>(.6)*@{>}},
   p+(2,0) *{\frac{\e_1 +\e_2 +\e_3 +\e_4}{2}} *+\frm{-} **@{-};
 \end{xy}
 \tag{$F_4$}\] There are 48 roots total. Remember that the dimension of the Lie
 algebra (which we have yet to construct) is the number of roots plus the dimension of the
 Cartan subalgebra (the rank of $\g$, which is 4 here), so the dimension is 52 in this
 case.

 To construct $E_8$\index{E8@$E_8$!construction of|idxbf}, look at $\RR^9$ with our
 usual orthonormal basis. The trick is that we are going to project on to the plane
 orthogonal to $\varepsilon_1+\cdots + \varepsilon_9$. The roots are
 \[
    \{\varepsilon_i - \varepsilon_j| i\neq j\} \cup \{\pm (\varepsilon_i+\varepsilon_j + \varepsilon_k)| i\neq j\neq k\}
 \]
 The total number of roots is $|\Delta| = 9\cdot 8 + 2\binom{9}{3} = 240$. So the
 dimension of the algebra is 248! The simple roots are
 \[\hspace*{-1em}\begin{xy}
   (0,0) *{\varepsilon_1 - \varepsilon_2} *+\frm{-};
   p+(1.7,0) *{\varepsilon_2 - \varepsilon_3} *+\frm{-} **@{-};
   p+(1.7,0) *{\varepsilon_3 - \varepsilon_4} *+\frm{-} **@{-};
   p+(1.7,0) *{\varepsilon_4 - \varepsilon_5} *+\frm{-} **@{-};
   p+(1.7,0) *{\varepsilon_5 - \varepsilon_6} *+\frm{-} **@{-};
   p+(0,-1) *{\varepsilon_6 + \varepsilon_7 + \varepsilon_8} *+\frm{-} **@{-},
   p+(1.7,0) *{\varepsilon_6 - \varepsilon_7} *+\frm{-} **@{-};
   p+(1.7,0) *{\varepsilon_7 - \varepsilon_8} *+\frm{-} **@{-};
 \end{xy}\tag{$E_8$}\]

 The root systems $E_6$ and $E_7$ are contained in the obvious way in the root system
 $E_8$.
 \begin{warning}\label{lec15Warn}
   Remember to project onto the orthogonal complement of $\e_1+\cdots+\e_9$. Thus,
   $\e_6+\e_7+\e_8$ is really $\frac{2}{3}(\e_6+\e_7+\e_8) -
   \frac{1}{3}(\e_1+\cdots+\e_5 + \e_9)$. There is another way to construct this root
   system, which is discussed in Lecture 25.
 \end{warning}
 \begin{exercise}
   Verify that $F_4$ and $E_8$ are root systems, and that the given sets are simple
   roots.
   \begin{solution}
     It is immediate to verify \hyperlink{RS1}{RS1} and \hyperlink{RS3}{RS3}. One may
     check that the proposed sets of simple roots are correct by checking that every
     root can be written as a non-positive or non-negative integer combination of the
     proposed simple roots. It is not hard to verify that the given root systems
     satisfy $r_\alpha(\Delta)=\Delta$ for each $\alpha\in \Delta$.

     Finally, it is enough to verify \hyperlink{RS2}{RS2} in the case where $\beta$ is
     a simple root. Since every root is an integer sum of simple roots, it is enough
     to consider the case where $\alpha$ is also a simple root. This amounts to
     checking that the given number of lines between $\alpha$ and $\beta$ is correct,
     which is relatively straightforward (keeping in mind Warning \ref{lec15Warn}).
   \end{solution}
 \end{exercise}

 We have now classified all indecomposable root systems. The diagram of the root
 system $\Delta_1\coprod \Delta_2$ is the disjoint union of the diagrams of $\Delta_1$
 and $\Delta_2$.

 \subsektion{Construction of the Lie algebras \texorpdfstring{$A_n$}{An},
 \texorpdfstring{$B_n$}{Bn}, \texorpdfstring{$C_n$}{Cn}, and
 \texorpdfstring{$D_n$}{Dn}}

 Next lecture, we will prove Serre's Theorem (Theorem
 \ref{lec16T:Serre})\index{Serre's Theorem}, which states that for each irreducible
 root system and for each algebraically closed field of characteristic zero, there is
 a unique simple Lie algebra with the given root system (it actually gives
 explicit generators and relations for this Lie algebra). Meanwhile, we will
 explicitly construct Lie algebras with the classical root systems.

 \underline{$A_n$}: Example \ref{lec14Eg:A_n} shows that $\sl(n+1)$ has root system
 $A_n$.

 \underline{$D_n$}:\index{Dn@$D_n$!and $\so(2n)$|idxbf}\index{so(2n)@$\so(2n)$}
 Consider $\so(2n)$, the Lie algebra of linear maps of $k^{2n}$ preserving some
 non-degenerate symmetric form. We can choose a basis for $k^{2n}$ so that the matrix
 of the form is $I = \matrix 0{1_n}{1_n}0$. Let $X\in \so(2n)$, then we have that $X^t
 I+IX=0$. It follows that $X$ is of the form
 \[
    X = \mat{A & B\\ C& -A^t}\text{, with $B^t=-B, C^t=-C$.}
 \]
 We guess\footnote{This will always be the right guess. The elements of the Cartan are
 simultaneously diagonalizable, so in some basis, the Cartan is exactly the set of
 diagonal matrices in the Lie algebra. The guess would be wrong if the Lie algebra did
 not have enough diagonal elements, but this would just mean that we had chosen the
 wrong basis.} that and element $H$ of the Cartan subalgebra should have the form
 $A=diag(t_1,\dots, t_n)$ and $B=C=0$ (to check this guess, it is enough to
 demonstrate that we get a root decomposition). To compute the root spaces,
 we try bracketing $H$ with various elements of $\so(2n)$. We have that $\matrix
 {E_{ij}}00{-E_{ji}}$ has eigenvalue $t_i-t_j$, that $\matrix 0{E_{ij}-E_{ji}}00$ has
 eigenvalue $t_i+t_j$, and that $\matrix 00{E_{ij}-E_{ji}}0$ eigenvalue $-t_i-t_j$.
 Since these matrices span $\so(2n)$, we know that we are done. Thus, $D_n$ is the
 root system of $\so(2n)$.

 \underline{$B_n$}:\index{Bn@$B_n$!and $\sp(2n)$|idxbf}\index{sp(2n)@$\sp(2n)$} Consider
 $\sp(2n)$, the linear operators on $k^{2n}$ which preserve a non-degenerate
 skew-symmetric form. In some basis, the form is $J = \matrix 0{1_n}{-1_n}0$, so an
 element $X\in \sp(2n)$ satisfies $X^tJ+JX=0$. It follows that $X$ is of the form
 \[
    X = \mat{A & B\\ C& -A^t}\text{, with $B^t=B, C^t=C$.}
 \]
 Let the Cartan subalgebra be the diagonal matrices. We get all the same roots as
 for $\so(2n)$, and a few more. $\matrix 0{E_{ii}}00$ has eigenvalue $2t_i$, and
 $\matrix 00{E_{ii}}0$ has eigenvalue $-2t_i$. Thus, $B_n$ is the root system of
 $\sp(2n)$.

 \underline{$C_n$}:\index{Cn@$C_n$!and $\so(2n+1)$|idxbf}\index{so(2n+1)@$\so(2n+1)$}
 Consider $\so(2n+1)$. Choose a basis so that the non-degenerate symmetric form is
 \[I = \left(\begin{array}{c|cc}
   1 & 0 & 0\\ \hline
   0 & 0 & 1_n\\
   0 & 1_n & 0\\
 \end{array}\right).
 \]
 Then an element $X\in \so(2n+1)$, satisfying $X^tI+IX=0$, has the form
 \[
  X = \left(\begin{array}{c|cc}
   0 & u & v \\ \hline
   -v^t & A & B\\
   -u^t & C & -A^t
  \end{array}\right)\text{, with $B^t=-B, C^t=-C$,}
 \]
 where $u$ and $v$ are row vectors of length $n$. Again, we take the Cartan subalgebra
 to be the diagonal matrices. We get all the same roots as we for $\so(2n)$, and a few
 more. If $e_i$ is the row vector with a one in the $i$-th spot and zeros elsewhere,
 then {\scriptsize $\left(\begin{array}{c|cc}
   0 & e_i & 0\\ \hline
   0 & 0 & 0 \\
   -e_i^t & 0 & 0
 \end{array}\right)$}
 has eigenvalue $t_i$, and
 {\scriptsize $\left(\begin{array}{c|cc}
   0 & 0 & e_i \\ \hline
   -e_i^t & 0 & 0 \\
   0 & 0 & 0
 \end{array}\right)$}
 has eigenvalue $-t_i$. Thus, $C_n$ is the root system of $\so(2n+1)$.

 \subsektion{Isomorphisms of small dimension} Let's say that we believe Serre's
 Theorem. Then you can see that for small $n$ some of the Dynkin diagrams coincide, so
 the corresponding Lie algebras are isomorphic.
 \[\begin{array}{ccc}
 B_2 = C_2 & D_2 = A_1\coprod A_1 & D_3=A_3 \\
 \begin{xy}
   (0,0) *{},
   (0,.5) *\cir<2pt>{};
   (1,.5) *\cir<2pt>{} **@{=} ?*@{>},
 \end{xy} & \begin{xy}
   (0,.1) *\cir<2pt>{},
   (0,.9) *\cir<2pt>{},
 \end{xy} &
 \begin{xy}
   (0,0) *\cir<2pt>{};
   a(30) *\cir<2pt>{} **@{-};
   (0,1) *\cir<2pt>{} **@{-},
 \end{xy}\\
 \quad \so(5)\simeq \sp(4) \quad & \quad \so(4)\simeq \sl(2)\times \sl(2)
 \quad & \quad \so(6)\simeq \sl(4) \quad
 \end{array}\]
 We can see some of these isomorphisms directly on the level of groups! Let's
 construct a map of groups $SL(2)\times SL(2) \to SO(4)$, whose kernel is discrete.
 Let $W$ be the space of $2\times 2$ matrices, then the $SL(2)\times SL(2)$ acts on
 $W$ by $(X,Y)w = XwY^{-1}$. This action preserves the determinant of $w=\matrix
 abcd$. That is, the quadratic form $ad-bc$ is preserved, so the corresponding
 non-degenerate bilinear form is preserved.\footnote{Given a quadratic form $Q$, one
 gets a symmetric bilinear form $(w,w'):=Q(w+w')-Q(w)-Q(w')$. In the case $Q(w)=\det
 w$, the form is non-degenerate. Indeed, assume $\det(w+w')=\det w + \det w$ for all
 $w$. Then by choosing a basis so that $w'$ is in Jordan form and letting $w$ vary
 over diagonal matrices, we see that $w'=0$.} The Lie group preserving such a form is
 $SO(4)$, so we have a map $SL_(2)\times SL(2)\to SO(4)$. It is easy to check that the
 kernel is the set $\{(I,I),(-I,-I)\}$, and since the domain and range each have
 dimension 6, we get $SL(2)\times SL(2)/(\pm I,\pm I)\cong SO(4)$ (we are also using
 that $SO(4)$ is connected). This yields an isomorphism on the level of Lie algebras.

 Now let's see that $\so(6)\simeq \sl(4)$. The approach is the same. Let $V$ be the
 standard 4 dimensional representation of $SL(4)$. Let $W = \Lambda^2 V$, which is a
 6 dimensional representation of $SL(4)$. Note that you have a pairing
 \[
    W\times W = \Lambda^2 V \times \Lambda^2 V \to \Lambda^4 V \overset{\det}{\simeq} k
 \]
 where the last map is an isomorphism of representations of $SL(4)$ (because the
 determinant of any element of $SL(4)$ is 1). Thus, $W=\Lambda^2 V$ has some
 $SL(4)$-invariant non-degenerate symmetric bilinear form, so we have a map $SL(4)\to
 SO(W)\simeq SO(6)$. It is not hard to check that the kernel is $\pm I$, and the
 dimensions match, so we get an isomorphism of Lie algebras.
}{   % Lilit Martirosyan, lilit@math
  \stepcounter{lecture}
 \setcounter{lecture}{16}
 \sektion{Lecture 16 - Serre's Theorem}\index{Serre's Theorem|(idxbf}
 \newcommand\jj{\mathfrak{j}}

 Start with a semisimple Lie algebra $\g$ over an algebraically closed field $k$ of
 characteristic zero, with Cartan subalgebra $\h\subset \g$. Then we have the root
 system $\Delta\subseteq \h^*$, with a fixed set of simple roots $\Pi =
 \{\alpha_1,\dots, \alpha_n\}$. We have a copy of $\sl_2$---generated by
 $X_i$, $Y_i$, and $H_i$---associated to each simple root.

 The \emph{Cartan matrix}\index{Cartan!matrix|idxbf} $(a_{ij})$ of $\g$ is given by
 $a_{ij} = \langle \check\alpha_i,\alpha_j\rangle = \alpha_j(H_i) =
 \frac{2(\alpha_i,\alpha_j)}{(\alpha_i,\alpha_i)}$. From the definition of coroots and
 from properties of simple roots, we know that $a_{ij} \in \ZZ_{\le 0}$ for $i\neq j$,
 that $a_{ii}=2$, and that $a_{ij}=0$ implies $a_{ji}=0$.

 \begin{claim}
   The following relations (called \emph{Serre relations}\index{Serre
   relations|idxbf}\footnote{Serre called them Weyl relations.}) are satisfied
   in $\g$. \hypertarget{Serre}
   {\begin{gather*}
     \begin{array}{lr}
       [H_i, X_j]=a_{ij} X_j & \text{\normalfont \quad(a)} \\
       {[H_i,Y_j]=-a_{ij}Y_j} & \text{\normalfont \quad(b)}
     \end{array}\qquad\qquad
     \begin{array}{lr}
       {[H_i, H_j] = 0} & \text{\normalfont \quad(c)}\\
       {[X_i,Y_j]=\delta_{ij}H_i} & \text{\normalfont \quad(d)}
     \end{array} \tag{Ser1}\\
     \begin{array}{l}
       \theta_{ij}^+ := (ad_{X_i})^{1-a_{ij}}X_j = 0 \\
       \theta_{ij}^- := (ad_{Y_i})^{1-a_{ij}}Y_j = 0
     \end{array}\text{, for }i\neq j. \tag{Ser2}
   \end{gather*}}
 \end{claim}
 \begin{proof}
   (Ser1a), (Ser1b), and (Ser1c) are immediate because $X_i\in \g_{\alpha_i}$, $Y_i\in
   \g_{-\alpha_i}$, and $H_i=[X_i,Y_i]\in \h$. To show (Ser1d), we need to show that
   $[X_i,Y_i]=0$ for $i\neq j$. This is because $[X_i,Y_j]\in \g_{\alpha_i-\alpha_j}$,
   which is not in $\Delta$ because every element of $\Delta$ is a non-negative or
   non-positive combination of the $\alpha_i$.

   Since $ad_{X_i}(Y_j)=0$, we get that $Y_j$ is a highest vector for the $\sl(2)$
   generated by $X_i$, $Y_i$, and $H_i$. We also have that $ad_{H_i}(Y_j)=
   -a_{ij}Y_j$. Thus, the $\alpha_i$-string\index{alpha-string@$\alpha$-string|idxit}
   through $Y_j$ is spanned by $Y_j$, $ad_{Y_i}Y_j$, \dots, $ad_{Y_i}^{-a_{ij}}Y_j$.
   In particular, $\theta^-_{ij}=ad_{Y_i}^{1-a_{ij}} Y_j=0$. Similarly,
   $\theta^+_{ij}=0$, so the relations (Ser2) hold.
 \end{proof}
 So far, all we know is that any Lie algebra with root system $\Delta$ satisfies these
 relations. We have yet to show that such an algebra exists, that it is unique, and
 that these relations define it.

% \begin{example}
%   $\g = \sl(n+1)$. We have $X_i = E_{i,i+1}, Y_i=E_{i+1,i}, H_i = [X_i,Y_i] = E_{ii} -
%   E_{i+1,i+1}$.
%
%  Let's check that the second group of relations hold. We want to show that
%  $(ad_{X_i})^{1-a_{ij}}X_j=0$. What is $a_{ij}$? We know that
%  $(ad_{X_i})^{1-a_{i,i+1}}X_{i+1} = 0$ by a calculation ... just do it.
% \end{example}

 \begin{theorem}[Serre's Theorem]\label{lec16T:Serre} Let $\Delta$ be a root system,
   with a fixed set of simple roots $\Pi=\{\alpha_1,\dots, \alpha_n\}$, yielding the
   Cartan matrix $a_{ij}=\frac{2(\alpha_i,\alpha_j)}{(\alpha_i,\alpha_i)}$. Let $\g$
   be the Lie algebra generated by $H_i, X_i,Y_i$ for $1\le i\le n$, with relations
   \textrm{(Ser1)} and \textrm{(Ser2)}. Then $\g$ is a finite dimensional semisimple Lie algebra with a
   Cartan subalgebra spanned by $H_1,\dots, H_n$, and with root system $\Delta$.
 \end{theorem}
 \begin{remark}\label{lec16Rmk:freelie}
   In order to talk about a Lie algebra given by certain generators and relations, it
   is necessary to understand the notion of a \emph{free Lie algebra}\index{Lie
   algebra!free} $L(X)$ on a set of generators $X$, which is non-trivial (because of
   the Jacobi identity). We define $L(X)$ as the Lie subalgebra of the tensor algebra
   $T(X)$ generated by the set $X$. This algebra has the universal property that for
   any Lie algebra $L'$ and for any function $f:X\to L'$, there is a \emph{unique}
   extension of $f$ to a Lie algebra homomorphism $\tilde f:L(X)\to L'$ (to
   prove this, one needs the PBW theorem\index{PBW|idxit}).

   To impose a set of relations $R$, quotient $L(X)$ by the smallest ideal containing
   $R$. The resulting Lie algebra $L(X,R)$ has the universal property that for any Lie
   algebra $L'$ and for any function $f:X\to L'$ such that the image satisfies the
   relations $R$, there is a \emph{unique} extension of $f$ to a Lie algebra
   homomorphism $\tilde f:L(X,R)\to L'$.
 \end{remark}
 \begin{remark}
   Serre's Theorem proves that for any root system $\Delta$ there is a finite
   dimensional semisimple Lie algebra $\g$ with root system $\Delta$. But since any
   other Lie algebra $\g'$ with root system $\Delta$ satisfies (Ser1) and (Ser2), and
   since $\g$ is the universal Lie algebra satisfying these relations, we get an
   induced Lie algebra homomorphism $\phi:\g\to \g'$. This homomorphism is surjective
   because $\g'$ is spanned by $\phi(X_i),\phi(Y_i)$, and $\phi(H_i)$. Moreover, both
   $\dim \g$ and $\dim \g'$ must be equal to $|\Delta|+$rank$(\Delta)$, so $\phi$ must
   be an isomorphism. Therefore, we get uniqueness of $\g$.
 \end{remark}

 \begin{proof}[Proof of Serre's Theorem]{\ }

    \underline{Step 1. Decompose $\tilde\g$}: Consider the free Lie algebra with
    generators $X_i$, $Y_i$, $H_i$ for $1\le i\le n$ and impose the relations (Ser1).
    Call the result $\tilde \g$. Let $\h$ be the abelian Lie subalgebra generated by
    $H_1,\dots, H_n$, and let $\tilde \n^+$ (resp.\ $\tilde \n^-$) be the Lie
    subalgebra generated by the $X_i$ (resp.\ $Y_i$). The goal is to show that $\tilde
    \g = \tilde \n^- \oplus \h \oplus \tilde \n^+$ as a vector space.

%    , where $\h$ is the abelian Lie algebra generated by $H_1,\dots,
%    H_n$, and $\tilde \n^{\pm}$ are free Lie algebras generated by the $X_i$ (resp.\
%    $Y_i$). They should be free because there are no relations among the $X_i$'s, but
%    it is not so easy.
%%%%%%%%%%%%%%%%%%%%%%%%%%%%%%%%%%%%%%%%%%%%%%%%%%%%%%%%%%%%%%%%%%%%%%%%%%%%%%%%%%%%

    There is a standard trick for doing such things. It is easy to see from
    \hyperlink{Serre}{(Ser1)} that $U\tilde \g = U\tilde \n^- \cdot U\h\cdot U\tilde
    \n^+$.\footnote{By $U\tilde\n^-\cdot U\h\cdot U\tilde\n^+$, we mean the set of
    linear combinations of terms of the form $y\cdot h\cdot x$, where $y\in
    U\tilde\n^-$, $h\in U\h$, and $x\in U\tilde\n^+$.} Let $T(X)$ be the tensor
    algebra on the $X_i$, let $T(Y)$ be the tensor algebra on the $Y_i$, and let $S\h$
    be the symmetric algebra on the $H_i$. We define a representation $U\tilde \g\to
    \End\bigl(T(Y) \otimes S\h\otimes T(X)\bigr)$. For $a\in T(Y)$, $b\in S\h$, and
    $c\in T(X)$, define
    \begin{align*}
      X_i(1\otimes 1\otimes c) &= 1\otimes 1\otimes X_ic,\\
      H_i(1\otimes b\otimes c) &= 1\otimes H_ib\otimes c,\text{ and}\\
      Y_i(a\otimes b\otimes c) &= (Y_ia)\otimes b\otimes c.
    \end{align*}
    Then extend inductively by
    \begin{align*}
      H_i(Y_ja\otimes b\otimes c) &= Y_jH_i(a\otimes b\otimes c) - a_{ij}Y_j(a\otimes
              b\otimes c)\\
      X_i(1\otimes H_jb\otimes c) &= H_jX_i(1\otimes b\otimes c) - a_{ji}X_i(1\otimes
              b\otimes c)\\
      X_i(Y_ja\otimes b\otimes c) &= Y_jX_i(a\otimes b\otimes c) +
              \delta_{ij}H_i(a\otimes b\otimes c).
    \end{align*}
    \begin{exercise}
      Check that this is a representation.
      \begin{solution}
        It is enough to check that the proposed endomorphisms of $T(Y)\otimes
        S\h\otimes T(X)$ satisfy \hyperlink{Serre}{(Ser1)}. Then the universal
        property $\tilde \g$ (from Remark \ref{lec16Rmk:freelie}) and the universal
        property of $U\tilde \g$ (from Proposition \ref{lec07P:Ug}) tell us exactly
        that there is a unique algebra homomorphism $U\tilde\g \to
        \End\bigl(T(Y)\otimes S\h\otimes T(X)\bigr)$ such that $X_i$, $Y_i$, and $H_i$
        act as described. We get (Ser1a), (Ser1b), and (Ser1d) by construction. We
        need only check that $H_iH_j$ acts in the same way as $H_jH_i$. It is clear
        that $H_iH_j(1\otimes b\otimes c)=H_jH_i(1\otimes b\otimes c)$. Now we induct
        on the degree of $a$.
        \begin{align*}
          H_iH_j(Y_ka\otimes b\otimes c)&= (H_iY_kH_j - a_{jk}H_iY_k)(a\otimes b\otimes
                    c) & \text{(Ser1b)}\\
                &= (Y_kH_iH_j - a_{ik}Y_kH_j \\
                &\phantom{= (H_iY_kH_j }\; - a_{jk}Y_kH_i +
                    a_{jk}a_{ik}Y_k)(a\otimes b\otimes c) &\text{(Ser1b)}\\
                &= H_jH_i(Y_ka\otimes b \otimes c) &\llap{($i$, $j$ symmetric)}
        \end{align*}
        This shows that the representation is well defined.
      \end{solution}
    \end{exercise}
    Observe that the canonical (graded vector space) homomorphism $T(Y)\otimes
    S\h\otimes T(X)\to U\tilde\n^-\cdot U\h\cdot U\tilde\n^+ = U\tilde \g$ is the
    inverse of the map $w\mapsto w(1\otimes 1\otimes 1)$, so $U\tilde\g \simeq
    T(Y)\otimes S\h\otimes T(X)$ as graded vector spaces.\footnote{$U\tilde\g$ is
    graded as a vector space, but only \emph{filtered} as an algebra.} Looking at the
    degree 1 parts, we get the vector space isomorphism
    $\tilde\g\simeq\tilde\n^-\oplus \h\oplus\tilde\n^+$.

%%%%%%%%%%%%%%%%%%%%%%%%%%%%%%%%%%%%%%%%%%%%%%%%%%%%%%%%%%%%%%%%%%%%%%%%%%%%%%%%%%%%
%{
%    There is a standard trick for doing such things. It is easy to see from the
%    relations (Ser1) that $U\tilde\g = U(\tilde \n^-)U(\h)U(\tilde \n^+)$. In fact, we
%    want $U(\tilde \g) =\underbrace{U(\tilde\n^-)}_{T(\tilde\n^-)}\otimes
%    \underbrace{U(\h)}_{S(\h)}\otimes \underbrace{U(\tilde\n^+)}_{T(\tilde\n^+)}$. We
%    construct a representation $U\tilde\g\to \End\bigl(T(\tilde \n^-)\otimes
%    S(\h)\bigr)$.
%
%    For $b\in S(\h)$, define $Y_i(1\otimes b)=Y_i\otimes b$, $H_i(1\otimes b)=1\otimes
%    H_ib$, and $X_i(1\otimes b)=0$. Then for $a\in T(\tilde n^-)$, define
%    \begin{align*}
%    Y_i (Y_ja\otimes b) &= (Y_iY_ja)\otimes b\\
%    H_i(Y_ja\otimes b)  &= Y_j(H_i(a\otimes b)) - a_{ij}Y_j(a\otimes b)\\
%    X_i(Y_ja\otimes b)  &= Y_j(X_i(a\otimes b)) + \delta_{ij} H_i(a\otimes b)
%    \end{align*}
%    \begin{exercise}
%      \mpar[\noindent \raggedright What we did in class \hfill \smash{\rule[-1.85in]{.6pt}{4.2in}}]{}
%      Check that this is a representation, and that $1\otimes 1$ generates the
%      representation.
%      \begin{solution}\anton{I can do this, but it is long}
%        It is clear that $1\otimes 1$ generates. To check that this is a
%        representation, we must check that the relations \hyperlink{Serre}{(Ser1)}
%        hold in the image. We get (Ser1b) and (Ser1d) immediately, and (Ser1c) holds
%        because the second factor is the \emph{symmetric} algebra on $\h$. It remains
%        to show (Ser1a), which is painful. First we check that
%        \begin{align*}
%          X_iH_j(1\otimes b) &= X_i(1\otimes H_jb) = 0\\
%                &= H_jX_i(1\otimes b) - a_{ji}X_i(1\otimes b).
%        \end{align*}
%        Now we induct on the degree of $a$,
%        {\newcommand{\ab}{\mbox{\relsize{-1} $(a\otimes b)$}}
%         \newcommand{\uthinsp}{\kern .055 em }
%        \begin{align*}%\hspace*{-2em}
%          \!\! X_iH_j(Y_ka\otimes b) &= ( X_iY_kH_j - a_{jk}X_iY_k )\ab & \text{(Ser1b)}\\
%            &= (Y_kX_iH_j+\delta_{ik}H_iH_j - a_{jk}Y_kX_i - \delta_{ik}a_{jk}H_i )\ab &
%              \text{(Ser1d)}\\
%            &= ( \underline{\underline{Y_kH_jX_i}}
%                 - a_{ji}Y_kX_i + &\llap{(induction)}\\
%            &\phantom{= (Y_kX_iH_j}\;
%                 +\underbrace{\delta_{ik}H_iH_j}\uthinsp
%                 \underline{-\: a_{jk}Y_kX_i}\uthinsp
%                 \underbracket{-\: \delta_{ik}a_{jk}H_i})\ab\\
%      \!\!(H_jX_i-a_{ji}X_i\rlap{$)(Y_k a\otimes b)$}\\
%            &= (H_jY_kX_i + \delta_{ik}H_jH_i - a_{ji}Y_kX_i -\delta_{ik}a_{ji}H_i)\ab
%              & \text{(Ser1d)}\\
%            &= ( \underline{\underline{Y_kH_jX_i}}\:
%                 \underline{-\: a_{jk}Y_kX_i} &\text{(Ser1b)}\\
%            &\phantom{= (H_jY_kX_i}\;
%                 +\underbrace{\delta_{ik}H_jH_i}\uthinsp
%                 -\: a_{ji}Y_kX_i\uthinsp
%                 \underbracket{-\:\delta_{ik}a_{ji}H_i}
%            )\ab
%        \end{align*}}
%        Note that $\delta_{ik}a_{jk}=\delta_{ik}a_{ji}$.
%      \end{solution}
%    \end{exercise}
%    So we get an isomorphism of $U(\tilde \n^-)$ and $T(\tilde \n^-)$. We define
%    $U(\tilde \n^- + \h) \twoheadrightarrow T(\tilde \n^-)\otimes S(\h)$ given by
%    $x\mapsto x(1\otimes 1)$. We do the same thing for $\tilde \n^+$.\mpar{\vspace{2cm}\anton{I don't see how this part works.}}
%   \begin{gather*}
%    0\to U(\tilde \n^+) \to  U(\tilde \g) \to T(\tilde \n^-)\otimes S(\h)\to 0\\
%    0\to U(\tilde \n^-) \to  U(\tilde \g) \to T(\tilde \n^+)\otimes S(\h)\to 0
%   \end{gather*}
%  Hence $U(\tilde \g) \simeq T(\tilde \n^-)\otimes S(\h)\otimes T(\tilde \n^+)$. So
%  $\tilde \g = \tilde \n^-\oplus \h\oplus \tilde \n^+$.
%}
%%%%%%%%%%%%%%%%%%%%%%%%%%%%%%%%%%%%%%%%%%%%%%%%%%%%%%%%%%%%%%%%%%%%%%%%%%%%%%%%%%%%%%%%

  \underline{Step 2. Construct $\g$}: We have that $\theta_{ij}^\pm \in \tilde \n^\pm$. Let
  $\jj^+$ (resp.\ $\jj^-$) be the ideal in $\tilde\n^+$ (resp.\
  $\tilde\n^-$) generated by the set $\{\theta_{ij}^+\}$ (resp. $\{\theta_{ij}^-\}$).
  \begin{exercise}
    Check that
    \[
        [Y_k,\theta_{ij}^+]=0 \text{\qquad and \qquad} [H_k,\theta_{ij}^+]=c_{kij}\theta_{ij}^+
    \]
    for some constants $c_{kij}$. Therefore, $\jj^\pm$ are ideals in $\tilde \g$.
    \begin{solution}
      It is easy to check by induction that in $U\tilde \g$,
      \begin{align*}
        H_k X_i^r &= X_i^r H_k + ra_{ki}X_i^r\text{, and}\\
        Y_kX_i^r &= X_i^rY_k - r\delta_{ik}\bigl(X_i^{r-1}H_i +(r-1)X_i^{r-1}\bigr).
      \end{align*}
      Since $ad$ is a representation, it follows that
      \begin{align*}
        [H_k,\theta_{ij}^+] &= ad_{H_k}ad_{X_i}^{1-a_{ij}}X_j\\
            &= ad_{X_i}^{1-a_{ij}} ad_{H_k}X_j +
            (1-a_{ij})a_{ki}ad_{X_i}^{1-a_{ij}}X_j\\
            &= (a_{kj}+a_{ki}-a_{ij}a_{ki})\theta_{ij}^+\\
        [Y_k,\theta_{ij}^+] &= ad_{X_i}^{1-a_{ij}}[Y_k,X_j] = 0&\text{(if $k\neq
                j$)}\\
        [Y_j,\theta_{ij}^+] &=
            ad_{X_i}^{1-a_{ij}}\overbrace{[Y_j,X_j]}^{-H_j}
            -(1-a_{ij})ad_{X_i}^{-a_{ij}}\overbrace{[H_i,X_j]}^{a_{ij}X_j} \\
            &\phantom{=ad_{X_i}^{1-a_{ij}}[Y_j,X_j]}
                +(1-a_{ij})a_{ij}ad_{X_i}^{-a_{ij}}X_j\\
            &=a_{ji}ad_{X_i}^{-a_{ij}}X_i
      \end{align*}
      which is zero if $a_{ij}=a_{ji}=0$, and is zero if $a_{ij}<0$.
    \end{solution}
  \end{exercise}
  Now define $\n^+=\tilde\n^+/\jj^+$, $\n^-=\tilde\n^-/\jj^-$, and $\g =
  \tilde\g/(\jj^++\jj^-) = \n^-\oplus \h\oplus \n^+$. From relations
  \hyperlink{Serre}{(Ser1)}, we know that $\h$ acts diagonalizably on $\n^+$, $\n^-$,
  and $\h$, so we get the decomposition $\g = \h\oplus \bigoplus_{\alpha\in \h^*}
  \g_\alpha$, where $\g_\alpha = \{x\in \g| [h,x]=\alpha(h)x\}$. Note that each
  $\g_\alpha$ is either in $\n^+$ or in $\n^-$.

  Define $R$ as the set of non-zero $\alpha\in \h^*$ such that $\g_\alpha\neq 0$. We
  know that $\pm \alpha_1,\dots,\pm \alpha_n\in R$ because $X_i\in \g_{\alpha_i}$ and
  $Y_i\in \g_{-\alpha_i}$. Since each $\g_\alpha$ is either in $\n^+$ or in $\n^-$,
  $\alpha$ must be a non-negative or a non-positive combination of the $\alpha_i$
  (recalling that $[\g_\alpha,\g_\beta]\subseteq \g_{\alpha+\beta}$). This gives us
  the decomposition $R = R^+\amalg R^-$.

  Since $\g$ is generated by the $X_i$ and $Y_i$, the relation
  $[\g_\alpha,\g_\beta]\subseteq \g_{\alpha+\beta}$ tells us that $R$ is contained in
  the lattice $\sum_{i=1}^n \ZZ\alpha_i$. Since $\n^+=\bigoplus_{\alpha\in
  R^+}\g_\alpha$ is a quotient of $\tilde\n^+$, it is generated as Lie algebra by the
  $X_i$. Together with the relation $[\g_\alpha,\g_{\beta}]\subseteq
  \g_{\alpha+\beta}$ and the linear independence of the $\alpha_i$, this tells us that
  $\g_{\alpha_i}$ is one dimensional, spanned by $X_i$, and that
  $\g_{n\alpha_i}=[\g_{\alpha_i},\g_{(n-1)\alpha_i}]=0$ for $n>1$.

 \underline{Step 3. $R$ is $\weyl$-invariant}: Let $\weyl$ be the Weyl group of the
 root system $\Delta$, generated by the simple reflections $r_i:\lambda\mapsto
 \lambda-\lambda(H_i)\alpha_i$. We would like to show that $R$ is invariant under the
 action of $\weyl$. To do this, we need to make sense of the element $s_i =
 \exp(ad_{X_i})\exp(-ad_{Y_i})\exp(ad_{X_i})\in \aut \g$.

 The main idea is that \hyperlink{Serre}{(Ser1)} and
 \hyperlink{Serre}{(Ser2)} imply that $ad_{X_i},ad_{Y_i}$ are locally nilpotent
 operators on $\g$.\footnote{An operator $A$ on $V$ is \emph{locally nilpotent} if for
 any vector $v\in V$, there is some $n(v)$ such that $A^{n(v)}v=0$.} The Serre
 relations say that $ad_{X_i}$ and $ad_{Y_i}$ are nilpotent on generators, and then
 the Jacobi identity implies that they are locally nilpotent. Thus,
 $s_i=\exp(ad_{X_i})\exp(-ad_{Y_i})\exp(ad_{X_i})$ is a well-defined automorphism of
 $\g$ because each power series is (locally) finite.

 As in Exercise \ref{lec14Ex:Weyl}, we get $s_i(\h)\subseteq \h$ and
 \begin{equation}\label{lec16dag}
  \lambda\bigl( s_i(h)\bigr) = \langle \lambda, s_i(h) \rangle =\langle r_i(\lambda),h\rangle
  = (r_i \lambda)(h)
 \end{equation}
 for any $h\in \h$ and any $\lambda\in \h^*$.

 Now we are ready to show that $R$ is $\weyl$-invariant. If $\alpha\in R$, with $X\in
 \g_\alpha$, then we will show that $s_i^{-1}X$ is a root vector for $r_i\alpha$. For
 $h\in \h$, we have
 \begin{align*}
   [h,s_i^{-1}X] &= s_i^{-1}([s_i h,X]) = s_i^{-1}\bigl( \alpha(s_i h)X\bigr) \\
        &= \alpha(s_ih)\,s_i^{-1}X = (r_i\alpha)(h)\,s_i^{-1}X, &\text{(by \ref{lec16dag})}
 \end{align*}
 so $r_i\alpha\in R$. So $\weyl$ preserves $R$.

 On the other hand, we know that $\pm \alpha_i \subseteq R$ from the end of Step 2, so
 we get $\Delta\subseteq R$. Moreover, for any $\alpha\in \Delta$, we have that $\dim
 \g_\alpha = 1$ because we can choose $w=r_{i_1}\cdots r_{i_k}$ and $s=s_{i_1}\cdots
 s_{i_k}$ so that $\alpha = w(\alpha_i)$ for some $i$; then $\g_\alpha =
 s(\g_{\alpha_i})$ has dimension one by the last sentence of Step 2.

 \underline{Step 4. Prove that $\Delta=R$}: Let $\lambda\in R\smallsetminus \Delta$.
 Then $\lambda$ is not proportional to any $\alpha\in \Delta$. One can find some $h$
 in the real span of the $H_i$ such that $\langle \lambda, h\rangle =0$ and $\langle
 \alpha, h\rangle \neq 0$ for all $\alpha\in \Delta$. This decomposes $\Delta$ as
 $\Delta^{+'}\coprod \Delta^{-'}$, and gives a new basis of simple roots
 $\{\beta_1,\dots, \beta_n\} = \Pi'\subseteq \Delta^{+'}$. By Proposition
 \ref{lec14P:tansitive}, $\weyl$ acts transitively on the sets of simple roots, so we can
 find some $w\in \weyl$ such that $w(\alpha_i)=\beta_i$ (after permutation of the
 $\beta_i$, if necessary). Then look at $w^{-1}(\lambda)\in R$.

 By construction $\lambda$ is neither in the non-negative span nor the non-positive
 span of the $\beta_i$, so $w^{-1}(\lambda)$ is neither in the non-negative nor the
 non-positive span of the $\alpha_i$. But we had the decomposition $R = R^+ \coprod
 R^-$ from Step 2, so this is a contradiction. Hence $\Delta = R$.

 \underline{Step 5. Check that $\g$ is semisimple}: It is enough to show that $\h$ has
 no nontrivial abelian ideals. We already know that $\g = \h\oplus
 \bigoplus_{\alpha\in \Delta} \g_\alpha$ and that each $\g_\alpha$ is 1 dimensional.
 In particular, $\g$ is \emph{finite dimensional}. We also know that the Serre
 relations hold. Notice that for any ideal $\a$, $ad_\h$-invariance implies that $\a =
 \h'\oplus_{\alpha \in S} \g_\alpha$ for some subspace $\h'\subseteq \h$ and some
 subset $S\subseteq \Delta$. If $\g_\alpha\subseteq \a$, then $X_\alpha\in \a$, so
 $[X_\alpha,Y_\alpha]=H_\alpha\in \a$ ($\a$ is an ideal), and
 $[Y_\alpha,H_\alpha]=2Y_\alpha\in \a$. Thus, we have the whole $\sl(2)$ in $\a$, so
 it cannot be abelian. So $\a = \h'\subseteq \h$. Take a nonzero element $h\in \h'$.
 Since $\{\alpha_1,\dots, \alpha_n\}$ spans $\h^*$, there is some $\alpha_i$ with
 $\alpha_i(h)\neq 0$, then $[h,X_i]=\alpha_i(h)X_i\in \a$, contradicting $\a\subseteq
 \h$.
 \end{proof}

 In the non-exceptional cases, we have nice geometric descriptions of these Lie
 algebras. Next time, we will explicitly construct the exceptional Lie algebras.

 \index{Serre's Theorem|)idxbf}
}{   % Santiago Canez, scanez@math
  \stepcounter{lecture}
 \setcounter{lecture}{17}
 \sektion{Lecture 17 - Constructions of Exceptional simple Lie Algebras}\anton{We
 skipped much of this section during editing ... come back to it}

 We'll begin with the construction of $G_2$.

 \anton{remark about octonians goes here. see FH pp 363}
 We saw here that $G_2$ is isomorphic to the Lie algebra of automorphisms of a generic 3-form in
 7 dimensional space. The picture of the projective plane is related to Cayley
 numbers,
%% If you try to come up with an 8 dimensional algebra which is not
%% associative, the multiplication is given by the lines in the plane. What is
%% interesting is that any line plus the 1 gives a copy of the quaternions. This is
 an important nonassociative division algebra, of which  $G_2$ is the algebra of automorphisms.


 Consider the picture
 \[ \newcommand\thirdrootthree{.57735027}
 \begin{xy}<3.75em,0em>:
   (0,0) ="v1"  *+!R{v_1} *{\bullet},
   (1,0) ="w2"  *+!U{w_2} *{\bullet},
   (2,0) ="v3"  *+!L{v_3} *{\bullet},
   a(60) ="w3"  *+!DR{w_3} *{\bullet},
   "w3"+a(60) ="v2" *+!D{v_2} *{\bullet},
   "v3"+a(120)="w1" *+!DL{w_1} *{\bullet},
   (1,\thirdrootthree)="u" *+!(-.15,-.27){u} *{\bullet},
   "v1";"v2" **@{-};{?(.75)*@{>}},
   "v3" **@{-};{?(.75)*@{>}},
   "v1" **@{-};{?(.75)*@{>}},
   "w1" **@{-}, {?(.5)*@{>}},
   "w3";"v3" **@{-}, {?(.5)*@{<}},
   "w2";"v2" **@{-}, {?(.5)*@{<}},
   "u", *\xycircle(\thirdrootthree,\thirdrootthree){},
    *{\rotatebox{20}{\xycircle(\thirdrootthree,\thirdrootthree){=<5em>@{<}}}}
 \end{xy}\]
 This is the projective plane over $\FF_2$.

 Consider the standard 7 dimensional representation of $\gl(7)$, call it
 $E$. Take a basis given by points in the projective plane above.

 \begin{exercise} \label{lec17Ex1}
    Consider the following element of $\Lambda^3 E$.
    \[
    \w=v_1\wedge v_2\wedge v_3 +
    w_1\wedge w_2\wedge w_3 + u\wedge v_1\wedge w_1 + u\wedge v_2 \wedge w_2 + u\wedge
    v_3\wedge w_3
    \]
    Prove that $\gl(7)\w=\Lambda^3 E$.
    \begin{solution}
      It is enough to show that each basis vector of $\Lambda^3 E$ is in the orbit of
      $\w$. Let $p_u$, $p_v$, and $p_1$ be the projections onto $span\{u\}$,
      $span\{v_1,v_2,v_3\}$, and $span\{v_1,w_1\}$ respectively. For $x,y\in
      S:=\{u,v_1,v_2,v_3,w_1,w_2,w_3\}$, let $\phi_{x\to y}$ be the element of
      $\gl(7)$ sending $x$ to $y$, and sending the rest of $S$ to zero. Then a little
      messing around produces
      \[\renewcommand\arraystretch{1.2}
      \begin{array}{cc|cc}
        x & x\cdot \w & x & x\cdot \w\\ \hline
        \frac{1}{3}(p_v-p_u) & v_1\wedge v_2\wedge v_3 & \phi_{v_1\to u} & u\wedge v_2\wedge v_3\\
        \half p_1 + \half p_u - \frac{1}{6}\id & u\wedge v_1\wedge w_1 & \phi_{w_1\to w_2} & u\wedge v_1\wedge w_2\\
        \phi_{v_3\to w_1}+\phi_{w_2\to u} & v_1\wedge v_2\wedge w_1 & \phi_{v_3\to w_3} & v_1\wedge v_2\wedge w_3
      \end{array}\]
      Any other basis vector can be obtained from one of these (up to a sign) by
      permuting indices and/or swapping $v$'s and $w$'s, so we can get all of
      them.\footnote{Since $\w$ is not quite invariant under permutations of the
      indices or swapping of $v$'s and $w$'s, you sometimes have to tweak a sign
      (e.g.\ to get $w_1\wedge w_2\wedge v_1$, take $x=\phi_{w_3\to v_1}-\phi_{v_2\to
      u}$).}
    \end{solution}
 \end{exercise}
 \begin{warning}
   Don't forget that $\gl(7)$ acts on $\Lambda^3 E$ as a \emph{Lie algebra}, not as an
   associative algebra. That is, $x(a\wedge b\wedge c)=(xa)\wedge b\wedge c + a\wedge
   xb\wedge c + a\wedge b\wedge xc$. In particular, the action of $x$ followed by the
   action of $y$ is \emph{not} the same as the action of $yx$.
 \end{warning}

 \begin{claim}
   $\g = \{x\in \gl(7)|x\w=0\}$ is a simple Lie algebra with root system $G_2$.
 \end{claim}
 \begin{proof}
   It is immediate that $\g$ is a Lie algebra\anton{How do we know it is simple?}.
   Let's pick a candidate for the Cartan subalgebra. Consider linear operators which
   are diagonal in the given basis $u,v_1,v_2,v_3,w_1,w_2,w_3$,
   take $h = diag(c,a_1,a_2,a_3,b_1,b_2,b_3)$. If we want $h\in \g$, we must have
   \begin{align*}
    h\w &= (a_1+a_2+a_3)v_1v_2v_3 + (b_1+b_2+b_3)w_1w_2w_3 + \\
        & \quad + (c+a_1+b_1)uv_1w_1 + (c+a_2+b_2)uv_2w_2 + (c+a_3+b_3)uv_3w_3 = 0
   \end{align*}
   which is equivalent to $c=0$, $a_1+a_2+a_3=0$,
   and $b_i=-a_i$. So our Cartan subalgebra is two dimensional.

   If you consider the root diagram for $G_2$ and look at only the long roots, you get
   a copy of $A_2$. This should mean that you have an embedding $A_2\subseteq
   G_2$,\footnote{Note that this is not true of the short roots because the bracket of
   elements in ``adjacent'' short root spaces produces an element in a long root
   space, so the short root spaces will generate all of $G_2$.} so we should look for
   a copy of $\sl(3)$ in our $\g$. We can write $E = ku\oplus V\oplus W$, where
   $V=\langle v_1,v_2,v_3\rangle$ and $W=\langle w_1,w_2,w_3\rangle$. Let's consider
   the subalgebra which not only kills $\w$, but also kills $u$. Let $\g_0 = \{x\in
   \g| xu=0\}$.

   \anton{begin my poor explanation}

   Say $x\in \g_0$ is of the form
   \[
    x = \left(\begin{array}{c|cc}
      0 & a & b\\ \hline
      0 & A & B\\
      0 & C & D\\
    \end{array}\right),
   \]
   where $a,b$ are row vectors, then
   \[
     0 = x\cdot \w =
       \underset{\shortstack{$vvv$\\
                      $\fbox{$uvv$}_3$\\
                      $\fbox{$wvv$}_1$}}{x(v_1v_2v_3)}
     + \underset{\shortstack{$www$\\
                      $\fbox{$uww$}_3$\\
                      $\fbox{$vww$}_1$}}{x(w_1w_2w_3)}
     + \underset{\shortstack{$\fbox{$uvw$}_4$\\
        \rlap{\rule[2.5pt]{4.5ex}{.4pt}}$uuw$ \\
        \rlap{\rule[2.5pt]{4ex}{.4pt}}$uuv$ \\
                 $\fbox{\shortstack{$uvv$ \\
                                    $uww$}}_{2}$}}{u\wedge x(v_1w_1+ v_2w_2+v_3w_3)}
   \]
   where each term lies in the span of the basis vectors below it. Since the terms in
   boxes labelled 1 appear in only one way, we must have $B=C=0$. From that, it
   follows that the terms boxed an labelled $2$ cannot appear. Thus, the terms in
   boxes labelled 3 only appear in one way, so we must have $a=b=0$. Since the terms
   in boxes labelled 2 appear in only one place (though in two ways), we must have
   $D=-A^t$. Finally, since $vvv$ only appears in one place (in three different ways),
   we must have $tr\, A=0$.

   \anton{end poor explanation. Is there a better way?}

   For $x\in \g_0$ we have $x(v_1\wedge v_2\wedge v_3)=0$ and
   $x(w_1\wedge w_2\wedge w_3)=0$, so $x$ preserves $V$ and $W$. It also must kill the
   2-form $\alpha = v_1\wedge w_1+ v_2\wedge w_2+ v_3\wedge w_3$, since
   $0=x(u\wedge\alpha)=xu\wedge\alpha+u\wedge x\alpha=u\wedge x\alpha$ forces $x\alpha
   =0$. This 2-form gives a pairing, so that $V^* \simeq W$. We can compute exactly
   what the pairing is, $\langle v_i,w_j\rangle =\delta_{ij}$. Therefore the operator
   $x$ must be of the form
   \[
    x = \left(\begin{array}{c|cc}
      0 & 0 & 0\\ \hline
      0 & A & 0\\
      0 & 0 & -A^t\\
    \end{array}\right), \text{ where }tr(A)=0.
   \]

  %% Why does $x$ preserve the form? It is a non-degenerate 2-form on $V\oplus W$, so it
  %% identifies it with $V^*\oplus W^*$. And $V$ and $W$ are isotropic with respect to
  %% this form (lets call the form $\alpha$).  Then we have that $\langle w,v\rangle =
  %% \alpha(v,w)$. $x$ preserves $\alpha$ ... to show that, we need
  %% $\alpha(Av,w)+\alpha(v,-A^t w)=0$.
  %% this is showing that anyone of that block matrix form preserves \alpha %%


   The total dimension of $G_2$ is 14, which we also know by the exercise is the
   dimension of $\g$. We saw the Cartan subalgebra has dimension 2, and this $\g_0$
   piece has dimension 8 (two of which are the Cartan). So we still need another 6
   dimensional piece.

   For each $v\in V$, define a linear operator $X_v$ which acts by $X_v(u)=v$.
   $\Lambda^2 V\simeq_\gamma W$ is dual to $V$ (since $\Lambda^3 V=\CC$). Then $X_v$
   acts by $X_v(v')= \gamma(v\wedge v')$ and $X_v(w)=2\langle v,w\rangle u$. Check
   that this kills $\w$, and hence is in $\g$.\anton{clarify how you'd think if this
   $X_v$\dots what constraints does being in a root space put on $X_v$?}

   Similarly, you can define $X_w$ for each $w\in W$ by $X_w(u)=w,
   X_w(w')=\gamma(w\wedge w'), X_w(v) = 2\langle w,v\rangle u$.

   If you think about a linear operator which takes $u\mapsto v_i$, it must be
   in some root space, this tells you about how it should act
   on $V$ and $W$. This is how we constructed $X_v$ and $X_w$, so we know that
   $X_{v_i}$ and $X_{w_i}$ are in some root spaces. We can check that their roots
   are the short roots in the diagram,

   \[\begin{xy}
   (0,0);(0,0);
   \ar a(30) ="a" *+!L{X_{v_1}},
   \ar a(150)-"a"="b",
   \ar "b"+"a" *+!R{X_{v_2}},
   \ar "b"+"a"+"a" *+!D{X_{w_3}},
   \ar "b"+"a"+"a"+"a",
   \ar "b"+"b"+"a"+"a"+"a",
   \ar -"a" *+!R{X_{w_1}},
   \ar -"b",
   \ar -"b"-"a" *+!L{X_{w_2}},
   \ar -"b"-"a"-"a" *+!U{X_{v_3}},
   \ar -"b"-"a"-"a"-"a",
   \ar -"b"-"b"-"a"-"a"-"a",
 \end{xy}\]
   and so they span the remaining 6 dimensions of $G_2$.
   To properly complete this construction, we should check that this is
   semisimple, but we're not going to.
 \end{proof}

 Let's analyze what we did with $G_2$, so that we can do a similar thing to construct $E_8$.
 We discovered certain phenomena,
 we can write $\g = \underbrace{\g_0}_{\sl(3)}\oplus \underbrace{\g_1}_V\oplus
 \underbrace{\g_2}_W$. This gives us a $\ZZ/3$-grading: $[\g_i,\g_j]\subseteq
 \g_{i+j (mod\ 3)}$. As an $\sl(3)$ representation, it has three components: $ad$, standard,
 and the dual to the standard. We get that $W\cong V^*\simeq \Lambda^2 V$. Similarly,
 $V\simeq \Lambda^2 W$. This is called \emph{Triality}\index{triality|idxbf}.

 More generally, say we have $\g_0$ a semisimple Lie algebra, and $V,W$ representations of
 $\g_0$, with intertwining maps
 \begin{align*}
   \alpha:\Lambda^2 V&\to W \\
   \beta: \Lambda^2 W&\to V \\
   V&\simeq W^* . \\
 \end{align*}
  We also have $\gamma: V\otimes W\simeq V\otimes V^*\to \g_0$ (representations are semisimple,
  so the map $\g_0\to \gl(V)\simeq V\otimes V^*$ splits).
  We normalize $\gamma$ in the following way. Let $B$ be the Killing form, and normalize $\gamma$ so that
  $B(\gamma(v\otimes w),X) = \langle w,Xv\rangle$. Make a Lie algebra $\g =
  \g_0\oplus V\oplus W$ by defining $[X,v]=Xv$ and $[X,w]=Xw$ for $X\in \g_0, v\in V,w\in
  W$. We also need to define  $[\ ,\,]$ on $V$,$W$ and between $V$ and $W$. These are actually
  forced, up to coefficients:
  \[ [v_1,v_2] = a\alpha(v_1\wedge v_2) \]
  \[ [w_1,w_2] = b\beta(w_1\wedge w_2) \]
  \[ [v,w] = c\gamma(v\otimes w). \]
 There are some conditions on the coefficients $a,b,c$
 imposed by the Jacobi identity;
 $[x,[y,z]] = [[x,y],z]+[y,[x,z]]$. Suppose $x\in \g_0$, with $y,z\in \g_i$ for
 $i=0,1,2$, then there is nothing to check, these identities come for free because
 $\alpha,\beta, \gamma$ are $\g_0$-invariant maps. There are only a few more cases to
 check, and only one of them gives you a condition. Look at
 \[
    [v_0,[v_1,v_2]] = ca\gamma(v_0\otimes \alpha(v_1\wedge v_2)) \tag{RHS}
 \]
 and it must be equal to
 \begin{align*}
   [[v_0,v_1],v_2]+&[v_1,[v_0,v_2]] = \\ & -ac\gamma(v_2\otimes \alpha(v_0\wedge v_1)) +
   ac\gamma(v_1\otimes \alpha(v_0\wedge v_1)) \tag{LHS}
 \end{align*}
 This doesn't give a condition on $ac$, but we need to check that it is satisfied.
 It suffices to check that $B(RHS,X)=B(LHS,X)$ for any $X\in \g_0$. This
 gives us the following condition:
 \begin{align*}
   \langle \alpha ( v_1\wedge v_2),Xv_0\rangle &= \langle \alpha(v_0\wedge
   v_2),Xv_1\rangle - \langle \alpha(v_0\wedge v_1),Xv_2\rangle
 \end{align*}
 The fact that $\alpha$ is an intertwining map for $\g_0$ gives us the identity:
 \begin{align*}
   \langle \alpha ( v_1\wedge v_2),Xv_0\rangle &= \langle \alpha(Xv_1\wedge
   v_2),v_0\rangle - \langle \alpha(v_1\wedge Xv_2),v_0\rangle
 \end{align*}
 and we also have that
 \[
   \langle \alpha ( v_1\wedge v_2),v_0\rangle = \langle \alpha(v_0\wedge
   v_2),v_1\rangle = \langle \alpha(v_0\wedge v_1),v_2\rangle
 \]
 With these two identities it is easy to show that the equation
 (and hence this Jacobi identity) is satisfied.

 We also get the Jacobi identity on $[w,[v_1,v_2]]$, which is equivalent to:
 \[ ab\beta (w\wedge \alpha(v_1\wedge v_2)) = c(\gamma(v_1\otimes w)v_2-\gamma(v_2\otimes w)v_1)
 \]
 It suffices to show for any $w'\in W$ that the pairings of each side with $w'$ are equal,
 \begin{align*}
   ab\langle w',\beta(w\wedge \alpha(v_1\wedge v_2)\rangle = c B&(\gamma(v_1\otimes
   w),\gamma(v_2\otimes w')) \\ &- c B(\gamma(v_2\otimes w),\gamma(v_1\otimes w'))
 \end{align*}
 This time we will get a condition on $a$, $b$, and $c$.
 You can check that any of the other cases of the Jacobi identity give you the same conditions.

 Now we will use this to construct $E_8$. Write $\g = \g_0\oplus V\oplus W$, where we
 take $\g_0=\sl(9)$. Let $E$ be the 9 dimensional representation of $\g_0$. Then take
 $V=\Lambda^3 E$ and $W=\Lambda^3 E^* \simeq \Lambda^6 E$. We have a pairing
 $\Lambda^3 E \otimes \Lambda^6 E\to k$, so we have $V\simeq W^*$. We would like to
 construct $\alpha: \Lambda^2 \to W$, but this is just given by $v_1\wedge v_2$
 including into $\Lambda^6 E\simeq W$. Similarly, we get $\beta:\Lambda^2 W\to V$. You
 get that the rank of $\g$ is 8 (= rank of $\g_0$). Notice that $\dim V =
 \binom{9}{3} = 84$, which is the same as $\dim W$, and $\dim \g _0 = \dim \sl(9)=80$.
Thus, we have that $\dim \g = 84 + 84+80=248$, which is the dimension of $E_8$, and
this is indeed $E_8$.

 Remember that we previously got $E_7$ and $E_6$ from $E_8$. Look at the diagram for $E_8$:
 \[ \begin{xy}
   (0,0) *+!D{\varepsilon_1-\varepsilon_2} *\cir<2pt>{};
   (1,0) *+!U{\varepsilon_2-\varepsilon_3} *\cir<2pt>{} **@{-};
   p+(1,0) *+!D{\varepsilon_3-\varepsilon_4} *\cir<2pt>{} **@{-};
   p+(1,0) *+!U{\varepsilon_4-\varepsilon_5} *\cir<2pt>{} **@{-};
   p+(1,0) *+!D{\varepsilon_5-\varepsilon_6} *\cir<2pt>{} **@{-};
   p+(0,-1) *+!U{\varepsilon_6+\varepsilon_7+\varepsilon_8} *\cir<2pt>{} **@{-},
   p+(1,0) *+!U{\varepsilon_6-\varepsilon_7} *\cir<2pt>{} **@{-};
   p+(1,0) *+!D{\varepsilon_7-\varepsilon_8} *\cir<2pt>{} **@{-};
 \end{xy}\]
 The extra guy, $\varepsilon_6+\varepsilon_7+\varepsilon_8$, corresponds to the 3-form. When you cut out
 $\varepsilon_1-\varepsilon_2$, you can figure out what is left and you get $E_7$.
 Then you can additionally cut out $\varepsilon_2-\varepsilon_3$ and get $E_6$.

 Fianally, we construct $F_4$:\qquad $\begin{xy}
   (0,0) *\cir<2pt>{};
   (1,0)  *\cir<2pt>{} **@{-};
   p+(1,0)="x" *\cir<2pt>{} **@{=} ?*@{>};
   "x" *{\hspace{4pt}};"x"+(1,0)  *\cir<2pt>{} **@{-};
 \end{xy}$

 We know that any simple algebra can be determined by generators and relations, with a
 $X_i,Y_i,H_i$ for each node $i$. But sometimes our diagram has a symmetry, like
 switching the horns on a $D_n$, which induces an automorphism of the Lie algebra
 given by $\gamma(X_i)=X_i$ for $i<n-1$ and switches $X_{n-1}$ and $X_n$. Because the arrows
 are preserved, you can check that the Serre relations still hold. Thus, in general,
 an automorphism of the diagram induces an automorphism of the Lie algebra (in a very
 concrete way).

 \begin{theorem}
   $(\aut \g)/(\aut_0 \g) = \aut \Gamma$. So the connected component of the identity
   gives some automorphisms, and the connected components are parameterized by
   automorphisms of the diagram.
 \end{theorem}

 $D_n$ is the diagram for $SO(2n)$. We have that $SO(2n)\subset O(2n)$, and the group
 of automorphisms of $SO(2n)$ is $O(2n)$.
 This isn't true of  $SO(2n+1)$, because the automorphisms given by
 $O(2n+1)$ are the same as those from $SO(2n+1)$. This corresponds the the fact that $D_n$
 has a nontrivial automorphism, but $B_n$ doesn't.

 Notice that $E_6$ has a symmetry; the involution:
 \[\begin{xy}
   (0,0)="1" *\cir<2pt>{};
   (1,0)="2"  *\cir<2pt>{} **@{-};
   p+(1,0) *\cir<2pt>{} **@{-};
   p+(0,-1) *\cir<2pt>{} **@{-},
   p+(1,0)="22" *\cir<2pt>{} **@{-};
   p+(1,0)="11" *\cir<2pt>{} **@{-};
   "1" *+{\ };"11" *+{\ } **\crv{(2,1.5)} ?<*@{<} ?>*@{>},
   "2" *+{\ };"22" *+{\ } **\crv{(2,1)} ?<*@{<} ?>*@{>},
 \end{xy}\]

 Define $X_1'=X_1+X_5, X_2' = X_2+X_4, X_3' = X_3, X_6'=X_6$, and the same with $Y$'s,
 the fixed elements of this automorphism. We
 have that $H_1'=H_1+H_5$ (you have to check that this works), and similarly for the
 other $H$'s. As the set of fixed elements, you get an algebra of rank 4 (which must
 be our $F_4$). You can check that
 $\alpha_1'(H_2')=-1,\alpha_2'(H_1')=-1,\alpha_3'(H_1')=0,\alpha_3'(H_2')=-2,\alpha_2'(H_3')=-1$,
 so this is indeed $F_4$ as desired. In fact, any diagram with multiple edges can be
 obtained as the fixed algebra of some automorphism:
 \begin{exercise}
   Check that $G_2$ is the fixed algebra of the automorphism of $D_4$:
   \[\begin{xy}
     (0,0) *\cir<2pt>{};
     a(60)="1" *\cir<2pt>{} **@{-},
     a(180)="2" *\cir<2pt>{} **@{-},
     a(-60)="3" *\cir<2pt>{} **@{-},
     \ar@/_2ex/ "1" *+{\ };"2" *+{\ }
     \ar@/_2ex/ "2" *+{\ };"3" *+{\ }
     \ar@/_2ex/ "3" *+{\ };"1" *+{\ }
   \end{xy}\]
   Check that $B_n,C_n$ can be obtained from $A_{2n}, A_{2n+1}$
   \[\begin{xy}
     (0,0)="1" *\cir<2pt>{};
     (1,0)="2"  *\cir<2pt>{} **@{-};
     p+(.6,0) **@{-};
     p+(.8,0) **{\,.\,};
     p+(.6,0)="22" *\cir<2pt>{} **@{-};
     p+(1,0)="11" *\cir<2pt>{} **@{-};
     "1" *+{\ };"11" *+{\ } **\crv{(2,1.5)} ?<*@{<} ?>*@{>},
     "2" *+{\ };"22" *+{\ } **\crv{(2,1)} ?<*@{<} ?>*@{>},
   \end{xy}\]
 \end{exercise}
}{   % Katie Liesinger, kate@math
  \stepcounter{lecture}
 \setcounter{lecture}{18}
 \sektion{Lecture 18 - Representations of Lie algebras}

 Let $\g$ be a semisimple Lie algebra over an algebraically closed field $k$ of
 characteristic 0. Then we have the root decomposition $\g = \h\oplus \bigoplus_\alpha
 \g_\alpha$. Let $V$ be a finite dimensional representation of $\g$. Because all
 elements of $\h$ are semisimple, and because Jordan decomposition is preserved, the
 elements of $\h$ can be simultaneously diagonalized. That is, we have a \emph{weight
 decomposition}\index{weight decomposition|idxbf} $V = \bigoplus_{\mu\in \h^*} V_\mu$,
 where $V_\mu = \{v\in V| hv=\mu(h)v \text{ for all } h\in \h\}$. We call $V_\mu$ a
 \emph{weight space}\index{weight space|idxbf}, and $\mu$ a
 \emph{weight}\index{weight|idxbf}. Define $P(V) = \{\mu \in \h^*| V_\mu\neq 0\}$. The
 multiplicity of a weight $\mu\in P(V)$ is $\dim V_\mu$, and is denoted $m_\mu$.
 \begin{example}
   You can take $V=k$ (the trivial representation). Then $P(V)=\{0\}$ and $m_0=1$.
 \end{example}
 \begin{example}
   If $V=\g$ and we take the adjoint representation\index{adjoint representation|idxit},
   then we have that $P(V)=\Delta\cup \{0\}$, with $m_\alpha=1$ for $\alpha\in
   \Delta$, and $m_0$ is equal to the rank of $\g$.
 \end{example}
 \begin{example}\index{sl(3)@$\sl(3)$|idxit} Let $\g=\sl(3)$. The weights of the
 adjoint representation are shown by the solid arrows (together with zero, which has
 multiplicity two).
   \[\begin{xy}
   (0,0)="c",
   \ar@{.>} a(30)     ="1" *+!LD{\varepsilon_1},
   \ar@{.>} a(150)   ="2" *+!RD{\varepsilon_2},
   \ar@{.>} a(-90)  ="3" *+!U{\varepsilon_3},
   \ar@{-->} a(-150) *+!RU{-\varepsilon_1},
   \ar@{-->} a(-30)  *+!LU{-\varepsilon_2},
   \ar@{-->} a(90) *+!D{-\varepsilon_3},
   \ar@{->} "1"-"2" *+!L{\varepsilon_1 - \varepsilon_2},
   \ar@{->} "1"-"3" *+!DL{\varepsilon_1 - \varepsilon_3},
   \ar@{->} "2"-"1" *+!R{\varepsilon_2 - \varepsilon_1},
   \ar@{->} "2"-"3" *+!DR{\varepsilon_2 - \varepsilon_3},
   \ar@{->} "3"-"1" *+!UR{\varepsilon_3 - \varepsilon_1},
   \ar@{->} "3"-"2" *+!UL{\varepsilon_3 - \varepsilon_2},
  \end{xy}\]
   The weights of the standard 3-dimensional representation are
   $\{\varepsilon_1,\varepsilon_2,\varepsilon_3\}$, shown in dotted lines.

   In general, the weights of the dual of a representation are the negatives of the
   original representation because $\langle h\phi, v\rangle$ is defined as $-\langle
   \phi,hv\rangle$. Thus, the dashed lines show the weights of the dual of the
   standard representation.
 \end{example}
%   \anton{Note that the sum of these is a $G_2$. (We noted earlier that $G_2$ is the sum of
%   an $\sl(3)$, a standard representation of $\sl(3)$, and a dual of the standard
%   representation).}
 If $V$ is a finite dimensional representation, then its weight decomposition has the
 following are properties.\index{weight decomposition!properties of|(idxbf}
 \begin{enumerate}
  \item \label{lec18weightp1} For any root $\alpha$ and $\mu\in P(V)$, $\mu(H_\alpha)\in \ZZ$.

    To see this, consider $V$ as a representation of the $\sl(2)$ spanned by
    $X_\alpha$, $H_\alpha$, and $Y_\alpha$. Our characterization of finite dimensional
    representations of $\sl(2)$ implies the result.

 \item \label{lec18weightp2} For $\alpha\in \Delta$ and $\mu\in P(V)$,  $\g_\alpha V_\mu\subseteq V_{\mu+\alpha}$

    This follows from the standard calculation:
    \begin{align*}
      h(x_\alpha v) &= x_\alpha hv + [h,x_\alpha]v\\
            &= x_\alpha \mu(h)v + \alpha(h) x_\alpha v\\
            &= (\mu+\alpha)(h)x_\alpha v.
    \end{align*}

   \item \label{lec18weightp3} If $\mu\in P(V)$ and $w\in \weyl$, then $w(\mu)\in P(V)$ and
   $m_\mu=m_{w(\mu)}$.

    It is sufficient to check this when $w$ is a simple reflection $r_i$. Consider $V$
    as a representation of the copy of $\sl(2)$ spanned by $X_i$, $Y_i$, and $H_i$. If
    $v\in V_\mu$, then we have that $h\cdot v = \mu(h) v$ for all $h\in \h$. By
    property \ref{lec18weightp1}, we know that $\mu(H_i)=l$ is a non-negative integer.
    From the characterization of finite dimensional representations of $\sl(2)$, we
    know that there is a corresponding vector with $H_i$-eigenvalue $-l$, namely $u =
    Y_i^l v$. By property \ref{lec18weightp2}, $u\in V_{\mu-l\alpha_i}$. But
    $\mu-l\alpha_i = \mu-\mu(H_i)\alpha_i = \mu -
    \frac{2(\mu,\alpha_i)}{(\alpha_i,\alpha_i)}\alpha_i = r_i\mu$. Putting it all
    together, if we consider the $\sl(2)$ subrepresentation of $V$ generated by
    $V_\mu$ and $V_{r_i\mu}$, the symmetry of finite dimensional representations of
    $\sl(2)$ tells us that $\dim V_{r_i\mu}=\dim V_\mu$, as desired.

%   To see this, it is sufficient to check for simple reflections $w=r_i$. Recall that
%   for each reflection, there is an element $S_i = \exp X_i \exp (-Y_i) \exp
%   X_i$ of the simply connected group $G$. Exercise \ref{lec14Ex:Weyl} proved that
%   $Ad_{S_i}(\h)=\h$.
%   \begin{exercise}
%     Show that $S_i^{-1}\cdot V_\mu = V_{r_i \mu}$. (Hint: look at the solution to
%     Exercise \ref{lec14Ex:Weyl})
%     \begin{solution}
%        Let $\rho$ be the name of the representation, and let $\tilde \rho$ denote the
%        corresponding representation of $G$. If $a\in G$ and $\gamma(t)$ is a path in $G$ with
%        $\gamma'(0)=x\in \g$, then we have that
%        \begin{align*}
%          \rho_{Ad_a x} &= \der{}{t}\tilde\rho_{a\gamma(t)a^{-1}} \Bigr|_{t=0} \\
%          &= \tilde\rho_a \Bigl(\der{}{t}\tilde\rho_{\gamma(t)}\Bigr|_{t=0}\Bigr)
%                \tilde\rho_{a^{-1}}\\
%          &= \tilde\rho_a \rho_x \tilde\rho_{a^{-1}}.
%        \end{align*}
%        This equips us to mirror the calculation in Exercise \ref{lec14Ex:Weyl}. Say
%        $x\in V_\mu$ and $h\in \h$, then
%        \begin{align*}
%          \rho_h \tilde\rho_{S_i^{-1}} x &= \tilde\rho_{S_i^{-1}} \rho_{Ad_{S_i}h} x\\
%                &= \tilde\rho_{S_i^{-1}}\bigl( \mu(Ad_{S_i}h)x \bigr)\\
%                &= \mu(Ad_{S_i}h)\,\tilde\rho_{S_i^{-1}}x\\
%                &= (r_i\mu)(h)\,\tilde\rho_{S_i^{-1}}x\\
%        \end{align*}
%        so $S_i^{-1} \cdot V_\mu = \tilde\rho_{S_i^{-1}}V_\mu = V_{r_i\mu}$.
%     \end{solution}
%   \end{exercise}
 \end{enumerate}
 \begin{remark}\label{lec18Rmk:findim}
   Note that the proof of property \ref{lec18weightp2} did not require that $V$ be
   finite dimensional. Properties \ref{lec18weightp1} and \ref{lec18weightp3} used
   finite dimensionality, but in a weak way. Consider the $\sl(2)$ spanned by $X_i$,
   $Y_i$, and $H_i$. It is enough for each vector $v$ in a weight space of $V$ to be
   contained in a finite dimensional $\sl(2)$ subrepresentation. In particular, if
   each $X_i$ and $Y_i$ act locally nilpotently,\footnote{We say that a linear
   operator $A$ is \emph{locally nilpotent} if for each vector $v$ there is an integer
   $n(v)$ such that $A^{n(v)}v=0$.} then all three properties hold.
 \end{remark}\index{weight decomposition!properties of|)idxbf}
 \begin{example}
   If $\g=\sl(3)$, then we get $\weyl=D_{2\cdot 3}=S_3$. The orbit of a point can have a
   couple of different forms. If the point is on a hyperplane orthogonal to a root,
   then you get a triangle. For a generic point, you get a hexagon (which is not
   regular, but still symmetric).
   \[
    \begin{xy}<1.75em,0em>:
      a(180)+a(180); a(0)+a(0) **@{-},
      a(120)+a(120); a(-60)+a(-60) **@{-},
      a(60)+a(60); a(-120)+a(-120) **@{-},
      (0,0);<2.5em,0em>:
      a(60);
      a(180) **@{--};
      a(-60) **@{--};
      a(60) **@{--},
    \end{xy}\qquad
    \begin{xy}<1.75em,0em>:
      a(180)+a(180); a(0)+a(0) **@{-},
      a(120)+a(120); a(-60)+a(-60) **@{-},
      a(60)+a(60); a(-120)+a(-120) **@{-},
      (0,0);<2.5em,0em>:
      a(40);
      a(80) **@{--};
      a(160) **@{--};
      a(-160) **@{--};
      a(-80) **@{--};
      a(-40) **@{--};
      a(40) **@{--};
    \end{xy}
   \]
 \end{example}
 It is pretty clear that knowing the weights and multiplicities gives us a lot of
 information about the representation, so we'd better find a good way to exploit this
 information.

 Let $V$ be a representation of $\g$. Then $V$ is also a representation of the
 associated simply connected group $G$, and we get the commutative square
 \[\xymatrix{
   G\ar[r] & GL(V)\\
   \g \ar[r] \ar[u]^{\exp} & \gl(V)\ar[u]_{\exp}
 }\]
 If $h\in \h$, then $\exp h\in G$, and we can evaluate the group character of the
 representation $V$ on $\exp h$ as
 \[
    \chi_V(\exp h) = tr(\exp h) = \sum_{\mu\in P(V)} m_\mu e^{\mu(h)}
 \]
 where the second equality is because every eigenvalue $\mu(h)$ of $h$ yields an
 eigenvalue $e^{\mu(h)}$ of $\exp h$. Since characters tell us a lot about finite
 dimensional representations, it makes sense to consider the following definition.
 \begin{definition}
   The \emph{character}\index{character|idxbf}\index{chV@$ch\, V$|see{character}} of
   the representation $V$ is the formal sum
   \[
      ch\, V = \sum_{\mu\in P(V)} m_\mu e^\mu.
   \]
 \end{definition}
 You can and multiply these (formal) expressions; $ch$ is additive with respect to
 direct sum and multiplicative with respect to tensor products:
 \begin{align*}
   ch(V\oplus W) &= ch\, V+ch\, W\\
   ch(V\otimes W) &= (ch\, V)(ch\, W)
 \end{align*}
 This is because $V_\mu\otimes W_\nu \subseteq (V\otimes W)_{\mu+\nu}$ (or you can use
 the relationship with group characters). You can also check that the $ch\, V^*$
 is $\sum m_\mu e^{-\mu}$.
% If you don't care about multiplication, then you can do
% infinite-dimensional stuff.
 \begin{remark}
   We only evaluated $\chi_V$ on the image of the Cartan subalgebra. Is it possible
   that we've lost some information about the behavior of $\chi_V$ on the rest of $G$?
   The answer is no. Since $\chi_V$ is constant on conjugacy classes, and any Cartan
   subalgebra is conjugate to any other Cartan subalgebra (Theorem \ref{lec14T:CSA}),
   we know how $\chi_V$ behaves on the union of all Cartan subalgebras. Since the
   union of all Cartan subalgebras is dense in $\g$, $\exp \g$ is dense in
   $G$\anton{this can be proven with Bruhat decomposition ... is there another way?},
   and $\chi_V$ is continuous, the behavior of $\chi_V$ on the image of a single
   Cartan determines it completely.
 \end{remark}
 \begin{exercise}
   Show that the union of all Cartan subalgebras is dense in $\g$.
   \begin{solution}
     Every regular semisimple element is in some Cartan subalgebra; namely, the Cartan
     subalgebra of elements that commute with it. We will show that regular semisimple
     elements are dense in $\g$.

     Choose a basis for $\g$, which gives you a corresponding basis for $\gl(\g)$. Say
     $\g$ has rank $r$. Let $I$ be an indexing set so that for a matrix $A\in
     \gl(\g)$, the set $\{M_\gamma(A)\}_{\gamma\in I}$ is the set of all $(n-r)\times
     (n-r)$ minors of $A$. Define $f_\gamma:\g\to k$ by $ f_\gamma(x) =
     \det\bigl(M_\gamma(ad_x)\bigr)$. Since $ad$ is linear, $f_\gamma$ is a polynomial
     map for each $\gamma$. Now consider union of all of the zero sets of all of the
     $f_\gamma$. This is a Zariski closed set, so its complement in $\g$ is a Zariski
     open set\index{Zariski open set}. Since $\g$ has a regular element (a semisimple
     element $h$, where $ad_h$ is rank $n-r$), that open set is non-empty, and since
     $\g\cong \mathbb{A}^{\dim \g}$ is irreducible, this set is dense. \anton{To
     complete the proof, we need to know that the regular semisimple elements are
     exactly the elements $x$ so that $ad_x$ has rank $n-r$, which is not true. How
     can this be fixed?}
   \end{solution}
 \end{exercise}

 \subsektion{Highest weights} Fix a set of simple roots $\Pi =
 \{\alpha_1,\dots,\alpha_n\}$. A \emph{highest weight}\index{weight!highest|idxbf} of
 a representation $V$ is a $\lambda\in P(V)$ such that $\alpha_i+\lambda \not\in
 P(V)$ for all $\alpha_i\in\Pi$. A \emph{highest weight vector} is a vector in
 $V_\lambda$.

 Let $V$ be irreducible, let $\lambda$ be a highest weight, and let $v\in V_\lambda$
 be a highest weight vector. Since $V$ is irreducible, $v$ generates: $V = (U\g)v$.
 We know that $\n^+ v=0$ and that $\h$ acts on $v$ by scalars. By PBW\index{PBW|idxit},
 $U\g = U\n^-\otimes U\h\otimes U\n^+$, so $V=U\n^- v$. Thus, $V$ is generated from
 $v$ by applying various $Y_\alpha$, where $\alpha\in \Delta^+$. In particular,
 the multiplicity $m_\lambda$ is one. This also tells us that any other weight $\mu$
 is ``less than'' $\lambda$ in the sense that $\lambda -\mu = \sum_{\alpha\in
 \Delta^+} l_\alpha \alpha$, where the $l_\alpha$ are non-negative.
%
% Thus, all weights of the irreducible representation are given by
% applying negative roots. There are some relations among the different ways of
% applying roots: $[Y_{\alpha_1},Y_{\alpha_2}] = Y_{\alpha_1+\alpha_2}$.  Note that in
% particular, we get that $m_\lambda =1$.
% \[\begin{xy}
%   (0,0)="c" *+!DL{\lambda},
%   \ar_{\alpha_1} "c";a(180)="1"
%   \ar^{\alpha_2} "c";a(-60)="2"
%   \ar^{\!\alpha_3} "c";"1"+"2" *+{\,}
%   \ar "1" *+{\,};"1"+"1"
%   \ar "1" *+{\,};"1"+"2" *+{\,}
%   \ar "2" *+{\,};"2"+"1" *+{\,}
%   \ar "2" *+{\,};"2"+"2"
%   \ar "1"+"2";"1"+"2"+"1"
%   \ar "1"+"2";"1"+"2"+"2"
%   \ar "1"+"2";"1"+"2"+"1"+"2"
% \end{xy}\]

 It follows that in an irreducible representation, the highest weight is unique. If
 $\mu$ is another highest weight, then  we get $\lambda \le \mu$ and $\mu\le \lambda$,
 which implies $\mu = \lambda$.

 \begin{remark}\label{lec18Rmk:finirrep}
   If $V$ is an irreducible \emph{finite dimensional} representation with highest
   weight $\lambda$, then for any $w\in \weyl$, property \ref{lec18weightp3} tells us
   that $w(\lambda)$ is a highest weight with respect to the set of simple roots
   $\{w\alpha_1,\dots, w\alpha_n\}$. So $P(V)$ is contained in the convex hull of the
   set $\{w\lambda\}_{w\in \weyl}$.

   We also know that $\lambda$ is a highest weight for each $\sl(2)$ spanned by
   $X_\alpha$, $Y_\alpha$, and $H_\alpha$, with $\alpha\in \Delta^+$ (from the
   definition of highest weight). So
   $\lambda(H_i)=(\lambda,\check\alpha_i)\in \ZZ_{\ge 0}$ for each $i$.
 \end{remark}

 \begin{definition}
   The lattice generated by the roots, $Q= \ZZ\alpha_1\oplus \cdots\oplus \ZZ
   \alpha_n$, is called the \emph{root lattice}\index{root!lattice|idxbf}.
 \end{definition}
 \begin{definition}
   The lattice $P=\{\mu\in \h^*|(\mu,\check\alpha_i)\in \ZZ\text{ for } 1\le i\le n\}$
   is called the \emph{weight lattice}\index{weight!lattice|idxbf}.
 \end{definition}
 \begin{definition}
   The set $\{\mu\in \h^*|(\mu,\check\alpha_i)\ge 0\text{ for }1\le i\le n\}$ is
   called the \emph{Weyl chamber}\index{Weyl chamber|idxbf}, and the intersection of
   the Weyl chamber with the weight lattice is called the set of \emph{dominant
   integral weights}\index{weight!dominant integral|idxbf}, and is denoted $P^+$.
 \end{definition}
 $P$ and $Q$ have the same rank. It is clear that $Q$ is contained in $P$, and in
 general this containment is strict.

 $P/Q$ is isomorphic to the center of the simply connected group corresponding to
 $\g$.\anton{prove this or give a ref please}

% If $\lambda$ is a highest weight of an irreducible representation with respect to
% some choice of simple roots $\alpha_1,\dots, \alpha_n$, then for any $w\in \weyl$,
% $w(\lambda)$ will be the highest weight with respect to $w(\alpha_1),\dots, w(\alpha_n)$.
% Thus, we get that $P(V)$ is a subset in the convex hull of the orbit of the highest
% weight under the Weyl group, $\weyl\lambda$.
%
% If $\lambda$ is a highest weight of an irreducible $V$, then we know that
% $\lambda(H_i)\in \ZZ_{\ge 0}$. Sometimes this is written $(\lambda,\check
% \alpha_i)=\lambda(H_i)\ge 0$. To see this, note that we are restricting to our little
% $\sl(2)$ again.
%
% There are a few lattices in $\h^*$. If you look at the lattice generated by the
% roots, then we call that the \emph{root lattice}, and it is denoted $Q$:
% \[
%    Q = \ZZ\alpha_1\oplus \cdots\oplus \ZZ\alpha_n.
% \]
% There is another one, called the \emph{weight lattice}, and is called $P$:
% \[
%    P = \{\mu\in \h^*| (\mu,\check \alpha_i)\in \ZZ\}.
% \]
% Let $P^+ = \{\lambda\in P| (\lambda,\check \alpha_i)\ge 0\}$. These are called the
% integral dominant elements. Both $P$ and $Q$ have the same rank, but $P$ is usually
% bigger (and never smaller, obviously).
 \begin{example}
   For $\g=\sl(2)$, the root lattice is $2\ZZ$ (because $[H,X]=2X$),
   and the weight lattice is $\ZZ$.
 \end{example}

 \begin{example}
   In the three rank two cases, the weight lattices and Weyl chambers are
  \[\begin{array}{ccc}
   \sl(3) & \so(5)\cong \sp(4) & G_2\\
   \renewcommand\latticebody{\ifnum\latticeA=3 \ifnum\latticeB=2
                             \else \drop{\bullet} \fi
                             \else \drop{\bullet} \fi}
   \quad\begin{xy}a(120):: %sets coordinate system to that of simple roots
     {\ar_(.8){\alpha_1} (0,0);(1,0) *+{\,}
      \ar^(.8){\alpha_2} (0,0);(0,1) *+{\,}},
    (.66667,.33333):(.5,\halfrootthree):: %sets coordinate system to that of fundamental weights (reversed)
       (0,0) *!LD{\begin{pspicture}(0,0)(2,2.2)
                  \pswedge[fillstyle=solid,fillcolor=lightgray,linewidth=0pt]{2.2}{30}{90}
                  \end{pspicture}},
       (-2.5,0);(4.2,0) **@{--},
       (0,-2.5);(0,4.2) **@{--},
       (3,2) *{P^+},
       (0,0);(0,0) \croplattice{-2}3{-3}5 {-2.1}{3.1}{-2.1}{3.6}
   \end{xy}&
   \renewcommand\latticebody{\ifnum\latticeB=1 \ifnum\latticeA=-1
                             \else \ifnum\latticeA=4 \else \drop{\bullet} \fi\fi
                             \else \drop{\bullet} \fi}
   \quad\begin{xy}<1.75em,0em>:(-1,1):: %sets coordinate system to that of simple roots
     {\ar_(.8){\alpha_1} (0,0);(1,0) *+{\,}
      \ar (0,0);(0,1) *++!U{\mbox{\scriptsize $\alpha_2$}} *+{\,}},
    (1,.5):(1,1):: %sets coordinate system to that of fundamental weights (reversed)
       (0,0) *!LD{\begin{pspicture}(0,0)(2,2.2)
                  \pswedge[fillstyle=solid,fillcolor=lightgray,linewidth=0pt]{2.2}{45}{90}
                  \end{pspicture}},
       (-3.5,0);(5,0) **@{--},
       (0,-2);(0,3.5) **@{--},
       (4,1) *{P^+},
       (0,0);(0,0) \croplattice{-3}4{-3}4 {-4}{4.1}{-1.6}{3.1}
   \end{xy}&
   \renewcommand\latticebody{\ifnum\latticeB=0 \ifnum\latticeA=4
                             \else \drop{\bullet} \fi
                             \else \drop{\bullet} \fi}
   \quad\begin{xy}<1.5em,0em>: a(120)+a(180):: %sets coordinate system to that of simple roots
     {\ar_(.8){\alpha_1} (0,0);(1,0) *+{\,}
      \ar^(.8){\alpha_2} (0,0);(0,1) *+{\,}},
    (2,1):(1,0)+a(60):: %sets coordinate system to that of fundamental weights (reversed)
       (0,0) *!LD{\begin{pspicture}(0,0)(2,2.2)
                  \pswedge[fillstyle=solid,fillcolor=lightgray,linewidth=0pt]{2.2}{60}{90}
                  \end{pspicture}},
       (-2.4,0);(4,0) **@{--},
       (0,-1.3);(0,2.4) **@{--},
       (3,.5) *!DL{P^+},
       (0,0);(0,0) \croplattice{-4}4{-3}4 {-4.1}{4.1}{-1.1}{2.1}
   \end{xy}
 \end{array}\]
 \end{example}
 \begin{exercise}
   Show that $P^+$ is the fundamental domain of the action of $\weyl$ on $P$. That is,
   show that for every $\mu\in P$, the $\weyl$-orbit of $\mu$ intersects $P^+$ in exactly
   one point. (Hint: use Proposition \ref{lec14P:simply})
   \begin{solution}
     \anton{add the solution}
   \end{solution}
 \end{exercise}

 We have already shown that the highest weight of an irreducible finite dimensional
 representation is an element of $P^+$ (this is exactly the second part of Remark
 \ref{lec18Rmk:finirrep}). The rest of the lecture will be devoted to proving the
 following remarkable theorem.
 \begin{theorem}\label{lec18Thm:hiweight}
   There is a bijection between $P^+$ and the set of (isomorphism classes of)
   finite dimensional irreducible representations of $\g$, in which an irreducible
   representation corresponds to its highest weight.
 \end{theorem}
 It remains to show that two non-isomorphic finite dimensional irreducible
 representations cannot have the same highest weight, and that any element of $P^+$
 appears as the highest weight of some finite dimensional representation. To prove
 these things, we will use Verma modules.

 Let $V$ be an irreducible representation with highest weight $\lambda$. Then
 $V_\lambda$ is a 1-dimensional representation of the subalgebra $\b^+ :=\h\oplus
 \n^+\subseteq \g$. There is an induced representation $U\g\otimes_{U\b^+}V_\lambda$
 of $\g$, and an induced homomorphism $U\g\otimes_{U\b^+}V_\lambda\to V$ given by
 $x\otimes v\mapsto x\cdot v$.
 \begin{definition} \index{Verma module|(}
   A \emph{Verma module} is $M(\lambda) = U\g\otimes_{U\b^+} V_\lambda$.
 \end{definition}
 The Verma module is universal in the sense that for any representation $V$ with highest
 weight vector $v$ of weight $\lambda$, there is a unique homomorphism of
 representations $M(\lambda)\to V$ sending the highest vector of $M(\lambda)$ to $v$.
 However, there is a problem: $M(\lambda)$ is infinite dimensional.

 To understand $M(\lambda)$ as a vector space, we use PBW\index{PBW|idxit} to get that
 $U\g = U\n^-\otimes_k U\h \otimes_k U\n^+ = U(\n^-)\otimes_k U\b^+$. Since $U\b^+$
 acts on $V_\lambda$ by scalars, we get
 \[
   M(\lambda) = U\n^-\otimes_k U\b^+\otimes_{U\b^+} V_\lambda = U\n^- \otimes_k V_\lambda.
 \]
 If $\Delta^+ = \{\alpha_1,\dots, \alpha_N\}$, with $\Pi=\{\alpha_1,\dots,
 \alpha_n\}$, then by PBW\index{PBW|idxit}, $\{Y_{\alpha_1}^{k_1}\cdots
 Y_{\alpha_N}^{k_N}\}$ is a basis for $U\n^-$, so $\{Y_{\alpha_1}^{k_1}\cdots
 Y_{\alpha_N}^{k_N}v\}$ is a basis for $M(\lambda)=U(\n^-)\otimes_k V_{\lambda}$.
 Thus, even though the Verma module is infinite dimensional, it still has a weight
 decomposition with finite dimensional weight spaces:
 \[
    h(Y_{\alpha_1}^{k_1}\cdots Y_{\alpha_N}^{k_N}v) = (\lambda - k_1\alpha_1
    -\cdots - k_N\alpha_N)(h)(Y_{\alpha_1}^{k_1}\cdots Y_{\alpha_N}^{k_N}v).
 \]
 In particular, we get a nice formula for the multiplicity of a weight. The
 multiplicity of $\mu$ is given by the number of different ways $\lambda-\mu$ can be
 written as a non-negative sum of positive roots, corresponding to the number of basis
 vectors $Y_{\alpha_1}^{k_1}\cdots Y_{\alpha_N}^{k_N}v$ lying in $V_\mu$.
 \[
    m_\mu = \#\Bigl\{\lambda - \mu = \sum_{\alpha_i \in \Delta^+}k_i \alpha_i\Bigm|
    k_i\in \ZZ_{\ge 0}\Bigr\}.
 \]
 This is called the Kostant partition function.\index{Kostant partition function}

 \begin{example}\label{lec18Eg:Verma}
   We are now in a position to calculate the characters of Verma modules. In the rank
   two cases, we get the characters below. For example, since $2\alpha_3 =
   \alpha_3+\alpha_2+\alpha_1 = 2\alpha_1+2\alpha_2$ can be written in these three
   ways as a sum of positive roots, the circled multiplicity (in the $\sl(3)$ case) is
   3.
   \[\begin{array}{cc}
    \sl(3) & \so(5)\cong \sp(4)\\
    \quad \begin{xy}<1.75em,0em>: (-1,0):a(120):: %set the coordinates to negative simple roots
      {\ar (0,0) *++{\,};(-1,0) *+!L{\alpha_1}
       \ar (0,0) *++{\,};(0,-1) *+!R{\alpha_2}
       \ar (0,0) *++{\,};(-1,-1) *+!L{\alpha_3}},
       (4,0) *{1},(3,0) *{1},(2,0) *{1},(1,0) *{1},(0,0) *{1},
       (4,1) *{2},(3,1) *{2},(2,1) *{2},(1,1) *{2},(0,1) *{1},
       (5,2) *{3},(4,2) *{3},(3,2) *{3},(2,2) *\cir<6pt>{} *{3},(1,2) *{2},(0,2) *{1},
       (5,3) *{4},(4,3) *{4},(3,3) *{4},(2,3) *{3},(1,3) *{2},(0,3) *{1},
       (6,4) *{5},(5,4) *{5},(4,4) *{5},(3,4) *{4},(2,4) *{3},(1,4) *{2},(0,4) *{1},
    \end{xy}
    \quad & \quad
    \begin{xy}<1.75em,0em>: (-1,0):(-1,1):: %set the coordinates to negative simple roots
      {\ar (0,0) *++{\,};(-1,0)  *+!L{\alpha_1}
       \ar (0,0) *++{\,};(0,-1) *+!R{\alpha_2}
       \ar (0,0) *++{\,};(-1,-1) *+!D{\alpha_3}
       \ar (0,0) *++{\,};(-2,-1) *+!L{\alpha_4}},
       (4,0) *{1},(3,0) *{1},(2,0) *{1},(1,0) *{1},(0,0) *{1},
       (5,1) *{3},(4,1) *{3},(3,1) *{3},(2,1) *{3},(1,1) *{2},(0,1) *{1},
       (6,2) *{6},(5,2) *{6},(4,2) *{6},(3,2) *{5},(2,2) *{4},(1,2) *{2},(0,2) *{1},
       (7,3) *{10},(6,3) *{10},(5,3) *{9},(4,3) *{8},(3,3) *{6},(2,3) *{4},(1,3) *{2},(0,3) *{1},
       (8,4) *{15},(7,4) *{14},(6,4) *{13},(5,4) *{11},(4,4) *{9},(3,4) *{6},(2,4) *{4},(1,4) *{2},(0,4) *{1},
    \end{xy}\quad
    \end{array}\]
    \[\raisebox{-3em}{\mbox{$G_2$}} \qquad \qquad
    \begin{xy}<1.75em,0em>: (-1,0):a(120)+(-1,0):: %set the coordinates to negative simple roots
      {\ar (0,0) *++{\,};(-1,0)  *+!L{\alpha_1}
       \ar (0,0) *++{\,};(0,-1) *+!R{\alpha_2}
       \ar (0,0) *++{\,};(-1,-1) *+!DR{\alpha_3}
       \ar (0,0) *++{\,};(-3,-2) *+!D{\alpha_4}
       \ar (0,0) *++{\,};(-2,-1) *+!DL{\alpha_5}
       \ar (0,0) *++{\,};(-3,-1) *+!L{\alpha_6}},
       (6,0) *{1},(5,0) *{1},(4,0) *{1},(3,0) *{1},(2,0) *{1},(1,0) *{1},(0,0) *{1},
       (7,1) *{4},(6,1) *{4},(5,1) *{4},(4,1) *{4},(3,1) *{4},(2,1) *{3},(1,1) *{2},(0,1) *{1},
       (9,2) *{11},(8,2) *{11},(7,2) *{11},(6,2) *{11},(5,2) *{10},(4,2) *{9},(3,2) *{7},(2,2) *{4},(1,2) *{2},(0,2) *{1},
       (10,3) *{24},(9,3) *{24},(8,3) *{23},(7,3) *{22},(6,3) *{20},(5,3) *{16},(4,3) *{12},(3,3) *{8},(2,3) *{4},(1,3) *{2},(0,3) *{1},
       (12,4) *{46},(11,4) *{45},(10,4) *{44},(9,4) *{42},(8,4) *{38},(7,4) *{33},(6,4) *{27},(5,4) *{19},(4,4) *{13},(3,4) *{8},(2,4) *{4},(1,4) *{2},(0,4) *{1},
    \end{xy}
   \]
 \end{example}
 \begin{exercise}
   Check that the characters in Example \ref{lec18Eg:Verma} are correct. (Hint: For
   $\so(5)$, at each lattice point, keep track of four numbers: the number of ways to
   write $\lambda-\mu$ as a non-negative sum in the sets $\{\alpha_1,\dots,
   \alpha_4\}$, $\{\alpha_2,\alpha_3,\alpha_4\}$, $\{\alpha_2,\alpha_3\}$, and
   $\{\alpha_2\}$)
   \begin{solution}
     It is not hard to set up a recursive calculation with the numbers in the hint.
     Alternatively, note that the Kostant partition function tells us exactly that
     \begin{align*}
        ch\, M(\lambda) &= e^\lambda\prod_{\alpha\in \Delta^+}(1+e^{-\alpha} +
        e^{-2\alpha}+\cdots)\\
        &= e^\lambda \prod_{\alpha\in\Delta^+}(1-e^{-\alpha})^{-1}.
     \end{align*}
     You can easily (have your computer) compute the coefficients of this power
     series. For example, to compute the character of a Verma module of $G_2$, I think
     of $e^{\alpha_1}$ as \verb!x! and of $e^{\alpha_2}$ as \verb!y!. Then the following
     Mathematica code returns the first 144 multiplicities.\\
     {\raggedright
      \verb!Nmax = 12;! \\
      \verb!mySeries=Series[!
      \hspace*{1em} \verb!((1-x)(1-y)(1- x y)(1- x^2 y)(1- x^3 y)(1- x^3 y^2))^(-1),! \\
      \qquad \verb!{x,0,Nmax},{y,0,Nmax}];! \\
      \verb!TableForm[Table[SeriesCoefficient[! \\
      \hspace{10em}\verb!mySeries,{i,j}],{i,0,Nmax},{j,0,Nmax}]]!
     }
   \end{solution}
 \end{exercise}

 \begin{lemma}\label{lec18Lem:maxlprop}
   A Verma module $M(\lambda)$ has a unique proper maximal submodule $N(\lambda)$.
 \end{lemma}
 \begin{proof}
   $N$ being proper is equivalent to $N\cap V_\lambda = 0$. This property is clearly
   preserved under taking sums, so you get a unique maximal submodule.
 \end{proof}
 \begin{remark}\label{lec18Rmk:injectivity}
   If $V$ and $W$ are irreducible representations with the same highest weight,
   then they are both isomorphic to the unique irreducible quotient
   $M(\lambda)/N(\lambda)$, so they are isomorphic.
 \end{remark}
 \begin{lemma}\label{lec18Lem:Vlfindim}
   If $\lambda\in P^+$, then the quotient $V(\lambda) = M(\lambda)/N(\lambda)$ is
   finite dimensional.
 \end{lemma}
 \begin{proof}
   If $w$ is a weight vector (but not the highest weight vector) in $M(\lambda)$
   such that $X_iw=0$ for $i=1,\dots, n$, then we claim that $w\in N(\lambda)$. To see
   this, you note that
   \[
      (U\g)w = (U\n^-\otimes U\h\otimes U\n^+)w = (U\n^-)w
   \]
   so the submodule generated by $w$ contains only lower weight spaces. In particular,
   the highest weight space $V_\lambda$ cannot be obtained from $w$.

   Fix an $i\le n$. By assumption, $\lambda(H_i)=\langle \lambda, \check
   \alpha_i\rangle = l_i\in \ZZ_{\ge 0}$. Letting $w = Y_i^{l_i+1}v$, we get
   that
   \begin{align*}
     X_iw &= (l_i+1)\bigl(l_i-(l_i+1)+1\bigr) Y_i^{l_i+1}w =0 &
            \text{(by Equation \ref{lec13dag})}\\
     X_jw &= Y_i^{l_i+1} X_j w = 0 & \text{(since $[X_j,Y_i]=0$)}
   \end{align*}
   so $w\in N(\lambda)$. It follows from the Serre relations\index{Serre
   relations|idxit} that in the quotient $V(\lambda)$, the $Y_i$ act locally
   nilpotently. The $X_i$ act locally nilpotently on $M(\lambda)$, so they act locally
   nilpotently on $V(\lambda)$. By Remark \ref{lec18Rmk:findim},
   $P\bigl(V(\lambda)\bigr)$ is invariant under $\weyl$, so it is contained in the convex
   hull of the orbit of $\lambda$. Since each weight space is finite dimensional, it
   follows that $V(\lambda)$ is finite dimensional.
 \end{proof}

 Putting it all together, we can prove the Theorem.
 \begin{proof}[Proof of Theorem \ref{lec18Thm:hiweight}]
   By Remark \ref{lec18Rmk:finirrep}, the highest weight of an irreducible finite
   dimensional representation is an element of $P^+$. By Remark
   \ref{lec18Rmk:injectivity}, non-isomorphic representations have distinct highest
   weights. Finally, by Lemmas \ref{lec18Lem:maxlprop} and \ref{lec18Lem:Vlfindim},
   every element of $P^+$ appears as the highest weight of some finite dimensional
   irreducible representation.
 \end{proof}
 \begin{corollary}
   If $V$ and $W$ are finite dimensional representations, and if $ch\, V = ch\, W$,
   then $V\simeq W$.
 \end{corollary}
 \begin{proof}
   Since their characters are equal, $V$ and $W$ have a common highest weight
   $\lambda$, so they both contain a copy of $V(\lambda)$. By complete reducibility
   (Theorem \ref{lec12Weyl}), $V(\lambda)$ is a direct summand in both $V$ and $W$. It
   is enough to show that the direct complements are isomorphic, but this follows from
   the fact that they have equal characters and fewer irreducible components.
 \end{proof}
 \anton{Do you really need complete reducibility? Does this hold for (some) infinite
 dimensional representations?}

 So it is desirable to be able to compute the character of $V(\lambda)$. This is what
 we will do next lecture.
}{   % Aaron McMillan, aaronfm@math
  \stepcounter{lecture}
 \setcounter{lecture}{19}
 \sektion{Lecture 19 - The Weyl character formula}

 If $\lambda\in P^+$ (i.e.\ $(\lambda, \check \alpha_i)\in \ZZ_{\ge 0}$ for all $i$),
 then we can construct an irreducible representation with highest weight $\lambda$,
 which we called $V(\lambda)$. We define the \emph{fundamental
 weights}\index{weight!fundamental|idxbf} $\w_1,\dots, \w_n$ of a Lie algebra to be
 those weights for which $(\w_i,\check \alpha_j)=\delta_{ij}$. It is clear that any
 dominant integral weight can be written as $\lambda = \lambda_1\w_1+\cdots +
 \lambda_n \w_n$ for $\lambda_i\ge 0$, so people often talk about $V(\lambda)$ by
 drawing the Dynkin diagram with the the $i$-th vertex labelled by $\lambda_i$.

 With this notation, the first fundamental representation $V(\w_1)$ for $\sl(n)$ is written
  $\begin{xy}
   (0,-.15) *+!D{1} *\cir<2pt>{};
   p+(1,0) *+!D{0} *\cir<2pt>{} **@{-};
   p+(.5,0) **@{-};
   p+(.6,0) **{\hspace{1pt}.\hspace{1pt}};
   p+(.5,0) *+!D{0} *\cir<2pt>{} **@{-};
   p+(1,0) *+!D{0} *\cir<2pt>{} **@{-};
 \end{xy}$, which happens to be the standard representation
 (see Example \ref{lec19Eg:Fund_sl(n+1)} below).
 Similarly, the adjoint representation is
 $\begin{xy}
   (0,-.15) *+!D{1} *\cir<2pt>{};
   p+(1,0) *+!D{0} *\cir<2pt>{} **@{-};
   p+(.5,0) **@{-};
   p+(.6,0) **{\hspace{1pt}.\hspace{1pt}};
   p+(.5,0) *+!D{0} *\cir<2pt>{} **@{-};
   p+(1,0) *+!D{1} *\cir<2pt>{} **@{-};
 \end{xy}$.
 \begin{warning}
   Another common notation (incompatible with this one) is to write $\lambda = \sum
   k_i\alpha_i$ and label the $i$-th vertex by $k_i$. In this notation, the standard
   representation is
   $\begin{xy}
   (0,0) *+{1};
   p+(1,0) *+{0} **@{-};
   p+(.5,0) **@{-};
   p+(.6,0) **{\hspace{1pt}.\hspace{1pt}};
   p+(.5,0) *+{0} **@{-};
   p+(1,0) *+{0} **@{-};
 \end{xy}$
 and the adjoint representation is
 $\begin{xy}
   (0,0) *+{1};
   p+(1,0) *+{1} **@{-};
   p+(.5,0) **@{-};
   p+(.6,0) **{\hspace{1pt}.\hspace{1pt}};
   p+(.5,0) *+{1} **@{-};
   p+(1,0) *+{1} **@{-};
 \end{xy}$. In these notes, we will draw the diagram differently to distinguish
 between the two notations.
 \end{warning}

 Observe that if $v\in V$ a highest vector of weight $\lambda$, and $w\in W$ another
 highest weight vector of weight $\mu$ in another representation, then $v\otimes w\in
 V\otimes W$ is a highest weight vector of weight $\lambda+\mu$. It follows that every
 finite dimensional irreducible representation can be realized as a subrepresentation
 of a tensor product of fundamental representations.
 \begin{example}\label{lec19Eg:Fund_sl(n+1)}\index{sl(n)@$\sl(n)$|idxit}
   Let's calculate the fundamental weights for $\sl(n+1)$. Recall that we have simple
   roots $\e_1-\e_2, \dots, \e_n-\e_{n+1}$, and they are equal to their coroots (since
   they have length $\sqrt 2$). It follows that $\w_i=\e_1+\cdots +\e_i$ for
   $i=1,\dots, n$.

   Let $E$ be the standard $(n+1)$-dimensional representation of $\sl(n+1)$. Let
   $e_1,\dots, e_{n+1}$ be a basis for $E$. Note that $e_i$ has weight $\e_i$, and
   $\e_i-\e_j$ can be written as a non-negative sum of positive roots exactly when
   $i\le j$. Thus, the weights of $E$, in decreasing order, are $\e_1$, $\e_2$, \dots,
   $\e_{n+1}$.

   Consider the representation $\Lambda^k E$. We'd like to write down its weights.
   Note that $\Lambda^k E$ is spanned by the vectors $e_{i_1}\wedge \cdots \wedge
   e_{i_k}$, which have weights $\e_{i_1}+\cdots +\e_{i_k}$. Thus, the highest weight
   is $\e_1+\cdots + \e_k=\w_k$, so we know that $V(\w_k)\subseteq \Lambda^k E$.

   Note also that $\weyl \cong S_{n+1}$ acts by permutation of the $e_i$, so it can
   take any weight space to any other weight space. Such a representation (where all
   the weight spaces form a single orbit of the Weyl group) is called
   \emph{minuscule}\index{minuscule representation|idxbf}. Since the character of any
   subrepresentation must be $\weyl$-invariant, minuscule representations are always
   irreducible. So $\Lambda^k E= V(\w_k)$ is a fundamental representation.
 \end{example}
 \begin{remark}[Highest weights of duals]
   One of the weights of $V(\lambda)^*$ is $-\lambda$, but to compute the highest
   weight, we need to get back into $P^+$, so we apply the longest word $w$ in the
   Weyl group. Thus, $-w(\lambda)$ is the highest weight of $V(\lambda)^*$. This means
   that there is a fixed involution of the Weyl chamber (namely, $-w$) which takes the
   highest weight of a representation to the highest weight of its dual. It is clear
   that $-w$ preserves the set of simple roots and preserves inner products, so it
   corresponds to an involution of the Dynkin diagram.

   In the case of $\sl(n+1)$, the involution is
   \begin{xy}
     (0,0)="1" *\cir<2pt>{};
     (1,0)="2"  *\cir<2pt>{} **@{-};
     p+(.6,0) **@{-};
     p+(.8,0) **{\,.\,};
     p+(.6,0)="22" *\cir<2pt>{} **@{-};
     p+(1,0)="11" *\cir<2pt>{} **@{-};
     "1" *+{\ };"11" *+{\ } **\crv{(2,.5)} ?<*@{<} ?>*@{>},
     "2" *+{\ };"22" *+{\ } **\crv{(2,.3)} ?<*@{<} ?>*@{>},
   \end{xy}. In particular, the dual of the standard representation $V(\w_1)$ is
   $V(\w_n)$.
 \end{remark}
% \begin{exercise}
%   Compute these involutions for the other simple Lie algebras.
%   \begin{solution}
%     $B_n$ and $C_n$ have no non-trivial diagram involutions, so
%   \end{solution}
% \end{exercise}

% How do you calculate $ch\, V$? Recall that $ch\, V = \sum m_{\mu} e^{\mu}$, where
% $m_\mu$ is the multiplicity of the weight $\mu$. You can do it with Verma modules.
% Remember that for Verma modules, you still get a weight decomposition.

 The key to computing the character of $V(\lambda)$ is to write it as a linear
 combination of characters of Verma modules, as in the following example.

 \index{Weyl character formula|(idxbf}
 \begin{example} \label{lec19Eg:2w1+w2}
   Let $\g = \sl(3)$ and let $\lambda = 2\w_1+\w_2$. We try to write $ch\, V(\lambda)$
   as a linear combination of characters of Verma modules in the na\"\i ve way. We
   know  that $M(\lambda)$ must appear once and that $ch\, V(\lambda)$ must end up
   symmetric with respect to the Weyl group. We must subtract off two Verma modules to
   keep the symmetry. Then we find that we must add back two more and subtract one in
   order to get zeros outside of the hexagon. In the picture below, each dot can be
   read as a zero.
 \[\newcommand\buff{.2} \newcommand\cbuff{.8}
   \renewcommand\latticebody{\drop{\raisebox{-8.6pt}{\kern-.1pt \mbox{\LARGE $\cdot$}}}}
   \begin{xy}<2em,0em>:a(30):a(60)::
     (6,0);(-8,0) **@{--},
     (0,5);(0,-7) **@{--},
     (6,-6);(-8,8) **@{--},
     {(2\buff,1\buff);p+(-1,.5):a(120)::
      (10\buff,0);(0,0) **@{-};(0,8\buff) **@{-}},
     0 *{\gdef\buff{.15} \gdef\cbuff{.85}},
     {(4\buff,-2\cbuff);p+(-1,.5):a(120)::
      (12\buff,0);(0,0) **@{-};(0,4\buff) **@{-}},
     {(-3\cbuff,4\buff);p+(-1,.5):a(120)::
      ( 4\buff,0);(0,0) **@{-};(0,12\buff) **@{-}},
     0 *{\gdef\buff{.1} \gdef\cbuff{.9}},
     {(-5\cbuff,2\buff);p+(-1,.5):a(120)::
      ( 2\buff,0);(0,0) **@{-};(0,8\buff) **@{-}},
     {(1\buff,-5\cbuff);p+(-1,.5):a(120)::
      ( 9\buff,0);(0,0) **@{-};(0,2\buff) **@{-}},
     0 *{\gdef\buff{.15} \gdef\cbuff{.85}},
     {(-2\cbuff,-3\cbuff);p+(-1,.5):a(120)::
      ( 5\buff,0);(0,0) **@{-};(0,2\buff) **@{-}},
     {0;0 \croplattice{-10}{10}{-10}{10} {-8.1}{6.1}{-7.1}{5.1}},
     @={(2,1),(-2,3),(-3,2),(-1,-2),(1,-3),(3,-1),(0,2),(-2,0),(2,-2)}
     @@{*{\psframe*[linecolor=white](-.1,-.15)(.1,.15)} *{1}},
     @i @={(1,0),(0,-1),(-1,1)}
     @@{*{\psframe*[linecolor=white](-.15,-.17)(.15,.17)} *{2}},
%      (0,2)  *{\psframe*[linecolor=white](-.05,-.14)(.05,.141)} *{1},
%      (-2,0) *{\pspolygon*[linecolor=white](.05,-.05)(.05,.05)(.03,.15)(-.08,.15)(-.08,-.05)} *{1},
%     (2,-2)  *{\psframe*[linecolor=white](-.07,-.14)(.07,.14)} *{1},
     @i @={(2,1),(-4,4),(-6,2),(-3,-4),(1,-6),(4,-3)}
     @@{*+++{\,};p+(1,1) *+\cir<4pt>{} **@{-}, {?<>(1)*@{>}},},
     @i @={(-4,4),(-3,-4),(4,-3)}
     @@{*!UR{-}}
     @i @={(-6,2),(1,-6)}
     @@{ *!UR{+}}
   \end{xy}
 \]
 For now, just observe that if we shift the weights that appear (by something we will
 call the Weyl vector), we get an orbit of the Weyl group, with signs alternating
 according to the length of the element of the Weyl group.
 \end{example}
% \[\def\buff{.3} \def\cbuff{.7}
%    \begin{xy}<-2.5em,0em>:a(120)::
%    (0,0) *+{1};(0,1) *+{1}, (0,2) *+{0} *\cir{};(0,3) *+{0};(0,4) *+{0};
%    (1,0) *+{1};(1,1) *+{2};(1,2) *+{1};(1,3) *+{0};(1,4) *+{0};
%    (2,0) *+{1};(2,1) *+{2};(2,2) *+{2};(2,3) *+{1};(2,4) *+{0};
%    (3,0) *+{0} *\cir{};(3,1) *+{1};(3,2) *+{1};(3,3) *+{1};(3,4) *+{0};
%    (4,0) *+{0};(4,1) *+{0};(4,2) *+{0};(4,3) *+{0};(4,4) *+{0};
%    @={(-\buff,-\buff),(2,-\buff),(3\buff,1),(3\buff,3\buff),(2,3\buff),(-\buff,1)},
%    s0="prev" @@{;"prev";**@{-}="prev"},
%    (4,-\buff); (2\cbuff,-\buff) **@{-}; (2\cbuff,4) **@{-};
%    (4,1\cbuff); (-\buff,1\cbuff) **@{-}; (-\buff,4\buff) **@{-};
% \end{xy}
% \]
% \mpar{Explain this picture better}
%   In the quotient, the shown weights disappear. What about the one below it? Its
%   multiplicity in the Verma module is 2. The multiplicity in the Verma module
%   centered at $2\w_1$ is 1, so in the quotient, you get multiplicity 1. How about
%   right below it? You had 3 from the big Verma module, and you remove 1 and 1, so you
%   get 1 left. Furthur down and to the left, you get all zeros. Similarly, we can fill
%   in the other numbers. You get the leftover hexagon.

 Some notation: if $w\in \weyl$, we define $(-1)^w := \det(w)$. Since each simple
 reflection has determinant $-1$, this is the same as $(-1)^{\text{length}(w)}$. Note
 that $(-1)^{w'w}=(-1)^{w'}(-1)^w$.

 The \emph{Weyl vector}\index{Weyl vector|idxbf} is $\rho = \frac{1}{2} \sum_{\alpha
 \in \Delta^+}\alpha$. Note that $r_i(\rho) = \rho-\alpha_i$ by Lemma
 \ref{lec14L:key}. On  the other hand, $r_i(\rho)= \rho - (\rho,\check
 \alpha_i)\alpha_i$, so we know that $(\rho,\check \alpha_i)=1$ for all $i$. Thus,
 $\rho$ is the sum of all the fundamental weights.

 \begin{theorem}[Weyl Character Formula]
   For $\lambda\in P^+$, the character of the irreducible finite dimensional
   representation with highest weight $\lambda$ is\footnote{ This formula may look
  ugly, but it is \emph{sweet}. It says that you can compute the character of
  $V(\lambda)$ in the following way. Translate the Weyl vector $\rho$ around by the
  Weyl group; this will form some polytope. Make a piece of cardboard shaped like this
  polytope (ok, so maybe this is only practical for rank 2), and put $(-1)^w$ at the
  vertex $w(\rho)$. This is your \emph{cardboard denominator}.\index{cardboard
  denominator|see{Weyl denominator}} Now the formula tells you that when you center
  your cardboard denominator around any weight, and then add the multiplicities of the
  weights of $V(\lambda)$ at the vertices with the appropriate sign, you'll get zero
  (unless you centered your cardboard
  \begin{window}[1,r,%
      {\begin{xy}<-1.75em,0em>:a(120)::
       (-1,0)*+{0};(-1,1)*+{0};(0,2) *+{0};
       (0,0) *+{1};(0,1) *+{1};
       (1,0) *+{1};(1,1) *+{?};(1,2) *+{1};
       (2,0) *+{1};(2,1) *+{?};(2,2) *+{?};(2,3) *+{1};
       (3,0)      ;(3,1) *+{1};(3,2) *+{1};(3,3) *+{1};
       @={(0,0),(1,1),(1,2),(0,2),(-1,1),(-1,0)},
       s0="prev" @@{;"prev" *=<2.5mm>{\ }; *=<2.5mm>{\ } **@{-}="prev"},
%       (1.65,1.35);(-1,0) **@{.}, (1,0) **@{.}
     \end{xy}},]
  \noindent at $w(\lambda+\rho)$, in which case only one non-zero multiplicity shows
  up in the sum, so you'll get $\pm 1$). Since we know that the highest weight has
  multiplicity 1, we can use this to compute the rest of the character.\\
  For $\sl(3)$, your cardboard denominator will be a hexagon, and one step of
  computing the character of $V_{2\w_1+\w_2}$ might look like:\\
  $0=0-1+\,?-1+0-0$, so $?=2$. Since $ch\, V$ is symmetric with respect to $\weyl$,
  all three of the $?$s must be 2.
  \end{window}}
 \[
  ch\, V(\lambda) = \frac{\sum_{w\in \weyl} (-1)^w e^{w(\lambda+\rho)}}{\sum_{w\in \weyl}
    (-1)^w e^{w(\rho)}}.
 \]
 \end{theorem}
 The denominator is called the \emph{Weyl denominator}\index{Weyl denominator|idxbf}.
 It is not yet obvious that the Weyl denominator divides the numerator (as formal
 sums), so one may prefer to rewrite the equation as $  ch\, V(\lambda)\cdot
 \sum_{w\in \weyl}(-1)^w e^{w(\rho)} = \sum_{w\in \weyl} (-1)^w e^{w(\lambda+\rho)}$.
 \begin{proof}
 %\begin{trivlist}\item \end{trivlist}
  \underline{Step 1. Compute $ch\,M(\gamma)$}: Recall from the previous lecture that
  the multiplicity of $\mu$ in $M(\gamma)$ is the number of ways $\gamma-\mu$ can be
  written as a sum of positive roots. Thus, it is easy to see that $ch\,M(\gamma)$
  is given by the following generating function.
  \begin{align*}
    ch\,M(\gamma) &= e^\gamma \prod_{\alpha\in
    \Delta^+}(1+e^{-\alpha}+e^{-2\alpha}+\cdots)\\
    &= \frac{e^\gamma}{\prod_{\alpha\in \Delta^+}(1-e^{-\alpha})} \\
    &= \frac{e^{\gamma+\rho}}{\prod_{\alpha\in \Delta^+}(e^{\alpha/2}-e^{-\alpha/2})}
    & (\textstyle\prod_{\alpha\in \Delta^+}e^{\alpha/2} = e^\rho )
  \end{align*}

  \underline{Step 2. The action of the Casimir operator}: Recall the Casimir operator
  \index{Casimir operator|idxit} from the proof of Whitehead's Theorem (Theorem
  \ref{lec12Whitehead}). If $\{e_i\}$ is a basis for $\g$, and $\{f_i\}$ is the dual
  basis (with respect to the Killing form), then $\W = \sum e_if_i\in U\g$. We showed
  that $\W$ is in the center of $U\g$ (i.e.\ that $\W x = x\W$ for all $x\in \g$).
  \begin{claim}
    $\W$ acts on $M(\gamma)$ as $(\gamma,\gamma+2\rho)\id$.
  \end{claim}
  \begin{proof}[Proof of Claim]  \renewcommand\qedsymbol{$\Box_\text{Claim}$}
  Since $\W$ is in the center of $U\g$, it is enough to show that
  $\W v=(\gamma+2\rho,\gamma) v$ for a highest weight vector $v\in V_\gamma$.

  Let $\{u_i\}$ be an orthonormal basis for $\h$, and let $\{X_\alpha\}_{\alpha\in
  \Delta}$ be a basis for the rest of $\g$. The dual basis is
  $\Big\{\frac{Y_\alpha}{(X_\alpha,Y_\alpha)}\Big\}$. Then we get
  \begin{align*}
    \W &= \sum_{i=1}^n u_i^2 + \sum_{\alpha\in \Delta}\frac{X_\alpha
    Y\alpha}{(X_\alpha,Y_\alpha)}\\
    &= \sum_{i=1}^n u_i^2 + \sum_{\alpha\in \Delta^+}\frac{X_\alpha
    Y\alpha}{(X_\alpha,Y_\alpha)}+\frac{Y_\alpha X_\alpha}{(X_\alpha,Y_\alpha)}
    & (X_{-\alpha} Y_{-\alpha} = Y_\alpha X_\alpha) .
  \end{align*}
  Using the equalities
  \begin{gather*}
   \begin{array}{rl}
     u_i v &\!\!\!= \gamma(u_i) v,\\
     X_\alpha v&\!\!\!=0,\\
     X_\alpha Y_\alpha v &\!\!\!= H_\alpha v - Y_\alpha X_\alpha v \\
       &\!\!\!= \gamma(H_\alpha) v,\\
     \gamma(H_\alpha) &\!\!\!= \frac{2(\gamma,\alpha)}{(\alpha,\alpha)},
   \end{array}\qquad
   \begin{array}{rl}
     (\gamma,\gamma) &\!\!\!= \sum_{i=1}^n \gamma(u_i)^2,\text{ and}\\
     (X_\alpha,Y_\alpha)&\!\!\!= \half ([H_\alpha,X_\alpha],Y_\alpha) \\
     &\!\!\!= \half(H_\alpha,[X_\alpha,Y_\alpha])\\
     &\!\!\!= \half (H_\alpha, H_\alpha) = \half \cdot \frac{2\alpha(H_\alpha)}{(\alpha,\alpha)}\\
     &\!\!\!= \frac{2}{(\alpha,\alpha)}
   \end{array}
  \end{gather*}
  we get
  \begin{align*}
    \W v &= \Bigl(\sum_{i=1}^n \gamma(u_i)^2\Bigr) v + \sum_{\alpha\in \Delta^+}
    \frac{\gamma(H_\alpha)}{(X_\alpha,Y_\alpha)} v\\
    &= (\gamma,\gamma) v + \sum_{\alpha\in \Delta^+}(\gamma,\alpha)v
    = (\gamma,\gamma+2\rho) v \qedhere
  \end{align*}
  \end{proof}

  Note that the universal property of Verma modules implies that the action of $\W$ on
  any representation generated by a highest vector of weight $\gamma$ is given by
  $(\gamma,\gamma+2\rho)\id$.
  \smallskip

  Finally, consider the set
  \[
    \W^\gamma = \{\mu\in P| (\mu+\rho,\mu+\rho) = (\gamma+\rho,\gamma+\rho)\}.
  \]
  This is the intersection of the weight lattice $P$ with the sphere of radius
  $\|\gamma+\rho\|$ centered at $-\rho$. In particular, it is a \emph{finite set}. On
  the other hand, since $(\gamma,\gamma+2\rho) =
  (\gamma+\rho,\gamma+\rho)-(\rho,\rho)$, it is also the set of weights $\mu$ such
  that $\W$ acts on $M(\mu)$ in the same way it acts on $M(\gamma)$.

  \underline{Step 3. Filter $M(\gamma)$ for another formula}: We say that a weight
  vector $v$ is a \emph{singular vector} if $\n^+v=0$. If a representation is
  generated by some highest vector $v$, and if all singular vectors are proportional to
  $v$, then the representation is irreducible. To see this, note that a highest weight
  vector of any proper subrepresentation must be singular, and it cannot be
  proportional to $v$, lest it generate the whole representation.

  Now let $w$ be a singular vector of weight $\mu$ in $M(\gamma)$. Then $w$ generates
  a subrepresentation which is a quotient of $M(\mu)$. By the claim in Step 2, $\W$
  acts on this subrepresentation by $(\mu,\mu+2\rho)$. On the other hand, since we are
  in $M(\gamma)$, $\W$ must act by $(\gamma,\gamma+2\rho)$. It follows that $\mu\in
  \W^\gamma$.

  In particular, since $\W^\gamma$ is finite, there is a minimal singular vector $w$,
  which generates some irreducible subrepresentation; we will call that representation
  $F_1M(\gamma)$. Mod out my $F_1M(\gamma)$ and repeat the process. Any singular
  vector in $M(\gamma)/F_iM(\gamma)$ must be in $\W^\gamma$, so there is a minimal
  one, $w$, which generates an irreducible subrepresentation. Define
  $F_{i+1}M(\gamma)\subseteq M(\gamma)$ to be the pre-image of that representation.
  Since $\W^\gamma$ is finite and each $V_\mu$ is finite dimensional, the process
  terminates. The result is a filtration
  \[
    0=F_0M(\gamma) \subseteq F_1M(\gamma)\subseteq \cdots \subseteq
    F_kM(\gamma)=M(\gamma)
  \]
  such that $F_iM(\gamma)/F_{i+1}M(\gamma)$ is isomorphic to the irreducible
  representation $V(\mu)$ for some $\mu\in \W^\gamma$.\footnote{We showed in Lecture
  18 that for every $\mu\in \h^*$, there is a unique irreducible representation
  $V(\mu)$ with highest weight $\mu$. However, we only showed that $V(\mu)$ is finite
  dimensional when $\mu\in P^+$. In general, it is infinite dimensional. In fact,
  sometimes it happens that $V(\mu)=M(\mu)$.\anton{Is it true that $V(\mu)=M(\mu)$ if
  and only if for all $i$, $(\mu,\check\alpha_i)\not\in \ZZ$?}} We also know that each
  $\mu$ that appears is less than or equal to $\lambda$.

  This gives us the nice formula
  \[
    ch\, M(\gamma) = \sum_{\makebox[0pt]{$\scriptstyle \mu\le \gamma,\, \mu\in
    \W^\gamma$}} b_{\gamma\mu}  ch\, V(\mu)
  \]
  for some non-negative integers $b_{\gamma\mu}$.\footnote{These $b_{\gamma\mu}$ are
  called \emph{Kazhdan-Luztig multiplicities}\index{Kazhdan-Luztig multiplicities},
  and they are hard to compute for general $\gamma$ and $\mu$.} Moreover, $V(\gamma)$
  appears as a quotient exactly once, so $b_{\gamma\gamma}=1$.

  \underline{Step 4. Invert and simplify the equation}: We've shown that the matrix
  $(b_{\gamma\mu})_{\gamma,\mu\in \W^\lambda}$ is lower triangular with ones on the
  diagonal, so it has a lower triangular inverse $(c_{\gamma\mu})_{\gamma,\mu\in
  \W^\lambda}$ with ones on the diagonal.\footnote{We will prove that each non-zero
  $c_{\gamma\mu}$ is $\pm 1$. It was once conjectured that even if $\lambda\not\in
  P^+$, each non-zero $c_{\gamma\mu}$ is $\pm 1$, but this is false.} This gives us the
  formula
  \[
    ch\, V(\lambda) = \sum_{\makebox[0pt]{$\scriptstyle \mu\le \lambda,\, \mu\in
    \W^\lambda$}} c_{\lambda\mu} ch\, M(\mu).
  \]
  Using Step 1, we can rewrite this as
  \[
    ch\,V(\lambda) \cdot \prod_{\alpha\in \Delta^+}(e^{\alpha/2}-e^{-\alpha/2}) =
    \sum_{\makebox[0pt]{$\scriptstyle \mu\le \lambda,\, \mu\in \W^\lambda$}} c_{\lambda\mu}e^{\mu+\rho}.
  \]
  For any element $w$ of the Weyl group, we know that $w($LHS$)=(-1)^w$LHS, so the
  same must be true of the RHS, i.e.
  \[
    \sum c_{\lambda\mu} e^{w(\mu+\rho)} = \sum (-1)^w c_{\lambda\mu} e^{\mu+\rho}.
  \]
  This is equivalent to the condition $c_{\lambda, w(\mu+\rho)-\rho}=c_{\lambda\mu}$.
  Since $P^+$ is the fundamental domain of $\weyl$, and since $c_{\lambda\lambda}=1$,
  we get
  \[
    ch\,V(\lambda) \cdot \prod_{\alpha\in \Delta^+}(e^{\alpha/2}-e^{-\alpha/2}) =
        \sum_{w\in \weyl}(-1)^w e^{w(\lambda+\rho)} +
        \sum_{\makebox[0pt]{$\substack{\mu< \lambda,\, \mu\in \W^\lambda\\ \mu+\rho
        \in P^+}$}} (-1)^w c_{\lambda\mu} e^{w(\mu+\rho)}.
  \]

  We would like to eliminate the second sum on the right hand side. The following
  claim does that nicely by showing that the sum is empty.
  \begin{claim}
    If  $\mu\le \lambda$, $\mu\in \W^\lambda$, and $\mu+\rho \ge 0$, then $\mu=\lambda$.
  \end{claim}
  \begin{proof}
  We assume that $(\mu+\rho,\mu+\rho)=(\lambda+\rho,\lambda+\rho)$ and $\lambda-\mu =
  \sum_{i=1}^n k_i \alpha_i$ for some non-negative $k_i$. Then we get
  \begin{align*}
    0 &= \big( (\lambda+\rho)-(\mu+\rho),(\lambda+\rho)+(\mu+\rho) \big)\\
      &= (\lambda-\mu, \lambda+\mu+2\rho)\\
      &= \sum_{i=1}^n k_i (\alpha_i, \lambda+\mu+2\rho)
  \end{align*}
  But $\lambda\ge 0$ and $\mu+\rho\ge 0$, so $(\alpha_i,\lambda+\mu+\rho)\ge 0$. Also,
  $(\alpha_i,\rho)>0$ for each $i$, so $(\alpha_i, \lambda+\mu+2\rho)>0$. It follows
  that each $k_i$ is zero.
  \renewcommand{\qedsymbol}{$\Box_\text{Claim}$\quad}
  \end{proof}

  Now we have
  \[
    ch\, V(\lambda) \cdot \prod_{\alpha\in \Delta^+}(e^{\alpha/2}-e^{-\alpha/2}) =
    \sum_{w\in \weyl}(-1)^w e^{w(\lambda+\rho)}.
  \]
  Specializing to the case $\lambda=0$, we know that $V(0)$ is the trivial
  representation, so $ch\, V(0)=1$. This tells us that
  \begin{equation}
    \prod_{\alpha\in \Delta^+}(e^{\alpha/2}-e^{-\alpha/2}) = \sum_{w\in \weyl}(-1)^w
    e^{w(\rho)}, \label{lec19WeylD}
  \end{equation}
  so we get the desired
  \[
    ch\, V(\lambda) =\frac{\sum_{w\in \weyl}(-1)^w e^{w(\lambda+\rho)}}{\sum_{w\in
    \weyl}(-1)^w e^{w(\rho)}}. \qedhere
  \]\index{Verma module|)}
  \end{proof}
 \begin{corollary}[Weyl dimension formula]\index{Weyl dimension formula|idxbf}
   $\displaystyle \dim V(\lambda) = \prod_{\alpha\in \Delta^+} \frac{(\lambda+\rho,
    \alpha)}{(\rho,\alpha)}$.
 \end{corollary}
 \begin{proof}\newcommand\f{6}
   The point is that $e^\mu$ is a formal expression. The only property that we use is
   $e^\mu e^\gamma = e^{\mu+\gamma}$, so everything we've ever done with characters
   works if we replace $e^\mu$ by any other expression satisfying that relation. In
   particular, if replace $e^\mu$ with $\f^{t(\gamma+\rho,\mu)}$, where $t$ is a real
   number,\footnote{Obviously, there is nothing special about the base $\f$; just
   about any number would work. It is important to understand that for any $\mu$,
   $t\mapsto \f^{t(\gamma+\rho,\mu)}$ is an honest real-valued function in $t$.
   Equation \ref{lec19ast} is an equality of \emph{real-valued functions} in $t$! Similarly,
   $ch\, V(\lambda)$ becomes a real-valued function in $t$.} then
   Equation \ref{lec19WeylD} says
   \begin{align}
     \prod_{\alpha\in \Delta^+} \Bigl(\f^{t(\gamma+\rho,\alpha/2)} -
     \f^{-t(\gamma+\rho,\alpha/2)}\Bigr) &=
     \sum_{w\in \weyl}(-1)^w \f^{t(\gamma+\rho,w(\rho))}\notag\\
     &=\sum_{w\in \weyl}(-1)^w \f^{t(w(\gamma+\rho),\rho)} \label{lec19ast}
   \end{align}
   where the second equality is obtained by replacing $w$ by $w^{-1}$ and observing
   that $(x,w^{-1}y)=(w\,x,y)$ and that $(-1)^{w^{-1}}=(-1)^w$.

   Now we switch things up and replace $e^\mu$ by $\f^{t(\mu,\rho)}$, so the character
   formula becomes
   \[
     ch\, V(\lambda) =\frac{\sum_{w\in \weyl}(-1)^w
     \f^{t(w(\lambda+\rho),\rho)}}{\sum_{w\in \weyl}(-1)^w
     \f^{t(w(\rho),\rho)}}.
   \]
   Applying Equation \ref{lec19ast} to the numerator (with $\gamma=\lambda$) and to the
   denominator (with $\gamma=0$), we get
   \[
     ch\, V(\lambda) = \prod_{\alpha\in \Delta^+}
     \frac{\bigl(\f^{t(\lambda+\rho,\alpha/2)} -
     \f^{-t(\lambda+\rho,\alpha/2)}\bigr)}{\bigl(\f^{t(\rho,\alpha/2)} -
     \f^{-t(\rho,\alpha/2)}\bigr)}.
   \]
   The dimension of $V(\lambda)$ is equal to the expression $ch\, V(\lambda)$ with
   $e^\mu$ replaced by $1$. We can obtain this by letting $t$ tend to zero in
   $\f^{t(\mu,\rho)}$. This gives
   \begin{align*}
     \dim V(\lambda) &= \lim_{t\to 0}\prod_{\alpha\in \Delta^+}
     \frac{\bigl(\f^{t(\lambda+\rho,\alpha/2)} -
     \f^{-t(\lambda+\rho,\alpha/2)}\bigr)}{\bigl(\f^{t(\rho,\alpha/2)}
     - \f^{-t(\rho,\alpha/2)}\bigr)}\\
     &= \prod_{\alpha\in \Delta^+}\frac{(\lambda+\rho,\alpha)}{(\rho,\alpha)}. &
     \text{(By l'H\^{o}pital's rule)\qquad} \qedhere
   \end{align*}
 \end{proof}
 \begin{example}
   Let $\g = \sl(n+1)$.\index{sl(n)@$\sl(n)$|idxit} We choose the standard set of
   simple roots $\Pi=\{\alpha_1,\dots, \alpha_n\}$ so that $\Delta^+ =
   \{\alpha_i+\alpha_{i+1}+\cdots+\alpha_j\}_{1\le i\le j\le n}$. Recall that
   $(\rho,\alpha_i)=1$ for $1\le i\le n$ and that $(\w_i,\alpha_j)=\delta_{ij}$. If
   $\lambda+\rho = \sum_{i=1}^n a_i\w_i$, the dimension formula tells us that
   \begin{align*}
     \dim V(\lambda) &= \prod_{\alpha\in \Delta^+}
            \frac{(\lambda+\rho,\alpha)}{(\rho,\alpha)}\\
         &= \prod_{1\le i\le j\le n}\frac{a_i+a_{i+1}+\cdots+a_{j-1}+a_j}{j-i+1}\\
         &= \frac{1}{n!!}\prod_{1\le i\le j\le n} \sum_{k=i}^j a_k
   \end{align*}
   where $n!!:= n!\,(n-1)!\cdots 3!\,2!\,1!$.

   If $\g=\sl(3)$, and if $\lambda+\rho=3\w_1+2\w_2$, we get $\dim V(\lambda) =
   \frac{1}{2!!}\cdot 2\cdot 3\cdot (2+3) =15$, computing the dimension of the
   representation in Example \ref{lec19Eg:2w1+w2}.
   This formula is nice because the calculation does not get big as $\lambda$ gets
   big. If $\lambda+\rho=20\w_1 + 91\w_2$, it would be really annoying to compute
   $ch\, V(\lambda)$ completely, but we can get $\dim V(\lambda) = \half
   20\cdot 91\cdot 111 = 101010$ easily.

   Even for larger $n$, this formula is pretty good. Say we want the dimension of
   $\begin{xy}
    (0,-.15) *+!D{1} *\cir<2pt>{};
    p+(1,0) *+!D{2} *\cir<2pt>{} **@{-};
    p+(1,0) *+!D{0} *\cir<2pt>{} **@{-};
    p+(1,0) *+!D{6} *\cir<2pt>{} **@{-};
  \end{xy}$, then $\lambda+\rho = 2\w_1+ 3\w_2+ 1\w_3+ 7\w_4$, so we get
  {\small \[\frac{1}{4!!}2\cdot 3\cdot 1\cdot 7\cdot
  (2+3)(3+1)(1+7)(2+3+1)(3+1+7)(2+3+1+7)=20020.\]}
 \end{example}
% \begin{exercise}
%   Find similar formulas for the representations of the other classical algebras.
%   \begin{solution}
%     We computed the root systems explicitly in Lecture 15. Choose $\alpha_1,\dots,
%     \alpha_{n-1}$ to look like the simple roots of an $A_{n-1}$, and let $\alpha_n$
%     be the strange simple root. Fix the notation $\lambda+\rho = \sum_{i=1}^n
%     a_i\w_i$.
%
%     \underline{$B_n$}: The positive roots are
%     \begin{gather*}
%     \begin{align*}
%       \{\e_i-\e_j\}_{i<j}\cup\{\e_i\} \cup \{\e_i+\e_j\}_{i<j} &=
%       \{\alpha_i+\cdots+\alpha_{j-1}\}_{i<j}\cup \{\alpha_i+\cdots +\alpha_n\}\cup\\
%            &\cup \{(\alpha_i+\cdots +\alpha_n)+(\alpha_j+\cdots +\alpha_n)\}_{i<j}
%            ,\text{ so}
%     \end{align*}\\
%     \begin{align*}
%       \dim V(\lambda) &= \prod_{1\le i<j\le n} \frac{\sum_{k=i}^{j-1} a_k}{j-i}
%       \times \prod_{1\le i\le n} \frac{\sum_{k=i}^n a_k}{n-i+1}
%       \times \prod_{1\le i<j\le n} \frac{\sum_{k=i}^n a_k + \sum_{k=j}^n a_k}{2n-i-j+2}\\
%       &= \frac{1}{n!!} \Bigl(\prod_{1\le i<j\le n} \sum_{k=i}^{j-1} a_k\Bigr)
%       \times \Bigl(\prod_{1\le i\le n}\sum_{k=i}^n a_k\Bigr).
%     \end{align*}
%     \end{gather*}
%
%     \underline{$C_n$}: The positive roots are
%     \[
%        \{\e_i-\e_j\}_{i<j}\cup \{2\e_i\} = \{\alpha_i+\cdots+\alpha_{j-1}\}_{i< j}\cup
%        \{2\alpha_i+\cdots+2\alpha_{n-1}+\alpha_n\},\text{ so}
%     \]
%     \begin{align*}
%       \dim V(\lambda) &= \prod_{1\le i<j\le n} \frac{\sum_{k=i}^{j-1} a_k}{j-1-i}
%       \times \prod_{1\le i\le n} \frac{a_n+\sum_{k=i}^{n-1} 2a_k}{2n-2i+1}\\
%       &= \frac{2^{n-2}\, (n-2)!}{n!!\, (2n-3)!} \Bigl(\prod_{1\le i<j\le n}
%       \sum_{k=i}^{j-1} a_k\Bigr) \times \biggl(\prod_{1\le i\le n}\Bigl(a_n+2\sum_{k=i}^{n-1}
%       a_k\Bigr)\biggr).
%     \end{align*}
%
%     \underline{$D_n$}: The positive roots are
%     \begin{align*}
%        \{\e_i-\e_j\}_{i<j}\cup \{\e_i&+\e_j\}_{i<j} =
%        \{\alpha_i+\cdots+\alpha_{j-1}\}_{i< j}\quad \cup\\
%           &\cup \quad \bigl\{(\alpha_i+\cdots+\alpha_{n-2}+\alpha_n)+(\alpha_j+\cdots
%           +\alpha_{n-1}) \bigr\},\text{ so}
%     \end{align*}
%     \begin{align*}
%       \dim& V(\lambda) = \prod_{1\le i<j\le n} \frac{\sum_{k=i}^{j-1} a_k}{j-1-i}
%       \times \prod_{1\le i\le n} \frac{a_n+\sum_{k=i}^{n-2} a_k +\sum_{k=j}^{n-1} a_k}{2n-i-j}\\
%       &= \frac{1}{\prod_{k=1}^{n-1} (2k-1)!} \Bigl(\prod_{1\le i<j\le n} \sum_{k=i}^{j-1}
%       a_k\Bigr) \times \biggl(\prod_{1\le i<j\le n}\Bigl(a_n+\sum_{k=i}^{n-2} a_k +\sum_{k=j}^{n-1}
%       a_k\Bigr)\biggr).
%     \end{align*}
%   \end{solution}
% \end{exercise}

% In the $\sl(n)$ case, you get $\e_1+\cdots+\e_n=0$, and you get $\rho = stuff$, so
% you can compute dimension as
% \[
%    \prod_{\alpha\in \Delta^+}(\rho,\alpha) = (n-1)!(n-2)!\cdots = (n-1)!!
% \]
%
% If $\lambda = \sum a_i\e_i$, then we get
% \[
%    \prod_{i< j} (a_i+a_j + j-i)
% \]
% We can  prove this from the character formula? We have to take a limit as $e\to 1$
% (since we get $0/0$ if we take $e=1$).
%
% \[
%    \lim_{u\to 0} \frac{\sum (-1)^w e^{(w(\lambda+\rho),u)}}{\sum (-1)^w e^{(w(\rho),u)}}
% \]
% Let $u=\rho t$ for $t\to 0$, then we get
% \[
%    \frac{\prod_{\alpha\in \Delta^+} e^{t(\lambda+\rho,\alpha/2)}-e^{-t(\lambda+\rho,\alpha/2)}}{\prod_{\alpha\in \Delta^+} e^{t(\rho,\alpha/2)} - e^{-t(\rho,\alpha/2)}}
% \]
% using $(w(\lambda+\rho),u) = (\lambda+\rho, w^{-1}(u))$, and from here it is just an
% exercise in L'hopital.

 \begin{remark}
 Given complete reducibility, knowing the characters of all irreducible
 representations allows you to decompose tensor products, just like in representation
 theory of finite groups. That is, we can now compute the coefficients in
 $V(\lambda)\otimes V(\mu) = \bigoplus b_{\lambda \mu}^\nu V(\nu)$. In the finite
 group case, we make this easier by choosing an inner product on class functions so
 that characters of irreducible representations form an orthonormal basis. Now we
 would like to come up with an inner product on formal expressions $\sum m_\mu e^\mu$
 so that characters of irreducible representations are orthonormal.

 The obvious inner product is $\langle e^\lambda, e^\mu\rangle =
 \delta_{\lambda,\mu}$, under which the $e^\mu$ are an orthonormal basis. There is no
 hope for the $ch\, V(\lambda)$ to be orthogonal, but we can tweak it. Another inner
 product is
 \[
    (e^\lambda,e^\mu) = \frac{1}{|\weyl|} \langle D\cdot e^\lambda, D\cdot
    e^\mu\rangle
 \]
 where $D$ is the Weyl denominator. The character formula tells us that under this
 inner product, the $ch\, V(\lambda)$ are orthonormal, and form a basis for
 \emph{$\weyl$-symmetric} expressions where $m_\mu=0$ for $\mu\not\in P$.

 As with the character formula, this may not look so impressive, but it makes
 decomposing tensor products very fast. We want to compute
 \[
   \bigl(ch\, V(\lambda) \cdot ch\, V(\mu),ch\, V(\gamma)\bigr) =
   \frac{1}{|\weyl|}\bigl\langle
   \underbrace{D\cdot ch\, V(\lambda)}_{
   \makebox[0pt]{\scriptsize $\sum (-1)^w e^{w(\lambda+\rho)}$}}
   \cdot ch\, V(\mu),
   \underbrace{D\cdot ch\,V(\gamma)}_{
   \makebox[0pt]{\scriptsize $\sum (-1)^w e^{w(\gamma+\rho)}$}}
   \bigr\rangle
 \]
 for all $\gamma\in P^+$. Since we know that the result must be $\weyl$-symmetric, we
 can remove the $\frac{1}{|\weyl|}$ and restrict our attention to the Weyl chamber.
 That is, we can just compute $\bigl\langle \sum (-1)^w e^{w(\lambda+\rho)}\cdot ch\,
 V(\mu),e^{\gamma+\rho}\bigr\rangle$, which is the multiplicity of $\gamma$ in
 $\sum (-1)^w e^{w(\lambda+\rho)-\rho}ch\, V(\mu)$. In practice, we choose $|\mu|\le
 |\lambda|$, so most of the summands lie outside of the Weyl chamber, so
 we can ignore them.
 \end{remark}

 \begin{example}[For those who know about $\gl(n)$]\index{gl(n)@$\gl(n)$|idxit} We
 know that $\gl(n+1)$ is the direct sum (as a Lie algebra) of its center, $k\cdot
 \id$, and $\sl(n+1)$.\footnote{ In general, a Lie algebra which is the the direct sum
 of its center and its semisimple part is called \emph{reductive}\index{reductive}.}
 \anton{what does that tell you about its irreps?} Let $\{\e_1,\dots, \e_{n+1}\}$
 be the image of an orthonormal basis of $k^{n+1}$ in $k^n$ (under the usual
 projection, so that $\sum \e_i=0$). Let $z_i = e^{\e_i}$, so $z_1\cdots z_{n+1}=1$.
 The Weyl group $W\simeq S_{n+1}$ acts on the $z_i$ by permutation. We have that
 \begin{align*}
   \rho &= \frac{1}{2} \sum_{i< j} \e_i-\e_j \\
   &= \frac{n}{2}\e_1 + \frac{n-2}{2}\e_2+\cdots + \frac{-n}{2}\e_{n+1} \\
   &= n\e_1 + (n-1)\e_2 + \cdots + 2\e_{n-1} + \e_n + 0\e_{n+1} &
      \bigl(\textstyle\sum \e_i=0\bigr)
 \end{align*}
 so
 \[
   e^\rho = z_1^n z_2^{n-1}\cdots z_n^1 z_{n+1}^0.
 \]
 If $\lambda = \sum_{i=1}^{n+1} a_i\e_i$ (with $\sum a_i=0$) is a dominant integral
 weight, we have $(\lambda, \check \alpha_i)=a_i-a_{i+1} \ge 0$. The character formula
 says that
 \begin{align*}
   ch\, V(\lambda) &= \frac{\sum_{\sigma\in S_{n+1}} (-1)^\sigma
  z_{\sigma(1)}^{a_1+n} \cdots z_{\sigma(n)}^{a_n +1} z_{\sigma(n+1)}^{a_{n+1}}}{\sum_{\sigma\in S_{n+1}}
  (-1)^\sigma z_{\sigma(1)}^n\cdots z_{\sigma(n)}^1 z_{\sigma(n+1)}^0}
 \end{align*}
 The denominator (call it $D$) is the famous Vandermonde determinant,\index{Vandermonde determinant}
 \[
  \renewcommand\arraystretch{1.3}
    \det\left( \begin{array}{cccc}
    z_1^n & z_2^n &\cdots & z_{n+1}^n\\
    z_1^{n-1} & z_2^{n-1} & \cdots & z_{n+1}^{n-1}\\
    \vdots & \vdots & \ddots & \vdots\\
    z_1 & z_2& \cdots & z_{n+1}\\
    1 & 1 & \cdots & 1
  \end{array}\right)
 \renewcommand\arraystretch{3}
   \raisebox{1.5ex}{\mbox{$\begin{array}{l}
    =\displaystyle\sum_{\sigma\in S_{n+1}} (-1)^\sigma z_{\sigma(1)}^n\cdots z_{\sigma(n)}^1
      z_{\sigma(n+1)}^0 \\
    =\displaystyle\prod_{1\le i<j\le n+1}(z_j-z_i)\\
  \end{array}$}}
 \]
 The numerator is
 \[
   D_\lambda =   \renewcommand\arraystretch{1.3}
    \det\left( \begin{array}{cccc}
    z_1^{a_1+n} & z_2^{a_1+n} &\cdots & z_{n+1}^{a_1+n}\\
    z_1^{a_2+n-1} & z_2^{a_2+n-1} & \cdots & z_{n-1}^{a_2+n+1}\\
    \vdots & \vdots & \ddots & \vdots\\
    z_1^{a_n+1} & z_2^{a_n+1}& \cdots & z_{n+1}^{a_n+1}\\
    z_1^{a_{n+1}} & z_1^{a_{n+1}} & \cdots & z_1^{a_{n+1}}
  \end{array}\right)
 \]
 So the character is the Schur polynomial.\index{Schur polynomial}

 Usually, the representations are encoded as Young diagrams. The marks on the dynkin
 diagram are the differences in consecutive rows in the young diagram.\anton{what does
 this part mean?}
 \end{example}
 \index{Weyl character formula|)idxbf}
}{   % Anton, geraschenko@gmail.com
  \stepcounter{lecture}
 \setcounter{lecture}{20}
 \sektion{Lecture 20 - Compact Lie groups}\index{compact groups|(}
 \anton{Things that should be here:
 \begin{enumerate}
   \item Rep theory of reductive Lie \rlap{\rule[3pt]{3.6em}{.4pt}}algebras groups
   \item fundamental group of a compact group is finite?
   \item That one paragraph should be expanded. $|Z(G)|\le |P/Q|=\det($Cartan), with
   equality if $G$ simply connected.
   \item stuff about maximal tori? e.g.\ $\exp$ is surjective for a compact group
   \item uniqueness of the compact real form?
   \item Peter-Weyl theorem?
   \item Finite-dimensional representations of a compact group are unitary.
 \end{enumerate}}

 So far we classified semisimple Lie algebras over an algebraically closed field
 characteristic 0. Now we will discuss the connection to compact groups.
 Representations of Lie groups are always taken to be smooth.
 \begin{example}\index{SU(n)@$SU(n)$|idxit}
   $SU(n) = \{X\in GL(n,\CC)| \bar X^tX=\id \text{ and } \det X=1\}$ is a compact
   connected Lie group over $\RR$. It is the group of linear transformations of
   $\CC^n$ preserving some hermitian form.

   You may already know that $SU(2)$ is topologically a 3-sphere.
 \end{example}
 \begin{exercise}
   If $G$ is an abelian compact connected Lie group, then it is a product of circles,
   so it is $\mathbb{T}^n$.
   \begin{solution}
     Since $G$ is abelian, $\g$ is the abelian Lie algebra $\RR^n$, whose simply
     connected Lie group is $\RR^n$. Thus, $G$ is a quotient of $\RR^n$ by a discrete
     subgroup (i.e.\ a lattice). Since $G$ is compact, this lattice must be full rank,
     so $G\cong \mathbb{T}^n$.
   \end{solution}
 \end{exercise}
 There exists the $G$-invariant volume form\footnote{A \emph{volume form} is a
 non-vanishing top degree form.} $\w$ satisfying
 \begin{enumerate}
 \item The volume of $G$ is one: $\int_G \w =1$, and \item $\w$ is left invariant:
 $\int_G f \w = \int_G L_h^* f\,\w$ for all $h\in G$. Recall that $L_h^*f$ is defined
 by $(L_h^* f)(g) = f(hg)$.
 \end{enumerate}
 To construct $\w$ pick $\w_e\in \Lambda^\text{top} (T_eG)^*$ and define $\w_g =
 L_{g^{-1}}^* \w_e$.
 \begin{exercise}
   If $G$ is connected, show that this $\w$ is also right invariant. Even if $G$ is
   not connected, show that the measure obtained from a right invariant form agrees
   with the measure obtained from a left invariant form.
   \begin{solution}
     Consider the representation $G\to \End (\Lambda^\text{top}T_eG)\simeq \End(\RR)=
     \RR^\times$ given by $h\mapsto \Lambda^\text{top}Ad_h$. Since $G$ is compact, its
     image must also be compact, but the only compact subgroups of $\RR^\times$ are
     $\{1\}$ and $\{\pm 1\}$.

     If $G$ is connected, the image must be $\{1\}$, so the adjoint action on
     $\Lambda^\text{top}T_eG$ is trivial. It follows that $R_h^* \w_e =
     L_h^*\w_e=\w_h$, i.e.\ that $\w$ is right invariant.

     If $G$ is not connected, then we may have $R_h^*\w_e=-\w_h$. That is, the left
     invariant form agrees with the right invariant form up to sign. Since the volume
     form determines the orientation, changing it by a sign does not change the
     measure.
   \end{solution}
 \end{exercise}
 \begin{theorem}
   If $G$ is a compact group and $V$ is a real representation of $G$, then there
   exists a positive definite $G$-invariant inner product on $V$. That is,
   $(gv,gw)=(v,w)$.
 \end{theorem}
 \begin{proof}
   Pick any positive definite inner product\footnote{Pick any basis, and declare it to
   be orthonormal.} $\langle v,w\rangle$, and define
   \[
    (v,w) = \int_G \langle gv,gw\rangle \w
   \]
   which is positive definite and invariant.
 \end{proof}
 It follows that any finite dimensional representation of a compact group $G$ is
 completely reducible (i.e.\ splits into a direct sum of irreducibles) because the
 orthogonal complement to a subrepresentation is a subrepresentation.

 In particular, the representation $Ad: G\to GL(\g)$ is completely reducible, and the
 irreducible subrepresentations are exactly the irreducible subrepresentations of the
 derivative, $ad:\g\to \gl(\g)$. Thus, we get the decomposition $\g=\g_1\oplus\cdots
 \g_k\oplus \a$, with each $\g_i$ is a one dimensional or simple ideal. We dump all
 the one dimensional $\g_i$ into $\a$, which is then the center of $\g$. Thus, the Lie
 algebra of a compact group is the direct sum of its center and a semisimple Lie
 algebra. Such a Lie algebra is called \emph{reductive}.\index{reductive}

 If $G$ is simply connected, then I claim that $\a$ is trivial. This is because the
 simply connected group connected to $\a$ must be a torus, so a center gives you some
 fundamental group\anton{why can't the rest of $\g$ somehow ``fill in'' the torus?}.
 Thus, if $G$ is simply connected, then $\g$ is semisimple.

 \begin{theorem}
   If the Lie group $G$ of $\g$ is compact, then the Killing form $B$ on $\g$ is
   negative semi-definite. If the Killing form on $\g$ is negative definite, then
   there is some compact group $G$ with Lie algebra $\g$. \anton{In fact, since
   $\pi_1$ of a compact group is finite, all groups with Lie algebra $\g$ are
   compact. Should this be in this lecture?}
 \end{theorem}
 \begin{proof}
   If you have $\g\to \gl(\g)$, and you know that $\g$ has an $ad$-invariant positive
   definite product, so it lies in $\so(\g)$. Here you have $A^t=-A$, so you have to
   check that $tr(A^2)< 0$. It is not hard to check that the eigenvalues of $A$ are
   imaginary (as soon as $A^t=-A$), so we have that the trace of the square is
   negative (or zero).

   If $B$ is negative definite, then it is non-degenerate, so $\g$ is semisimple by
   Theorem \ref{lec12Cartan}, and $-B$ is an inner product. Moreover, we have that
   \[
      -B(ad_X Y,Z) = B(Y,ad_X Z)
   \]
   so $ad_X = -ad_X^t$ with respect to this inner product. That is, the image of $ad$
   lies in $\so(\g)$. It follows that the image under $Ad$ of the simply connected
   group $\tilde G$ with Lie algebra $\g$ lies in $SO(\g)$. Thus, the image is a
   closed subgroup of a compact group, so it is compact. Since $Ad$ has a discrete
   kernel, the image has the same Lie algebra.
 \end{proof}


 How to classify compact Lie algebras?  We know the classification over $\CC$, so we
 can always take $\g\rightsquigarrow \g_\CC=\g\otimes_\RR \CC$, which remains
 semisimple. However, this process might not be injective. For example, take $\mathfrak{su}(2)
 = \{\matrix{a}{b}{-\bar b}{a}| a\in \RR i, b\in \CC\}$ and $\sl(2,\RR)$, then they
 complexify to the same thing.

 $\g$ in this case is called a \emph{real form}\index{real form} of $\g_\CC$. So you
 can start with $\g_\CC$ and classify all real forms.
 \begin{theorem}[Cartan]\index{Cartan}
   Every semisimple Lie algebra has exactly one (up to isomorphism) compact real
   form.\index{real form!compact|idxbf}
 \end{theorem}
 For example, for $\sl(2)$ it is $\mathfrak{su}(2)$.

 Classical Lie groups: $SL(n,\CC), SO(n,\CC)$ ($SO$ has lots of real forms of this,
because in the real case, you get a signiture of a form; in the complex case, all
forms are isomorphic), $Sp(2n,\CC)$. What are the corresponding compact simple Lie
groups?

 \underline{Compact real forms}: $SU(n)=$ the group of linear operators on $\CC^n$
 preserving a positive definite Hermitian form. $SL(n)=$ the group of linear operators
 on $\RR^n$ preserving a positive definite symmetric bilinear form. $Sp(2n)=$the group
 of linear operators on $\mathbb{H}^n$ preserving a positive definite Hermitian form

 We're not going to prove this theorem because we don't have time, but let's show
 existence.

 \begin{proof}[Proof of existence]
   Suppose $\g_\CC = \g\otimes_\RR \CC = \g\oplus i\g$. Then you can construct
   $\sigma: \g_\CC\to \g_\CC$ ``complex conjugation''. Then $\sigma$ preserves the
   commutator, but it is an anti-linear involution. Classifying real forms amounts to
   classifying all anti-linear involutions. There should be one that corresponds to
   the compact algebra. Take $X_1,\dots, X_n, H_1,\dots, H_n,Y_1,\dots, Y_n$
   generators for the algebra. Then we just need to define $\sigma$ on the generators:
   $\sigma(X_i)=-Y_i, \sigma(Y_i)=-X_i, \sigma(H_i)=-H_i$, and extend anti-linearly.
   This particular $\sigma$ is called the \emph{Cartan
   involution}\index{Cartan!involution|idxbf}.

   Now we claim that $\g = (\g_\CC)^\sigma = \{X|\sigma(X)=X\}$ is a compact simple
   Lie algebra. We just have to check that the Killing form is negative definite. If
   you take $h\in \h$, written as $h = \sum a_iH_i$, then $\sigma(h)=h$ implies that
   all the $a_i$ are purely imaginary. This implies that the eigenvalues of $h$ are
   imaginary, which implies that $B(h,h)<0$. You also have to check it on $X_i,Y_i$.
   The fixed things will be of the form $(aX_i-\bar aY_i)\in \g$. The Weyl group
   action shows that $B$ is negative on all of the root space.
 \end{proof}

   Look at $\exp \h^\sigma \subset G$ (simply connected), which is called the maximal
   torus $T$. I'm going to tell you several facts now. You can always think of $T$ as
   $\RR^n/L$. The point is that $\RR^n$ can be identified with $\h_{re}$, and
   $\h_{re}^*$ has two natural lattices: $Q$ (the root lattice) and $P$ (the weight
   lattice). So one can identify $T = \RR^n/L = \h_{re}/\check P$, where $\check P$ is
   the natural dual lattice to $P$, the set of $h\in \h$ such that $\langle
   \w,h\rangle \in \ZZ$ for all $\w\in P$. $G$ is simply connected, and when you
   quotient by the center, you get $Ad\, G$, and all other groups with the same
\anton{make sense of this paragraph}
   algebra are in between. $Ad\, T = \h_{re}/\check Q$. We have the sequence
   $\{1\}\to Z(G)\to T \to Ad\, T \to \{1\}$. You can check that any element is
   semisimple in a compact group, so the center of $G$ is the quotient $P/Q\simeq
   \check Q/\check P$. Observe that $|P/Q| = $ the determinant of the Cartan matrix.
   For example, if $\g=\sl(3)$, then we have $\det\matrix 2{-1}{-1}2=3$, and the
   center of $SU(3)$ is the set of all elements of the form $diag(\w,\w,\w)$ where
   $\w^3=1$.

   $G_2$ has only one real form because the det is 1?

  Orthogonality relations for compact groups:
  \[
    \int_G \chi(g)\bar\psi(g^{-1}) \w = \delta_{\chi,\psi}
  \]
  where $\chi$ and $\psi$ are characters of irreducible representations. You know that
  the character is constant on conjugacy classes, so you can integrate over the
  conjugacy classes. There is a nice picture for $SU(2)$.

  \[\begin{xy}
    (0,0) *\xycircle(2,2){},
    (-2,0);(2,0) **\crv{(-2,-1)&(2,-1)}
        **\crv{~*=<2pt>@{.} (-2,1)&(2,1)},(.5,-1) *{T},
    (-1,1.732);(-1,-1.732) **\crv{(-1.5,1.6)&(-1.5,-1.6)}
        **\crv{~*=<2pt>@{.} (-.7,1.6)&(-.7,-1.6)},
    (1.4142136,1.4142136);(1.4142136,-1.4142136) **\crv{(1.2,1.4142136)&(1.2,-1.4142136)}
        **\crv{~*=<2pt>@{.} (1.6,1.4142136)&(1.6,-1.4142136)},
    (1.55,.49) *{\bullet}, (1.28,-.61) *{\bullet},
    (-1.35,-.57) *{\bullet}, (-.79,.68) *{\bullet},
  \end{xy}\]

  The integral can be written as
  \[
    \frac{1}{|\weyl|}\int_T \chi(t)\bar\psi(t)Vol(C(t))dt
  \]
  And $Vol(C(t)) = \D(t)\bar \D(t)$. You divide by $|\weyl|$ because that is how many
  times each class hits $T$.

 \index{Serganova, Vera|)}
}{   % Hanh Duc Do, ddhanh@math
  \stepcounter{lecture}
 \setcounter{lecture}{21}
 \sektion{Lecture 21 - An overview of Lie groups} \index{Borcherds, Richard E.|(}

 The (unofficial) goal of the last third of the course is to prove no theorems. We'll
 talk about
 \begin{enumerate}
 \item Lie groups in general,
 \item Clifford algebras and Spin groups,
 \item Construction of all Lie groups and all representations. You might say this
 is impossible, so let's just try to do all simple ones, and in particular
 $E_8,E_7,E_6$.
 \item Representations of $SL_2(\RR)$. % and of the Heisenberg group ... didn't happen
 \end{enumerate}

% Today we'll give an overview of Lie algebras and Lie groups. We'll compare with
% algebraic groups (matrix groups), and finite groups.

 \subsektion{Lie groups in general}

 In general, a Lie group $G$ can be broken up into a number of pieces.

 The connected component of the identity, $G_\text{conn}\subseteq G$, is a
 normal subgroup, and $G/G_\text{conn}$ is a discrete group.
 \[
      1\longrightarrow G_\text{conn} \longrightarrow G \longrightarrow
      G_\text{discrete}\longrightarrow 1
 \]

 The maximal connected normal solvable subgroup of $G_\text{conn}$ is called
 $G_\text{sol}$. Recall that a group is \emph{solvable}\index{solvable!group} if there
 is a chain of subgroups $G_\text{sol}\supseteq \cdots\supseteq 1$, where consecutive
 quotients are abelian. The Lie algebra of a solvable group is solvable (by Exercise
 \ref{lec11hardEx}), so Lie's theorem (Theorem \ref{lec11Lie}) tells us that
 $G_\text{sol}$ is isomorphic to a subgroup of the group of upper triangular
 matrices.\anton{almost ... it tells us that $G_\text{sol}/$(some discrete subgroup)
 is a subgroup of upper triangular matrices}

 Every normal solvable subgroup of $G_\text{conn}/G_\text{sol}$ is discrete, and
 therefore in the center (which is itself discrete). We call the pre-image of the
 center $G_*$. Then $G/G_*$ is a product of simple groups (groups with no normal
 subgroups).\anton{is this obvious? it is clear that this is the adjoint form of a
 group with semisimple Lie algebra}
 \[
    G_\text{sol}\subseteq
     \left\{ \mat{\ast & & & \smash{\raisebox{-1ex}{\llap{\LARGE $\ast$}}} \\
                   & \ddots &  & \\
                   & & \ast & \\
                 \smash{\rlap{\LARGE 0}} & & & \ast} \right\}
    \qquad
    G_\text{nil}\subseteq
     \left\{ \mat{1 & & & \smash{\raisebox{-1ex}{\llap{\LARGE $\ast$}}} \\
                   & \ddots &  & \\
                   & & 1 & \\
                 \smash{\rlap{\LARGE 0}} & & & 1} \right\}
 \]
 Since $G_\text{sol}$ is solvable, $G_\text{nil}:=[G_\text{sol},G_\text{sol}]$ is
 nilpotent,\index{nilpotent!group} i.e.\ there is a chain of subgroups
 $G_\text{nil}\supseteq G_1\supseteq\cdots \supseteq G_k=1$ such that $G_i/G_{i+1}$ is
 in the center of $G_\text{nil}/G_{i+1}$. In fact, $G_\text{nil}$ must be isomorphic
 to a subgroup of the group of upper triangular matrices with ones on the diagonal.
 Such a group is called \emph{unipotent}.\index{unipotent group|idxbf}

 We have the picture
 \[ \vfuzz=5pt
 \begin{xy}
   (0,5) *+{G};
   (0,4) *+{G_\text{conn}} **@{-};
   (0,3) *+{G_{*}} **@{-};
   (0,2) *+{G_\text{sol}} **@{-};
   (0,1) *+{G_\text{nil}} **@{-};
   (0,0) *+{1} **@{-};
   (.7,4.5) *=<0em,1.75em>\frm{)} *+!L{\text{discrete; classification hopeless}};
   (.7,3.5) *=<0em,1.75em>\frm{)} *+!L{\prod \text{connected simples; classified}};
   (.7,2.5) *=<0em,1.75em>\frm{)} *+!L{\text{abelian discrete}};
   (.7,1.5) *=<0em,1.75em>\frm{)} *+!L{\text{abelian}};
        (3.7,2) *=<0em,2.5em>\frm{\}} *++!L{\text{classification trivial}};
   (.7,0.5) *=<0em,1.75em>\frm{)} *+!L{\text{nilpotent; classification a mess}};
   (-1,2) *=<0em,10em>\frm{\{} *++!R{\text{connected}}
 \end{xy}
 \]
 The classification of connected simple Lie groups\index{connected} is quite long.
 There are many infinite series and a lot of exceptional cases. Some infinite series
 are $PSU(n)$, $PSL_n(\RR)$, and $PSL_n(\CC)$.\footnote{The $P$ means ``mod out by the
 center''.}

 One way to get many connected simple Lie groups is not observe that there is a unique
 connected simple Lie group for each simple Lie algebra. We've already classified
 complex Lie algebras, and it turns out that there a finite number of real Lie
 algebras which complexify to any given complex Lie algebra. We will classify all such
 real forms\index{real form} in Lecture~29.

 For example, $\sl_2(\RR)\not\simeq \mathfrak{su}_2(\RR)$, but $\sl_2(\RR)\otimes \CC
 \simeq \mathfrak{su}_2(\RR)\otimes \CC \simeq \sl_2(\CC)$. By the way, $\sl_2(\CC)$
 is simple as a \emph{real} Lie algebra, but its complexification is $\sl_2(\CC)\oplus
 \sl_2(\CC)$, which is not simple. Thus, we cannot obtain all connected simple groups
 this way.
 \begin{example}
   Let $G$ be the group of all shape-preserving transformations of $\RR^4$ (i.e.\
   translations, reflections, rotations, and scaling). It is sometimes called
   $\RR^4\cdot\nobreak GO_4(\RR)$. The $\RR^4$ stands for translations, the $G$ means
   that you can multiply by scalars, and the $O$ means that you can reflect and
   rotate. The $\RR^4$ is a normal subgroup.  In this case, we have
   \[
     \renewcommand\arraycolsep{.3ex}
     \raisebox{-.4\baselineskip}{\shortstack{$G_\text{conn}/G_\text{sol}$\\ $=SO_4(\RR)$}}
     \raisebox{.1\baselineskip}{
       $\left\{ \rule{0pt}{2.6\baselineskip} \right.$}
     \begin{array}{rl}
      \RR^4\cdot GO_4(\RR) &= G\\ \\
      \RR^4\cdot GO_4^+(\RR) &= G_\text{conn}\\ \\
      \RR^4\cdot \RR^\times &= G_*\\ \\
      \RR^4\cdot \RR^+ &= G_\text{sol} \\ \\
      \RR^4 &= G_\text{nil}
    \end{array}
    \begin{array}{rl}
      G/G_\text{conn} &= \ZZ/2\ZZ\\ \\
      G_\text{conn}/G_* &= PSO_4(\RR) \\
       & \quad \big(\simeq SO_3(\RR)\times SO_3(\RR)\big)\\
      G_*/G_\text{sol} &= \ZZ/2\ZZ \\ \\
      G_\text{sol}/G_\text{nil} &=\RR^+
    \end{array}
  \]
  where $GO_4^+(\RR)$ is the connected component of the identity (those
  transformations that preserve orientation), $\RR^\times$ is scaling by something
  other than zero, and $\RR^+$ is scaling by something positive. Note that
  $SO_3(\RR) = PSO_3(\RR)$ is simple.

  $SO_4(\RR)$ is ``almost'' the product $SO_3(\RR)\times SO_3(\RR)$. To see this,
  consider the associative (but not commutative) algebra of quaternions, $\HH$. Since
  $q\bar q = a^2+b^2+c^2+d^2 >0$ whenever $q\neq 0$, any non-zero quaternion has an
  inverse (namely, $\bar q/q\bar q$). Thus, $\HH$ is a division algebra. Think of
  $\HH$ as $\RR^4$ and let $S^3$ be the unit sphere, consisting of the quaternions
  such that $\|q\|=q\bar q=1$. It is easy to check that $\|pq\|=\|p\|\cdot \|q\|$,
  from which we get that left (right) multiplication by an element of $S^3$ is a
  norm-preserving transformation of $\RR^4$. So we have a map $S^3\times S^3\to
  O_4(\RR)$. Since $S^3\times S^3$ is connected, the image must lie in $SO_4(\RR)$. It
  is not hard to check that $SO_4(\RR)$ is the image. The kernel is
  $\{(1,1),(-1,-1)\}$. So we have $S^3\times S^3/\{(1,1),(-1,-1)\}\simeq SO_4(\RR)$.

  Conjugating a purely imaginary quaternion by some $q\in S^3$ yields a purely
  imaginary quaternion of the same norm as the original, so we have a homomorphism
  $S^3\to O_3(\RR)$. Again, it is easy to check that the image is $SO_3(\RR)$ and that
  the kernel is $\pm 1$, so $S^3/\{\pm 1\}\simeq SO_3(\RR)$.

  So the universal cover of $SO_4(\RR)$ (a double cover) is the cartesian square of
  the universal cover of $SO_3(\RR)$ (also a double cover). Orthogonal groups in
  dimension 4 have a strong tendency to split up like this.\anton{Really? especially
  in dimension 4?} Orthogonal groups in general tend to have these double covers, as
  we shall see in Lectures 23~and~24. These double covers are important if you want
  to study fermions.\index{fermions}
 \end{example}

% \begin{example}
% Let $G$ be the group of all shape-preserving (translations, rotations, reflections,
% dilations) transformations of $\RR^4$, sometimes called $\RR^4\cdot\nobreak
% GO_4(\RR)$.\footnote{$\RR^4$ is translations, $G$ is multiplication by scalers, $O$
% is reflections and rotations} This is pretty much the smallest possible Lie group with
% all the properties of a general Lie group.
% \begin{itemize}
% \item[(1)] $G$ has a connected component $G_\text{conn}$, which is a normal subgroup, and
% $G/G_\text{conn}$ is a discrete group.
%
% Any Lie group can be built from a connected Lie group and a discrete group
% (0 dimensional group). Classifying discrete groups is completely hopeless.
%
% \item[(2)] We let $G_\text{sol}$ be the maximal normal connected solvable subgroup of
% $G_\text{conn}$. Recall that solvable means that there is a chain
% $G_\text{sol}\supseteq G_\text{sol}' \supseteq \cdots \supset 1$ where all
% consecutive quotients are abelian. So you build solvable groups out of abelian
% groups.
%
% $G_\text{conn}/G_\text{sol}$ is connected and has no solvable normal subgroups. It
% turns out that it is much easier to deal with the group if you kill off all the
% solvable subgroups like this. Groups like this are semisimple. They are almost a
% product of simple groups. A semisimple group $G$ modulo its discrete center is a
% product of simple groups.
%
% The nice thing is that connected simple Lie groups are all completely classified. The
% list is quite long, and it has a lot of infinite series and a lot of exceptional
% guys. Some examples are $PSU(n)$, $PSL_n(\RR)$, $PSL_n(\CC)$. You can go from simple
% Lie groups to simple Lie algebras over $\RR$, then complexify to get Lie algebras
% over $\CC$. This last step is finite-to-one, and they can be classified fairly
% easily. For example, the Lie algebras of $\sl_2(\RR)$ and $\mathfrak{su}_2(\RR)$ are
% different, but they have the same complexification, $\sl_2(\CC)$, which, by the way,
% is actually simple as a \emph{real} Lie algebra. Its complexification is not simple,
% it is $\sl_2(\CC)\oplus \sl_2(\CC)$.
%
% Suppose $G_\text{sol}$ is solvable. The only thing you can say is that its derived
% subgroup $G'_\text{sol}$ is nilpotent. That is, $G'\supseteq [G',G']\supseteq
% [G',[G',G']]\supseteq \cdots\supseteq 1$, or rather there is a chain such that
% $G_i/G_{i+1}\subseteq $ the center of $G_1/G_{i+1}$. Thus, nilpotent groups can be
% built up by taking central extensions\index{central extension}. Nilpotent groups are in practice not
% classifiable. The simply connected ones are all isomorphic to subgroups of
% upper-triangular matrices
% \[
%%    \left\{ \mat{1 & & & \smash{\raisebox{-1ex}{\llap{\Huge $\ast$}}} \\
%%                   & \ddots &  & \\
%%                   & & 1 & \\
%%                 \smash{\rlap{\Huge 0}} & & & 1} \right\}
%    \left\{ \mat{1 & & & \smash{\raisebox{-1ex}{\llap{\LARGE $\ast$}}} \\
%                   & \ddots &  & \\
%                   & & 1 & \\
%                 \smash{\rlap{\LARGE 0}} & & & 1} \right\}
%%    \left\{ \mat{1 & & & \ast \\
%%                   & \ddots &  & \\
%%                   & & 1 & \\
%%                 0 & & & 1} \right\}
% \]
% If you put arbitrary things on the diagonal, that is what the solvable groups look
%like.
%
% \end{itemize}
%
% In our case, $G$ is $\RR^4\cdot\nobreak GO_4(\RR)$. The connected component
% $G_\text{conn}$ is $\RR^4\cdot\nobreak GO^+_4(\RR)$, which is the stuff that
% preserves orientation. $G_\text{sol}$ is $\RR^4\cdot\nobreak \RR^+$ (rescaling).
% $G_\text{nil}$ is $\RR^4$. $G_\text{conn}/G_\text{sol}$ is $SO_4(\RR)$,
% $G_\text{sol}/G_\text{nil}$ is $\RR^+$. $G_\text{conn}/G_*$ is $PSO_4(\RR)\simeq
% SO_3(\RR)\times SO_3(\RR)$. $G_*/G_\text{sol}$ is $\ZZ/2\ZZ$.
%
% Note: What is $O_4(\RR)$? It almost is the product $O_3(\RR)\times O_3(\RR)$. To see
% this is to look at the quaternions $\HH$, which is an associative, but not
% commutative algebra. Notice that $q\bar q = a^2+b^2+c^2+d^2$, which is $>0$ if $q\neq
% 0$, so any non-zero quaternion has an inverse. Thus, $\HH$ is a division algebra. Put
% $\RR^4$ to be the quaternions, and $S^3$ is the unit sphere in this $\RR^4$
% consisting of the quaternions such that $\|q\|=q\bar q=1$. It is easy to check that
% $\|pq\|=\|p\|\, \|q\|$. So left or right multiplication by an element of $S^3$ give
% rotations of $\RR^4$. Thus, we have that $S^3\times S^3\to O_4(\RR)$, and the image
% is $SO_4(\RR)$. This is not quite injective because the kernel is
% $\{(1,1),(-1,-1)\}$. So you get $S^3\times S^3/(-1,-1)\simeq SO_4(\RR)$.
%
% Let $\RR^3$ be the purely imaginary quaternions. Then $q\in S^3$ acts by conjugation
% on $\RR^3$, so we get $S^3\to O_3(\RR)$, and again it is not quite injective or
% surjective. The image is $SO_3(\RR)$ and the kernel is $\{1,-1\}$, so
% $S^3/\{1,-1\}\simeq SO_3(\RR)$. So we almost have that $SO_4$ isomorphic to
% $SO_3\times SO_3$. If you take a double cover of $SO_4(\RR)$, it is the product of
% two copies of a double cover of $SO_3(\RR)$. Orthogonal groups in dimension 4 have a
% strong tendency to split up like this.
% \end{example}

 \subsektion{Lie groups and Lie algebras}

 Let $\g$ be a Lie algebra. We can set $\g_\text{sol} = \text{rad}\, \g$ to be the
 maximal solvable ideal (normal subalgebra), and $\g_\text{nil} =
 [\g_\text{sol},\g_\text{sol}]$. Then we get the chain
 \[\vfuzz=5pt
 \begin{xy}
   (0,3) *+{\g};
   (0,2) *+{\g_\text{sol}} **@{-};
   (0,1) *+{\g_\text{nil}} **@{-};
   (0,0) *+{0} **@{-};
   (.5,2.7);(.5,2.3) **\frm{)}; (.5,2.5) *+!L{\prod\text{simples; classification known}};
   (.5,1.7);(.5,1.3) **\frm{)}; (.5,1.5) *+!L{\text{abelian; easy to classify}};
   (.5,0.7);(.5,0.3) **\frm{)}; (.5,0.5) *+!L{\text{nilpotent; classification a mess}};
 \end{xy}
 \]
 We have an equivalence of categories between simply connected Lie groups and Lie
 algebras. The correspondence cannot detect
 \begin{itemize}
   \item Non-trivial components of $G$. For example, $SO_n$ and $O_n$ have the same
   Lie algebra.

   \item Discrete normal (therefore central, Lemma \ref{lec05L:discCentral}) subgroups
   of $G$. If $Z\subseteq G$ is any discrete normal subgroup, then $G$ and $G/Z$ have
   the same Lie algebra. For example, $SU(2)$ has the same Lie algebra as
   $PSU(2)\simeq SO_3(\RR)$.
 \end{itemize}
 If $\tilde G$ is a connected and simply connected Lie group with Lie algebra $\g$,
 then any other connected group $G$ with Lie algebra $\g$ must be isomorphic to
 $\tilde G/Z$, where $Z$ is some discrete subgroup of the center. Thus, if you know
 all the discrete subgroups of the center of $\tilde G$, you can read off all the
 connected Lie groups with the given Lie algebra.

 Let's find all the groups with the algebra $\so_4(\RR)$. First let's find a simply
 connected group with this Lie algebra. You might guess $SO_4(\RR)$, but that isn't
 simply connected. The simply connected one is $S^3\times S^3$ as we saw earlier (it
 is a product of two simply connected groups, so it is simply connected). The center
 of $S^3$ is generated by $-1$, so the center of $S^3\times S^3$ is $(\ZZ/2\ZZ)^2$,
 the Klein four group. There are three subgroups of order 2
 \[\xymatrix @!0 @R=3em @C=4em {
   & (\ZZ/2\ZZ)^2 \ar@{-}[dl] \ar@{-}[d] \ar@{-}[dr]\\
   (-1,1) \ar@{-}[dr]& (-1,-1) \ar@{-}[d]& (1,-1) \ar@{-}[dl]\\
   & 1
 }\qquad
 \xymatrix @!0 @R=3em @C=5.8em{
   & PSO_4(\RR) \ar@{-}[dl] \ar@{-}[d] \ar@{-}[dr]\\
   SO_3(\RR)\times S^3 \ar@{-}[dr]& SO_4(\RR) \ar@{-}[d]& S^3\times SO_3(\RR) \ar@{-}[dl]\\
   & S^3\times S^3
 }\]
 Therefore, there are 5 groups with Lie algebra $\so_4$.

 \subsektion{Lie groups and finite groups}
 \begin{enumerate}
 \item The classification of finite simple groups resembles the classification of
 connected simple Lie groups when $n\ge 2$. \anton{what is $n$?}

 For example, $PSL_n(\RR)$ is a simple Lie group, and $PSL_n(\FF_q)$ is a finite
 simple group except when $n=q=2$ or $n=2,q=3$. Simple finite groups form about 18
 series similar to Lie groups, and 26 or 27 exceptions, called sporadic groups, which
 don't seem to have any analogues for Lie groups.\anton{what about exceptional Lie
 groups?}

 \item Finite groups and Lie groups are both built up from simple and abelian groups.
 However, the way that finite groups are built is much more complicated than the way
 Lie groups are built. Finite groups can contain simple subgroups in very complicated
 ways; not just as direct factors.

 For example, there are \emph{wreath products}.\index{wreath product|idxbf} Let $G$
 and $H$ be finite simple groups with an action of $H$ on a set of $n$ points. Then
 $H$ acts on $G^n$ by permuting the factors. We can form the semi-direct product
 $G^n\ltimes H$, sometimes denoted $G\wr H$. There is no analogue for (finite
 dimensional) Lie groups. There \emph{is} an analogue for infinite dimensional Lie
 groups, which is why the theory becomes hard in infinite dimensions.

 \item The commutator subgroup of a solvable finite group need not be a nilpotent
 group. For example, the symmetric group $S_4$ has commutator subgroup $A_4$, which is
 not nilpotent.

% \item[(4)] (non-trivial) Nilpotent finite groups are never usually subgroups of upper
% triangular matrices (with ones on the diagonal).
 \end{enumerate}

 \subsektion{Lie groups and Algebraic groups (over \texorpdfstring{$\RR$}{the reals})}
 By algebraic group, we mean an algebraic variety which is also a group, such as
 $GL_n(\RR)$. Any algebraic group is a Lie group. Probably all the Lie groups you've
 come across have been algebraic groups. Since they are so similar, we'll list some
 differences.
 \begin{enumerate}
 \item Unipotent and semisimple abelian algebraic groups are totally different, but
 for Lie groups they are nearly the same. For example $\RR\simeq
 \left\{\matrix{1}{\ast}{0}{1}\right\}$ is unipotent and $\RR^\times \simeq
 \left\{\matrix{a}00{a^{-1}}\right\}$ is semisimple. As Lie groups, they are closely
 related (nearly the same), but the Lie group homomorphism $\exp: \RR\to \RR^\times$
 is not algebraic (polynomial), so they look quite different as algebraic groups.

 \item Abelian varieties are different from affine algebraic groups. For example,
 consider the (projective) elliptic curve $y^2=x^3+x$ with its usual group operation
 and the group of matrices of the form $\matrix{a}b{-b}a$ with $a^2+b^2=1$. Both are
 isomorphic to $S^1$ as Lie groups, but they are completely different as algebraic
 groups; one is projective and the other is affine.

 \item Some Lie groups do not correspond to ANY algebraic group. We give two examples
 here.

 The \emph{Heisenberg group}\index{Heisenberg group|idxit} is the subgroup of
 symmetries of $L^2(\RR)$ generated by translations ($f(t)\mapsto f(t+x)$),
 multiplication by $e^{2\pi ity}$ ($f(t)\mapsto e^{2\pi ity} f(t)$), and
 multiplication by $e^{2\pi iz}$ ($f(t)\mapsto e^{2\pi iz}f(t)$). The general element
 is of the form $f(t)\mapsto e^{2\pi i(yt+z)}f(t+x)$. This can also be modelled as
 \[
 \left.\left\{\mat{1 & x & z \\ 0 & 1 & y \\ 0 & 0 & 1}\right\}\right/
 \left\{\left.\mat{1 & 0 & n \\ 0 & 1 & 0 \\ 0 & 0 & 1} \right| n\in \ZZ\right\}
 \]
 It has the property that in any finite dimensional representation, the center
 (elements with $x=y=0$) acts trivially, so it cannot be isomorphic to any algebraic
 group.

 The \emph{metaplectic group}.\index{metaplectic group|idxbf} Let's try to find all
 connected groups with Lie algebra $\sl_2(\RR) = \{\matrix{a}{b}{c}{d}| a+d=0\}$.
 There are two obvious ones: $SL_2(\RR)$ and $PSL_2(\RR)$. There aren't any other ones
 that can be represented as groups of finite dimensional matrices. However, if you
 look at $SL_2(\RR)$, you'll find that it is not simply connected. To see this, we
 will use Iwasawa decomposition (without proof).
 \begin{theorem}[Iwasawa decomposition]\label{lec21T:Iwasawa}\index{Iwasawa
 decomposition!idxbf}
   If $G$ is a (connected) semisimple Lie group, then there are closed subgroups $K$,
   $A$, and $N$, with $K$ compact, $A$ abelian, and $N$ unipotent, such that the
   multiplication map $K\times A\times N\to G$ is a surjective diffeomorphism.
   Moreover, $A$ and $N$ are simply connected.
 \end{theorem}
 In the case of $SL_n$, this is the statement that any basis can be obtained uniquely
 by taking an orthonormal basis ($K=SO_n$), scaling by positive reals ($A$ is the
 group of diagonal matrices with positive real entries), and shearing ($N$ is the
 group $\Bigl( \begin{array}{cc}
        1 \smash{\raisebox{-1.5ex}{\rlap{\scriptsize $\ddots$}}}& \ast\\ 0 & 1
        \end{array}\Bigr)$). This is exactly the result of the Gram-Schmidt
 process.\index{Gram-Schmidt}

 The upshot is that $G\simeq K\times A \times N$ (topologically), and $A$ and $N$ do
 not contribute to the fundamental group, so the fundamental group of $G$ is the same
 as that of $K$. In our case, $K=SO_2(\RR)$ is isomorphic to a circle, so the
 fundamental group of $SL_2(\RR)$ is $\ZZ$.

 So the universal cover $\widetilde{SL_2(\RR)}$ has center $\ZZ$. Any
 finite dimensional representation of $\widetilde{SL_2(\RR)}$ factors through
 $SL_2(\RR)$, so none of the covers of $SL_2(\RR)$ can be written as a group of
 finite dimensional matrices. Representing such groups is a pain.

 The most important case is the metaplectic group $Mp_2(\RR)$, which is the connected
 double cover of $SL_2(\RR)$. It turns up in the theory of modular forms of
 half-integral weight and has a representation called the metaplectic representation.
 \end{enumerate}

 \subsektion{Important Lie groups}

 \underline{Dimension 1}: There are just $\RR$ and $S^1 = \RR/\ZZ$.

 \underline{Dimension 2}: The abelian groups are quotients of $\RR^2$ by some discrete
 subgroup; there are three cases: $\RR^2$, $\RR^2/\ZZ = \RR\times S^1$, and
 $\RR^2/\ZZ^2 = S^1\times S^1$.

 There is also a non-abelian group, the group of all matrices of the form $\matrix
 ab0{a^{-1}}$, where $a>0$. The Lie algebra is the subalgebra of $2\times 2$ matrices
 of the form $\matrix hx0{-h}$, which is generated by two elements $H$ and $X$, with
 $[H,X]=2X$.

 \underline{Dimension 3}: There are some boring abelian and solvable groups, such as
 $\RR^2\ltimes \RR^1$, or the direct sum of $\RR^1$ with one of the two dimensional
 groups. As the dimension increases, the number of boring solvable groups gets huge,
 and nobody can do anything about them, so we ignore them from here on.

 You get the group $SL_2(\RR)$, which is the most important Lie group of all. We saw
 earlier that $SL_2(\RR)$ has fundamental group $\ZZ$. The double cover $Mp_2(\RR)$ is
 important. The quotient $PSL_2(\RR)$ is simple, and acts on the open upper half plane
 by linear fractional transformations

 Closely related to $SL_2(\RR)$ is the compact group $SU_2$. We know that $SU_2\simeq
 S^3$, and it covers $SO_3(\RR)$, with kernel $\pm 1$. After we learn about Spin
 groups, we will see that $SU_2 \cong \spin_3(\RR)$. The Lie algebra $\mathfrak{su}_2$
 is generated by three elements $X$, $Y$, and $Z$ with relations $[X,Y]=2Z$,
 $[Y,Z]=2X$, and $[Z,X]=2Y$.\footnote{An explicit representation is given by
 $X=\matrix 01{-1}0$, $Y=\matrix 0ii0$, and $Z=\matrix i00{-i}$. The cross product on
 $\RR^3$ gives it the structure of this Lie algebra.}

 The Lie algebras $\sl_2(\RR)$ and
 $\mathfrak{su}_2$ are non-isomorphic, but when you complexify, they both become
 isomorphic to $\sl_2(\CC)$.

 There is another interesting 3 dimensional algebra. The Heisenberg algebra is the Lie
 algebra of the Heisenberg group. It is generated by $X,Y,Z$, with $[X,Y]=Z$ and $Z$
 central. You can think of this as strictly upper triangular matrices.

 \underline{Dimension 6}: (nothing interesting happens in dimensions 4,5) We get the
 group $SL_2(\CC)$. Later, we will see that it is also called $\spin_{1,3}(\RR)$.

 \underline{Dimension 8}: We have $SU_3(\RR)$ and $SL_3(\RR)$. This is the first time
 we get a non-trivial root system.

 \underline{Dimension 14}: $G_2$, which we will discuss a little.

 \underline{Dimension 248}: $E_8$, which we will discuss in detail.

 \smallskip
 This class is mostly about finite dimensional algebras, but let's mention some
 infinite dimensional Lie groups or Lie algebras.
 \begin{enumerate}
   \item Automorphisms of a Hilbert space form a Lie group.

   \item Diffeomorphisms of a manifold form a Lie group. There is some physics stuff
   related to this.

   \item \emph{Gauge groups}\index{Guage groups!idxbf} are (continuous, smooth,
   analytic, or whatever) maps from a manifold $M$ to a group $G$.

   \item The \emph{Virasoro algebra}\index{Virasoro algebra} is generated by $L_n$ for
   $n\in \ZZ$ and $c$, with relations $[L_n,L_m]=(n-m) L_{n+m} +
   \delta_{n+m,0}\frac{n^3-n}{12}c$, where $c$ is central (called the \emph{central
   charge}). If you set $c=0$, you get (complexified) vector fields on $S^1$, where we
   think of $L_n$ as $ie^{in\theta}\pder{}{\theta}$. Thus, the Virasoro algebra is a
   central extension
   \[
     0\to c\CC \to \text{Virasoro}\to \text{Vect}(S^1)\to 0.
   \]

   \item Affine Kac-Moody algebras, which are more or less central extensions of
   certain gauge groups over the circle.
 \end{enumerate}
}{   % Hanh Duc Do, ddhanh@math
  \stepcounter{lecture}
 \setcounter{lecture}{22}
 \sektion{Lecture 22 - Clifford algebras}\index{Clifford algebra|(idxbf}

 With Lie algebras of small dimensions, there are accidental isomorphisms. Almost all
 of these can be explained with Clifford algebras and Spin groups.

 Motivational examples that we'd like to explain:
 \begin{enumerate}
   \item $SO_2(\RR) = S^1$: $S^1$ can double cover $S^1$ itself.
   \item $SO_3(\RR)$: has a simply connected double cover $S^3$.
   \item $SO_4(\RR)$: has a simply connected double cover $S^3\times S^3$.

   \item $SO_5(\CC)$: Look at $Sp_4(\CC)$, which acts on $\CC^4$ and on
   $\Lambda^2(\CC^4)$, which is 6 dimensional, and decomposes as $5\oplus 1$.
   $\Lambda^2(\CC^4)$ has a symmetric bilinear form given by $\Lambda^2(\CC^4)\otimes
   \Lambda^2(\CC^4)\to \Lambda^4(\CC^4)\simeq \CC$, and $Sp_4(\CC)$ preserves this form.
    You get that $Sp_4(\CC)$ acts on
   $\CC^5$, preserving a symmetric bilinear form, so it maps to $SO_5(\CC)$. You can
   check that the kernel is $\pm 1$. So $Sp_4(\CC)$ is a double cover of $SO_5(\CC)$.

   \item $SO_5(\CC)$: $SL_4(\CC)$ acts on $\CC^4$, and we still have our 6 dimensional
   $\Lambda^2(\CC^4)$, with a symmetric bilinear form. So you get a homomorphism
   $SL_4(\CC) \to SO_6(\CC)$, which you can check is surjective, with kernel $\pm 1$.
 \end{enumerate}
 So we have double covers $S^1$, $S^3$, $S^3\times S^3$, $Sp_4(\CC)$, $SL_4(\CC)$ of
 the orthogonal groups in dimensions 2,3,4,5, and 6, respectively. All of these look
 completely unrelated. By the end of the next lecture, we will have an understanding
 of these groups, which will be called $\spin_2(\RR)$, $\spin_3(\RR)$, $\spin_4(\RR)$,
 $\spin_5(\CC)$, and $\spin_6(\CC)$, respectively.

 \begin{example}
   We have not yet defined Clifford algebras, but here are some examples of Clifford
   algebras over $\RR$.
  \begin{itemize}
   \item $\CC$ is generated by $\RR$, together with $i$, with $i^2=-1$
   \item $\HH$ is generated by $\RR$, together with $i,j$, each squaring to $-1$, with
   $ij+ji=0$.

   \item Dirac wanted a square root for the operator $\nabla =
   \frac{\partial^2}{\partial x^2} + \frac{\partial^2}{\partial y^2}+
   \frac{\partial^2}{\partial z^2} - \frac{\partial^2}{\partial t^2}$ (the wave
   operator in 4 dimensions). He supposed that the square root is of the form $A =
   \gamma_1\pder{}{x}+\gamma_2\pder{}{y}+\gamma_3\pder{}{z}+\gamma_4\pder{}{t}$ and
   compared coefficients in the equation $A^2=\nabla$. Doing this yields
   $\gamma_1^2=\gamma_2^2=\gamma_3^2=1$, $\gamma_4^2=-1$, and
   $\gamma_i\gamma_j+\gamma_j\gamma_i=0$ for $i\neq j$.

   Dirac solved this by taking the $\gamma_i$ to be $4\times 4$ complex matrices. $A$
   operates on vector-valued functions on space-time.
 \end{itemize}
\end{example}
\begin{definition}
   A general \emph{Clifford algebra} over $\RR$ should be generated by elements
   $\gamma_1,\dots, \gamma_n$ such that $\gamma_i^2$ is some given real, and
   $\gamma_i\gamma_j+\gamma_j\gamma_i=0$ for $i\neq j$.
 \end{definition}
 \begin{definition}[better definition]
   Suppose $V$ is a vector space over a field $K$, with some quadratic
   form\footnote{$N$ is a \emph{quadratic form}\index{quadratic form} if it is a
   homogeneous polynomial of degree 2 in the coefficients with respect to some basis.}
   $N:V\to K$. Then the \emph{Clifford algebra} $C_V(K)$ is generated by the vector
   space $V$, with relations $v^2 = N(v)$.
 \end{definition}
 We know that $N(\lambda v) = \lambda^2N(v)$ and that the expression $(a,b) :=
 N(a+b)-N(a)-N(b)$ is bilinear. If the characteristic of $K$ is not 2, we have
 $N(a)=\frac{(a,a)}{2}$. Thus, you can work with symmetric bilinear forms instead of
 quadratic forms so long as the characteristic of $K$ is not 2. We'll use quadratic
 forms so that everything works in characteristic 2.
 \begin{warning}
   A few authors (mainly in index theory) use the relations $v^2=-N(v)$.
%
%   This is a terrible convention introduced by Atiyah and Bott.
%
   Some people add a factor of 2, which usually doesn't matter, but is wrong in
   characteristic 2.
 \end{warning}
 \begin{example}
   Take $V=\RR^2$ with basis $i,j$, and with $N(xi+yj)=-x^2-y^2$. Then the relations
   are $(xi+yj)^2=-x^2-y^2$ are exactly the relations for the quaternions:
   $i^2=j^2=-1$ and $(i+j)^2 = i^2+ij+ji+j^2=-2$, so $ij+ji=0$.
 \end{example}

 \begin{remark}\label{lec22Rmk:QuadForm}
   If the characteristic of $K$ is not 2,  a ``completing the square'' argument shows
   that any quadratic form is isomorphic to $c_1x_1^2+\cdots +
   c_nx_n^2$, and if one can be obtained from another other by permuting the $c_i$ and
   multiplying each $c_i$ by a non-zero square, the two forms are isomorphic.

   It follows that every quadratic form on a vector space over $\CC$
   is isomorphic to $x_1^2+\cdots +x_n^2$, and that every quadratic form on a
   vector space over $\RR$ is isomorphic to $x_1^2+\cdots + x_m^2 - x_{m+1}^2 - \cdots
   - x_{m+n}^2$ ($m$ pluses and $n$ minuses) for some $m$ and $n$. One can check that
   these forms over $\RR$ are non-isomorphic.

   We will always assume that $N$ is non-degenerate (i.e.\ that the associated bilinear
   form is non-degenerate), but one could study Clifford algebras arising from degenerate
   forms.
 \end{remark}
 \begin{warning}
   The criterion in the remark is not sufficient for classifying quadratic forms. For
   example, over the field $\FF_3$, the forms $x^2+y^2$ and $-x^2-y^2$ are isomorphic
   via the isomorphism $\matrix 111{-1}:\FF_3^2\to \FF_3^2$, but $-1$ is not a square
   in $\FF_3$.  Also, completing the square doesn't work in characteristic 2.
 \end{warning}
 \begin{remark}
   The tensor algebra $TV$ has a natural $\ZZ$-grading, and to form the Clifford
   algebra $C_V(K)$, we quotient by the ideal generated by the even elements
   $v^2-N(v)$. Thus, the algebra $C_V(K)=C_V^0(K)\oplus C_V^1(K)$ is
   $\ZZ/2\ZZ$-graded. A $\ZZ/2\ZZ$-graded algebra is called a
   \emph{superalgebra}\index{superalgebra}.
%   In light of Remark
%   \ref{lec22Rmk:QuadForm}, the quadratic form can be ``diagonalized'' when the
%   characteristic of $K$ is not 2. That is, there is a basis $v_1,\dots, v_n$ so that
%   $v_iv_j=-v_jv_i$ for $i\neq j$, so for homogeneous elements $x,y\in C_V(K)$,
%   $xy=(-1)^{|x|\, |y|}yx$, where $|x|$ and $|y|$ are the degrees of $x$ and $y$. A
%   superalgebra satisfying this relation is called \emph{supercommutative}.
%
%   odd elements do not supercommute with themselves unless $N=0$
 \end{remark}

 \underline{Problem}: Find the structure of $C_{m,n}(\RR)$, the Clifford algebra over
 $\RR^{n+m}$ with the form $x_1^2+\cdots + x_m^2 - x_{m+1}^2 - \cdots - x_{m+n}^2$.

 \begin{example}
   \begin{itemize}
   \item[]
   \item $C_{0,0}(\RR)$ is $\RR$.

   \item $C_{1,0}(\RR)$ is $\RR[\e]/(\e^2-1) = \RR(1+\e)\oplus \RR(1-\e) = \RR\oplus
   \RR$. Note that the given basis, this is a direct sum of \emph{algebras} over $\RR$.

   \item $C_{0,1}(\RR)$ is $\RR[i]/(i^2+1)=\CC$, with $i$ odd.\\

   \item $C_{2,0}(\RR)$ is
   $\RR[\alpha,\beta]/(\alpha^2-1,\beta^2-1,\alpha\beta+\beta\alpha)$. We get a
   homomorphism $C_{2,0}(\RR)\to \MM_2(\RR)$, given by $\alpha \mapsto \matrix 100{-1}$
   and $\beta \mapsto \matrix 0110$. The homomorphism is onto because the two given
   matrices generate $\MM_2(\RR)$ as an algebra. The dimension of $\MM_2(\RR)$ is 4, and
   the dimension of $C_{2,0}(\RR)$ is at most 4 because it is spanned by $1$, $\alpha$,
   $\beta$, and $\alpha\beta$. So we have that $C_{2,0}(\RR)\simeq \MM_2(\RR)$.

   \item $C_{1,1}(\RR)$ is
   $\RR[\alpha,\beta]/(\alpha^2-1,\beta^2+1,\alpha\beta+\beta\alpha)$. Again, we get an
   isomorphism with $\MM_2(\RR)$, given by $\alpha\mapsto \matrix 100{-1}$ and
   $\beta\mapsto\matrix 01{-1}0$
   \end{itemize}
   Thus, we've computed the Clifford algebras
   \[\begin{array}{cccccc}
    m\backslash n  & 0 & 1 & 2 \\
   0 & \RR & \CC & \HH \\
   1 & \RR\oplus \RR & \MM_2(\RR) \\
   2 & \MM_2(\RR)
   \end{array}\]
 \end{example}
 \begin{remark}
   If $\{v_1,\dots, v_n\}$ is a basis for $V$, then $\{v_{i_1}\cdots v_{i_k}|i_1<\cdots
   <i_k,\ k\le n\}$ spans $C_V(K)$, so the dimension of $C_V(K)$ is less than or equal to
   $2^{\dim V}$. The tough part of Clifford algebras is showing that it cannot be smaller.
 \end{remark}

 Now let's try to analyze larger Clifford algebras more systematically. What is
 $C_{U\oplus V}$ in terms of $C_U$ and $C_V$? One might guess $C_{U\oplus V} \cong
 C_U\otimes C_V$. For the usual definition of tensor product, this is false (e.g.\
 $C_{1,1}(\RR) \neq C_{1,0}(\RR)\otimes C_{0,1}(\RR)$). However, for the
 \emph{superalgebra} definition of tensor product, this is correct. The superalgebra
 tensor product is the regular tensor product of vector spaces, with the product given by
 $(a\otimes b)(c\otimes d) = (-1)^{\deg b\cdot \deg c} ac\otimes bd$ for homogeneous
 elements $a$, $b$, $c$, and $d$.

 \smallskip
 Let's specialize to the case $K=\RR$ and try to compute $C_{U\oplus V}(K)$. Assume for
 the moment that $\dim U=m$ is even. Take $\alpha_1$, $\dots$, $\alpha_m$ to be
 an orthogonal basis for $U$ and let $\beta_1,\dots, \beta_n$ to be an orthogonal basis
 for $V$. Then set $\gamma_i = \alpha_1\alpha_2\cdots \alpha_m \beta_i$. What are the
 relations between the $\alpha_i$ and the $\gamma_j$? We have
 \[
   \alpha_i\gamma_j = \alpha_i \alpha_1\alpha_2\cdots \alpha_m \beta_j
   = \alpha_1\alpha_2\cdots \alpha_m \beta_i \alpha_i = \gamma_j\alpha_i
 \]
 since $\dim U$ is even, and $\alpha_i$ anti-commutes with everything except itself.
 \begin{align*}
    \gamma_i\gamma_j &= \gamma_i\alpha_1\cdots \alpha_m \beta_j
    = \alpha_1\cdots \alpha_m \gamma_i \beta_j\\
    &= \alpha_1\cdots \alpha_m \alpha_1\cdots \alpha_m \underbrace{\beta_i \beta_j}_{-\beta_j\beta_i}
    = -\gamma_j\gamma_i\\
  \gamma_i^2 &= \alpha_1\cdots \alpha_m\alpha_1\cdots \alpha_m \beta_i \beta_i
    = (-1)^{\frac{m(m-1)}{2}} \alpha_1^2\cdots \alpha_m^2 \beta_i^2 \\
    &= (-1)^{m/2} \alpha_1^2\cdots \alpha_m^2 \beta_i^2 & \text{($m$ even)}
 \end{align*}
 So the $\gamma_i$'s commute with the $\alpha_i$ and satisfy the relations of some
 Clifford algebra. Thus, we've shown that $C_{U\oplus V} (K) \cong C_U(K)\otimes C_W(K)$,
 where $W$ is $V$ with the quadratic form multiplied by $(-1)^{\half\dim U}\alpha_1^2
 \cdots \alpha_m^2 = (-1)^{\half\dim U}\cdot\,$discriminant$(U)$, and this is the usual
 tensor product of algebras over $\RR$.

 Taking $\dim U=2$, we find that
 \begin{align*}
   C_{m+2,n}(\RR) &\cong \MM_2(\RR)\otimes C_{n,m}(\RR)\\
   C_{m+1,n+1}(\RR) &\cong \MM_2(\RR) \otimes C_{m,n}(\RR)\\
   C_{m,n+2}(\RR) &\cong \HH\otimes C_{n,m}(\RR)
 \end{align*}
 where the indices switch whenever the discriminant is positive. Using these formulas, we
 can reduce any Clifford algebra to tensor products of things like $\RR$, $\CC$, $\HH$,
 and $\MM_2(\RR)$.

 Recall the rules for taking tensor products of matrix algebras (all tensor products are
 over $\RR$).
 \begin{itemize}
   \item $\RR\otimes X \cong X$.

   \item $\CC\otimes \HH \cong \MM_2(\CC)$.

   This follows from the isomorphism $\CC\otimes C_{m,n}(\RR)\cong C_{m+n}(\CC)$.

   \item $\CC\otimes \CC\cong \CC\oplus \CC$.

   \item $\HH\otimes \HH \cong \MM_4(\RR)$.

   You can see by thinking of the action on
   $\HH\cong \RR^4$ given by $(x\otimes y)\cdot z = xzy^{-1}$.

   \item $\MM_m\bigl(\MM_n(X)\bigr) \cong \MM_{mn}(X)$.
   \item $\MM_m(X)\otimes \MM_n(Y) \cong \MM_{mn}(X\otimes Y)$.
 \end{itemize}

 Filling in the middle of the table is easy because you can move diagonally by tensoring
 with $\MM_2(\RR)$. It is easy to see that $C_{8+m,n}(\RR)\cong C_{m,n+8}(\RR) \cong
 C_{m,n}\otimes \MM_{16}(\RR)$, which gives the table a kind of mod 8 periodicity. There
 is a more precise way to state this: $C_{m,n}(\RR)$ and $C_{m',n'}(\RR)$ are \emph{super
 Morita equivalent}\index{super Morita equivalence} if and only if $m-n\equiv m'-n'\mod
 8$.

 \newpage
 \[
 \mbox{\footnotesize
 \rotatebox{-90}{
 \iflilbook
 \xymatrix @!0 @R=3em @C=5em{
  \ar@{}[d]|(.4){\mbox{\large $m$}} \ar@{}[r]|(.4){\mbox{\large $n$}}
        & 0 & 1 & 2 & 3 & 4 & 5 & 6 & 7 & 8 \\
 0 & \RR \ar@(l,l)[dddddddd]_(.82){\otimes\, \MM_{16}} \ar@(ur,ul)[rrrrrrrr]^(.85){\otimes\, \MM_{16}} &
     \CC\ar@(d,r)[dddl]^(.7){\otimes\, \MM_2} |(.47)\hole &
     \HH \ar@(d,r)[lldddd] |(.213)\hole & \HH\oplus \HH &
     \MM_2(\HH) & \MM_4(\CC) & **[l] \MM_8(\RR) & \MM_8(\RR)\oplus \MM_8(\RR) & **[r] \MM_{16}(\RR) \\
 1 & \RR\oplus \RR \ar@(r,d)[urrr]_(.7){\quad\otimes\, \HH} |(.287)\hole  \\
 2 & \MM_2(\RR) \ar@(r,d)[uurrrr] |(.378)\hole \\
 3 & \MM_2(\CC) \\
 4 & \MM_2(\HH) \\
 5 & \qquad\quad\MM_2(\HH)\oplus \MM_2(\HH) \ar[dr]^(.65){\otimes\, \MM_2} \\
 6 & \MM_4(\HH) & \qquad \MM_4(\HH)\oplus \MM_4(\HH) \ar[dr]^(.65){\otimes\, \MM_2} \\
 7 & \MM_8(\CC) & & \text{etc.}\\
 8 & \MM_{16}(\RR) \\
 9 & \MM_{16}(\RR)\oplus \MM_{16}(\RR) \\
 }
 \else
 \xymatrix @!0 @R=3em @C=5em{
  \ar@{}[d]|(.4){\mbox{\large $m$}} \ar@{}[r]|(.4){\mbox{\large $n$}}
        & 0 & 1 & 2 & 3 & 4 & 5 & 6 & 7 & 8 \\
 0 & \RR \ar@(l,l)[dddddddd]_(.82){\otimes\, \MM_{16}} \ar@(ur,ul)[rrrrrrrr]^(.85){\otimes\, \MM_{16}} &
     \CC\ar@(d,r)[dddl]^(.7){\otimes\, \MM_2} |(.485)\hole &
     \HH \ar@(d,r)[lldddd] |(.213)\hole & \HH\oplus \HH &
     \MM_2(\HH) & \MM_4(\CC) & **[l] \MM_8(\RR) & \MM_8(\RR)\oplus \MM_8(\RR) & **[r] \MM_{16}(\RR) \\
 1 & \RR\oplus \RR \ar@(r,d)[urrr]_(.7){\quad\otimes\, \HH} |(.293)\hole  \\
 2 & \MM_2(\RR) \ar@(r,d)[uurrrr] |(.379)\hole \\
 3 & \MM_2(\CC) \\
 4 & \MM_2(\HH) \\
 5 & \qquad\quad\MM_2(\HH)\oplus \MM_2(\HH) \ar[dr]^(.65){\otimes\, \MM_2} \\
 6 & \MM_4(\HH) & \qquad \MM_4(\HH)\oplus \MM_4(\HH) \ar[dr]^(.65){\otimes\, \MM_2} \\
 7 & \MM_8(\CC) & & \text{etc.}\\
 8 & \MM_{16}(\RR) \\
 9 & \MM_{16}(\RR)\oplus \MM_{16}(\RR) \\
 }
 \fi
 }}
 \]
}{   % An Huang, huangan@math
  \stepcounter{lecture}
 \setcounter{lecture}{23}
 \sektion{Lecture 23}

 Last time we defined the Clifford algebra $C_V(K)$, where $V$ is a vector space over $K$
 with a quadratic form $N$. $C_V(K)$ is generated by $V$ with $x^2=N(x)$.
 $C_{m,n}(\RR)$ uses the form $x_1^2+\cdots + x_m^2-x_{m+1}^2-\cdots - x_{m+n}^2$. We
 found that the structure depends heavily on $m-n \mod 8$.
 \begin{remark}
   This mod 8 periodicity turns up in several other places:
 \begin{enumerate}
 \item Real Clifford algebras $C_{m,n}(\RR)$ and $C_{m',n'}(\RR)$ are super Morita
 equivalent if and only if $m-n\equiv m'-n'\mod 8$.

 \item \emph{Bott periodicity},\index{Bott periodicity} which says that stable
 homotopy groups of orthogonal groups are periodic mod 8.

 \item Real $K$-theory is periodic with a period of 8.

 \item Even unimodular lattices (such as the $E_8$ lattice) exist in $\RR^{m,n}$
 if and only if $m-n\equiv 0\mod 8$.

 \item The Super Brauer group\index{super Brauer group} of $\RR$ is $\ZZ/8\ZZ$. The
 Super Brauer group consists of super division algebras over $\RR$ (algebras in which
 every non-zero homogeneous element is invertible) with the operation of tensor product
 modulo super Morita equivalence.\footnote{See
 \url{http://math.ucr.edu/home/baez/trimble/superdivision.html}}
 \[
 \begin{xy}<3.75em,0em>:
   (-1,0) *++!R{\RR} *{\bullet},
   (-\halfroottwo,\halfroottwo) *++!RD{\RR[\e_+]} *{\bullet},
   (0,1) *++!D{\CC[\e_+]} *{\bullet},
   (\halfroottwo,\halfroottwo) *+!LD{\HH[\e_-]} *{\bullet},
   (1,0) *++!L{\HH} *{\bullet},
   (\halfroottwo,-\halfroottwo) *+!LU{\HH[\e_+]} *{\bullet},
   (0,-1) *++!U{\CC[\e_-]} *{\bullet},
   (-\halfroottwo,-\halfroottwo) *+!RU{\RR[\e_-]} *{\bullet},
   (0,0), *\xycircle(1,1){}
 \end{xy}\]
 where $\e_{\pm}$ are odd with $\e_\pm^2 = \pm 1$, and $i\in \CC$ is odd,\footnote{One
 could make $i$ even since $\RR[i,\e_\pm]=\RR[\mp \e_\pm i, \e_\pm]$, and $\RR[\mp\e_\pm
 i]\cong \CC$ is entirely even.} but $i,j,k\in \HH$ are even.
 \end{enumerate}
 \end{remark}
 Recall that $C_V(\RR)=C^0_V(\RR)\oplus C^1_V(\RR)$, where $C^1_V(\RR)$ is the odd part
 and $C^0_V(\RR)$ is the even part. It turns out that we will need to know the structure
 of $C^0_{m,n}(\RR)$. Fortunately, this is easy to compute in terms of smaller Clifford
 algebras. Let $\dim U=1$, with $\gamma$ a basis for $U$ and let $\gamma_1,\dots,
 \gamma_n$ an orthogonal basis for $V$. Then $C^0_{U\oplus V}(K)$ is generated by
 $\gamma\gamma_1,\dots, \gamma\gamma_n$. We compute the relations
 \[
    \gamma\gamma_i\cdot \gamma\gamma_j = -\gamma\gamma_j\cdot \gamma\gamma_i
 \]
 for $i\neq j$, and
 \[
    (\gamma\gamma_i)^2 = (-\gamma^2)\gamma_i^2
 \]
 So $C_{U\oplus V}^0(K)$ is itself the Clifford algebra $C_{W}(K)$, where $W$ is $V$ with
 the quadratic form multiplied by $-\gamma^2 = -\text{disc}(U)$. Over $\RR$, this tells
 us that
 \begin{align*}
   C^0_{m+1,n}(\RR) &\cong C_{n,m}(\RR) & \text{(mind the indices)} \\
   C^0_{m,n+1}(\RR) &\cong C_{m,n}(\RR).
 \end{align*}
% So we've worked out all the real algebras and all the even and odd parts of them.

 \begin{remark}
   For complex Clifford algebras, the situation is similar, but easier. One finds that
   $C_{2m}(\CC)\cong \MM_{2^m}(\CC)$ and $C_{2m+1}(\CC) \cong \MM_{2^m}(\CC)\oplus
   \MM_{2^m}(\CC)$, with $C_n^0(\CC)\cong C_{n-1}(\CC)$. You could figure these out by
   tensoring the real algebras with $\CC$ if you wanted. We see a mod 2 periodicity now.
   Bott periodicity for the unitary group is mod 2.
 \end{remark}

 \subsektion{Clifford groups, Spin groups, and Pin groups}\index{Clifford groups|idxbf}
 In this section, we define Clifford groups, denoted $\Gamma_V(K)$, and find an exact
 sequence
 \[
    1\to K^\times \xrightarrow{\mathrm{central}} \Gamma_V(K) \to O_V(K)\to 1.
 \]
 \begin{definition}
   $\Gamma_V(K) = \{x\in C_V(K) \text{ homogeneous\footnotemark} |
   xV\alpha(x)^{-1}\subseteq V\}$ \footnotetext{We assume that $\Gamma_V(K)$ consists of
   homogeneous elements, but this can actually be proven.\anton{ref for this stuff?}}
   (recall that $V\subseteq C_V(K)$), where $\alpha$ is the automorphism of $C_V(K)$
   induced by $-1$ on $V$ (i.e.\ the automorphism which acts by $-1$ on odd elements and
   1 on even elements).
 \end{definition}
   Note that $\Gamma_V(K)$ acts on $V$ by $x\cdot v = xv\alpha(x)^{-1}$.

   Many books leave out the $\alpha$, which is a mistake, though not a serious one.
   They use $xVx^{-1}$ instead of $xV\alpha(x)^{-1}$.
   Our definition is better for the following reasons:
   \begin{enumerate}
   \item It is the correct superalgebra sign. The superalgebra convention
   says that whenever you exchange two elements of odd degree, you pick up a minus sign, and
   $V$ is odd.

   \item Putting $\alpha$ in makes the theory much cleaner in odd dimensions. For
   example, we will see that the described action gives a map $\Gamma_V(K)\to O_V(K)$
   which is onto if we use $\alpha$, but not if we do not. (You get $SO_V(K)$ without the
   $\alpha$, which isn't too bad, but is still annoying.)
   \end{enumerate}
 \begin{lemma}\label{lec23L} \hspace*{-1ex}\footnote{I promised no Lemmas or Theorems, but I was
   lying to you.} The elements of $\Gamma_V(K)$ which act trivially on
   $V$ are the elements of $K^\times \subseteq \Gamma_V(K)\subseteq C_V(K)$.
 \end{lemma}
 \begin{proof}
   Suppose $a_0+a_1\in \Gamma_V(K)$ acts trivially on $V$, with $a_0$ even and $a_1$ odd.
   Then $(a_0+a_1)v=v\alpha(a_0+a_1)=v(a_0-a_1)$. Matching up even and odd parts, we get
   $a_0v=va_0$ and $a_1 v=-va_1$. Choose an orthogonal basis $\gamma_1,\dots, \gamma_n$
   for $V$.\footnote{All these results are true in characteristic 2, but you have to work
   harder ... you can't go around choosing orthogonal bases because they may not exist.}
   We may write
   \[
      a_0 = x + \gamma_1 y
   \]
    where $x \in C_V^0(K)$ and $y \in C_V^1(K)$ and neither $x$ nor $y$ contain a factor
    of $\gamma_1$, so $\gamma_1 x=x\gamma_1$ and $\gamma_1y=y\gamma_1$. Applying the
    relation $a_0v = va_0$ with $v=\gamma_1$, we see that $y=0$, so $a_0$
    contains no monomials with a factor $\gamma_1$.

    Repeat this procedure with $v$ equal to the other basis elements to show that $a_0\in
    K^\times$ (since it cannot have any $\gamma$'s in it). Similarly, write
    $a_1=y+\gamma_1 x$, with $x$ and $y$ not containing a factor of $\gamma_1$. Then the
    relation $a_1 \gamma_1=-\gamma_1a_1$ implies that $x=0$. Repeating with the other
    basis vectors, we conclude that $a_1=0$.

    So $a_0 + a_1 = a_0 \in K \cap \Gamma_V(K) = K^\times$.
 \end{proof}
   Now we define ${-}^T$ to be the identity on $V$, and extend it to an anti-automorphism
   of $C_V(K)$ (``anti'' means that $(ab)^T=b^Ta^T$). Do not confuse $a\mapsto \alpha(a)$
   (automorphism), $a\mapsto a^T$ (anti-automorphism), and $a\mapsto \alpha(a^T)$
   (anti-automorphism).

   Notice that on $V$, $N$ coincides with the quadratic form $N$. Many authors seem not
   to have noticed this, and use different letters. Sometimes they use a sign
   convention which makes them different.

   Now we define the \emph{spinor norm}\index{spinor norm|idxbf} of $a\in C_V(K)$ by
   $N(a)=aa^T$. We also define a twisted version: $N^\alpha(a)=a\alpha(a)^T$.
   \begin{proposition}
   \begin{enumerate}\item[]
     \item The restriction of $N$ to $\Gamma_V(K)$ is a homomorphism whose image lies in
     $K^\times$. $N$ is a mess on the rest of $C_V(K)$.

     \item The action of $\Gamma_V(K)$ on $V$ is orthogonal. That is, we have a
     homomorphism $\Gamma_V(K)\to O_V(K)$.
   \end{enumerate}
   \end{proposition}
   \begin{proof}
    First we show that if $a\in \Gamma_V(K)$, then $N^\alpha(a)$ acts trivially on $V$.
    \begin{align*}
        N^\alpha(a)\, v\, \alpha\bigl(&N^\alpha(a)\bigr)^{-1} =
            a\alpha(a)^T v \Bigl( \alpha(a)\underbrace{\alpha\bigl(\alpha(a)^T\bigr)}_{\smash{=a^T}}
            \Bigr)^{-1}\\
        &= a\underbrace{\alpha(a)^T v (a^{-1})^T}_{
            \smash{=(a^{-1}v^T \alpha(a))^T}} \alpha(a)^{-1} \\
        &= a a^{-1} v \alpha(a)\alpha(a)^{-1} & \hspace*{6em}
            \llap{($T|_V=\id_V$ and $a^{-1}v\alpha(a)\in V$)}\\
        &= v
    \end{align*}
    So by Lemma \ref{lec23L}, $N^\alpha(a)\in K^\times$. This implies that $N^\alpha$ is
    a homomorphism on $\Gamma_V(K)$ because
    \begin{align*}
      N^\alpha(a)N^\alpha(b) &= a\alpha(a)^T N^\alpha(b) \\
       & = aN^\alpha(b) \alpha(a)^T & (N^\alpha(b) \text{ is central})\\
       & = ab\alpha(b)^T\alpha(a)^T\\
       &=(ab)\alpha(ab)^T=N^\alpha(ab).
    \end{align*}
    After all this work with $N^\alpha$, what we're really interested is $N$. On the even
    elements of $\Gamma_V(K)$, $N$ agrees with $N^\alpha$, and on the odd elements,
    $N=-N^\alpha$. Since $\Gamma_V(K)$ consists of homogeneous elements, $N$ is also a
    homomorphism from $\Gamma_V(K)$ to $K^\times$. This proves the first statement of the
    Proposition.

    Finally, since $N$ is a homomorphism on $\Gamma_V(K)$, the action on $V$ preserves
    the quadratic form $N|_V$. Thus, we have a homomorphism $\Gamma_V(K)\to O_V(K)$.
 \end{proof}
 Now let's analyze the homomorphism $\Gamma_V(K)\to O_V(K)$. Lemma \ref{lec23L} says
 exactly that the kernel is $K^\times$. Next we will show that the image is all of
 $O_V(K)$. Say $r\in V$ and $N(r)\neq 0$.
 \begin{align}
   r v\alpha(r)^{-1} &= -rv\frac{r}{N(r)} = v - \frac{vr^2+rvr}{N(r)} \notag\\
    &= v - \frac{(v,r)}{N(r)}r \label{lec23star}\\
    &= \begin{cases}
       -r &\text{if }v=r\\
        v &\text{if }(v,r)=0
    \end{cases}
 \end{align}
 Thus, $r$ is in $\Gamma_V(K)$, and it acts on $V$ by reflection through the hyperplane
 $r^\perp$. One might deduce that the homomorphism $\Gamma_V(K)\to O_V(K)$ is surjective
 because $O_V(K)$ is generated by reflections. This is wrong; $O_V(K)$ is \emph{not}
 always generated  by reflections!\index{orthogonal group!not generated by reflections}
 \begin{exercise}
   Let $H=\FF_2^2$, with the quadratic form $x^2+y^2+xy$, and let $V=H\oplus H$. Prove
   that $O_V(\FF_2)$ is not generated by reflections.
   \begin{solution}
     In $H$, the norm of any non-zero vector is 1. It is immediate to check that the
     reflection of a non-zero vector $v$ through another non-zero vector $u$ is
     \[
        r_u(v) = \begin{cases}
          u &\text{if }u=v\\
          v+u & \text{if }u\ne v
        \end{cases}
     \]
     so reflection through a non-zero vector fixes that vector and swaps the two other
     non-zero vectors. Thus, the reflection in $H$ generate the symmetric group on three
     elements $S_3$, acting on the three non-zero vectors.

     If $u$ and $v$ are non-zero vectors, then $(u,v)\in H\oplus H$ has norm $1+1=0$, so
     one cannot reflect through it. Thus, every reflection in $V$ is ``in one of the
     $H$'s,'' so the group generated by reflections is $S_3\times S_3$. However,
     swapping the two $H$'s is clearly an  orthogonal transformation, so reflections do
     not generate $O_V(\FF_2)$.
   \end{solution}
 \end{exercise}
 \begin{remark}\label{lec23Rmk:allOK}
   It turns out that this is the \emph{only} counterexample. For any other vector space
   and/or any other non-degenerate quadratic form, $O_V(K)$ is generated by reflections.
   The map $\Gamma_V(K)\to O_V(K)$ is surjective even in the example above. Also, in
   every case except the example above, $\Gamma_V(K)$ is generated as a group by non-zero
   elements of $V$ (i.e.\ every element of $\Gamma_V(K)$ is a monomial).\anton{ref for
   this stuff?}
 \end{remark}
 \begin{remark}
   Equation \ref{lec23star} is the definition of the reflection of $v$ through $r$. It is
   only possible to reflect through vectors of non-zero norm. Reflections in
   characteristic 2 are strange; strange enough that people don't call them reflections,
   they call them \emph{transvections}\index{transvections}.
 \end{remark}

 Thus, we have the diagram
 \begin{equation}
  \xymatrix{
    &1 \ar[r] & K^\times \ar[r] \ar@{}[d]|{\|} & \Gamma_V(K) \ar[d]^N
        \ar[r] & O_V(K) \ar[r] \ar@{.>}[d]^N & 1\\
    1 \ar[r] & \pm 1 \ar[r] & K^\times \ar[r]^{x\mapsto x^2} & K^\times \ar[r] &
        K^\times/(K^\times)^2 \ar[r] & 1}
  \label{lec23dag}
 \end{equation}
 where the rows are exact, $K^\times$ is in
 the center of $\Gamma_V(K)$ (this is obvious, since $K^\times$ is in the center of
 $C_V(K)$), and $N:O_V(K)\to K^\times/(K^\times)^2$ is the unique homomorphism sending
 reflection through $r^\perp$ to $N(r)$ modulo $(K^\times)^2$.


% Summarizing: We have the exact sequence
% \[
%    1\to K^\times \to \Gamma_V(K) \to O_V(K)\to 1
% \]
% where $K^\times$ lands in the center of $\Gamma_V(K)$.
% \[\xymatrix{
% 1 \ar[r] & K^\times \ar[r] \ar@{}[d]|{\|} & \Gamma_V(K) \ar[d]^N \ar[r] & O_V(K)
% \ar[r] \ar@{.>}[d]^N & 1\\
% 1 \ar[r] & K^\times \ar[r]^{x\mapsto x^2} & K^\times \ar[r] & K^\times/(K^\times)^2
% \ar[r] & 1
% }\]

 \begin{definition}
   $\pin_V(K) = \{x\in \Gamma_V(K)| N(x)=1\}$, and $\spin_V(K) =\pin_V^0(K)$, the even
   elements of $\pin_V(K)$.
 \end{definition}
 On $K^\times$, the spinor norm is given by $x\mapsto x^2$, so the elements of spinor
 norm 1 are $=\pm 1$. By restricting the top row of (\ref{lec23dag}) to elements of norm
 1 and even elements of norm 1, respectively, we get exact sequences
 \[
 \xymatrix @R=1em{
    1\ar[r] &\pm 1 \ar[r] & \pin_V(K) \ar[r] & O_V(K) \ar@{.>}[r]^(.45)N & K^\times/(K^\times)^2 \\
    1\ar[r] &\pm 1 \ar[r] & \spin_V(K) \ar[r] & SO_V(K) \ar@{.>}[r]^(.45)N & K^\times/(K^\times)^2
 }
 \]
 To see exactness of the top sequence, note that the kernel of $\phi$ is $K^\times\cap
 \pin_V(K)=\pm 1$, and that the image of $\pin_V(K)$ in $O_V(K)$ is exactly the elements
 of norm 1. The bottom sequence is similar, except that the image of $\spin_V(K)$ is not
 all of $O_V(K)$, it is only $SO_V(K)$; by Remark \ref{lec23Rmk:allOK}, every element of
 $\Gamma_V(K)$ is a product of elements of $V$, so every element of $\spin_V(K)$ is a
 product of an even number of elements of $V$. Thus, its image is a product of an even
 number of reflections, so it is in $SO_V(K)$.

 ?????????????????????????????????????????????????????????????

 These maps are NOT always onto, but there are many important cases when they are,
 like when $V$ has a positive definite quadratic form. The image is the set of
 elements of $O_V(K)$ or $SO_V(K)$ which have spinor norm 1 in
 $K^\times/(K^\times)^2$.

 What is $N:O_V(K) \to K^\times/(K^\times)^2$? It is the UNIQUE homomorphism such that
 $N(a)=N(r)$ if $a$ is reflection in $r^\perp$, and $r$ is a vector of norm $N(r)$.

 \begin{example}
   Take $V$ to be a positive definite vector space over $\RR$. Then $N$ maps to 1 in
   $\RR^\times/(\RR^\times)^2=\pm 1$ (because $N$ is positive definite). So the spinor
   norm on $O_V(\RR)$ is TRIVIAL.
 \end{example}
 So if $V$ is positive definite, we get double covers
 \[
    1\to \pm 1 \to \pin_V(\RR) \to O_V(\RR)\to 1
 \]
 \[
    1\to \pm 1 \to \spin_V(\RR) \to SO_V(\RR)\to 1
 \]
 This will account for the weird double covers we saw before.

 What if $V$ is negative definite. Every reflection now has image $-1$ in
 $\RR^\times/(\RR^\times)^2$, so the spinor norm $N$ is the same as the determinant
 map $O_V(\RR)\to \pm 1$.

 So in order to find interesting examples of the spinor norm, you have to look at
 cases that are neither positive definite nor negative definite.

 Let's look at Losrentz space: $\RR^{1,3}$.

 \[\begin{xy}
   (-1,-1);(1,1) **@{-},
   (1,-1);(-1,1) **@{-},
   (0,1.08) *{{}_{\text{norm}>0}},
   (1.5,0) *{{}_{\text{norm}<0}},
   (1,1);(-1,1) **\crv{(1.2,1.4)&(-1.2,1.4)},
   (1,1);(-1,1) **\crv{(.7,.7)&(-.7,.7)},
   (1,-1);(-1,-1) **\crv{(1.2,-1.4)&(-1.2,-1.4)},
   (1,-1);(-1,-1) **\crv{~*=<2pt>@{.} (.7,-.7)&(-.7,-.7)},
   \ar@/_1mm/ (2.5,1.2) *+{{}_{\text{norm}=0}}; (1,1) *+{\,},
 \end{xy}\]

 Reflection through a vector of norm $<0$ (spacelike vector, $P$: parity reversal) has
 spinor norm $-1$, det $-1$ and reflection through a vector of norm $>0$ (timelike
 vector, $T$: time reversal) has spinor norm $+1$, det $-1$. So $O_{1,3}(\RR)$ has 4
 components (it is not hard to check that these are all the components), usually
 called $1$, $P$, $T$, and $PT$.

 \begin{remark}
   For those who know Galois cohomology. We get an exact sequence of algebraic groups
   \[
   1\to GL_1\to \Gamma_V\to O_V \to 1
   \]
   (algebraic group means you don't put a field). You do not necessarily get an exact
   sequence when you put in a field.

   If
   \[
    1\to A\to B \to C\to 1
    \]
    is exact,
   \[
    1\to A(K) \to B(K)\to C(K)
   \]
   is exact. What you really get is
   \begin{align*}
    1&\to H^0(\mathrm{Gal}(\bar K/K),A) \to H^0(\mathrm{Gal}(\bar K/K), B) \to
    H^0(\mathrm{Gal}(\bar K/K), C)\to\\
    &\to H^1(\mathrm{Gal}(\bar K/K), A)\to \cdots
   \end{align*}
   It turns out that $H^1(Gal(\bar K/K),GL_1)=1$. However, $H^1(Gal(\bar K/K),\pm
   1)=K^\times / (K^\times)^2$.

   So from
   \[
   1\to GL_1\to \Gamma_V\to O_V \to 1
   \]
   you get
   \[
    1\to K^\times \to \Gamma_V(K) \to O_V(K) \to 1= H^1(Gal(\bar K/K),GL_1)
   \]
   However, taking
   \[
    1\to \mu_2 \to \spin_V \to SO_V \to 1
   \]
   you get
   \[
    1\to \pm 1\to \spin_V(K) \to SO_V(K) \xrightarrow{N}
    K^\times/(K^\times)^2=H^1(\bar K/K,\mu_2)
   \]
   so the non-surjectivity of $N$ is some kind of higher Galois cohomology.
   \begin{warning}
     $\spin_V \to SO_V$ is onto as a map of ALGEBRAIC GROUPS, but $\spin_V(K)\to
     SO_V(K)$ need NOT be onto.
   \end{warning}
 \end{remark}

 \begin{example}
   Take $O_3(\RR)\cong SO_3(\RR)\times \{\pm 1\}$ as 3 is odd (in general
   $O_{2n+1}(\RR)\cong SO_{2n+1}(\RR)\times \{\pm 1\}$). So we have a sequence
   \[
    1\to \pm 1\to \spin_3(\RR) \to SO_3(\RR) \to 1.
   \]
   Notice that $\spin_3(\RR)\subseteq C_3^0(\RR)\cong \HH$, so $\spin_3(\RR)\subseteq
   \HH^\times$, and in fact we saw that it is $S^3$.
 \end{example}
}{   % Santiago Canez, scanez@math
  \stepcounter{lecture}
 \setcounter{lecture}{24}
 \sektion{Lecture 24}

 Last time we constructed the sequences
 \[
    1\to K^\times \to \Gamma_V(K)\to O_V(K)\to 1
 \]
 \[
    1\to \pm 1\to \pin_V(K)\to O_V(K)\xrightarrow{N} K^\times/(K^\times )^2
 \]
 \[
    1\to \pm 1 \to \spin_V(K) \to SO_V(K) \xrightarrow{N} K^\times/(K^\times)^2
 \]

 \subsektion{Spin representations of Spin and Pin groups} Notice that
 $\pin_V(K)\subseteq C_V(K)^\times$, so any module over $C_V(K)$ gives a representation of
 $\pin_V(K)$. We already figured out that $C_V(K)$ are direct sums of matrix algebras
 over $\RR,\CC$, and $\HH$.

 What are the representations (modules) of complex Clifford algebras? Recall that
 $C_{2n}(\CC)\cong \MM_{2^n}(\CC)$, which has a representations of dimension $2^n$,
 which is called the spin representation of $\pin_V(K)$ and $C_{2n+1}(\CC)\cong
 \MM_{2^n}(\CC)\times \MM_{2^n}(\CC)$, which has 2 representations, called the spin
 representations of $\pin_{2n+1}(K)$.

 What happens if we restrict these to $\spin_V(\CC)\subseteq \pin_V(\CC)$? To do that,
 we have to recall that $C^0_{2n}(\CC)\cong \MM_{2^{n-1}}(\CC)\times
 \MM_{2^{n-1}}(\CC)$ and $C^0_{2n+1}(\CC)\cong \MM_{2^n}(\CC)$. So in EVEN dimensions
 $\pin_{2n}(\CC)$ has 1 spin representation of dimension $2^n$ splitting into 2 HALF
 SPIN representations of dimension $2^{n-1}$ and in ODD dimensions, $\pin_{2n+1}(\CC)$
 has 2 spin representations of dimension $2^n$ which become the same on restriction to
 $\spin_V(\CC)$.

 Now we'll give a second description of spin representations. We'll just do the
 even dimensional case (odd is similar). Say $\dim V=2n$, and say we're over $\CC$.
 Choose an orthonormal basis $\gamma_1,\dots, \gamma_{2n}$ for $V$, so that
 $\gamma_i^2=1$ and $\gamma_i\gamma_j = -\gamma_j\gamma_i$. Now look at the group $G$
 generated by $\gamma_1,\dots, \gamma_{2n}$, which is finite, with order $2^{1+2n}$
 (you can write all its elements explicitly). You can see that representations of
 $C_V(\CC)$ correspond to representations of $G$, with $-1$ acting as $-1$ (as opposed
 to acting as 1). So another way to look at representations of the Clifford algebra,
 you can look at representations of $G$.

 Let's look at the structure of $G$:
 \begin{itemize}
 \item[(1)] The center is $\pm 1$. This uses the fact that we are in even dimensions,
 lest $\gamma_1\cdots \gamma_{2n}$ also be central.

 \item[(2)] The conjugacy classes: 2 of size 1 (1 and $-1$), $2^{2n}-1$ of size 2
 ($\pm \gamma_{i_1}\cdots \gamma_{i_{n}}$), so we have a total of $2^{2n}+1$ conjugacy
 classes, so we should have that many representations. $G/$center is abelian,
 isomorphic to $(\ZZ/2\ZZ)^{2n}$, which gives us $2^{2n}$ representations of dimension 1, so there is
 only one more left to find! We can figure out its dimension by recalling that the sum
 of the squares of the dimensions of irreducible representations gives us the order of
 $G$, which is $2^{2n+1}$. So $2^{2n}\times 1^1+1\times d^2=2^{2n+1}$, where $d$ is
 the dimension of the mystery representation. Thus, $d=\pm 2^n$, so $d=2^n$. Thus,
 $G$, and therefore $C_V(\CC)$, has an irreducible representation of dimension $2^n$
 (as we found earlier in another way).
 \end{itemize}

 \begin{example}
   Consider $O_{2,1}(\RR)$. As before, $O_{2,1}(\RR)\cong SO_{2,1}(\RR)\times (\pm
   1)$, and $SO_{2,1}(\RR)$ is not connected: it has two components, separated by the
   spinor norm $N$. We have maps
   \[
    1\to \pm 1 \to \spin_{2,1}(\RR)\to SO_{2,1}(\RR) \xrightarrow{N} \pm 1.
    \]
    $\spin_{2,1}(\RR)\subseteq C^*_{2,1}(\RR) \cong \MM_2(\RR)$, so $\spin_{2,1}(\RR)$
   has one 2 dimensional spin representation. So there is a map $\spin_{2,1}(\RR)\to
   SL_2(\RR)$; by counting dimensions and such, you can show it is an isomorphism. So
   $\spin_{2,1}(\RR)\cong SL_2(\RR)$.
 \end{example}
 Now let's look at some 4 dimensional orthogonal groups
 \begin{example}
   Look at $SO_4(\RR)$, which is compact. It has a complex spin representation of
   dimension $2^{4/2}=4$, which splits into two half spin representations of dimension
   2. We have the sequence
   \[
    1\to \pm 1\to \spin_4(\RR)\to SO_4(\RR)\to 1 \qquad (N=1)
   \]
   $\spin_4(\RR)$ is also compact, so the image in any complex representation is
   contained in some unitary group. So we get two maps $\spin_4(\RR)\to SU(2)\times
   SU(2)$, and both sides have dimension 6 and centers of order 4. Thus, we find that
   $\spin_4(\RR)\cong SU(2)\times SU(2) \cong S^3\times S^3$, which give you the two
   half spin representations.
 \end{example}
 So now we've done the positive definite case.
 \begin{example}
   Look at $SO_{3,1}(\RR)$. Notice that $O_{3,1}(\RR)$ has four components
   distinguished by the maps $\det,N\to \pm 1$. So we get
   \[
    1\to \pm 1 \to \spin_{3,1}(\RR)\to SO_{3,1}(\RR)\xrightarrow{N} \pm 1 \to 1
   \]
   We expect 2 half spin representations, which give us two homomorphisms
   $\spin_{3,1}(\RR)\to SL_2(\CC)$. This time, each of these homomorphisms is an
   isomorphism (I can't think of why right now). The $SL_2(\CC)$s are double covers of
   simple groups. Here, we don't get the splitting into a product as in the positive
   definite case. This isomorphism is heavily used in quantum field theory because
   $\spin_{3,1}(\RR)$ is a double cover of the connected component of the Lorentz
   group (and $SL_2(\CC)$ is easy to work with). Note also that the center of
   $\spin_{3,1}(\RR)$ has order 2, not 4, as for $\spin_{4,0}(\RR)$. Also note that
   the group $PSL_2(\CC)$ acts on the compactified $\CC\cup \{\infty\}$ by $\matrix
   abcd (\tau) = \frac{a\tau + b}{c\tau + d}$. Subgroups of this group are called
   KLEINIAN groups. On the other hand, the group $SO_{3,1}(\RR)^+$ (identity
   component) acts on $\HH^3$ (three dimensional hyperbolic space). To see this, look
   at
 \[\begin{xy}
   (1,1);(.29,.29) **@{-}; (-.75,-.75) **@{.}; (-1,-1) **@{-},
   (-1,1);(-.29,.29) **@{-}; (.75,-.75) **@{.}; (1,-1) **@{-},
   %%%%%%%%%%%%%%%%%%
   (1,1.5);(-1,1.5) **\crv{(1.2,1.9)&(-1.2,1.9)},
   (1,1.5);(-1,1.5) **\crv{(.7,1.2)&(-.7,1.2)},
   (1,1.5);(-1,1.5) **\crv{(.5,.7)&(-.5,.7)},
   %%%%%%%%%%%%%%%%%%
   (1,1);(.78,1.22) **\crv{(1.06,1.12)},
   (-1,1);(-.78,1.22) **\crv{(-1.06,1.12)},
   (1,1);(-1,1) **\crv{(.7,.7)&(-.7,.7)},
   (1,-1);(-1,-1) **\crv{(1.2,-1.4)&(-1.2,-1.4)},
   %%%%%%%%%%%%%%%%%%
   (1,-1.5);(-1,-1.5) **\crv{(1.2,-1.9)&(-1.2,-1.9)},
   (1,-1.5);(.78,-1.22) **\crv{(.9,-1.325)};
   (-.78,-1.22) **\crv{~*=<2pt>@{.} (.34,-.8)&(-.34,-.8)};
   (-1,-1.5) **\crv{(-.9,-1.325)},
   %%%%%%%%%%%%%%%%%%
   (1.1,.5);(.75,.74) **\crv{(1.18,.66)},
   (-1.1,.5);(-.75,.74) **\crv{(-1.18,.66)},
   (1.1,.5);(-1.1,.5) **\crv{(.8,.2)&(-.8,.2)},
   (1.1,-.5);(-1.1,-.5) **\crv{(1.3,-.9)&(-1.3,-.9)},
   (1.1,-.5);(1.1,.5) **\crv{(.5,0)},
   (-1.1,-.5);(-1.1,.5) **\crv{(-.5,0)},
   %%%%%%%%%%%%%%%%%%
   \ar (2.5,-1.5) *{{}_{\text{ norm=}-1}}; (1,-1.5) *+{\,},
   \ar@/_1mm/ (2.5,1.2) *+{{}_{\text{norm}=0}}; (1,1) *+{\,},
   \ar@/_1mm/ (2.5,1.7) *+{{}_{\text{norm}=-1}}; (1,1.5) *+{\,},
   \ar@/_1mm/ (2.5,.7) *+{{}_{\text{norm}=1}}; (1.1,.5) *+{\,},
 \end{xy}\]
   One sheet of norm $-1$ hyperboloid is isomorphic to $\HH^3$ under the induced
   metric. In fact, we'll define hyperbolic space that way. If you're a topologist,
   you're very interested in hyperbolic 3-manifolds, which are $\HH^3/$(discrete
   subgroup of $SO_{3,1}(\RR)$). If you use the fact that $SO_{3,1}(\RR)\cong
   PSL_2(\RR)$, then you see that these discrete subgroups are in fact Klienian
   groups.
 \end{example}

 There are lots of exceptional isomorphisms in small dimension, all of which are very
 interesting, and almost all of them can be explained by spin groups.

 \begin{example}
   $O_{2,2}(\RR)$ has 4 components (given by $\det, N$); $C^0_{2,2}(\RR)\cong
   \MM_2(\RR)\times \MM_2(\RR)$, which induces an isomorphism $\spin_{2,2}(\RR)\to
   SL_2(\RR)\times SL_2(\RR)$, which give you the two half spin representations. Both
   sides have dimension 6 with centers of order 4. So this time we get two non-compact
   groups. Let's look at the fundamental group of $SL_2(\RR)$, which is $\ZZ$, so the
   fundamental group of $\spin_{2,2}(\RR)$ is $\ZZ\oplus \ZZ$. As we recall,
   $\spin_{4,0}(\RR)$ and $\spin_{3,1}(\RR)$ were both simply connected. This shows
   that SPIN GROUPS NEED NOT BE SIMPLY CONNECTED. So we can take covers of it. What do
   the corresponding covers (e.g.\ the universal cover) of $\spin_{2,2}(\RR)$ look
   like? This is hard to describe because for FINITE dimensional complex
   representations, you get finite dimensional representations of the Lie algebra $L$,
   which correspond to the finite dimensional representations of $L\otimes \CC$, which
   correspond to the finite dimensional representations of $L'=$ Lie algebra of
   $\spin_{4,0}(\RR)$, which correspond to the finite dimensional representations of
   $\spin_{4,0}(\RR)$, which has no covers because it is simply connected. This means
   that any finite dimensional representation of a cover of $\spin_{2,2}(\RR)$
   actually factors through $\spin_{2,2}(\RR)$. So there is no way you can talk about
   these things with finite matrices, and infinite dimensional representations are
   hard.

   To summarize, the ALGEBRAIC GROUP $\spin_{2,2}$ is simply connected (as an
   algebraic group) (think of an algebraic group as a functor from rings to groups),
   which means that it has no algebraic central extensions. However, the LIE GROUP
   $\spin_{2,2}(\RR)$ is NOT simply connected; it has fundamental group $\ZZ\oplus
   \ZZ$. This problem does not happen for COMPACT Lie groups (where every finite cover
   is algebraic).
 \end{example}

 We've done $O_{4,0}, O_{3,1},$ and $O_{2,2}$, from which we can obviously get
 $O_{1,3}$ and $O_{0,4}$. Note that $O_{4,0}(\RR)\cong O_{0,4}(\RR)$,
 $SO_{4,0}(\RR)\cong SO_{0,4}(\RR)$, $\spin_{4,0}(\RR)\cong \spin_{0,4}(\RR)$.
 However, $\pin_{4,0}(\RR)\not\cong \pin_{0,4}(\RR)$. These two are hard to
 distinguish. We have
 \[\xymatrix{
    \pin_{4,0}(\RR)\ar[d] & \pin_{0,4}(\RR)\ar[d]\\
    O_{4,0}(\RR) \ar@{}[r]|{=} & O_{0,4}(\RR)
 }\]
 Take a reflection (of order 2) in $O_{4,0}(\RR)$, and lift it to the $\pin$ groups.
 What is the order of the lift? The reflection vector $v$, with $v^2=\pm 1$ lifts to
 the element $v\in \Gamma_V(\RR)\subseteq C^*_V(\RR)$. Notice that $v^2=1$ in the case of  $\RR^{4,0}$
 and $v^2=-1$ in the case of $\RR^{0,4}$, so in $\pin_{4,0}(\RR)$, the reflection lifts to
 something of order 2, but in $\pin_{0,4}(\RR)$, you get an element of order 4!. So
 these two groups are different.

 Two groups are \emph{isoclinic} if they are confusingly similar. A similar
 phenomenon is common for groups of the form $2\cdot G\cdot 2$, which means it has a
 center of order 2, then some group $G$, and the abelianization has order $2$. Watch
 out.

 \begin{exercise}
   $\spin_{3,3}(\RR)\cong SL_4(\RR)$.
 \end{exercise}

 \subsektion{Triality} This is a special property of 8 dimensional orthogonal groups.
 Recall that $O_8(\CC)$ has the Dynkin diagram $D_4$, which has a symmetry of order
 three:

    \[\begin{xy}
     (0,0) *\cir<2pt>{};
     a(60)="1" *\cir<2pt>{} **@{-},
     a(180)="2" *\cir<2pt>{} **@{-},
     a(-60)="3" *\cir<2pt>{} **@{-},
     \ar@/_2ex/ "1" *+{\ };"2" *+{\ }
     \ar@/_2ex/ "2" *+{\ };"3" *+{\ }
     \ar@/_2ex/ "3" *+{\ };"1" *+{\ }
   \end{xy}\]

 But $O_8(\CC)$ and $SO_8(\CC)$ do NOT have corresponding symmetries of order three.
 The thing that does have the symmetry of order three is the spin group! The group
 $\spin_8(\RR)$ DOES have ``extra'' order three symmetry. You can see it as follows.
 Look at the half spin representations of $\spin_8(\RR)$. Since this is a spin group
 in even dimension, there are two. $C_{8,0}(\RR)\cong \MM_{2^{8/2 -1}}(\RR)\times \MM_{2^{8/2
 -1}}(\RR) \cong \MM_8(\RR)\times \MM_8(\RR)$. So $\spin_8(\RR)$ has two 8 dimensional
 real half spin representations. But the spin group is compact, so it preserves some
 quadratic form, so you get 2 homomorphisms $\spin_8(\RR)\to SO_8(\RR)$. So
 $\spin_8(\RR)$ has THREE 8 dimensional representations: the half spins, and the one
 from the map to $SO_8(\RR)$. These maps $\spin_8(\RR)\to SO_8(\RR)$ lift to Triality automorphisms $\spin_8(\RR)\to \spin_8(\RR)$.  The center of
 $\spin_8(\RR)$ is $(\ZZ/2)+ (\ZZ/2)$ because the center of the Clifford group is $\pm
 1, \pm \gamma_1\cdots\gamma_8$. There are 3 non-trivial elements of the center, and
 quotienting by any of these gives you something isomorphic to $SO_8(\RR)$. This is
 special to 8 dimensions.

 \subsektion{More about Orthogonal groups} Is $O_V(K)$ a simple group? NO, for the
 following reasons:
 \begin{itemize}
 \item[(1)] There is a determinant map $O_V(K)\to \pm 1$, which is usually onto, so it
 can't be simple.

 \item[(2)] There is a spinor norm map $O_V(K)\to K^\times/(K^\times)^2$

 \item[(3)] $-1\in $ center of $O_V(K)$.

 \item[(4)] $SO_V(K)$ tends to split if $\dim V=4$, abelian if $\dim V=2$, and trivial
 if $\dim V=1$.
 \end{itemize}
 It turns out that they are usually simple apart from these four reasons why they're
 not. Let's mod out by the determinant, to get to $SO$, then look at $\spin_V(K)$,
 then quotient by the center, and assume that $\dim V\ge 5$. Then this is usually
 simple. The center tends to have order 1,2, or 4. If $K$ is a FINITE field, then this
 gives many finite simple groups.

 Note that $SO_V(K)$ is NOT a subgroup of $O_V(K)$, elements of determinant 1 in
 general, it is the image of $\Gamma^0_V(K)\subseteq \Gamma_V(K)\to O_V(K)$, which is
 the correct definition. Let's look at why this is right and the definition you know
 is wrong. There is a homomorphism $\Gamma_V(K)\to \ZZ/2\ZZ$, which takes
 $\Gamma^0_V(K)$ to 0 and $\Gamma^1_V(K)$ to 1 (called the DICKSON INVARIANT). It is
 easy to check that $\det(v) = (-1)^{\text{dickson invariant}(v)}$. So if the
 characteristic of $K$ is not 2, $\det=1$ is equivalent to dickson $=0$, but in
 characteristic 2, determinant is the wrong invariant (because determinant is always
 1).

 Special properties of $O_{1,n}(\RR)$ and $O_{2,n}(\RR)$. $O_{1,n}(\RR)$ acts on
 hyperbolic space $\HH^n$, which is a component of norm $-1$ vectors in $\RR^{n,1}$.
 $O_{2,n}(\RR)$ acts on the ``Hermitian symmetric space'' (Hermitian means it has a
 complex structure, and symmetric means really nice).
There are three ways to construct this space:
 \begin{itemize}
 \item[(1)] It is the set of positive definite 2 dimensional subspaces of $\RR^{2,n}$

 \item[(2)] It is the norm 0 vectors $\w$ of $\mathbb{P}\CC^{2,n}$ with $(\w,\bar \w)=0$.

 \item[(3)] It is the vectors $x+iy\in \RR^{1,n-1}$ with $y\in C$, where the cone $C$
 is the interior of the norm 0 cone.
 \end{itemize}

 \begin{exercise}
   Show that these are the same.
 \end{exercise}

 Next week, we'll mess around with $E_8$.
}{   % Lilit Martirosyan, lilit@math
  \stepcounter{lecture}
 \setcounter{lecture}{25}
 \sektion{Lecture 25 - \texorpdfstring{$E_8$}{E8}}

In this lecture we use a vector notation in which powers represent repetitions:  so
$(1^8)=(1,1,1,1,1,1,1,1)$ and $(\pm \half^2 , 0^6)=(\pm \half, \pm \half,
0,0,0,0,0,0)$.

  Recall that $E_8$ has the Dynkin diagram
 \[\begin{xy}
   (0,0) *+!U{e_1-e_2} *\cir<2pt>{};
   (1,0) *+!D{e_2-e_3} *\cir<2pt>{} **@{-};
   p+(1,0) *+!U{e_3-e_4} *\cir<2pt>{} **@{-};
   p+(1,0) *+!D{e_4-e_5} *\cir<2pt>{} **@{-};
   p+(1,0) *+!U{e_5-e_6} *\cir<2pt>{} **@{-};
   p+(0,1) *+!D{(-\half^5 , \, \half^3)} *\cir<2pt>{} **@{-},
   p+(1,0) *+!D{e_6-e_7} *\cir<2pt>{} **@{-};
   p+(1,0) *+!U{e_7-e_8} *\cir<2pt>{} **@{-};
 \end{xy} \]
  where each vertex is a root $r$ with $(r,r)=2$; $(r,s)=0$ when $r$ and $s$ are
  not joined, and $(r,s)=-1$ when $r$ and $s$ are joined. We choose an orthonormal basis
   $e_1,\dots, e_8$, in which the roots are as given.

  We want to figure out what the root lattice $L$ of $E_8$ is (this is the lattice
  generated by the roots). If you take $\{e_i-e_{i+1}\} \cup (-1^5 , \, 1^3)$ (all the
  $A_7$ vectors plus twice the strange vector), they generate the $D_8$ lattice
  $=\{(x_1,\dots, x_8)|x_i\in \ZZ, \quad \sum x_i \text{ even}\}$. So the $E_8$
  lattice consists of two cosets of this lattice, where the other coset is
  $\{(x_1,\dots, x_8)|x_i\in \ZZ+\frac{1}{2},\quad \sum x_i\text{ odd}\}$.

  Alternative version: If you reflect this lattice through the hyperplane $e_1^\perp$,
  then you get the same thing except that $\sum x_i$ is always even.  We will freely
  use both characterizations, depending on which is more convenient for the
  calculation at hand.

  We should also work out the weight lattice, which is the vectors $s$ such that
  $(r,r)/2$ divides $(r,s)$ for all roots $r$. Notice that the weight lattice of $E_8$
  is contained in the weight lattice of $D_8$, which is the union of four cosets of
  $D_8$:  $D_8$, $D_8+(1 , \, 0^7)$, $D_8+(\half^8)$ and $ D_8+(-\half, \, \half^7)$.
  Which of these have integral inner product with the vector $(-\half^5 , \,
  \half^3)$? They are the first and the last, so the weight lattice of $E_8$ is
  $D_8\cup D_8+(-\half, \half^7)$, which is equal to the root lattice of $E_8$.

  In other words, the $E_8$ lattice $L$ is UNIMODULAR (equal to its dual $L'$), where
  the dual is the lattice of vectors having integral inner product with all lattice
  vectors.  This is also true of $G_2$ and $F_4$, but is not in general true of Lie
  algebra lattices.

  The $E_8$ lattice is EVEN, which means that the inner product of any vector with
  itself is always even.

  Even unimodular lattices in $\RR^n$ only exist if $8|n$ (this $8$ is the same $8$
 that shows up in the periodicity of Clifford groups). The $E_8$ lattice is the only
 example in dimension equal to $8$ (up to isomorphism, of course). There are two in
 dimension 16 (one of which is $L\oplus L$, the other is $D_{16}\cup$ some coset).
 There are 24 in dimension 24, which are the Niemeier lattices. In 32 dimensions,
 there are more than a billion!

 The Weyl group of $E_8$ is generated by the reflections through $s^\perp$ where $s
 \in L$ and $(s,s)=2$ (these are called roots). First, let's find all the roots:
 $(x_1,\dots,x_8)$ such that $\sum x_i^2=2$ with $x_i\in \ZZ$ or $\ZZ+\half$ and $\sum
 x_i$ even. If $x_i \in \ZZ$, obviously the only solutions are permutations of $(\pm
 1, \pm 1, 0^6)$, of which there are $\binom{8}{2}\times 2^2=112$ choices. In the
 $\ZZ+\half$ case, you can choose the first 7 places to be $\pm \half$, and the last
 coordinate is forced, so there are $2^7$ choices. Thus, you get $240$ roots.

 Let's find the orbits of the roots under the action of the Weyl group. We don't yet
 know what the Weyl group looks like, but we can find a large subgroup that is easy to
 work with. Let's use the Weyl group of $D_8$, which consists of the following: we can
 apply all permutations of the coordinates, or we can change the sign of an even
 number of coordinates (e.g., reflection in $(1 , {-1} , 0^6)$ swaps the first two
 coordinates, and reflection in $(1 , \, {-1} , \, 0^6)$ followed by reflection in $(1
 ,  1, 0^6)$ changes the sign of the first two coordinates.)

 Notice that under the Weyl group of $D_8$, the roots form two orbits: the set which
 is all permutations of $(\pm 1^2 , 0^6)$, and the set $(\pm \half^8)$. Do these
 become the same orbit under the Weyl group of $E_8$? Yes; to show this, we just need
 one element of the Weyl group of $E_8$ taking some element of the first orbit to the
 second orbit. Take reflection in $(\half^8)^\perp$ and apply it to $(1^2 , 0^6)$: you
 get $(\half^2 , -\half^6)$, which is in the second orbit. So there is just one orbit
 of roots under the Weyl group.

 What do orbits of $W(E_8)$ on other vectors look like? We're interested in this
 because we might want to do representation theory. The character of a representation
 is a map from weights to integers, which is $W(E_8)$-invariant. Let's look at vectors
 of norm $4$ for example. So $\sum x_i^2=4$, $\sum x_i$ even, and $x_i\in \ZZ$ or
 $x_i\in \ZZ+\half$. There are $8\times 2$ possibilities which are permutations of
 $(\pm 2 ,  0^7)$. There are $\binom 84 \times 2^4$ permutations of $(\pm 1^4 ,
 0^4)$, and there are $8\times 2^7$ permutations of  $(\pm \frac{3}{2}, \pm \half^7)$.
 So there are a total of $240\times 9$ of these vectors. There are 3 orbits under
 $W(D_8)$, and as before, they are all one orbit under the action of $W(E_8)$. Just
 reflect $(2 , \, 0^7)$ and $(1^3 , -1, 0^4)$ through $(\half^8)$.

 \begin{exercise}
   Show that the number of norm 6 vectors is $240\times 28$, and they form one orbit
 \end{exercise}
(If you've seen a course on modular forms, you'll know that the number of vectors of
norm
 $2n$ is given by $240\times \sum_{d|n} d^3$. If you let call these $c_n$, then
 $\sum c_n q^n$ is a modular form of level 1 ($E_8$ even, unimodular), weight 4 ($\dim
 E_8/2$).)

 For norm $8$ there are two orbits, because you have vectors that are twice a norm 2
 vector, and vectors that aren't.  As the norm gets bigger, you'll get a large number
 of orbits.

 What is the order of the Weyl group of $E_8$? We'll do this by 4 different methods,
 which illustrate the different techniques for this kind of thing:
 \begin{itemize}
 \item[(1)] This is a good one as a mnemonic. The order of $E_8$ is given by
 \begin{align*}
   |W(E_8)| &= 8! \times \prod\left(
        \genfrac{}{}{0em}{}{\text{numbers on the}}{\text{affine $E_8$
        diagram\footnotemark}}
    \right) \times \frac{\text{Weight lattice of $E_8$}}{\text{Root lattice of $E_8$}}\\
    &= 8! \times
        \biggr( \begin{xy}
       <1em,0em>:
       (0,0) *+!U{1} *\cir<2pt>{};
       (1,0) *+!U{2} *\cir<2pt>{} **@{-};
       p+(1,0) *+!U{3} *\cir<2pt>{} **@{-};
       p+(1,0) *+!U{4} *\cir<2pt>{} **@{-};
       p+(1,0) *+!U{5} *\cir<2pt>{} **@{-};
       p+(1,0) *+!U{6} *\cir<2pt>{} **@{-};
       p+(0,1) *+!R{3} *\cir<2pt>{} **@{-},
       p+(1,0) *+!U{4} *\cir<2pt>{} **@{-};
       p+(1,0) *+!U{2} *\cir<2pt>{} **@{-};
       \end{xy}\biggr)
       \times 1\\
    &= 2^{14} \times 3^5\times 5^2\times 7
 \end{align*}
 \footnotetext{These are the numbers giving highest root.}

 We can do the same thing for any other Lie algebra, for example,
 \begin{align*}
   |W(F_4)| &= 4!\times (
   \begin{xy}
   <2.25em,0em>:
   (0,0) *+!D{1} *\cir<2pt>{};
   (1,0) *+!D{2} *\cir<2pt>{} **@{-};
   p+(1,0) *+!D{3} *\cir<2pt>{} **@{-};
   p+(1,0)="x" *+!D{4} *\cir<2pt>{} **@{=} ?*@{>};
   "x" *{\hspace{4pt}};p+(1,0) *+!D{2} *\cir<2pt>{} **@{-};
   \end{xy}
   ) \times 1\\
   &=2^7 \times 3^2
 \end{align*}

 \item[(2)] The order of a reflection group is equal to the products of degrees of the
 fundamental invariants. For $E_8$, the fundamental invariants are of degrees
 2,8,12,14,18,20,24,30 (primes $+1$).

 \item[(3)] This one is actually an honest method (without quoting weird facts). The
 only fact we will use is the following: suppose $G$ acts transitively on a set $X$
 with $H=$ the group fixing some point; then $|G|=|H|\cdot |X|$.

 This is a general purpose method for working out the orders of groups. First, we need
 a set acted on by the Weyl group of $E_8$. Let's take the root vectors (vectors of
 norm 2). This set has 240 elements, and the Weyl group of $E_8$ acts transitively on it.
 So $|W(E_8)|=240\times |$subgroup fixing $(1 , -1 , 0^6)|$. But what is the order
 of this subgroup (call it $G_1$)? Let's find a set acted on by this group. It acts on
 the  set of norm 2 vectors, but the action is NOT transitive. What are the orbits?
 $G_1$ fixes $s=(1 , -1 , 0^6)$. For other roots $r$, $G_1$ obviously fixes
 $(r,s)$. So how many roots are there with a given inner product with $s$?
 \[\begin{array}{c|c|c}
   (s,r) & \text{number} & \text{choices}\\ \hline
   2 & 1 & s\\
   1 & 56 & (1 ,  0 , \pm 1^6), (0,-1,\pm 1^6), (\half,-\half, \half^6)\\
   0 & 126 & \\
   -1 & 56 & \\
   -2 & 1 & -s\\
 \end{array}\]
 So there are at least 5 orbits under $G_1$. In fact, each of these sets is a single
 orbit under $G_1$. How can we see this? Find a large subgroup of $G_1$. Take
 $W(D_6)$, which is all permutations of the last 6 coordinates and all even sign
 changes of the last 6 coordinates. It is generated by reflections associated to the roots
 orthogonal to $e_1$ and $e_2$ (those that start with two 0s). The three cases with
 inner product 1 are three orbits under $W(D_6)$. To see that there is a single orbit
 under $G_1$, we just need some reflections that mess up these orbits. If you take a vector
 $(\half,\half,\pm \half^6)$ and reflect norm $2$ vectors through it, you will get
 exactly $5$ orbits. So $G_1$ acts transitively on these orbits.

 We'll use the orbit of vectors $r$ with $(r,s)=-1$. Let $G_2$ be the vectors fixing
 $s$ and $r$:
 \begin{xy}
   (0,0) *+!D{s} *\cir<2pt>{};
   (1,0) *+!D{r} *\cir<2pt>{} **@{-};
 \end{xy}
 We have that $|G_1| = |G_2|\cdot 56$.

 Keep going ... it gets tedious, but here are the answers up to the last step:

 Our plan is to chose vectors acted on by $G_i$, fixed by $G_{i+1}$ which give us the
Dynkin diagram of $E_8$.  So the next step is to try to find vectors $t$ that give us
the picture
 \begin{xy}
   (0,0) *+!D{s} *\cir<2pt>{};
   (1,0) *+!D{r} *\cir<2pt>{} **@{-};
   (2,0) *+!D{t} *\cir<2pt>{} **@{.};
 \end{xy},
 i.e, they have inner product $-1$ with $r$ and $0$ with $s$. The possibilities for
 $t$ are $(-1, -1,0,0^5)$ (one of these), $(0,0,1, \pm 1, 0^4)$ and permutations of
 its last five coordinates (10 of these), and $(-\half,-\half,\half,\pm \half^5)$
 (there are 16 of these), so we get 27 total. Then we could check that they form one
 orbit, which is boring.

 Next find vectors which go next to $t$ in our picture:\\
  $\begin{xy}
   (0,0) *+!D{s} *\cir<2pt>{};
   (1,0) *+!D{r} *\cir<2pt>{} **@{-};
   p+(1,0) *+!D{t} *\cir<2pt>{} **@{-};
   p+(1,0) *\cir<2pt>{} **@{.};
 \end{xy}\ $,
 i.e., whose inner product is $-1$ with $t$ and zero with $r,s$. The possibilities are
 permutations of the last four coords of $(0,0,0,1,\pm 1, 0^3)$ (8 of these) and
 $(-\half,-\half,-\half,\half,\pm \half^4)$ (8 of these), so there are 16 total. Again
 check transitivity.

 Find a fifth vector; the possibilities are $(0^4,1,\pm 1, 0^2)$ and perms of the last
 three coords (6 of these), and $(-\half^4,\half, \pm \half^3)$ (4 of these) for a
 total of 10.

 For the sixth vector, we can have $(0^5, 1,\pm 1, 0)$ or $(0^5, 1, 0, \pm 1)$ (4
possibilites) or $(-\half^5,\half,\pm \half^2)$ (2 possibilities), so we get 6 total.

 NEXT CASE IS TRICKY: finding the seventh one, the possibilities are $(0^6,1,\pm 1)$
 (2 of these) and $((-\half)^6,\half,\half)$ (just 1). The proof of transitivity fails
 at this point. The group we're using by now doesn't even act transitively on the pair
 (you can't get between them by changing an even number of signs). What elements of
 $W(E_8)$ fix all of these first 6 points $\begin{xy}
   (0,0) *+!D{s} *\cir<2pt>{};
   (1,0) *+!D{r} *\cir<2pt>{} **@{-};
   p+(1,0) *+!D{t} *\cir<2pt>{} **@{-};
   p+(1,0) *\cir<2pt>{} **@{-};
   p+(1,0) *\cir<2pt>{} **@{-};
   p+(1,0) *\cir<2pt>{} **@{-};
 \end{xy}\ $
 ? We want to find roots perpendicular to all of these vectors, and the only
 possibility is $((\half)^8)$. How does reflection in this root act on the three vectors
 above? $(0^6, 1^2)\mapsto ((-\half)^6,\half^2)$ and $(0^6,1,-1)$ maps to
 itself. Is this last vector in the same orbit? In fact they are in different orbits.
 To see this, look for vectors
 \[\begin{xy}
   (0,0) *+!D{s} *\cir<2pt>{};
   (1,0) *+!D{r} *\cir<2pt>{} **@{-};
   p+(1,0) *+!D{t} *\cir<2pt>{} **@{-};
   p+(1,0) *\cir<2pt>{} **@{-};
   p+(1,0) *\cir<2pt>{} **@{-};
    p+(0,1) *+!D{?} *\cir<2pt>{} **@{.},
   p+(1,0) *\cir<2pt>{} **@{-};
   p+(1,0) *+!L{(0^6,1,\pm 1)} *\cir<2pt>{} **@{-};
 \end{xy}\]

 completing the $E_8$ diagram. In the $(0^6, 1,1)$ case, you can take the vector
 $((-\half)^5,\half,\half,-\half)$. But in the other case, you can show that there are no
 possibilities. So these really are different orbits.

 Use the orbit with 2 elements, and you get
 \[
    |W(E_8)| = 240\times \underbrace{56\times \overbrace{27 \times 16\times 10 \times 6\times 2\times
    1}^{\text{order of $W(E_6)$}}}_{\text{order of $W(E_7)$}}
 \]
 because the group fixing all 8 vectors must be trivial.
 You also get that
 \[
    |W(\text{``}E_5\text{''})| = 16\times \underbrace{10 \times \overbrace{6\times
            2\times 1}^{|W(A_2\times A_1)|}}_{|W(A_4)|}
 \]
 where ``$E_5$'' is the algebra with diagram
 $\begin{xy}<1.75em,0em>:
   (0,0) *\cir<2pt>{};
   (1,0) *\cir<2pt>{} **@{-};
    p+(0,1) *\cir<2pt>{} **@{-},
   p+(1,0) *\cir<2pt>{} **@{-};
   p+(1,0) *\cir<2pt>{} **@{-};
 \end{xy}$ (that is, $D_5$). Similarly, $E_4$ is $A_4$ and $E_3$ is $A_2\times A_1$.

 We got some other information. We found that the Weyl group of $E_8$ acts
 transitively on all the configurations

 \[\begin{array}{l}
    \begin{xy}
      (0,0) *\cir<2pt>{};
    \end{xy} \\
    \begin{xy}
      (0,0) *\cir<2pt>{}; p+(1,0) *\cir<2pt>{} **@{-};
    \end{xy} \\
    \begin{xy}
      (0,0) *\cir<2pt>{}; p+(1,0) *\cir<2pt>{} **@{-};
      p+(1,0) *\cir<2pt>{} **@{-};
    \end{xy} \\
    \begin{xy}
      (0,0) *\cir<2pt>{}; p+(1,0) *\cir<2pt>{} **@{-};
      p+(1,0) *\cir<2pt>{} **@{-}; p+(1,0) *\cir<2pt>{} **@{-};
    \end{xy} \\
    \begin{xy}
      (0,0) *\cir<2pt>{}; p+(1,0) *\cir<2pt>{} **@{-};
      p+(1,0) *\cir<2pt>{} **@{-}; p+(1,0) *\cir<2pt>{} **@{-};
      p+(1,0) *\cir<2pt>{} **@{-};
    \end{xy} \\
    \begin{xy}
      (0,0) *\cir<2pt>{}; p+(1,0) *\cir<2pt>{} **@{-};
      p+(1,0) *\cir<2pt>{} **@{-}; p+(1,0) *\cir<2pt>{} **@{-};
      p+(1,0) *\cir<2pt>{} **@{-}; p+(1,0) *\cir<2pt>{} **@{-};
    \end{xy} \\
 \end{array}\]
 but not on
 \[
     \begin{xy}
      (0,0) *\cir<2pt>{}; p+(1,0) *\cir<2pt>{} **@{-};
      p+(1,0) *\cir<2pt>{} **@{-}; p+(1,0) *\cir<2pt>{} **@{-};
      p+(1,0) *\cir<2pt>{} **@{-}; p+(1,0) *\cir<2pt>{} **@{-};
      p+(1,0) *\cir<2pt>{} **@{-};
    \end{xy}
 \]
 \item[(4)] We'll slip this in to next lecture
 \end{itemize}
Also, next time we'll construct the Lie algebra $E_8$.
}{   % Emily Peters, eep@math
  \stepcounter{lecture}
 \setcounter{lecture}{26}
 \sektion{Lecture 26}

 Today we'll finish looking at $W(E_8)$, then we'll construct $E_8$.

 Remember that we still have a fourth method of finding the order of $W(E_8)$. Let $L$
 be the $E_8$ lattice. Look at $L/2L$, which has 256 elements. Look at this as a set
 acted on by $W(E_8)$. There is an orbit of size 1 (represented by 0). There is an
 orbit of size $240/2=120$, which are the roots (a root is congruent mod $2L$ to it's
 negative). Left over are 135 elements. Let's look at norm 4 vectors. Each norm 4
 vector, $r$, satisfies $r\equiv -r \mod 2$, and there are $240\cdot 9$ of them, which
 is a lot, so norm 4 vectors must be congruent to a bunch of stuff. Let's look at
 $r=(2,0,0,0,0,0,0,0)$. Notice that it is congruent to vectors of the form $(0\dots
 \pm 2\dots 0)$, of which there are 16. It is easy to check that these are the only
 norm 4 vectors congruent to $r$ mod 2. So we can partition the norm 4 vectors into
 $240\cdot 9/16=135$ subsets of 16 elements. So $L/2L$ has 1+120+135 elements, where
 $1$ is the zero, 120 is represented by 2 elements of norm 2, and 135 is represented
 by 16 elements of norm 4. A set of 16 elements of norm 4 which are all congruent is
 called a FRAME. It consists of elements $\pm e_1,\dots, \pm e_8$, where $e_i^2=4$ and
 $(e_i,e_j)=1$ for $i\neq j$, so up to sign it is an orthogonal basis.

 Then we have
 \[
    |W(E_8)| = (\text{\# frames})\times |\text{subgroup fixing a frame}|
 \]
 because we know that $W(E_8)$ acts transitively on frames. So we need to know what
 the automorphisms of an orthogonal base are. A frame is 8 subsets of the form
 $(r,-r)$, and isometries of a frame form the group $(\ZZ/2\ZZ)^8\cdot S_8$, but these
 are not all in the Weyl group. In the Weyl group, we found a $(\ZZ/2\ZZ)^7\cdot S_8$,
 where the first part is the group of sign changes of an EVEN number of coordinates.
 So the subgroup fixing a frame must be in between these two groups, and since these
 groups differ by a factor of 2, it must be one of them. Observe that changing an odd
 number of signs doesn't preserve the $E_8$ lattice, so it must be the group
 $(\ZZ/2\ZZ)^7\cdot S_8$, which has order $2^7\cdot 8!$. So the order of the Weyl
 group is
 \[
    135\cdot 2^7\cdot 8! = |2^7\cdot S_8| \times \frac{\text{\# norm 4
    elements}}{2\times \dim L}
 \]
 \begin{remark}
 Similarly, if $\Lambda$ is the Leech lattice, you actually get the order of Conway's
 group to be
 \[
    |2^{12}\cdot M_{24}|\cdot \frac{\text{\# norm 8 elements}}{2\times \dim\Lambda}
 \]
 where $M_{24}$ is the Mathieu group (one of the sporadic simple groups). The Leech
 lattice seems very much to be trying to be the root lattice of the monster group, or
 something like that. There are a lot of analogies, but nobody can make sense of it.
 \end{remark}

 $W(E_8)$ acts on $(\ZZ/2\ZZ)^8$, which is a vector space over $\FF_2$, with quadratic
 form $N(a)=\frac{(a,a)}{2} \mod 2$, so you get a map
 \[
    \pm 1\to W(E_8) \to O^+_8(\FF_2)
 \]
 which has kernel $\pm 1$ and is surjective. $O^+_8$ is one of the $8$ dimensional
 orthogonal groups over $\FF_2$. So the Weyl group is very close to being an
 orthogonal group of a vector space over $\FF_2$.

 What is inside the root lattice/Lie algebra/Lie group $E_8$? One obvious way to find
 things inside is to cover nodes of the $E_8$ diagram:
 \[\begin{xy}
   (0,0) *\cir<2pt>{};
   p+(1,0) *\cir<2pt>{} **@{-};
   p+(1,0) *{\times} *\cir<2pt>{} **@{-};
   p+(1,0) *\cir<2pt>{} **@{-};
   p+(1,0) *\cir<2pt>{} **@{-};
   p+(0,1) *\cir<2pt>{} **@{-},
   p+(1,0) *\cir<2pt>{} **@{-};
   p+(1,0) *\cir<2pt>{} **@{-};
 \end{xy} \]
 If we remove the shown node, you see that $E_8$ contains $A_2\times D_5$. We can do
 better by showing that we can embed the affine $\tilde E_8$ in the $E_8$ lattice.
 \[\begin{xy}
   (0,0) *+!UR{-\text{highest root}} *\cir<2pt>{};
   p+(1,0) *\cir<2pt>{} **@{-};
   p+(1,0) *\cir<2pt>{} **@{-};
   p+(1,0) *\cir<2pt>{} **@{-};
   p+(1,0) *\cir<2pt>{} **@{-};
   p+(1,0) *\cir<2pt>{} **@{-};
   p+(0,1) *\cir<2pt>{} **@{-},
   p+(1,0) *\cir<2pt>{} **@{-};
   p+(1,0) *\cir<2pt>{} **@{-};
   (4,-.3) *=(6.3,0)\frm{_\}} *+!U{\text{simple roots}}
 \end{xy} \]
 Now you can remove nodes here and get some bigger sub-diagrams. For example, if we
 cover
 \[\begin{xy}
   (0,0) *\cir<2pt>{};
   p+(1,0) *{\times} *\cir<2pt>{} **@{-};
   p+(1,0) *\cir<2pt>{} **@{-};
   p+(1,0) *\cir<2pt>{} **@{-};
   p+(1,0) *\cir<2pt>{} **@{-};
   p+(1,0) *\cir<2pt>{} **@{-};
   p+(0,1) *\cir<2pt>{} **@{-},
   p+(1,0) *\cir<2pt>{} **@{-};
   p+(1,0) *\cir<2pt>{} **@{-};
 \end{xy} \]
 you get that an $A_1\times E_7$ in $E_8$. The $E_7$ consisted of 126
 roots orthogonal to a given root. This gives an easy construction of $E_7$ root
 system, as all the elements of the $E_8$ lattice perpendicular to $(1,-1,0\dots)$

 We can cover
 \[\begin{xy}
   (0,0) *\cir<2pt>{};
   p+(1,0) *\cir<2pt>{} **@{-};
   p+(1,0) *{\times} *\cir<2pt>{} **@{-};
   p+(1,0) *\cir<2pt>{} **@{-};
   p+(1,0) *\cir<2pt>{} **@{-};
   p+(1,0) *\cir<2pt>{} **@{-};
   p+(0,1) *\cir<2pt>{} **@{-},
   p+(1,0) *\cir<2pt>{} **@{-};
   p+(1,0) *\cir<2pt>{} **@{-};
 \end{xy} \]
 Then we get an $A_2\times E_6$, where the $E_6$ are all the vectors with the first 3
 coordinates equal. So we get the $E_6$ lattice for free too.

 If you cover
 \[\begin{xy}
   (0,0) *\cir<2pt>{};
   p+(1,0) *\cir<2pt>{} **@{-};
   p+(1,0) *\cir<2pt>{} **@{-};
   p+(1,0) *\cir<2pt>{} **@{-};
   p+(1,0) *\cir<2pt>{} **@{-};
   p+(1,0) *\cir<2pt>{} **@{-};
   p+(0,1) *\cir<2pt>{} **@{-},
   p+(1,0) *\cir<2pt>{} **@{-};
   p+(1,0) *{\times} *\cir<2pt>{} **@{-};
 \end{xy} \]
 you see that there is a $D_8$ in $E_8$, which is all vectors of the $E_8$ lattice
 with integer coordinates. We sort of constructed the $E_8$ lattice this way in the
 first place.

 We can ask questions like: What is the $E_8$ Lie algebra as a representation of
 $D_8$? To answer this, we look at the weights of the $E_8$ algebra, considered as a
 module over $D_8$, which are the 112 roots of the form $(\dots\pm 1\dots \pm 1\dots)$
 and the 128 roots of the form $(\pm 1/2,\dots)$ and 1 vector 0, with multiplicity 8.
 These give you the Lie algebra of $D_8$. Recall that $D_8$ is the Lie algebra of
 $SO_{16}$. The double cover has a half spin representation of dimension $2^{16/2
 -1}=128$. So $E_8$ decomposes as a representation of $D_8$ as the adjoint
 representation (of dimension 120) plus a half spin representation of dimension 128.
 This is often used to construct the Lie algebra $E_8$. We'll do a better construction
 in a little while.

 We've found that the Lie algebra of $D_8$, which is the Lie algebra of $SO_{16}$, is
 contained in the Lie algebra of $E_8$. Which \emph{group} is contained in the the compact
 form of the $E_8$? We found that there were groups
 \[\xymatrix @R=.75em {
   & \spin_{16}(\RR) \ar@{-}[dl] \ar@{-}[d] \ar@{-}[dr]\\
   SO_{16}(\RR) \ar@{-}[dr]& \spin_{16}(\RR)/(\ZZ/2\ZZ) \ar@{-}[d] \ar@{}[r]|{\cong}
   & *+[F-:<10pt>]{\spin_{16}(\RR)/(\ZZ/2\ZZ)}  \ar@{-}[dl]\\
   & PSO_{16}(\RR)
 }\]
 corresponding to subgroups of the center $(\ZZ/2\ZZ)^2$:
 \[\xymatrix @R=.75em {
   & 1 \ar@{-}[dl] \ar@{-}[d] \ar@{-}[dr]\\
   \ZZ/2\ZZ \ar@{-}[dr]& \ZZ/2\ZZ \ar@{-}[d]& \ZZ/2\ZZ \ar@{-}[dl]\\
   & (\ZZ/2\ZZ)^2
 }\]
 We have a homomorphism $\spin_{16}(\RR)\to E_8$(compact). What is the kernel? The
 kernel are elements which act trivially on the Lie algebra of $E_8$, which is equal
 to the Lie algebra $D_8$ plus the half spin representation. On the Lie algebra of
 $D_8$, everything in the center is trivial, and on the half spin representation, one
 of the elements of order 2 is trivial. So the subgroup that you get is the circled
 one.

 \begin{exercise}
   Show $SU(2)\times E_7$(compact)$/(-1,-1)$ is a subgroup of $E_8$ (compact).
   Similarly, show that $SU(9)/(\ZZ/3\ZZ)$ is also. These are similar to the example
   above.
 \end{exercise}

 \subsektion{Construction of \texorpdfstring{$E_8$}{E8}} Earlier in the course, we had some constructions:
 \begin{enumerate}
   \item using the Serre relations, but you don't really have an idea of what it looks
   like
   \item Take $D_8$ plus a half spin representation
 \end{enumerate}
 Today, we'll try to find a natural map from root lattices to Lie algebras.
 The idea is as follows: Take a basis element $e^\alpha$ (as a formal symbol)
 for each root $\alpha$; then take the Lie algebra to be the direct sum of
   1 dimensional spaces generated by each $e^\alpha$ and $ L$ ($L$ root lattice
   $\cong$ Cartan subalgebra) . Then we have to define the Lie bracket by setting
   $[e^\alpha,e^\beta]=e^{\alpha+\beta}$, but then we have a sign problem because
   $[e^\alpha,e^\beta]\neq -[e^\beta,e^\alpha]$.  Is there some way to resolve the
   sign problem? The answer is that there is no good way to solve this problem (not
   true, but whatever). Suppose we had a nice functor from root lattices to Lie
   algebras. Then we would get that the automorphism group of the lattice has to be
   contained in the automorphism group of the Lie algebra (which is contained in the
   Lie group), and the automorphism group of the Lattice contains the Weyl group of
   the lattice. But the Weyl group is NOT usually a subgroup of the Lie group.

 We can see this going wrong even in the case of $\sl_2(\RR)$. Remember that the Weyl
 group is $N(T)/T$ where $T=\matrix a00{a^{-1}}$ and $N(T)=T\cup \matrix
 0b{-b^{-1}}0$, and this second part is stuff having order 4, so you cannot possibly
 write this as a semi-direct product of $T$ and the Weyl group.

 So the Weyl group is not usually a subgroup of $N(T)$. The best we can do is to find
 a group of the form $2^n\cdot W\subseteq N(T)$ where $n$ is the rank. For example,
 let's do it for $SL(n+1,\RR)$ Then $T = diag(a_1,\dots, a_n)$ with $a_1\cdots a_n=1$.
 Then we take the normalizer of the torus to be $N(T)=$all permutation matrices with
 $\pm 1$'s with determinant 1, so this is $2^n\cdot S_n$, and it does not split. The
 problem we had with signs can be traced back to the fact that this group doesn't
 split.

 We can construct the Lie algebra from something acted on by $2^n\cdot W$ (but not
 from something acted on by $W$). We take a CENTRAL EXTENSION\index{central extension}
 of the lattice by a group of order 2. Notation is a pain because the lattice is
 written additively and the extension is nonabelian, so you want it to be written
 multiplicatively. Write elements of the lattice in the form $e^\alpha$ formally, so
 we have converted the lattice operation to multiplication. We will use the central
 extension
 \[
    1\to \pm 1 \to \hat e^L\to \underbrace{e^L}_{\cong L}\to 1
 \]
 We want $\hat e^L$ to have the property that $\hat e^\alpha \hat e^\beta =
 (-1)^{(\alpha,\beta)} \hat e^\beta \hat e^\alpha$, where $\hat e^\alpha$ is something
 mapping to $e^\alpha$. What do the automorphisms of $\hat e^L$ look like? We get
 \[
    1\to \underbrace{(L/2L)}_{(\ZZ/2)^{\mathrm{rank}(L)}} \to \aut (\hat e^L) \to \aut (e^L)
 \]
 for $\alpha\in L/2L$, we get the map $\hat e^\beta \to (-1)^{(\alpha,\beta)}\hat
 e^\beta$. The map turns out to be onto, and the group $\aut(e^L)$ contains the
 reflection group of the lattice. This extension is usually non-split.

 Now the Lie algebra is $L\oplus \{$1 dimensional spaces spanned by $(\hat e^\alpha,-\hat
 e^\alpha)\}$ for $\alpha^2=2$ with the convention that $-\hat e^\alpha$ ($-1$ in the
 vector space) is $-\hat e^\alpha$ (-1 in the group $\hat e^L$). Now define a Lie
 bracket by the ``obvious rules'' $[\alpha,\beta]=0$ for $\alpha,\beta \in L$ (the
 Cartan subalgebra is abelian), $[\alpha,\hat e^\beta] = (\alpha,\beta)\hat e^\beta$ ($\hat
 e^\beta$ is in the root space of $\beta$), and $[\hat e^\alpha,\hat e^\beta]=0$ if
 $(\alpha,\beta)\ge 0$ (since $(\alpha+\beta)^2>2$), $[\hat e^\alpha,\hat e^\beta]
 = \hat e^\alpha \hat e^\beta$ if $(\alpha,\beta)<0$ (product in the group $\hat e^L$), and $[\hat
 e^\alpha,(\hat e^\alpha)^{-1}]=\alpha$.

 \begin{theorem}
   Assume $L$ is positive definite. Then this Lie bracket forms a Lie algebra (so it
   is skew and satisfies Jacobi).
 \end{theorem}
 \begin{proof}
   Easy but tiresome, because there are a lot of cases; let's do them (or most of
   them).

   We check the Jacobi identity: We want $[[a,b],c]+[[b,c],a]+[[c,a],b]=0$
   \begin{enumerate}
     \item all of $a,b,c$ in $L$. Trivial because all brackets are zero.
     \item two of $a,b,c$ in $L$. Say $\alpha,\beta,e^\gamma$
     \[
        \underbrace{[[\alpha,\beta],e^\gamma]}_0+\underbrace{[[\beta,e^\gamma],\alpha]}_{(\beta,\alpha)(-\alpha,\beta)e^\gamma}+[[e^\gamma,\alpha],\beta]
     \]
     and similar for the third term, giving a sum of 0.

     \item one of $a,b,c$ in $L$. $\alpha,e^\beta,e^\gamma$. $e^\beta$ has weight
     $\beta$ and $e^\gamma$ has weight $\gamma$ and $e^\beta e^\gamma$ has weight
     $\beta+\gamma$. So check the cases, and you get Jacobi:
\begin{align*}
  [[\alpha,e^\beta],e^\gamma] &= (\alpha,\beta)[e^\beta,e^\gamma] \\
  [ [e^\beta,e^\gamma],\alpha] &= -[\alpha,[e^\beta,e^\gamma]] =
  -(\alpha,\beta+\gamma)[e^\beta,e^\gamma] \\
  [ [e^\gamma,\alpha],e^\beta] &= -[ [\alpha,e^\gamma],e^\beta] =
  (\alpha,\gamma)[e^\beta,e^\gamma],
\end{align*}
  so the sum is zero.

     \item none of $a,b,c$ in $L$. This is the really tiresome one,
     $e^\alpha,e^\beta,e^\gamma$. The main point of going through this is to show that
     it isn't as tiresome as you might think. You can reduce it to two or three cases.
     Let's make our cases depending on $(\alpha,\beta)$, $(\alpha,\gamma)$,
     $(\beta,\gamma)$.
     \begin{enumerate}
       \item if 2 of these are 0, then all the $[[\ast,\ast],\ast]$ are zero.

       \item $\alpha=-\beta$. By case a, $\gamma$ cannot be orthogonal to them, so say
       $(\alpha,\gamma)=1$ $(\gamma,\beta)=-1$; adjust so that $e^\alpha e^\beta=1$,
       then calculate
\begin{align*}
  [ [e^\gamma,e^\beta],e^\alpha] - [ [e^\alpha,e^\beta],e^\gamma] + [
  [e^\alpha,e^\gamma],e^\beta]
  &= e^\alpha e^\beta e^\gamma - (\alpha,\gamma)e^\gamma + 0\\
  &= e^\gamma - e^\gamma = 0.
\end{align*}

       \item $\alpha=-\beta=\gamma$, easy because $[e^\alpha,e^\gamma]=0$ and
       $[[e^\alpha,e^\beta],e^\gamma] = -[ [e^\gamma,e^\beta],e^\alpha] $

       \item We have that each of the inner products is 1, 0 or $-1$. If some
        $(\alpha,\beta)=1$, all brackets are 0.
     \end{enumerate}
     This leaves two cases, which we'll do next time
   \end{enumerate}

 \end{proof}
}{   % Santiago Canez, scanez@math
  \stepcounter{lecture}
 \setcounter{lecture}{27}
 \sektion{Lecture 27}

 Last week we talked about $\hat e^L$, which was a double cover of $e^L$. $L$ is the
 root lattice of $E_8$. We had the sequence
 \[
    1\to \pm 1\to \hat e^L\to e^L\to 1.
 \]
 The Lie algebra structure on $\hat e^L$ was given by
 \begin{align*}
 [\alpha, \beta]     &= 0\\
 [\alpha, e^\beta]   &= (\alpha, \beta) e^\beta \\
 [e^\alpha, e^\beta] &= \begin{cases}
                       0 & \text{if $(\alpha, \beta) \ge 0$}\\
                       e^\alpha e^\beta & \text{if $(\alpha, \beta) = -1$}\\
                       \alpha & \text{if $(\alpha, \beta) = -2$}
                     \end{cases}
 \end{align*}
 The Lie algebra is $L\oplus \bigoplus_{\alpha^2=2} \hat e^\alpha$.

 Let's finish checking the Jacobi identity. We had two cases left:
 \[
 [[e^\alpha,e^\beta],e^\gamma] + [[e^\beta,e^\gamma],e^\alpha] + [[e^\gamma,e^\alpha],e^\beta]=0
 \]
 \begin{itemize}
   \item[$-$] $(\alpha,\beta)=(\beta,\gamma)=(\gamma,\alpha)=-1$, in which case
   $\alpha+\beta+\gamma=0$. then $[[e^\alpha,e^\beta],e^\gamma] = [e^\alpha
   e^\beta,e^\gamma] = \alpha+\beta$. By symmetry, the other two terms are
   $\beta+\gamma$ and $\gamma+\alpha$;the sum of all three terms is
   $2(\alpha+\beta+\gamma)=0$.

   \item[$-$] $(\alpha,\beta)=(\beta,\gamma)=-1$, $(\alpha,\gamma)=0$, in which case
   $[e^\alpha,e^\gamma]=0$. We check that $[[e^\alpha,e^\beta],e^\alpha]=[e^\alpha
   e^\beta, e^\gamma] = e^\alpha e^\beta e^\gamma$ (since
   $(\alpha+\beta,\gamma)=-1$).
   Similarly, we have $[[e^\beta,
   e^\gamma],e^\alpha] = [e^\beta e^\gamma,e^\alpha] = e^\beta e^\gamma e^\alpha$.
   We notice that $e^\alpha e^\beta = -e^\beta e^\alpha$
   and $e^\gamma e^\alpha = e^\alpha e^\gamma$ so
   $e^\alpha e^\beta e^\gamma = -e^\beta e^\gamma e^\alpha$; again, the sum
   of all three terms in the Jacobi identity is 0.
 \end{itemize}
 This concludes the verification of the Jacobi identity, so we have a Lie algebra.

 Is there a proof avoiding case-by-case check? Good news: yes! Bad news: it's actually
 more work. We really have functors as follows:
 \[\xymatrix @R=3.75em @C=3.75em{
   \txt{Dynkin\\ diagrams} \ar[r] \ar[dr] & \txt{Double\\ cover $\hat L$} \ar[d]
   \ar[rr]^{\mbox{\scriptsize\txt{elementary,\\ but tedious}}}_{\mbox{\scriptsize\txt{only for positive\\ definite lattices}}}
   \ar@/_1.25em/[rrd]_{}="a"
   & & \text{Lie algebras}\\
   & \txt{Root lattice $L$} & & \text{Vertex algebras} \ar[u]^{}="b"\\
   & & & {}\save[]*\txt<6pc>{these work\\ for any\\ \emph{even} lattice} \ar@/^1.25em/
 "a" \ar@/_4.75em/ "b" \restore }
 \]
 where $\hat L$ is generated by $\hat e^{\alpha_i}$
 (the $i$'s are the dots in your Dynkin diagram), with $\hat e^{\alpha_i}\hat
 e^{\alpha_j}=(-1)^{(\alpha_i,\alpha_j)}\hat e^{\alpha_j}\hat e^{\alpha_i}$, and $-1$
 is central of order 2.

 Unfortunately, you have to spend several weeks learning vertex algebras. In fact, the
 construction we did was the vertex algebra approach, with all the vertex algebras
 removed. So there is a more general construction which gives a much larger class of
 infinite dimensional Lie algebras.

 Now we should study the double cover $\hat L$, and in particular prove its existence.
 Given a Dynkin diagram, we can construct $\hat L$ as generated by the elements
 $e^{\alpha_i}$ for $\alpha_i$ simple roots with the given relations. It is easy to
 check that we get a surjective homomorphism $\hat L \to L$ with kernel
 generated by $z$ with $z^2=1$.  What's a little harder to show is that
 $z\neq 1$ (i.e., show that $\hat L\neq L$).
 The easiest way to do it is to use cohomology of groups, but since we have such an
 explicit case, we'll do it bare hands:\\
 Problem: Given $Z$, $H$ groups with $Z$ abelian, construct central extensions
 \[
    1\to Z\to G\to H\to 1
 \]
 (where $Z$ lands in the center of $G$). Let $G$ be the set of pairs $(z,h)$, and set
 the product $(z_1,h_1)(z_2,h_2) = (z_1z_2 c(h_1,h_2),h_1h_2)$, where $c(h_1,h_2)\in
 Z$ ($c(h_1,h_2)$ will be a cocycle in group cohomology). We obviously get a
 homomorphism by mapping $(z,h)\mapsto h$. If $c(1,h)=c(h,1)=1$ (normalization), then
 $z\mapsto (z,1)$ is a homomorphism mapping $Z$ to the center of $G$. In particular,
 $(1,1)$ is the identity. We'll leave it as an exercise to figure out what
 the inverses are. When is this thing \emph{associative}?
 Let's just write everything out:
 \begin{align*}
   \big( (z_1,h_1)(z_2,h_2)\big)(z_3,h_3) &= (z_1z_2z_3 c(h_1,h_2)c(h_1h_2,h_3),
   h_1h_2h_3)\\
   (z_1,h_1)\big( (z_2,h_2)(z_3,h_3)\big) &= (z_1z_2z_3 c(h_1,h_2h_3)c(h_2,h_3),
   h_1h_2h_3)\\
 \end{align*}
 so we must have
 \[
    c(h_1,h_2)c(h_1h_2,h_3) = c(h_1h_2,h_3)c(h_2,h_3).
 \]
 This identity is actually very easy to satisfy in one particular case: when $c$ is
 bimultiplicative: $c(h_1,h_2h_3)=c(h_1,h_2)c(h_1,h_3)$ and
 $c(h_1h_2,h_3)=c(h_1,h_3)c(h_2,h_3)$. That is, we have a map $H\times H\to Z$. Not
 all cocycles come from such maps, but this is the case we care about.

 To construct the double cover, let $Z=\pm 1$ and $H=L$ (free abelian). If we write
 $H$ additively, we want $c$ to be a bilinear map $L\times L \to \pm 1$. It is really
 easy to construct bilinear maps on free abelian groups. Just take any basis
 $\alpha_1,\dots, \alpha_n$ of $L$, choose $c(\alpha_1,\alpha_j)$ arbitrarily for each $i,j$
 and extend $c$ via bilinearity to $L\times L$. In our case, we want to find a double
 cover $\hat L$ satisfying $\hat e^\alpha \hat e^\beta = (-1)^{(\alpha,\beta)} \hat
 e^\beta \hat e^\alpha$ where $\hat e^\alpha$ is a lift of $e^\alpha$. This just means
 that $c(\alpha,\beta) = (-1)^{(\alpha,\beta)} c(\beta,\alpha)$. To satisfy this, just
 choose $c(\alpha_i,\alpha_j)$ on the basis $\{\alpha_i\}$ so that
 $c(\alpha_i,\alpha_j) = (-1)^{(\alpha_i,\alpha_j)} c(\alpha_j,\alpha_i)$. This is
 trivial to do as $(-1)^{(\alpha_i,\alpha_i)}=1$. Notice that this uses the fact that
 the lattice is even. There is no canonical way to choose this 2-cocycle (otherwise,
 the central extension would split as a product), but all the different double covers
 are isomorphic because we can specify $\hat L$ by generators and relations. Thus, we
 have constructed $\hat L$ (or rather, verified that the kernel of $\hat L \to L$ has
 order 2, not 1).

 Let's now look at lifts of automorphisms of $L$ to $\hat L$.
 \begin{exercise}
 Any automorphism of $L$ preserving $(\ ,\,)$ lifts to an automorphism
 of $\hat L$
 \end{exercise}
 There are two special cases:
 \begin{enumerate}
   \item $-1$ is an automorphism of $L$, and we want to lift it to $\hat L$
   explicitly. First attempt: try sending $\hat e^\alpha$ to $\hat e^{-\alpha}:=(\hat
   e^\alpha)^{-1}$, which doesn't work because $a\mapsto a^{-1}$ is not an
   automorphism on non-abelian groups.

   Better: $\w: \hat e^\alpha \mapsto (-1)^{\alpha^2/2}(\hat e^\alpha)^{-1}$ is an
   automorphism of $\hat L$. To see this, check
   \begin{align*}
     \w(\hat e^\alpha) \w(\hat e^\beta) &= (-1)^{(\alpha^2+\beta^2)/2}(\hat
     e^\alpha)^{-1}(\hat e^\beta)^{-1}\\
     \w(\hat e^\alpha \hat e^\beta) &= (-1)^{(\alpha+\beta)^2/2} (\hat e^\beta)^{-1}
     (\hat e^\alpha)^{-1}
   \end{align*}
   which work out just right

   \item If $r^2=2$, then $\alpha\mapsto \alpha - (\alpha,r)r$ is an automorphism of
   $L$ (reflection through $r^\perp$). You can lift this by $\hat e^\alpha \mapsto
   \hat e^\alpha (\hat e^r)^{-(\alpha,r)} \times (-1)^{\binom{(\alpha,r)}{2}}$. This
   is a homomorphism (check it!) of order (usually) 4!
   \begin{remark}
     Although automorphisms of $L$ lift to automorphisms of $\hat L$, the lift might
     have larger order.
   \end{remark}
 \end{enumerate}

 This construction works for the root lattices of $A_n$, $D_n$, $E_6$, $E_7$, and
 $E_8$; these are the lattices which are even, positive definite, and generated by
 vectors of norm 2 (in fact, all such lattices are sums of the given ones). What about
 $B_n$, $C_n$, $F_4$ and $G_2$? The reason the construction doesn't work for these
 cases is because there are roots of different lengths. These all occur as fixed
 points of diagram automorphisms of $A_n$, $D_n$ and $E_6$. In fact, we have a
 \emph{functor}
 from Dynkin diagrams to Lie algebras, so and automorphism of the diagram gives an
 automorphism of the algebra
 \[\begin{tabular}{cc|cc}
  Involution & Fixed points & Involution & Fixed Points \\
  \begin{xy}<1.75em,0em>:
     (0,0)="1" *\cir<2pt>{};
     p+(1,0)="2"  *\cir<2pt>{} **@{-};
     p+(.6,0) **@{-};
     p+(.8,0) **{\hspace{.7pt}.\hspace{.7pt}};
     p+(.6,0)="22" *\cir<2pt>{} **@{-};
     p+(1,0)="11" *\cir<2pt>{} **@{-};
     "1" *+{\ };"11" *+{\ } **\crv{(2,1.5)} ?<*@{<} ?>*@{>},
     "2" *+{\ };"22" *+{\ } **\crv{(2,1)} ?<*@{<} ?>*@{>},
     (2,-.5) *{=A_{2n+1}}
   \end{xy} &
   \begin{xy}<1.75em,0em>:
   (0,2) *\cir<2pt>{};
   p-(0,1) *\cir<2pt>{} **@{-};
   p-(0,.5) **@{-};
   p-(0,.6) **+<-3.4pt,2pt>{.};
   p-(0,.5) *\cir<2pt>{} **@{-};
   p-(0,1) *\cir<2pt>{} **@{=}?(.5)*@{>};
  \end{xy} $= C_{n+1}$ &
  \begin{xy}<1.75em,0em>:
     (0,0) *\cir<2pt>{};
     a(60)="1" *\cir<2pt>{} **@{-},
     a(180)="2" *++!R{D_4=} *\cir<2pt>{} **@{-},
     a(-60)="3" *\cir<2pt>{} **@{-},
     \ar@/_1.4ex/ "1" *+{\ };"2" *+{\ }
     \ar@/_1.4ex/ "2" *+{\ };"3" *+{\ }
     \ar@/_1.4ex/ "3" *+{\ };"1" *+{\ }
   \end{xy} &
   \begin{xy}
   (.5,0) *++!U{=G_2};
   (0,0)="1" *\cir<2pt>{};
   (1,0)="2" *\cir<2pt>{} **@{-}?*@{>},
   \ar@{-} "1" *{\hspace{3pt}};"2" *{\hspace{3pt}} <1.5pt>
   \ar@{-} "1" *{\hspace{3pt}};"2" *{\hspace{3pt}} <-1.5pt>
   \end{xy}\\
   \begin{xy}<1.75em,0em>:
   (0,0) *++!R{D_n=} *\cir<2pt>{};
   p+(.5,0) **@{-};
   p+(.6,0) **{.};
   p+(.5,0)  *\cir<2pt>{} **@{-};
   p+a(45)="a"  *\cir<2pt>{} **@{-},
   p+a(-45)="b"  *\cir<2pt>{} **@{-};
   \ar@{<->}@/^/ "a" *+{\,};"b" *+{\,}
 \end{xy} &
 \begin{xy}<1.75em,0em>:
   (0,0) *\cir<2pt>{};
%   (1,0) *\cir<2pt>{} **@{-};
   p+(.5,0) **@{-};
   p+(.6,0) **{.};
   p+(.5,0) *\cir<2pt>{} **@{-};
   p+(1,0) *\cir<2pt>{} **@{=}?(.5)*@{<};
   (1.3,0) *++!U{=B_n}
  \end{xy} &
  \begin{xy}<1.75em,0em>:
   (0,.8) *{E_6=};
   (0,0)="1" *\cir<2pt>{};
   (1,0)="2"  *\cir<2pt>{} **@{-};
   p+(1,0) *\cir<2pt>{} **@{-};
   p+(0,-1) *\cir<2pt>{} **@{-},
   p+(1,0)="22" *\cir<2pt>{} **@{-};
   p+(1,0)="11" *\cir<2pt>{} **@{-};
   "1" *+{\ };"11" *+{\ } **\crv{(2,1.5)} ?<*@{<} ?>*@{>},
   "2" *+{\ };"22" *+{\ } **\crv{(2,1)} ?<*@{<} ?>*@{>},
 \end{xy} &
 \begin{xy} <1.75em,0em>:
   (0,-1.5) *\cir<2pt>{};
   p+(0,1)  *\cir<2pt>{} **@{-};
   p+(0,1) *\cir<2pt>{} **@{=};
   p+(0,1)  *\cir<2pt>{} **@{-};
 \end{xy} $= F_4$
 \end{tabular}\]

 $A_{2n}$ doesn't really give you a new algebra: it corresponds to some
 superalgebra stuff.

 \subsektion{Construction of the Lie group of \texorpdfstring{$E_8$}{E8}} It is the
 group of automorphisms of the Lie algebra generated by the elements $\exp(\lambda
 Ad(\hat e^\alpha))$, where $\lambda$ is some real number, $\hat e^\alpha$ is one of
 the basis elements of the Lie algebra corresponding to the root $\alpha$, and
 $Ad(\hat e^\alpha)(a) = [\hat e^\alpha, a]$. In other words,
 \begin{align*}
   \exp(\lambda Ad(\hat e^\alpha))(a) &= 1+ \lambda [\hat e^\alpha, a] +
   \frac{\lambda^2}{2} [\hat e^\alpha, [\hat e^\alpha, a]].
 \end{align*}
 and all the higher terms are zero. To see that $Ad(\hat e^\alpha)^3 = 0$, note that
 if $\beta$ is a root, then $\beta+3\alpha$ is not a root (or 0).
 \begin{warning}
   In general, the group generated by these automorphisms is NOT the whole
   automorphism group of the Lie algebra. There might be extra diagram automorphisms,
   for example.
 \end{warning}

 We get some other things from this construction. We can get simple groups over finite
 fields: note that the construction of a Lie algebra above works over any commutative
 ring (e.g.\ over $\ZZ$). The only place we used division is in $\exp(\lambda Ad(\hat
 e^\alpha))$ (where we divided by 2). The only time this term is non-zero is when we
 apply $\exp (\lambda Ad(\hat e^\alpha))$ to $\hat e^{-\alpha}$, in which case we find
 that $[\hat e^\alpha,[\hat e^\alpha,\hat e^{-\alpha}]]= [\hat e^\alpha, \alpha] =
 -(\alpha,\alpha) \hat e^\alpha$, and the fact that $(\alpha,\alpha)=2$ cancels the
 division by 2. So we can in fact construct the $E_8$ group over \emph{any} commutative ring.
 You can mumble something about group schemes over $\ZZ$ at this point. In particular,
 we have groups of type $E_8$ over \emph{finite fields}, which are actually finite
 simple groups (these are called Chevalley groups; it takes work to show that they are
 simple, there is a book by Carter called \textsl{Finite Simple Groups} which you
 can look at).

 \subsektion{Real forms}So we've constructed the Lie group and Lie algebra of type
 $E_8$. There are in fact several \emph{different} groups of type $E_8$. There is one
 \emph{complex} Lie algebra of type $E_8$, which corresponds to several different real
 Lie algebras of type $E_8$.

 Let's look at some smaller groups:
 \begin{example}
   $\sl_2(\RR) = \matrix abcd$ with $a,b,c,d$ real $a+d=0$; this is not compact.
   On the other hand, $\mathfrak{su}_2(\RR) = \matrix abcd$ with $d=-a$ imaginary $b=-\bar c$, is compact.
   These have the same Lie algebra over $\CC$.
 \end{example}

 Let's look at what happens for $E_8$. In general, suppose $L$ is a Lie algebra with
 complexification $L\otimes \CC$. How can we find another Lie algebra $M$ with the
 same complexification? $L\otimes \CC$ has an anti-linear involution $\w_L: l\otimes
 z\mapsto l\otimes \bar z$. Similarly, it has an anti-linear involution $\w_M$. Notice
 that $\w_L\w_M$ is a linear involution of $L\otimes \CC$. Conversely, if we know this
 involution, we can reconstruct $M$ from it. Given an involution $\w$ of $L\otimes
 \CC$, we can get $M$ as the fixed points of the map $a\mapsto \w_L \w(a)$``=''
 $\overline{\w(a)}$. Another way is to put $L=L^+\oplus L^-$, which are the $+1$ and
 $-1$ eigenspaces, then $M=L^+\oplus iL^-$.

 Thus, to find other real forms, we have to study the involutions of the
 complexification of $L$. The exact relation is kind of subtle, but this is a good way
 to go.

 \begin{example}
   Let $L=\sl_2(\RR)$. It has an involution $\w(m) = -m^T$. $\mathfrak{su}_2(\RR)$ is
   the set of fixed points of the involution $\w$ times complex conjugation on
   $\sl_2(\CC)$, by definition.
 \end{example}

 So to construct real forms of $E_8$, we want some involutions of the Lie algebra
 $E_8$ which we constructed. What involutions do we know about? There are two obvious
 ways to construct involutions:
 \begin{enumerate}
   \item Lift $-1$ on $L$ to $\hat e^\alpha \mapsto (-1)^{\alpha^2/2}(\hat
   e^\alpha)^{-1}$, which induces an involution on the Lie algebra.

   \item Take $\beta\in L/2L$, and look at the involution $\hat e^\alpha\mapsto
   (-1)^{(\alpha,\beta)} \hat e^\alpha$.
 \end{enumerate}
 (2) gives nothing new ... you get the Lie algebra you started with. (1) only gives
 you one real form. To get all real forms, you multiply these two kinds of involutions
 together.

 Recall that $L/2L$ has 3 orbits under the action of the Weyl group, of size 1, 120,
 and 135. These will correspond to the three real forms of $E_8$. How do we
 distinguish different real forms? The answer was found by Cartan: look at the
 signature of an invariant quadratic form on the Lie algebra!

 A bilinear form $(\ ,\,)$ on a Lie algebra is called \emph{invariant} if
 $([a,b],c)+(b[a,c])=0$ for all $a,b,c$. This is called invariant because it
 corresponds to the form being invariant under the corresponding group action. Now we
 can construct an invariant bilinear form on $E_8$ as follows:
 \begin{enumerate}
   \item $(\alpha,\beta)_\text{in the Lie algebra} = (\alpha,\beta)_\text{in the lattice}$
   \item $(\hat e^\alpha,(\hat e^\alpha)^{-1}) = 1$
   \item $(a,b)=0$ if $a$ and $b$ are in root spaces $\alpha$ and $\beta$ with
   $\alpha+\beta \neq 0$.
 \end{enumerate}
 This gives an invariant inner product on $E_8$, which you prove by case-by-case check
 \begin{exercise}
   do these checks
 \end{exercise}

 Next time, we'll use this to produce bilinear forms on all the real forms and then
 we'll calculate the signatures.
}{   % Martin Vito-Cruz, vitocruz@math
  \stepcounter{lecture}
 \setcounter{lecture}{28}
 \sektion{Lecture 28}

 Last time, we constructed a Lie algebra of type $E_8$, which was $L\oplus \bigoplus
 \hat e^\alpha$, where $L$ is the root lattice and $\alpha^2=2$. This gives a double
 cover of the root lattice:
 \[
    1\to \pm 1\to \hat e^L\to e^L\to 1.
 \]
 We had a lift for $\w(\alpha)=-\alpha$, given by $\w(\hat
 e^\alpha)=(-1)^{(\alpha^2/2)}(\hat e^\alpha)^{-1}$. So $\w$ becomes an automorphism of
 order 2 on the Lie algebra. $e^\alpha\mapsto (-1)^{(\alpha,\beta)} e^\alpha$
 is also an automorphism of the Lie algebra.

 Suppose $\sigma$ is an automorphism of order 2 of the real Lie algebra $L=L^+ +L^-$
 (eigenspaces of $\sigma$). We saw that you can construct another real form given by $L^+
 +iL^-$. Thus, we have a map from conjugacy classes of automorphisms with $\sigma^2=1$
 to real forms of $L$. This is not in general in isomorphism.

 Today we'll construct some more real forms of $E_8$. $E_8$ has an invariant symmetric
 bilinear form $(e^\alpha,(e^\alpha)^{-1})=1$, $(\alpha,\beta)=(\beta,\alpha)$. The
 form is unique up to multiplication by a constant since $E_8$ is an irreducible
 representation of $E_8$. So the \emph{absolute value of the signature} is an
 invariant of the Lie algebra.

 For the split form of $E_8$, what is the signature of the invariant bilinear form
 (the split form is the one we just constructed)? On the Cartan subalgebra $L$, $(\
 ,\,)$ is positive definite, so we get $+8$ contribution to the signature. On
 $\{e^\alpha,(e^\alpha)^{-1}\}$, the form is $\matrix 0110$, so it has signature
 $0\cdot 120$. Thus, the signature is 8. So if we find any real form with a different
 signature, we'll have found a new Lie algebra.

 Let's first try involutions $e^\alpha\mapsto (-1)^{(\alpha,\beta)}e^\alpha$. But this
 doesn't change the signature. $L$ is still positive definite, and you still have
 $\matrix 0110$ or $\matrix 0{-1}{-1}0$ on the other parts. These Lie algebras
 actually turn out to be isomorphic to what we started with (though we haven't shown
 that they are isomorphic).

 Now try $\w:e^\alpha\mapsto (-1)^{\alpha^2/2}(e^\alpha)^{-1}$, $\alpha\mapsto
 -\alpha$. What is the signature of the form? Let's write down the $+$ and $-$
 eigenspaces of $\w$. The $+$ eigenspace will be spanned by $e^\alpha - e^{-\alpha}$,
 and these vectors have norm $-2$ and are orthogonal. The $-$ eigenspace will be
 spanned by $e^\alpha + e^{-\alpha}$ and $L$, which have norm 2 and are orthogonal,
 and $L$ is positive definite. What is the Lie algebra corresponding to the involution
 $\w$? It will be spanned by $e^\alpha - e^{-\alpha}$ where $\alpha^2=2$ (norm $-2$), and
 $i(e^\alpha + e^{-\alpha})$ (norm $-2$), and $iL$ (which is now negative definite).
 So the bilinear form is \emph{negative definite}, with signature $-248 (\neq \pm 8)$.

 With some more work, you can actually show that this is the Lie algebra of the
 \emph{compact} form of $E_8$. This is because the automorphism group of $E_8$ preserves the
 invariant bilinear form, so it is contained in $O_{0,248}(\RR)$, which is compact.

 Now let's look at involutions of the form $e^\alpha\mapsto
 (-1)^{(\alpha,\beta)}\w(e^\alpha)$. Notice that $\w$ commutes with $e^\alpha\mapsto
 (-1)^{(\alpha,\beta)}e^\alpha$. The $\beta$'s in $(\alpha,\beta)$ correspond to
 $L/2L$ modulo the action of the Weyl group $W(E_8)$. Remember this has three orbits,
 with 1 norm 0 vector, 120 norm 2 vectors, and 135 norm 4 vectors. The norm 0 vector
 gives us the compact form. Let's look at the other cases and see what we get.

 Suppose $V$ has a negative definite symmetric inner product $(\ ,\,)$, and suppose
 $\sigma$ is an involution of $V=V_+\oplus V_-$ (eigenspaces of $\sigma$). What is the
 signature of the invariant inner product on $V_+\oplus iV_-$? On $V_+$, it is
 negative definite, and on $iV_-$ it is positive definite. Thus, the signature is
 $\dim V_- - \dim V_+= -\mathrm{tr}(\sigma)$. So we want to work out the traces of these
 involutions.

 Given some $\beta \in L/2L$, what is $\mathrm{tr}(e^\alpha\mapsto
 (-1)^{(\alpha,\beta)}e^\alpha)$? If $\beta =0$, the traces is obviously 248 because
 we just have the identity map. If $\beta^2=2$, we need to figure how many roots have
 a given inner product with $\beta$. Recall that this was determined before:
 \begin{center}
 \begin{tabular}{|c|c|c|}
 \hline
 $(\alpha,\beta)$ & \# of roots $\alpha$ with given inner product & eigenvalue\\
 \hline
 2                & 1                                        & 1  \\
 1                & 56                                       & -1 \\
 0                & 126                                      & 1  \\
 -1               & 56                                       & -1 \\
 -2               & 1                                        & 1  \\
 \hline
 \end{tabular}
 \end{center}
 Thus, the trace is $1-56+126-56+1+8=24$ (the $8$
 is from the Cartan subalgebra). So the signature of the corresponding form on the Lie
 algebra is $-24$. We've found a third Lie algebra.

 If we also look at the case when $\beta^2=4$, what happens? How many $\alpha$ with
 $\alpha^2=2$ and with given $(\alpha,\beta)$ are there?  In this case,
 we have:
 \begin{center}
 \begin{tabular}{|c|c|c|}
 \hline
 $(\alpha,\beta)$ & \# of roots $\alpha$ with given inner product & eigenvalue\\
 \hline
 2                & 14                                       & 1  \\
 1                & 64                                       & -1 \\
 0                & 84                                       & 1  \\
 -1               & 64                                       & -1 \\
 -2               & 14                                       & 1  \\
 \hline
 \end{tabular}
 \end{center}
 The trace will be $14-64+84-64+14+8=-8$. This is just the split form again.

 Summary: We've found $3$ forms of $E_8$, corresponding to 3 classes in $L/2L$, with
 signatures 8, $-24$, $-248$, corresponding to involutions $e^\alpha\mapsto
 (-1)^{(\alpha,\beta)}e^{-\alpha}$ of the \emph{compact} form. If $L$ is the
 \emph{compact} form of a simple Lie algebra, then Cartan\index{Cartan} showed that
 the other forms correspond exactly to the conjugacy classes of involutions in the
 automorphism group of $L$ (this doesn't work if you don't start with the compact form
 --- so always start with the compact form).

 In fact, these three are the \emph{only} forms of $E_8$, but we won't prove that.

 \subsektion{Working with simple Lie groups}
 As an example of how to work with simple Lie groups, we will look at the
 general question: Given a simple Lie group, what is its homotopy type?
 Answer: $G$ has a unique conjugacy class of maximal compact subgroups $K$, and $G$ is
 homotopy equivalent to $K$.
 \begin{proof}[Proof for $GL_n(\RR)$]
   First pretend $GL_n(\RR)$ is simple, even though it isn't; whatever. There is an
   obvious compact subgroup: $O_n(\RR)$.  Suppose $K$ is \emph{any} compact subgroup of
   $GL_n(\RR)$. Choose any positive definite form $(\ ,\,)$ on $\RR^n$. This will
   probably not be invariant under $K$, but since $K$ is compact, we can average it
   over $K$ get one that is: define a new form
   $(a,b)_{\mathrm{new}} = \int_K (ka,kb)\, dk$. This gives an
   invariant positive definite bilinear form (since integral of something
   positive definite is
   positive definite). Thus, any compact subgroup preserves some positive definite
   form. But the subgroup fixing some positive definite bilinear form is conjugate to
   a subgroup of $O_n(\RR)$ (to see this, diagonalize the form). So $K$ is contained
   in a conjugate of $O_n(\RR)$.

   Next we want to show that $G=GL_n(\RR)$ is homotopy equivalent to $O_n(\RR)=K$. We
   will show that $G=KAN$, where $K$ is $O_n$, $A$ is all diagonal matrices with
   positive coefficients, and $N$ is matrices which are upper triangular with 1s on
   the diagonal. This is the \emph{Iwasawa decomposition}. In general, we get $K$ compact,
   $A$ semisimple abelian, and $N$ is unipotent. The proof of this you saw before was
   called the Grahm-Schmidt process for orthonormalizing a basis. Suppose $v_1,\dots,
   v_n$ is any basis for $\RR^n$.
   \begin{enumerate}
     \item Make it orthogonal by subtracting some stuff, you'll get $v_1$,
     $v_2-\ast v_1$, $v_3 - \ast v_2 - \ast v_1$, $\dots$.
     \item Normalize by multiplying each basis vector so that it has norm 1. Now we
     have an orthonormal basis.
   \end{enumerate}
   This is just another way to say that $GL_n$ can be written as $KAN$. Making things
   orthogonal is just multiplying by something in $N$, and normalizing is just
   multiplication by some diagonal matrix with positive entries. An orthonormal basis
   is an element of $O_n$. Tada! This decomposition is just a topological one, not
   a decomposition as groups. Uniqueness is easy to check.

   Now we can get at the homotopy type of $GL_n$. $N\cong \RR^{n(n-1)/2}$, and $A\cong
   (\RR^+)^n$, which are contractible. Thus, $GL_n(\RR)$ has the same homotopy type as
   $O_n(\RR)$, its maximal compact subgroup.
 \end{proof}
 If you wanted to know $\pi_1(GL_3(\RR))$, you could calculate $\pi_1(O_3(\RR))\cong
 \ZZ/2\ZZ$, so $GL_3(\RR)$ has a double cover. Nobody has shown you this double cover
 because it is \emph{not algebraic}.

 \begin{example}
  Let's go back to various forms of $E_8$ and figure out (guess) the fundamental
  groups. We need to know the maximal compact subgroups.

  \begin{enumerate}
  \item One of them is easy: the
  compact form is its own maximal compact subgroup. What is the fundamental group?
  Remember or quote the fact that for compact simple groups, $\pi_1\cong \frac{\text{weight
  lattice}}{\text{root lattice}}$, which is 1. So this form is simply connected.

  \item $\beta^2=2$ case (signature $-24$).
  Recall that there were 1, 56, 126, 56, and 1 roots $\alpha$
  with $(\alpha,\beta)=2,1,0,-1$, and -2 respectively,
  and there are another $8$ dimensions for the Cartan subalgebra.
   On the $1,126,1,8$ parts, the form is negative definite. The sum of these root
   spaces gives a Lie algebra of type $E_7 A_1$ with a negative definite bilinear form
   (the $126$ gives you the roots of an $E_7$, and the $1$s are the roots of an
   $A_1$). So it a reasonable guess that the maximal compact subgroup has something to
   do with $E_7A_1$. $E_7$ and $A_1$ are not simply connected: the compact form of
   $E_7$ has $\pi_1$ = $\ZZ/2$ and the compact form of $A_1$ also has $\pi_1 = \ZZ/2$.
   So the universal cover of $E_7A_1$ has center $(\ZZ/2)^2$. Which part of this acts
   trivially on $E_8$? We look at the $E_8$ Lie algebra as a representation of
   $E_7\times A_1$. You can read off how it splits form the picture above: $E_8\cong
   E_7\oplus A_1 \oplus 56 \otimes 2$, where $56$ and $2$ are irreducible, and the
   centers of $E_7$ and $A_1$ both act as $-1$ on them. So the maximal compact
   subgroup of this form of $E_8$ is the simply connected compact form of $E_7\times
   A_1/(-1,-1)$. This means that $\pi_1(E_8)$ is the same as $\pi_1$ of the compact
   subgroup, which is $(\ZZ/2)^2/(-1,-1)\cong \ZZ/2$. So this simple group has a
   nontrivial double cover (which is non-algebraic).

   \item For the other (split) form of $E_8$ with signature 8, the maximal compact
   subgroup is $\spin_{16}(\RR)/(\ZZ/2)$, and $\pi_1(E_8)$ is $\ZZ/2$.
  \end{enumerate}
 You can compute any other homotopy invariants with this method.
 \end{example}

 Let's look at the $56$ dimensional representation of $E_7$ in more detail. We had the
 picture
 \[\begin{tabular}{c|c}
   $(\alpha,\beta)$ & \# of $\alpha$'s\\
   \hline
   2 & 1\\
   1 & 56\\
   0 & 126\\
   -1 & 56\\
   -2 & 1\\
 \end{tabular}\]
 The Lie algebra $E_7$ fixes these 5 spaces of $E_8$ of dimensions $1,56,126+8,56,1$.
 From this we can get some representations of $E_7$. The $126+8$ splits as
 $1+(126+7)$. But we also get a 56 dimensional representation of $E_7$. Let's show
 that this is actually an irreducible representation. Recall that in calculating
 $W(E_8)$, we showed that $W(E_7)$ acts transitively on this set of $56$ roots of
 $E_8$, which can be considered as weights of $E_7$.

 An irreducible representation is called \emph{minuscule} if the Weyl group acts
 transitively on the weights. This kind of representation is particularly easy to work
 with. It is really easy to work out the character for example: just translate the
 1 at the highest weight around, so every weight has multiplicity 1.

 So the 56 dimensional representation of $E_7$ must actually be the irreducible
 representation with whatever highest weight corresponds to one of the vectors.

 \subsektion{Every possible simple Lie group} We will construct them as follows:
  Take an involution $\sigma$ of the compact form $L=L^+ + L^-$ of the Lie
  algebra, and form $L^+ + iL^-$. The way we constructed these was to first construct
  $A_n$, $D_n$, $E_6$, and $E_7$ as for $E_8$. Then construct the involution
  $\w:e^\alpha\mapsto -e^{-\alpha}$. We get $B_n$, $C_n$, $F_4$, and $G_2$ as fixed
  points of the involution $\w$.

 Kac classified all automorphisms of finite order of any compact simple Lie group. The
 method we'll use to classify involutions is extracted from his method. We can
 construct lots of involutions as follows:
 \begin{enumerate}
 \item Take any Dynkin diagram, say $E_8$, and select some of its vertices,
 corresponding to simple roots. Get an involution by taking $e^\alpha\mapsto \pm
 e^\alpha$ where the sign depends on whether $\alpha$ is one of the simple roots we've
 selected. However, this is not a great method. For one thing, you get a lot of
 repeats (recall that there are only 3, and we've found $2^8$ this way).
 \[
 \begin{xy}
   (0,0) *\cir<2pt>{};
   p+(1,0)="a" *\cir<2pt>{} **@{-};
   p+(1,0) *\cir<2pt>{} **@{-};
   p+(1,0)="b" *\cir<2pt>{} **@{-};
   p+(1,0) *\cir<2pt>{} **@{-};
   p+(0,1)="c" *\cir<2pt>{} **@{-},
   p+(1,0) *\cir<2pt>{} **@{-};
   p+(1,0) *\cir<2pt>{} **@{-};
   "a" *\cir<5pt>{} *+++!D{1};
   "b" *\cir<5pt>{} *+++!D{1};
   "c" *\cir<5pt>{} *+++!L{1};
 \end{xy}
 \]
 \item Take any diagram automorphism of order 2, such as
 \[\begin{xy}
   (0,0)="1" *\cir<2pt>{};
   (1,0)="2"  *\cir<2pt>{} **@{-};
   p+(1,0) *\cir<2pt>{} **@{-};
   p+(0,-1) *\cir<2pt>{} **@{-},
   p+(1,0)="22" *\cir<2pt>{} **@{-};
   p+(1,0)="11" *\cir<2pt>{} **@{-};
   "1" *+{\ };"11" *+{\ } **\crv{(2,1.5)} ?<*@{<} ?>*@{>},
   "2" *+{\ };"22" *+{\ } **\crv{(2,1)} ?<*@{<} ?>*@{>},
 \end{xy}\]
 This gives you more involutions.
 \end{enumerate}

 Next time, we'll see how to cut down this set of involutions.
}{   % Martin Vito-Cruz, vitocruz@math
  \stepcounter{lecture}
 \setcounter{lecture}{29}
 \sektion{Lecture 29}

 Split form of Lie algebra (we did this for $A_n$, $D_n$, $E_6$, $E_7$, $E_8$): $A=\bigoplus
 \hat e^\alpha \oplus L$. Compact form $A^++iA^-$, where $A^\pm$ eigenspaces of
 $\w:\hat e^\alpha\mapsto (-1)^{\alpha^2/2}\hat e^{-\alpha}$.

 We talked about other involutions of the compact form. You get all the other forms
 this way.

 The idea now is to find ALL real simple Lie algebras by listing all involutions of
 the compact form. We will construct all of them, but we won't prove that we have all
 of them.

 We'll use Kac's method for classifying all automorphisms of order $N$ of a compact
 Lie algebra (and we'll only use the case $N=2$). First let's look at inner
 automorphisms. Write down the AFFINE Dynkin diagram
 \[\begin{xy}
   (0,0) *+!U{2} *\cir<2pt>{};
   p-(1,0) *+!U{4} *\cir<2pt>{} **@{-};
   p-(1,0) *+!U{6} *\cir<2pt>{} **@{-};
       p+(0,+1) *+!L{3} *\cir<2pt>{} **@{-},
   p-(1,0) *+!U{5} *\cir<2pt>{} **@{-};
   p-(1,0) *+!U{4} *\cir<2pt>{} **@{-};
  p-(1,0) *+!U{3} *\cir<2pt>{} **@{-};
   p-(1,0) *+!U{2} *\cir<2pt>{} **@{-};
   p-(1,0) *+!U{1} *++!R{-\text{highest weight} =} *\cir<2pt>{} **@{-};
 \end{xy}\]
 Choose $n_i$ with $\sum n_im_i=N$ where the $m_i$ are the numbers on the diagram. We
 have an automorphism $e^{\alpha_j}\mapsto e^{2\pi i n_j/N}e^{\alpha_j}$ induces an
 automorphism of order dividing $N$. This is obvious. The point of Kac's theorem is
 that all inner automorphisms of order dividing $N$ are obtained this way and are
 conjugate if and only if they are conjugate by an automorphism of the Dynkin diagram.
 We won't actually prove Kac's theorem because we just want to get a bunch of
 examples. See \cite{Kac:IDLA} or \cite{Helgason}.

 \begin{example}
   Real forms of $E_8$. We've already found three, and it took us a long time. We can
   now do it fast. We need to solve $\sum n_i m_i =2$ where $n_i\ge 0$; there are only
   a few possibilities:
   \[
   \begin{tabular}{clcl}
     $\sum n_im_i=2$ & \# of ways & how to do it & \txt{maximal compact\\ subgroup $K$}\\
     $2\times 1$ & one way &
        $\begin{xy}<1.25em,0em>:
         (0,0) *\cir<2pt>{};
         p-(1,0) *\cir<2pt>{} **@{-};
         p-(1,0) *\cir<2pt>{} **@{-};
             p+(0,+1) *\cir<2pt>{} **@{-},
         p-(1,0) *\cir<2pt>{} **@{-};
         p-(1,0) *\cir<2pt>{} **@{-};
         p-(1,0) *\cir<2pt>{} **@{-};
         p-(1,0) *\cir<2pt>{} **@{-};
         p-(1,0) *{\times} *\cir<2pt>{} **@{-};
        \end{xy}$ & $E_8$ (compact form)\\
     $1\times 2$ & two ways&
        $\begin{xy}<1.25em,0em>:
         (0,0) *\cir<2pt>{};
         p-(1,0) *\cir<2pt>{} **@{-};
         p-(1,0) *\cir<2pt>{} **@{-};
             p+(0,+1) *\cir<2pt>{} **@{-},
         p-(1,0) *\cir<2pt>{} **@{-};
         p-(1,0) *\cir<2pt>{} **@{-};
         p-(1,0) *\cir<2pt>{} **@{-};
         p-(1,0) *{\times} *\cir<2pt>{} **@{-};
         p-(1,0) *\cir<2pt>{} **@{-};
        \end{xy}$ & $A_1 E_7$\\
      & &
        $\begin{xy}<1.25em,0em>:
         (0,0) *{\times} *\cir<2pt>{};
         p-(1,0) *\cir<2pt>{} **@{-};
         p-(1,0) *\cir<2pt>{} **@{-};
             p+(0,+1) *\cir<2pt>{} **@{-},
         p-(1,0) *\cir<2pt>{} **@{-};
         p-(1,0) *\cir<2pt>{} **@{-};
         p-(1,0) *\cir<2pt>{} **@{-};
         p-(1,0) *\cir<2pt>{} **@{-};
         p-(1,0) *\cir<2pt>{} **@{-};
        \end{xy}$ & $D_8$ (split form)\\
     $1\times 1+1\times 1$ & no ways
   \end{tabular}
   \]
%   $2\times 1=2$, $1\times 2=2$, and $1\times 1+1\times 1=2$.
%   There is only one node of weight 1, and two nodes of weight 2, so we can do the
%   first way one way, the second way two ways, and we can't do the third way. These
%   correspond to the three things we found last time
%
%   pictures with comments
   The points NOT crossed off form the Dynkin diagram of the maximal compact subgroup.
   Thus, by just looking at the diagram, we can see what all the real forms are!
 \end{example}
 \begin{example}
   Let's do $E_7$. Write down the affine diagram:
   \[
   \begin{xy}
      (0,0) *+!D{1} *\cir<2pt>{};
      p+(1,0) *+!D{2} *\cir<2pt>{} **@{-};
      p+(1,0) *+!D{3} *\cir<2pt>{} **@{-};
      p+(1,0) *+!D{4} *\cir<2pt>{} **@{-};
           p+(0,-1) *+!L{2} *\cir<2pt>{} **@{-},
      p+(1,0) *+!D{3} *\cir<2pt>{} **@{-};
      p+(1,0) *+!D{2} *\cir<2pt>{} **@{-};
      p+(1,0) *+!D{1} *\cir<2pt>{} **@{-};
    \end{xy}
   \]
   We get the possibilities
    \[
   \begin{tabular}{clcl}
     $\sum n_im_i=2$ & \# of ways & how to do it & \txt{maximal compact\\ subgroup $K$}\\
     $2\times 1$ & one way* &
      \begin{xy}<1.25em,0em>:
         (0,0) *{\times} *\cir<2pt>{};
         p+(1,0) *\cir<2pt>{} **@{-};
         p+(1,0) *\cir<2pt>{} **@{-};
         p+(1,0) *\cir<2pt>{} **@{-};
              p+(0,-1) *\cir<2pt>{} **@{-},
         p+(1,0) *\cir<2pt>{} **@{-};
         p+(1,0) *\cir<2pt>{} **@{-};
         p+(1,0) *\cir<2pt>{} **@{-};
       \end{xy} & $E_7$ (compact form)\\
     $1\times 2$ & two ways*&
      \begin{xy}<1.25em,0em>:
         (0,0) *\cir<2pt>{};
         p+(1,0) *{\times} *\cir<2pt>{} **@{-};
         p+(1,0) *\cir<2pt>{} **@{-};
         p+(1,0) *\cir<2pt>{} **@{-};
              p+(0,-1) *\cir<2pt>{} **@{-},
         p+(1,0) *\cir<2pt>{} **@{-};
         p+(1,0) *\cir<2pt>{} **@{-};
         p+(1,0) *\cir<2pt>{} **@{-};
       \end{xy} & $A_1 D_6$\\
     & &
      \begin{xy}<1.25em,0em>:
         (0,0) *\cir<2pt>{};
         p+(1,0) *\cir<2pt>{} **@{-};
         p+(1,0) *\cir<2pt>{} **@{-};
         p+(1,0) *\cir<2pt>{} **@{-};
              p+(0,-1) *{\times} *\cir<2pt>{} **@{-},
         p+(1,0) *\cir<2pt>{} **@{-};
         p+(1,0) *\cir<2pt>{} **@{-};
         p+(1,0) *\cir<2pt>{} **@{-};
       \end{xy} & $A_7$ (split form)**\\
     $1\times 1+1\times 1$ & one way &
      \begin{xy}<1.25em,0em>:
         (0,0) *{\times} *\cir<2pt>{};
         p+(1,0) *\cir<2pt>{} **@{-};
         p+(1,0) *\cir<2pt>{} **@{-};
         p+(1,0) *\cir<2pt>{} **@{-};
              p+(0,-1) *\cir<2pt>{} **@{-},
         p+(1,0) *\cir<2pt>{} **@{-};
         p+(1,0) *\cir<2pt>{} **@{-};
         p+(1,0) *{\times} *\cir<2pt>{} **@{-};
       \end{xy} & $E_6\oplus \RR$ \ ***\\
   \end{tabular}
   \]
   (*) The number of ways is counted up to automorphisms of the diagram.\\
   (**) In the split real form, the maximal compact subgroup has dimension equal to
        half the number of roots. The roots of $A_7$ look like $\e_i-\e_j$ for $i,j\le
        8$ and $i\neq j$, so the dimension is $8\cdot 7 + 7 = 56 = \frac{112}{2}$.\\
%   We can delete one node of weight 1, giving the compact form (the two ways are
%   conjugate by an automorphism of the affine Dynkin diagram):
%
%   picture
%
%   Next, you could delete one node of weight 2 ... three ways, but two are conjugate:
%
%   two pictures
%
%   The split form is something ... the maximal compact of the split for has dimension
%   half the number of roots.
%
%   Finally, you could delete two nodes of weight 1:
%
%   picture
%
   (***) The maximal compact subgroup is $E_6\oplus \RR$ because the fixed subalgebra
   contains the whole Cartan subalgebra, and the $E_6$ only accounts for $6$ of the $7$
   dimensions. You can use this to construct some interesting representations of $E_6$
   (the minuscule ones). How does the algebra $E_7$ decompose as a representation of
   the algebra $E_6\oplus \RR$?

   We can decompose it according to the eigenvalues of $\RR$. The $E_6\oplus \RR$ is
   the zero eigenvalue of $\RR$ [why?], and the rest is 54 dimensional. The easy way to see
   the decomposition is to look at the roots. Remember when we computed the Weyl group
   we looked for vectors like
   \[\begin{xy}
     (0,0) *\cir<2pt>{}; (1,0) *\cir<2pt>{} **@{-}; (2,0) *\cir<2pt>{} **@{.}
   \end{xy} \qquad\qquad \text{or} \qquad\qquad
   \begin{xy}
     (0,0) *\cir<2pt>{}; (1,0) *\cir<2pt>{} **@{.}; (2,0) *\cir<2pt>{} **@{-}
   \end{xy}\]
   The 27 possibilities (for each) form the weights of a 27 dimensional representation
   of $E_6$. The orthogonal complement of the two nodes is an $E_6$ root system whose
   Weyl group acts transitively on these 27 vectors (we showed that these form a
   single orbit, remember?). Vectors of the $E_7$ root system are the vectors of the
   $E_6$ root system plus these 27 vectors plus the other 27 vectors. This splits up the
   $E_7$ explicitly. The two 27s form single orbits, so they are irreducible. Thus,
   $E_7\cong E_6\oplus \RR\oplus 27\oplus 27$, and the 27s are minuscule.
 \end{example}
 Let $K$ be a maximal compact subgroup, with Lie algebra $\RR + E_6$. The factor of
 $\RR$ means that $K$ has an $S^1$ in its center. Now look at the space $G/K$, where
 $G$ is the Lie group of type $E_7$, and $K$ is the maximal compact subgroup. It is a
 \emph{Hermitian symmetric space}\index{symmetric space|idxbf}. Symmetric space means
 that it is a (simply connected) Riemannian manifold $M$ such that for each point
 $p\in M$, there is an automorphism fixing $p$ and acting as $-1$ on the tangent
 space. This looks weird, but it turns out that all kinds of nice objects you know
 about are symmetric spaces. Typical examples you may have seen: spheres $S^n$,
 hyperbolic space $\HH^n$, and Euclidean space $\RR^n$. Roughly speaking, symmetric
 spaces have nice properties of these spaces. Cartan\index{Cartan} classified all
 symmetric spaces: they are non-compact simple Lie groups modulo the maximal compact
 subgroup (more or less ... depending on simply connectedness hypotheses 'n such).
 Historically, Cartan classified simple Lie groups, and then later classified
 symmetric spaces, and was surprised to find the same result. Hermitian symmetric
 spaces are just symmetric spaces with a complex structure. A standard example of this
 is the upper half plane $\{x+iy|y>0\}$. It is acted on by $SL_2(\RR)$, which acts by
 $\matrix abcd \tau = \frac{a\tau + b}{c\tau + d}$.

 Let's go back to this $G/K$ and try to explain why we get a Hermitian symmetric space
 from it. We'll be rather sketchy here. First of all, to make it a symmetric space, we
 have to find a nice invariant Riemannian metric on it. It is sufficient to find a
 positive definite bilinear form on the tangent space at $p$ which is invariant under
 $K$ ... then you can translate it around. We can do this as $K$ is compact (so you
 have the averaging trick). Why is it Hermitian? We'll show that there is an almost
 complex structure. We have $S^1$ acting on the tangent space of each point because we
 have an $S^1$ in the center of the stabilizer of any given point. Identify this $S^1$
 with complex numbers of absolute value 1. This gives an invariant almost complex
 structure on $G/K$. That is, each tangent space is a complex vector space. Almost
 complex structures don't always come from complex structures, but this one does (it
 is integrable). Notice that it is a little unexpected that $G/K$ has a complex
 structure ($G$ and $K$ are odd dimensional in the case of $G=E_7$, $K=E_6\oplus \RR$,
 so they have no hope of having a complex structure).

 \begin{example}
   Let's look at $E_6$, with affine Dynkin diagram
   \[\begin{xy}
   (0,0) *+!D{1} *\cir<2pt>{};
   (1,0) *+!D{2} *\cir<2pt>{} **@{-};
   p+(1,0)="y" *+!D{3} *\cir<2pt>{} **@{-};
       p+(0,-1) *+!L{2} *\cir<2pt>{} **@{-};
       p+(0,-1) *+!L{1} *\cir<2pt>{} **@{-};
   "y" *{\hspace{4pt}};p+(1,0) *+!D{2} *\cir<2pt>{} **@{-};
   p+(1,0) *+!D{1} *\cir<2pt>{} **@{-};
  \end{xy}\]
  We get the possibilities
    \[
   \begin{tabular}{clcl}
     $\sum n_im_i=2$ & \# of ways & how to do it & \txt{maximal compact\\ subgroup $K$}\\
     $2\times 1$ & one way &
       \begin{xy}<1.75em,0em>:
        (0,0) *\cir<2pt>{};
        p+(1,0) *\cir<2pt>{} **@{-};
        p+(1,0)="y" *\cir<2pt>{} **@{-};
            p+(0,-1) *\cir<2pt>{} **@{-};
            p+(0,-1) *{\times} *\cir<2pt>{} **@{-};
        "y" *{\hspace{4pt}};p+(1,0) *\cir<2pt>{} **@{-};
        p+(1,0) *\cir<2pt>{} **@{-};
       \end{xy} & $E_6$ (compact form)\\
     $1\times 2$ & one way&
       \begin{xy}<1.75em,0em>:
        (0,0) *\cir<2pt>{};
        p+(1,0) *\cir<2pt>{} **@{-};
        p+(1,0)="y" *\cir<2pt>{} **@{-};
            p+(0,-1) *{\times} *\cir<2pt>{} **@{-};
            p+(0,-1) *\cir<2pt>{} **@{-};
        "y" *{\hspace{4pt}};p+(1,0) *\cir<2pt>{} **@{-};
        p+(1,0) *\cir<2pt>{} **@{-};
       \end{xy} & $A_1A_5$\\
     $1\times 1+1\times 1$ & one way &
       \begin{xy}<1.75em,0em>:
        (0,0) *{\times} *\cir<2pt>{};
        p+(1,0) *\cir<2pt>{} **@{-};
        p+(1,0)="y" *\cir<2pt>{} **@{-};
            p+(0,-1) *\cir<2pt>{} **@{-};
            p+(0,-1) *{\times} *\cir<2pt>{} **@{-};
        "y" *{\hspace{4pt}};p+(1,0) *\cir<2pt>{} **@{-};
        p+(1,0) *\cir<2pt>{} **@{-};
       \end{xy} & $D_5\oplus \RR$\\
   \end{tabular}
   \]
%   We can delete one point of weight 1, which gives the compact form
%
%   pictures
%
%   or one point of weight 2, which is blah. Or we can delete two points of weight 1:
%
%   picture
%
   In the last one, the maximal compact subalgebra is $D_5\oplus \RR$. Just as before,
   we get a Hermitian symmetric space. Let's compute its dimension (over $\CC$). The
   dimension will be the dimension of $E_6$ minus the dimension of $D_5\oplus\RR$, all
   divided by 2 (because we want complex dimension), which is $(78-46)/2=16$.

   So we have found two non-compact simply connected Hermitian symmetric spaces of
   dimensions 16 and 27. These are the only ``exceptional'' cases; all the others fall
   into infinite families!

   There are also some OUTER automorphisms of $E_6$ coming from the diagram
   automorphism
   \[\begin{xy}
     (0,0)="1" *\cir<2pt>{};
     (1,0)="2"  *\cir<2pt>{} **@{-};
     p+(1,0) *\cir<2pt>{} **@{-};
     p+(0,-1) *\cir<2pt>{} **@{-},
     p+(1,0)="22" *\cir<2pt>{} **@{-};
     p+(1,0)="11" *\cir<2pt>{} **@{-};
     "1" *+{\ };"11" *+{\ } **\crv{(2,3.3)} ?<*@{<} ?>*@{>} ?(.5)*+!U{\sigma},
     "2" *+{\ };"22" *+{\ } **\crv{(2,2.6)} ?<*@{<} ?>*@{>},
   \end{xy}\qquad \qquad \longrightarrow \qquad \qquad
   \begin{xy}
      (0,2) *\cir<2pt>{};
      p+(0,-1)  *\cir<2pt>{} **@{-};
      p+(0,-1)="x" *\cir<2pt>{} **@{=} ?*@{>};
      "x" *+<4pt>{\ };"x"+(0,-1) *\cir<2pt>{} **@{-};
   \end{xy}
   \]
   The fixed point subalgebra has Dynkin diagram
   obtained by folding the $E_6$ on itself. This is the $F_4$ Dynkin diagram. The
   fixed points of $E_6$ under the diagram automorphism is an $F_4$ Lie algebra. So we
   get a real form of $E_6$ with maximal compact subgroup $F_4$. This is probably the
   easiest way to construct $F_4$, by the way. Moreover, we can decompose $E_6$ as a
   representation of $F_4$. $\dim E_6=78$ and $\dim F_4=52$, so $E_6=F_4\oplus 26$,
   where $26$ turns out to be irreducible (the smallest non-trivial representation of
   $F_4$ ... the only one anybody actually works with). The roots of $F_4$ look like
   $(\dots, \pm 1, \pm 1\dots)$ (24 of these) and $(\pm \half \dots \pm \half)$ (16 of
   these), and $(\dots, \pm 1\dots)$ (8 of them) ... the last two types are in the
   same orbit of the Weyl group.

   The 26 dimensional representation has the following character: it has all norm 1
   roots with multiplicity 1 and 0 with multiplicity 2 (note that this is not
   minuscule).

   There is one other real form of $E_6$. To get at it, we have to talk about Kac's
   description of non-inner automorphisms of order $N$. The non-inner automorphisms
   all turn out to be related to diagram automorphisms. Choose a diagram automorphism
   of order $r$, which divides $N$. Let's take the standard thing on $E_6$. Fold the
   diagram (take the fixed points), and form a TWISTED affine Dynkin diagram (note
   that the arrow goes the wrong way from the affine $F_4$)
   \[
      \begin{xy}
      (0,-1.4)="1" *+!L{1} *\cir<2pt>{};
      p+(0,.7)="2" *+!L{2} *\cir<2pt>{} **@{-};
      p+(0,.7)="y" *+!R{3} *\cir<2pt>{} **@{-};
          p+(.7,0) *+!D{2} *\cir<2pt>{} **@{-};
          p+(.7,0) *+!D{1} *\cir<2pt>{} **@{-};
      "y" *=<4pt,4pt>{};p+(0,.7)="22" *+!L{2} *\cir<2pt>{} **@{-};
      p+(0,.7)="11" *+!L{1} *\cir<2pt>{} **@{-};
       "1" *+{\ };"11" *+{\ } **\crv{"y"+(-1.6,0)} ?<*@{<} ?>*@{>} ?(.5)*+!R{r},
       "2" *+{\ };"22" *+{\ } **\crv{"y"+(-1.2,0)} ?<*@{<} ?>*@{>},
      %%%%%%%%%%%
      (3,.5) *+!D{1} *\cir<2pt>{};
      p+(1,0) *+!D{2} *\cir<2pt>{} **@{-};
      p+(1,0)="x" *+!D{3} *\cir<2pt>{} **@{=} ?*@{>};
      "x" *{\hspace{4pt}};p+(1,0) *+!D{2} *\cir<2pt>{} **@{-};
      p+(1,0) *+!D{1} *\cir<2pt>{} *+<2.5em>!L{\text{Twisted Affine }F_4} **@{-};
      %%%%%%%%%%%
      (3,-1) *+!D{1} *+!R{\biggr(} *\cir<2pt>{};
      p+(1,0) *+!D{2} *\cir<2pt>{} **@{-};
      p+(1,0) *+!D{3} *\cir<2pt>{} **@{-};
      p+(1,0)="x" *+!D{4} *\cir<2pt>{} **@{=} ?*@{>};
      "x" *{\hspace{4pt}};p+(1,0) *+!D{2} *+<2.5em>!L{\text{Affine }F_4 \biggr)} *\cir<2pt>{} **@{-};
      %%%%%%%%%%%
      \ar (1.7,.1);(2.5,.3)
     \end{xy}
   \]
   There are also numbers on the twisted diagram, but nevermind them. Find $n_i$ so
   that $r\sum n_i m_i=N$. This is Kac's general rule. We'll only use the case $N=2$.

   If $r>1$, the only possibility is $r=2$ and one $n_1$ is 1 and the corresponding
   $m_i$ is 1. So we just have to find points of weight 1 in the twisted affine Dynkin
   diagram. There are just two ways of doing this in the case of $E_6$
   \[
    \begin{xy}
      (0,0) *\cir<2pt>{};
      p+(1,0) *\cir<2pt>{} **@{-};
      p+(1,0)="x" *\cir<2pt>{} **@{=} ?*@{>};
      "x" *{\hspace{4pt}};p+(1,0) *\cir<2pt>{} **@{-};
      p+(1,0) *{\times} *\cir<2pt>{} **@{-};
    \end{xy}\qquad\qquad \text{and} \qquad\qquad
    \begin{xy}
      (0,0) *{\times} *\cir<2pt>{};
      p+(1,0) *\cir<2pt>{} **@{-};
      p+(1,0)="x" *\cir<2pt>{} **@{=} ?*@{>};
      "x" *{\hspace{4pt}};p+(1,0) *\cir<2pt>{} **@{-};
      p+(1,0) *\cir<2pt>{} **@{-};
    \end{xy}
   \]
   one of these gives us $F_4$, and the other has maximal compact subalgebra $C_4$,
   which is the split form since $\dim C_4=\#\text{roots of }F_4/2 =24$.
 \end{example}

 \begin{example}
   $F_4$. The affine Dynkin is
    \begin{xy}
      (0,0) *{};
      p+(0,.1) *+!D{1} *\cir<2pt>{};
      p+(1,0) *+!D{2} *\cir<2pt>{} **@{-};
      p+(1,0) *+!D{3} *\cir<2pt>{} **@{-};
      p+(1,0)="x" *+!D{4} *\cir<2pt>{} **@{=} ?*@{>};
      "x" *{\hspace{4pt}};p+(1,0) *+!D{2} *\cir<2pt>{} **@{-};
    \end{xy}
   We can cross out one node of weight 1, giving the compact form (split form), or a
   node of weight 2 (in two ways), giving maximal compacts $A_1C_3$ or $B_4$. This
   gives us three real forms.
 \end{example}
 \begin{example}
   $G_2$. We can actually draw this root system ... UCB won't supply me with a
   four dimensional board. The construction is to take the $D_4$ algebra and look at
   the fixed points of:
   \[
   \begin{xy}
     (0,0) *\cir<2pt>{};
     a(60)="1" *\cir<2pt>{} **@{-},
     a(180)="2" *\cir<2pt>{} **@{-},
     a(-60)="3" *\cir<2pt>{} **@{-},
     \ar@/_2ex/ "1" *+{\ };"2" *+{\ }
     \ar@/_2ex/ "2" *+{\ };"3" *+{\ }
     \ar@/_2ex/_{\rho} "3" *+{\ };"1" *+{\ }
   \end{xy}
   \]
   We want to find the fixed point subalgebra.

   Fixed points on Cartan subalgebra: $\rho$ fixes a two dimensional space,
   and has 1 dimensional eigenspaces corresponding to $\w$ and $\bar \w$, where
   $\w^3=1$. The 2 dimensional space will be the Cartan subalgebra of $G_2$.

   Positive roots of $D_4$ as linear combinations of simple roots (not fundamental weights):
   \[\def\myDfour#1#2#3#4{
                 \begin{xy}
                   (0,0) *+{#1};
                   (-1,0) *+{#2} **@{-},
                    a(60) *+{#3} **@{-},
                    a(-60) *+{#4} **@{-},
                 \end{xy}}
     \def\myframecurve{2em}
    \begin{xy}
      (0,6) *++{\myDfour 0100 \qquad
               \myDfour 0010 \qquad
               \myDfour 0001} *\frm<\myframecurve>{-};
      (6,6) *++{\myDfour 1000} *\frm<\myframecurve>{-};
      (0,3) *++{\myDfour 1100 \qquad
               \myDfour 1010 \qquad
               \myDfour 1001} *\frm<\myframecurve>{-};
      (6,3) *++{\myDfour 1111} *\frm<\myframecurve>{-};
      (0,0) *++{\myDfour 1110 \qquad
               \myDfour 1011 \qquad
               \myDfour 1101} *\frm<\myframecurve>{-};
      (6,0) *++{\myDfour 2111} *\frm<\myframecurve>{-};
      (0,-1.8) *=<20em,0em>\frm{_\}} *+!U{\text{projections of norm }2/3};
      (6,-1.8) *=<6.25em,0em>\frm{_\}} *+!U{\text{projections of norm }2};
    \end{xy}
   \]
   There are six orbits under $\rho$, grouped above. It obviously acts on the negative
   roots in exactly the same way. What we have is a root system with six roots of norm
   2 and six roots of norm $2/3$. Thus, the root system is $G_2$:
   \[\begin{xy}
     (0,0) *+!DR{2} *{\bullet};
     a(0) *++!L{1} *{\bullet};
     a(60) *+!DL{1} *{\bullet};
     a(120) *+!DR{1} *{\bullet};
     a(180) *++!R{1} *{\bullet};
     a(240) *+!UR{1} *{\bullet};
     a(300) *+!UL{1} *{\bullet};
     a(60)+a(120) *+!D{1} *{\bullet};
     a(180)+a(240) *+!R{1} **@{-} *{\bullet};
     a(300)+a(0) *+!L{1} **@{-} *{\bullet};
     a(60)+a(120) **@{-};
     a(60)+a(0) *+!L{1} *{\bullet};
     a(180)+a(120) *+!R{1} **@{-} *{\bullet};
     a(300)+a(240) *+!U{1} **@{-} *{\bullet};
     a(60)+a(0) **@{-};
   \end{xy}\]
   One of the only root systems to appear on a country's national flag. Now let's work
   out the real forms. Look at the affine:
   \begin{xy}
   (0,0) *{};
   p+(0,.05) *+!D{1} *\cir<2pt>{};
   p+(1,0)="1" *+!D{2} *\cir<2pt>{} **@{-};
   p+(1,0)="2" *+!D{3} *\cir<2pt>{} **@{-} ?*@{>},
   \ar@{-} "1" *{\hspace{3pt}};"2" *{\hspace{3pt}} <1.5pt>
   \ar@{-} "1" *{\hspace{3pt}};"2" *{\hspace{3pt}} <-1.5pt>
   \end{xy}.
   we can delete the node of weight 1, giving the compact form:
   \begin{xy}
   (0,0) *{};
   p+(0,.05) *{\times} *\cir<2pt>{};
   p+(1,0)="1" *\cir<2pt>{} **@{-};
   p+(1,0)="2" *\cir<2pt>{} **@{-} ?*@{>},
   \ar@{-} "1" *{\hspace{3pt}};"2" *{\hspace{3pt}} <1.5pt>
   \ar@{-} "1" *{\hspace{3pt}};"2" *{\hspace{3pt}} <-1.5pt>
   \end{xy}
   . We can delete the node
   of weight 2, giving $A_1A_1$ as the compact subalgebra:
   \begin{xy}
   (0,0) *{};
   p+(0,.05) *\cir<2pt>{};
   p+(1,0)="1" *{\times} *\cir<2pt>{} **@{-};
   p+(1,0)="2" *\cir<2pt>{} **@{-} ?*@{>},
   \ar@{-} "1" *{\hspace{3pt}};"2" *{\hspace{3pt}} <1.5pt>
   \ar@{-} "1" *{\hspace{3pt}};"2" *{\hspace{3pt}} <-1.5pt>
   \end{xy}
   ... this must be the split
   form because there is nothing else the split form can be.

   Let's say some more about the split form. What does the Lie algebra of $G_2$ look
   like as a representation of the maximal compact subalgebra $A_1 \times A_1$? In
   this case, it is small enough that we can just draw a picture:
   \[\begin{xy}
     (0,0) *{2};
     a(0) *++!L{1} *{\bullet};
     a(60) *+!DL{1} *{\bullet};
     a(120) *+!DR{1} *{\bullet};
     a(180) *++!R{1} *{\bullet};
     a(240) *+!UR{1} *{\bullet};
     a(300) *+!UL{1} *{\bullet};
     a(60)+a(120) *+!D{1} *{\bullet};
     a(180)+a(240) *+!R{1} **@{.} *{\bullet};
     a(300)+a(0) *+!L{1} **@{.} *{\bullet};
     a(60)+a(120) **@{.};
     a(60)+a(0) *+!L{1} *{\bullet};
     a(180)+a(120) *+!R{1} **@{.} *{\bullet};
     a(300)+a(240) *+!U{1} **@{.} *{\bullet};
     a(60)+a(0) **@{.};
     (0,0) *=<8.75em,1.5em>\frm<8pt>{-} *=<1.5em,11.25em>\frm<8pt>{-};
   \end{xy} \qquad \qquad \longrightarrow \qquad \qquad
   \begin{xy}
     (0,0);
     a(60) *+!D{1} *{\bullet};
     a(120) *+!D{1} *{\bullet};
     a(240) *+!U{1} *{\bullet};
     a(300) *+!U{1} *{\bullet};
     a(60)+a(120);
     a(180)+a(240) *+!U{1} *{\bullet} **@{.};
     a(300)+a(0) *+!U{1} *{\bullet} **@{.};
     a(60)+a(120) **@{.};
     a(60)+a(0) *+!D{1} *{\bullet};
     a(180)+a(120) *+!D{1} *{\bullet} **@{.};
     a(300)+a(240) **@{.};
     a(60)+a(0) **@{.};
     (0,\halfrootthree) *=<11.25em,2.2em>\frm<8pt>{-};
     (1.5,0) *=<2em,7.5em>\frm<8pt>{-};
   \end{xy}\]
   We have two orthogonal $A_1$s, and we have leftover the stuff on the right. This
   thing on the right is a tensor product of the 4 dimensional irreducible
   representation of the horizontal and the 2 dimensional of the vertical. Thus, $G_2=
   3\times 1 + 1\otimes 3 + 4\otimes 2$ as irreducible representations of
   $A_1^{(\mathrm{horizontal})} \otimes A_1^{(\mathrm{vertical})}$.

   Let's use this to determine exactly what the maximal compact subgroup is. It is a
   quotient of the simply connected compact group $SU(2)\times SU(2)$, with Lie
   algebra $A_1\times A_1$. Just as for $E_8$, we need to identify which elements of
   the center act trivially on $G_2$. The center is $\ZZ/2\times \ZZ/2$. Since we've
   decomposed $G_2$, we can compute this easily. A non-trivial element of the center
   of $SU(2)$ acts as 1 (on odd dimensional representations) or $-1$ (on even
   dimensional representations). So the element $z\times z\in SU(2)\times SU(2)$ acts
   trivially on $3\otimes 1 + 1\otimes 3 + 4\times 2$. Thus the maximal compact
   subgroup of the non-compact simple $G_2$ is $SU(2)\times SU(2)/(z\times z)\cong
   SO_4(\RR)$, where $z$ is the non-trivial element of $\ZZ/2$.
 \end{example}

 So we have constructed $3+4+5+3+2$ (from $E_8$, $E_7$, $E_6$, $F_4$, $G_2$) real
 forms of exceptional simple Lie groups.

 There are another 5 exceptional real Lie groups: Take COMPLEX groups $E_8(\CC)$,
 $E_7(\CC)$, $E_6(\CC)$, $F_4(\CC)$, and $G_2(\CC)$, and consider them as REAL. These
 give simple real Lie groups of dimensions $248\times 2$, $133\times 2$, $78\times 2$,
 $52 \times 2$, and $14\times 2$.
}{   % Anton, geraschenko@gmail.com
  \stepcounter{lecture}
 \setcounter{lecture}{30}
 \sektion{Lecture 30 - Irreducible unitary representations of
                       \texorpdfstring{$SL_2(\RR)$}{SL(2,R)}}

 $SL_2(\RR)$ is non-compact. For compact Lie groups, all unitary representations are
 finite dimensional, and are all known well. For non-compact groups, the theory is
 much more complicated. Before doing the infinite dimensional representations, we'll
 review finite dimensional (usually not unitary) representations of $SL_2(\RR)$.

 \subsektion{Finite dimensional representations} Finite dimensional complex
 representations of the following are much the same: $SL_2(\RR)$, $\sl_2\RR$, $\sl_2
 \CC$ [branch $SL_2(\CC)$ as a complex Lie group] (as a complex Lie algebra),
 $\mathfrak{su}_2\RR$ (as a real Lie algebra), and $SU_2$ (as a real Lie group). This
 is because finite dimensional representations of a simply connected Lie group are in
 bijection with representations of the Lie algebra. Complex representations of a REAL
 Lie algebra $L$ correspond to complex representations of its complexification
 $L\otimes \CC$ considered as a COMPLEX Lie algebra.

 Note: Representations of a COMPLEX Lie algebra $L\otimes \CC$ are not the same
 as representations of the REAL Lie algebra $L\otimes \CC\cong L+ L$. The
 representations of the real Lie algebra correspond roughly to (reps of
 $L)\otimes$(reps of $L$).

 Strictly speaking, $SL_2(\RR)$ is not simply connected, which is not important for
 finite dimensional representations.

 Recall the main results for representations of $SU_2$:
 \begin{enumerate}
   \item For each positive integer $n$, there is one irreducible representation of
   dimension $n$.

   \item The representations are completely reducible (every representation is a sum
   of irreducible ones). This is perhaps the most important fact.

   The finite dimensional representation theory of $SU_2$ is EASIER than the
   representation theory of the ABELIAN Lie group $\RR^2$, and that is because
   representations of $SU_2$ are completely reducible.

   For example, it is very difficult to classify pairs of commuting nilpotent
   matrices.
 \end{enumerate}

 Completely reducible representations:
 \begin{enumerate}
   \item Complex representations of finite groups.

   \item Representations of compact groups (Weyl character formula)

   \item More generally, unitary representations of anything (you can take orthogonal
   complements of subrepresentations)

   \item Finite dimensional representations of semisimple Lie groups.
 \end{enumerate}
 Representations which are not completely reducible:
 \begin{enumerate}
   \item Representations of a finite group $G$  over fields of characteristic
   $p|\,|G|$.

   \item Infinite dimensional representations of non-compact Lie groups (even if they
   are semisimple).
 \end{enumerate}

 We'll work with the Lie algebra $\sl_2\RR$, which has basis $H=\matrix 100{-1}$,
 $E=\matrix 0100$, and $F=\matrix 0010$. $H$ is a basis for the Cartan subalgebra
 $\matrix a00{-a}$. $E$ spans the root space of the simple root. $F$ spans the root
 space of the negative of the simple root. We find that $[H,E]=2E$, $[H,F]=-2F$ (so
 $E$ and $F$ are eigenvectors of $H$), and you can check that $[E,F]=H$.
 \[\begin{xy}
   (1,0) *++!D{0} *++!U{H} *{\bullet};
   \ar@/^2.5em/@{<->}^{\text{Weyl group of order 2}}(0,0) *++!D{-2} *{\bullet} *++!U{F};
   (2,0) *++!D{2} *{\bullet} *++!U{E};
   \ar (3,.3) *+!L{\text{weights $=$ eigenvalues under }H};(2.2,.3)
 \end{xy}\]
 The Weyl group is generated by $\w = \matrix 01{-1}0$ and $\w^2=\matrix {-1}00{-1}$.

 Let $V$ be a finite dimensional irreducible complex representation of $\sl_2\RR$.
 First decompose $V$ into eigenspaces of the Cartan subalgebra (weight spaces) (i.e.\
 eigenspaces of the element $H$). Note that eigenspaces of $H$ exist because $V$ is
 FINITE-DIMENSIONAL (remember this is a complex representation). Look at the LARGEST
 eigenvalue of $H$ (exists since $V$ is finite dimensional), with eigenvector $v$. We
 have that $Hv=nv$ for some $n$. Compute
 \begin{align*}
   H(Ev) &= [H,E]v + E(Hv)\\
        &= 2Ev + Env = (n+2)Ev
 \end{align*}
 So $Ev=0$ (lest it be an eigenvector of $H$ with higher eigenvalue). $[E,-]$
 increases weights by 2 and $[F,-]$ decreases weights by 2, and $[H,-]$ fixes weights.

 We have that $E$ kills $v$, and $H$ multiplies it by $n$. What does $F$ do to $v$?
 \[\xymatrix{
  & nv & (n-2)Fv & (n-4)F^2v & (n-6)F^3v & \dots\\
  0 & v\ar@/^/[l]^E \ar[u]_H \ar@/^/[r]^F &
  Fv \ar@/^/[l]^{\substack{E\\ \times n}} \ar[u]_H \ar@/^/[r]^F &
  F^2v \ar@/^/[l]^{\substack{E \\ \times (2n-2)}} \ar[u]_H \ar@/^/[r]^F &
  F^3v \ar@/^/[l]^{\substack{E\\ \times (3n-6)}} \ar[u]_H & \dots
 }\]
 What is $E(Fv)$? Well,
 \begin{align*}
   EFv &= FEv + [E,F]v\\
        &= 0 + Hv = nv
 \end{align*}
 In general, we have
 \begin{align*}
   H(F^i v) &= (n-2i)F^i v\\
   E(F^i v) &= (ni-i(i-1))F^{i-1}v\\
   F(F^i v) &= F^{i+1} v
 \end{align*}
 So the vectors $F^i v$ span $V$ because they span an invariant subspace. This gives
 us an infinite number of vectors in distinct eigenspaces of $H$, and $V$ is
 finite dimensional. Thus, $F^k v=0$ for some $k$. Suppose $k$ is the SMALLEST integer
 such that $F^kv=0$. Then
 \[
    0 = E(F^k v) = (nk-k(k-1))\underbrace{EF^{k-1}v}_{\neq 0}
 \]
 So $nk-k(k-1)=0$, and $k\neq 0$, so $n-(k-1)=0$, so \fbox{$k=n+1$}\,. So $V$ has a
 basis consisting of $v,Fv,\dots, F^n v$. The formulas become a little better if we
 use the basis $w_n=v,w_{n-2}=Fv, w_{n-4}=\frac{F^2v}{2!}, \frac{F^3 v}{3!}, \dots,
 \frac{F^nv}{n!}$.
 \[\xymatrix{
  w_{-6} \ar@/^1.25em/[r]^1 \ar@/_1.25em/@{<-}[r]_6 &
  w_{-4} \ar@/^1.25em/[r]^2 \ar@/_1.25em/@{<-}[r]_5 &
  w_{-2} \ar@/^1.25em/[r]^3 \ar@/_1.25em/@{<-}[r]_4 &
  w_{0}  \ar@/^1.25em/[r]^4 \ar@/_1.25em/@{<-}[r]_3 &
  w_{2}  \ar@/^1.25em/[r]^5 \ar@/_1.25em/@{<-}[r]_2 &
  w_{4}  \ar@/^1.25em/[r]^6 \ar@/_1.25em/@{<-}[r]_1 &
  w_{6} & \hspace{-3.75em} \shortstack{$E$\\ \vspace{1.75em} \\$F$}
 }\]
 This says that $E(w_2)=5w_4$ for example. So we've found a complete description of
 all finite dimensional irreducible complex representations of $\sl_2 \RR$. This is as
 explicit as you could possibly want.

 These representations all lift to the group $SL_2(\RR)$: $SL_2(\RR)$ acts on
 homogeneous polynomials of degree $n$ by $\matrix abcd f(x,y)=f(ax+by,cx+dy)$. This
 is an $n+1$ dimensional space, and you can check that the eigenspaces are $x^i
 y^{n-i}$.

 We have implicitly constructed VERMA MODULES. We have a basis $w_n,w_{n-2},\dots,
 w_{n-2i},\dots$ with relations $H(w_{n-2i})=(n-2i)w_{n-2i}$,
 $Ew_{n-2i} = (n-i+1)w_{n-2i+2}$, and $Fw_{n-2i} = (i+1)w_{n-2i-2}$. These are
 obtained by copying the formulas from the finite dimensional case, but allow it to be
 infinite dimensional. This is the universal representation generated by the highest
 weight vector $w_n$ with eigenvalue $n$ under $H$ (highest weight just means
 $E(w_n)=0$).

 Let's look at some things that go wrong in infinite dimensions.
 \begin{warning}
   Representations corresponding to the Verma modules do NOT lift to representations of
   $SL_2(\RR)$, or even to its universal cover. The reason: look at the Weyl group
   (generated by $\matrix 01{-1}0$) of $SL_2(\RR)$ acting on $\langle H\rangle$; it
   changes $H$ to $-H$. It maps eigenspaces with eigenvalue $m$ to eigenvalue $-m$.
   But if you look at the Verma module, it has eigenspaces $n,n-2,n-4,\dots$, and this
   set is obviously not invariant under changing sign. The usual proof that
   representations of the Lie algebra lifts uses the exponential map of matrices,
   which doesn't converge in infinite dimensions.
 \end{warning}
 \begin{remark}
   The universal cover $\widetilde{SL_2(\RR)}$ of $SL_2(\RR)$, or even the double
   cover $Mp_2(\RR)$, has NO faithful finite dimensional representations.
   \begin{proof}
     Any finite dimensional representation comes from a
     finite dimensional representation of the Lie algebra $\sl_2\RR$. All such
     finite dimensional representations factor through $SL_2(\RR)$.
   \end{proof}
 \end{remark}
 All finite dimensional representations of $SL_2(\RR)$ are completely reducible. Weyl
 did this by Weyl's unitarian trick:

 Notice that finite dimensional representations of $SL_2(\RR)$ are isomorphic (sort
 of) to finite dimensional representations of the COMPACT group $SU_2$ (because they
 have the same complexified Lie algebras. Thus, we just have to show it for $SU_2$.
 But representations of ANY compact group are completely reducible. Reason:
 \begin{enumerate}
   \item All unitary representations are completely reducible (if $U\subseteq V$, then
   $V=U\oplus U^\perp$).

   \item Any representation $V$ of a COMPACT group $G$ can be made unitary: take any
   unitary form on $V$ (not necessarily invariant under $G$), and average it over $G$
   to get an invariant unitary form. We can average because $G$ is compact, so we can
   integrate any continuous function over $G$. This form is positive definite since it
   is the average of positive definite forms (if you try this with non-(positive
   definite) forms, you might get zero as a result).
 \end{enumerate}

 \subsektion{The Casimir operator}\index{Casimir operator|idxbf}Set $\W = 2EF + 2FE +
 H^2\in U(\sl_2\RR)$. The main point is that $\W$ commutes with $\sl_2\RR$. You can
 check this by brute force:
 \begin{align*}
   [H,\W] &= 2\underbrace{([H,E]F+E[H,F])}_0+\cdots\\
   [E,\W] &= 2[E,E]F + 2E[F,E] + 2[E,F]E \\
          & \qquad + 2F[E,E] + [E,H]H + H[E,H] = 0\\
   [F,\W] &= \text{Similar}
 \end{align*}
 Thus, $\W$ is in the center of $U(\sl_2\RR)$. In fact, it generates the center. This
 doesn't really explain where $\W$ comes from.
 \begin{remark}
   Why does $\W$ exist? The answer is that it comes from a symmetric invariant
   bilinear form on the Lie algebra $\sl_2\RR$ given by $(E,F)=1$,
   $(E,E)=(F,F)=(F,H)=(E,H)=0$, $(H,H)=2$. This bilinear form is an invariant map
   $L\otimes L\to \CC$, where $L=\sl_2\RR$, which by duality gives an invariant
   element in $L\otimes L$, which turns out to be $2E\otimes F + 2F\otimes E +
   H\otimes H$. The invariance of this element corresponds to $\W$ being in the center
   of $U(\sl_2\RR)$.
 \end{remark}
 Since $\W$ is in the center of $U(\sl_2\RR)$, it acts on each irreducible representation
 as multiplication by a constant. We can work out what this constant is for the
 finite dimensional representations.
 Apply $\W$ to the highest vector $w_n$:
 \begin{align*}
   (2EF + 2FE + HH)w_n &= (2n+0+n^2)w_n\\
            &= (2n+n^2)w_n
 \end{align*}
 So $\W$ has eigenvalue $2n+n^2$ on the irreducible representation of dimension $n+1$.
 Thus, $\W$ has DISTINCT eigenvalues on different irreducible representations, so it
 can be used to separate different irreducible representations. The main use of $\W$
 will be in the next lecture, where we'll use it to deal with infinite dimensional
 representation.

 To finish today's lecture, let's look at an application of $\W$. We'll sketch an
 algebraic argument that the representations of $\sl_2 \RR$ are completely reducible.
 Given an exact sequence of representations
 \[
    0\to U\to V\to W\to 0
 \]
 we want to find a splitting $W\to V$, so that $V=U\oplus W$.

 \underline{Step 1}: Reduce to the case where $W=\CC$. The idea is to look at
 \[
    0\to \hom_\CC(W,U)\to \hom_\CC(W,V)\to \hom_\CC(W,W)\to 0
 \]
 and $\hom_\CC(W,W)$ has an obvious one dimensional subspace, so we can get a smaller
 exact sequence
 \[
    0\to \hom_\CC(W,U)\to \text{subspace of }\hom_\CC(W,V)\to \CC \to 0
 \]
 and if we can split this, the original sequence splits.

 \underline{Step 2}: Reduce to the case where $U$ is irreducible. This is an easy
 induction on the number of irreducible components of $U$.
 \begin{exercise}
   Do this.
 \end{exercise}

 \underline{Step 3}: This is the key step. We have
 \[
    0\to U\to V\to \CC\to 0
 \]
 with $U$ irreducible. Now apply the Casimir operator $\W$. $V$ splits as eigenvalues
 of $\W$, so is $U\oplus \CC$ UNLESS $U$ has the same eigenvalue as $\CC$ (i.e.\
 unless $U=\CC$).

 \underline{Step 4}: We have reduced to
 \[
    0\to \CC\to V\to \CC\to 0
 \]
 which splits because $\sl_2(\RR)$ is perfect\footnote{$L$ is \emph{perfect} if
 $[L,L]=L$} (no homomorphisms to the abelian algebra $\matrix 0{\ast}00$).

 Next time, in the final lecture, we'll talk about infinite dimensional unitary
 representations.
}{   % Lilit Martirosyan, lilit@math
  \stepcounter{lecture}
 \setcounter{lecture}{31}
 \sektion{Lecture 31 - Unitary representations of \texorpdfstring{$SL_2(\RR)$}{SL(2,R)}}

 Last lecture, we found the finite dimensional (non-unitary) representations of
 $SL_2(\RR)$.

 \subsektion{Background about infinite dimensional representations} (of a Lie group
 $G$) What is an finite dimensional representation?
 \begin{itemize}
   \item[1st guess] Banach space acted on by $G$?

   This is no good for some reasons: Look at the action of $G$ on the functions on $G$
   (by left translation). We could use $L^2$ functions, or $L^1$ or $L^p$. These are
   completely different Banach spaces, but they are essentially the same
   representation.

   \item[2nd guess] Hilbert space acted on by $G$? This is sort of okay.

   The problem is that finite dimensional representations of $SL_2(\RR)$ are NOT
   Hilbert space representations, so we are throwing away some interesting
   representations.

   \item[Solution] (Harish-Chandra) Take $\g$ to be the Lie algebra of $G$, and let
   $K$ be the maximal compact subgroup. If $V$ is an infinite dimensional
   representation of $G$, there is no reason why $\g$ should act on $V$.

   The simplest example fails. Let $\RR$ act on $L^2(\RR)$ by left translation. Then
   the Lie algebra is generated by $\der{}{x}$ (or $i\der{}{x}$) acting on $L^2(\RR)$,
   but $\der{}{x}$ of an $L^2$ function is not in $L^2$ in general.

   Let $V$ be a Hilbert space. Set $V_\w$ to be the $K$-finite vectors of $V$, which are
   the vectors contained in a finite dimensional representation of $K$. The point is
   that $K$ is compact, so $V$ splits into a Hilbert space direct sum finite dimensional
   representations of $K$, at least if $V$ is a Hilbert space. Then $V_\w$ is a
   representation of the Lie algebra $\g$, not a representation of $G$. $V_\w$ is a
   representation of the group $K$. It is a $(\g,K)$-module, which means that it is
   acted on by $\g$ and $K$ in a ``compatible'' way, where compatible means that
   \begin{enumerate}
     \item they give the same representations of the Lie algebra of $K$.
     \item $k(u)v = k(u(k^{-1} v))$ for $k\in K$, $u\in \g$, and $v\in V$.
   \end{enumerate}
   The $K$-finite vectors of an irreducible unitary representation of $G$ is
   ADMISSIBLE, which means that every representation of $K$ only occurs a
   \emph{finite} number of times. The GOOD category of representations is the
   representations of admissible $(\g,K)$-modules. It turns out that this is a really
   well behaved category.
 \end{itemize}

 We want to find the unitary irreducible representations of $G$. We will do this in
 several steps:
 \begin{enumerate}
   \item Classify all irreducible admissible representations of $G$. This was solved
   by Langlands, Harish-Chandra et.\ al.

   \item Find which have hermitian inner products $(\ ,\,)$. This is easy.

   \item Find which ones are positive definite. This is VERY HARD. We'll only do this
   for the simplest case: $SL_2(\RR)$.
 \end{enumerate}

 \subsektion{The group \texorpdfstring{$SL_2(\RR)$}{SL(2,R)}} We found some generators (in $Lie(SL_2(\RR))\otimes
 \CC$ last time: $E$, $F$, $H$, with $[H,E]=2E$, $[H,F]=-2F$, and $[E,F]=H$. We have
 that $H = -i \matrix 01{-1}0$, $E= \half \matrix 1ii{-1}$, and $F=\half \matrix
 1{-i}{-i}{-1}$. Why not use the old $\matrix 100{-1}$, $\matrix 0100$, and $\matrix
 0010$?

 Because $SL_2(\RR)$ has two different classes of Cartan subgroup: $\matrix
 a00{a^{-1}}$, spanned by $\matrix 100{-1}$, and $\matrix {\cos\theta}{\sin
 \theta}{-\sin\theta}{\cos\theta}$, spanned by $\matrix 01{-1}0$, and the second one
 is COMPACT. The point is that non-compact (abelian) groups need not have eigenvectors
 on infinite dimensional spaces. An eigenvector is the same as a weight space. The
 first thing you do is split it into weight spaces, and if your Cartan subgroup is not
 compact, you can't get started. We work with the compact subalgebra so that the
 weight spaces exist.

 Given the representation $V$, we can write it as some direct sum of eigenspaces of
 $H$, as the Lie group $H$ generates is compact (isomorphic to $S^1$). In the
 finite dimensional case, we found a HIGHEST weight, which gave us complete control
 over the representation. The trouble is that in infinite dimensions, there is no
 reason for the highest weight to exist, and in general they don't. The highest weight
 requires a finite number of eigenvalues.

 A good substituted for the highest weight vector: Look at the Casimir operator $\W =
 2EF+2FE + H^2+1$. The key point is that $\W$ is in the center of the universal
 enveloping algebra. As $V$ is assumed admissible, we can conclude that $\W$ has
 eigenvectors (because we can find a finite dimensional space acted on by $\W$). As
 $V$ is irreducible and $\W$ commutes with $G$, all of $V$ is an eigenspace of $\W$.
 We'll see that this gives us about as much information as a highest weight vector.

 Let the eigenvalue of $\W$ on $V$ be $\lambda^2$ (the square will make the
 interesting representations have integral $\lambda$; the $+1$ in $\W$ is for the same
 reason).

 Suppose $v\in V_n$, where $V_n$ is the space of vectors where $H$ has eigenvalue $n$.
 In the finite dimensional case, we looked at $Ev$, and saw that $HEv=(n+2)Ev$. What
 is $FEv$? If $v$ was a highest weight vector, we could control this. Notice that
 $\W=4FE + H^2 + 2H +1$ (using $[E,F]=H$), and $\W v=\lambda^2 v$. This says that
 $4FEv + n^2 v + 2nv + v = \lambda^2 v$. This shows that $FEv$ is a multiple of $v$.

 Now we can draw a picture of what the representation looks like:
 \newsavebox{\upright}\savebox{\upright}{\raisebox{2ex}{\xymatrix @!0 @C=3em {\ar@/^1em/[r] & }}}
 \newsavebox{\downleft}\savebox{\downleft}{\raisebox{-2ex}{\xymatrix @!0 @C=3em {\ar@/_1em/@{<-}[r] & }}}
 \newsavebox{\uprights}\savebox{\uprights}{\raisebox{2ex}{\xymatrix @!0 @C=3em {\ar@/^1em/@{.>}[r] & }}}
 \newsavebox{\downlefts}\savebox{\downlefts}{\raisebox{-2ex}{\xymatrix @!0 @C=3em {\ar@/_1em/@{<.}[r] & }}}
 \newsavebox{\pright}\savebox{\pright}{\raisebox{1ex}{\xymatrix @!0 @C=2.5em {\ar@/^.5em/[r] & }}}
 \newsavebox{\nleft}\savebox{\nleft}{\raisebox{-1ex}{\xymatrix @!0 @C=2.5em {\ar@/_.5em/@{<-}[r] & }}}
% \newsavebox{\prights}\savebox{\prights}{\raisebox{1ex}{\xymatrix @!0 @C=2.5em {\ar@/^.5em/@{.>}[r] & }}}
% \newsavebox{\nlefts}\savebox{\nlefts}{\raisebox{-1ex}{\xymatrix @!0 @C=2.5em {\ar@/_.5em/@{<-}[r] & }}}
 \[
  \begin{xy}<3.5em,0em>:
   (0,0) *!R{\cdots} *++{\,}; p+(.5,0) *{\usebox{\uprights}} *{\usebox{\downleft}};
   p+(.5,0) *{v_{n-4}};       p+(.5,0) *{\usebox{\uprights}} *{\usebox{\downleft}};
   p+(.5,0) *{v_{n-2}};       p+(.5,0) *{\usebox{\uprights}} *{\usebox{\downleft}};
   p+(.5,0) *{v_{n}};         p+(.5,0) *{\usebox{\upright}} *{\usebox{\downleft}};
     p+(0,-.4) *+!U{\mbox{\scriptsize $\left(\frac{n^2+2n+1-\lambda^2}{4}\right)$}},
   p+(.5,0) *{v_{n+2}};       p+(.5,0) *{\usebox{\upright}} *{\usebox{\downlefts}};
   p+(.5,0) *{v_{n+4}};       p+(.5,0) *{\usebox{\upright}} *{\usebox{\downlefts}};
   p+(.5,0) *!L{\cdots} *++{\,};
   p+(.7,0) *{\shortstack{$E$\\ \vspace{.25em} \\ $H$ \\ \vspace{.25em} \\$F$}}
 \end{xy}
 \]
% \[\xymatrix @!0 @C=14mm{
%  **[l] \cdots \ar@{.>}@/^5mm/[r] \ar@/_5mm/@{<-}[r] &
%  v_{n-4} \ar@{.>}@/^5mm/[r] \ar@/_5mm/@{<-}[r] &
%  v_{n-2} \ar@{.>}@/^5mm/[r] \ar@/_5mm/@{<-}[r] &
%  v_{n}  \ar@/^5mm/[r] \ar@/_5mm/@{<-}[r]_{\left(\frac{n^2+2n+1-\lambda^2}{4}\right)} &
%  v_{n+2}  \ar@/^5mm/[r] \ar@/_5mm/@{<.}[r] &
%  v_{n+4}  \ar@/^5mm/[r] \ar@/_5mm/@{<.}[r] &
%  **[r] \cdots
%  & \shortstack{$E$\\ \vspace{1mm} \\ $H$ \\ \vspace{1mm} \\$F$}
% }\]
 Thus, $V_\w$ is spanned by $V_{n+2k}$, where $k$ is an integer. The non-zero elements
 among the $V_{n+2k}$ are linearly independent as they have different eigenvalues. The
 only question remaining is whether any of the $V_{n+2k}$ vanish.

 There are four possible shapes for an irreducible representation
 \begin{itemize}
   \item infinite in both directions:
    $\begin{xy}
       (0,0) *!R{\cdots} *++{\,};p+(.5,0) *{\usebox{\pright}} *{\usebox{\nleft}};
       p+(.5,0) *{\udot};        p+(.5,0) *{\usebox{\pright}} *{\usebox{\nleft}};
       p+(.5,0) *{\udot};        p+(.5,0) *{\usebox{\pright}} *{\usebox{\nleft}};
       p+(.5,0) *{\udot};        p+(.5,0) *{\usebox{\pright}} *{\usebox{\nleft}};
       p+(.5,0) *{\udot};        p+(.5,0) *{\usebox{\pright}} *{\usebox{\nleft}};
       p+(.5,0) *!L{\cdots} *++{\,};
       p+(.7,0) *{\shortstack{$E$\\ $H$ \\$F$}}
    \end{xy}$

%    $\xymatrix @!0 @C=10mm{
%     **[l] \cdots \ar@/^3mm/[r] \ar@/_3mm/@{<-}[r] &
%     \udot \ar@/^3mm/[r] \ar@/_3mm/@{<-}[r] &
%     \udot \ar@/^3mm/[r] \ar@/_3mm/@{<-}[r] &
%     \udot \ar@/^3mm/[r] \ar@/_3mm/@{<-}[r] &
%     \udot \ar@/^3mm/[r] \ar@/_3mm/@{<-}[r] & **[r] \cdots
%     & \shortstack{$E$\\ $H$ \\$F$}
%    }$
%    \mpar{I don't know if these pictures are really needed.}

   \item a lowest weight, and infinite in the other direction:
   \[\begin{xy}
       (0,0) *!R{\cdots} *++{\,};p+(.5,0) *{\usebox{\pright}} *{\usebox{\nleft}};
       p+(.5,0) *{\udot};        p+(.5,0) *{\usebox{\pright}} *{\usebox{\nleft}};
       p+(.5,0) *{\udot};        p+(.5,0) *{\usebox{\pright}} *{\usebox{\nleft}};
       p+(.5,0) *{\udot};        p+(.5,0) *{\usebox{\pright}} *{\usebox{\nleft}};
       p+(.5,0) *{\udot};        p+(.5,0) *{\usebox{\pright}} *{\usebox{\nleft}};
       p+(.5,0) *{\udot};
       p+(.7,0) *{\shortstack{$E$\\ $H$ \\$F$}}
    \end{xy}\]

%    \[\xymatrix @!0 @C=12mm{
%     **[l] \cdots \ar@/^5mm/[r] \ar@/_5mm/@{<-}[r] &
%     \udot \ar@/^5mm/[r] \ar@/_5mm/@{<-}[r] &
%     \udot \ar@/^5mm/[r] \ar@/_5mm/@{<-}[r] &
%     \udot \ar@/^5mm/[r] \ar@/_5mm/@{<-}[r] &
%     \udot \ar@/^5mm/[r] \ar@/_5mm/@{<-}[r] & \udot
%     & \shortstack{$E$\\ \vspace{1mm} \\ $H$ \\ \vspace{1mm} \\$F$}
%    }\]

   \item a highest weight, and infinite in the other direction:
    \[\begin{xy}
       (0,0)    *{\udot} *++{\,};p+(.5,0) *{\usebox{\pright}} *{\usebox{\nleft}};
       p+(.5,0) *{\udot};        p+(.5,0) *{\usebox{\pright}} *{\usebox{\nleft}};
       p+(.5,0) *{\udot};        p+(.5,0) *{\usebox{\pright}} *{\usebox{\nleft}};
       p+(.5,0) *{\udot};        p+(.5,0) *{\usebox{\pright}} *{\usebox{\nleft}};
       p+(.5,0) *{\udot};        p+(.5,0) *{\usebox{\pright}} *{\usebox{\nleft}};
       p+(.5,0) *!L{\cdots} *++{\,};
       p+(.7,0) *{\shortstack{$E$\\ $H$ \\$F$}}
    \end{xy}\]

%    \[\xymatrix @!0 @C=12mm{
%     \udot \ar@/^5mm/[r] \ar@/_5mm/@{<-}[r] &
%     \udot \ar@/^5mm/[r] \ar@/_5mm/@{<-}[r] &
%     \udot \ar@/^5mm/[r] \ar@/_5mm/@{<-}[r] &
%     \udot \ar@/^5mm/[r] \ar@/_5mm/@{<-}[r] &
%     \udot \ar@/^5mm/[r] \ar@/_5mm/@{<-}[r] & **[r] \cdots
%     & \shortstack{$E$\\ \vspace{1mm} \\ $H$ \\ \vspace{1mm} \\$F$}
%    }\]

   \item we have a highest weight and a lowest weight, in which case it is
    finite dimensional
    $\begin{xy}
       (0,0)    *{\udot} *++{\,};p+(.5,0) *{\usebox{\pright}} *{\usebox{\nleft}};
       p+(.5,0) *{\udot};        p+(.5,0) *{\usebox{\pright}} *{\usebox{\nleft}};
       p+(.5,0) *!L{\cdots};
       p+(.9,0) *{\usebox{\pright}} *{\usebox{\nleft}};
       p+(.5,0) *{\udot};        p+(.5,0) *{\usebox{\pright}} *{\usebox{\nleft}};
       p+(.5,0) *{\udot} *++{\,};
       p+(.7,0) *{\shortstack{$E$\\ $H$ \\$F$}}
    \end{xy}$

%    \[\xymatrix @!0 @C=12mm{
%     \udot \ar@/^5mm/[r] \ar@/_5mm/@{<-}[r] &
%     \udot \ar@/^5mm/[r] \ar@/_5mm/@{<-}[r] &
%     \cdots \ar@/^5mm/[r] \ar@/_5mm/@{<-}[r] &
%     \udot \ar@/^5mm/[r] \ar@/_5mm/@{<-}[r] &
%     \udot
%     & \shortstack{$E$\\ \vspace{1mm} \\ $H$ \\ \vspace{1mm} \\$F$}
%    }\]

 \end{itemize}
 We'll see that all these show up. We also see that an irreducible representation is
 completely determined once we know $\lambda$ and some $n$ for which $V_n\neq 0$. The
 remaining question is to construct representations with all possible values of
 $\lambda\in \CC$ and $n\in \ZZ$. $n$ is an integer because it must be a
 representations of the circle.

 If $n$ is even, we have
 \savebox{\upright}{\raisebox{2ex}{\xymatrix @!0 @C=2.5em {\ar@/^1em/[r] & }}}
 \savebox{\downleft}{\raisebox{-2ex}{\xymatrix @!0 @C=2.5em {\ar@/_1em/@{<-}[r] & }}}
 \[
  \begin{xy}<3em,0em>:
   (0,0) *!R{\cdots} *++{\,}; p+(.5,0) *{\usebox{\upright}} *{\usebox{\downleft}};
   p+(.5,0) *{-6};       p+(.5,0) *{\usebox{\upright}} *{\usebox{\downleft}};
   p+(.5,0) *{-4};       p+(.5,0) *{\usebox{\upright}} *{\usebox{\downleft}};
   p+(.5,0) *{-2};       p+(.5,0) *{\usebox{\upright}} *{\usebox{\downleft}};
   p+(.5,0) *{0};         p+(.5,0) *{\usebox{\upright}} *{\usebox{\downleft}};
   p+(.5,0) *{2};       p+(.5,0) *{\usebox{\upright}} *{\usebox{\downleft}};
   p+(.5,0) *{4};       p+(.5,0) *{\usebox{\upright}} *{\usebox{\downleft}};
   p+(.5,0) *{6};       p+(.5,0) *{\usebox{\upright}} *{\usebox{\downleft}};
   p+(.5,0) *!L{\cdots} *++{\,};
   p+(.7,0) *{\shortstack{$E$\\ \vspace{.25em} \\ $H$ \\ \vspace{.25em} \\$F$}};
   (.5,.65) *{{}^\frac{\lambda-7}{2}};
   p+(1,0) *{{}^\frac{\lambda-5}{2}};
   p+(1,0) *{{}^\frac{\lambda-3}{2}};
   p+(1,0) *{{}^\frac{\lambda-1}{2}};
   p+(1,0) *{{}^\frac{\lambda+1}{2}};
   p+(1,0) *{{}^\frac{\lambda+3}{2}};
   p+(1,0) *{{}^\frac{\lambda+5}{2}};
   p+(1,0) *{{}^\frac{\lambda+7}{2}};
   (.5,-.65) *{{}^\frac{\lambda+7}{2}};
   p+(1,0) *{{}^\frac{\lambda+5}{2}};
   p+(1,0) *{{}^\frac{\lambda+3}{2}};
   p+(1,0) *{{}^\frac{\lambda+1}{2}};
   p+(1,0) *{{}^\frac{\lambda-1}{2}};
   p+(1,0) *{{}^\frac{\lambda-3}{2}};
   p+(1,0) *{{}^\frac{\lambda-5}{2}};
   p+(1,0) *{{}^\frac{\lambda-7}{2}};
 \end{xy}
 \]
%    \[\xymatrix @!0 @C=12mm{
%     **[l] \cdots \ar@/^5mm/[r]^{\frac{\lambda-7}{2}} \ar@/_5mm/@{<-}[r]_{\frac{\lambda+7}{2}} &
%     {-6} \ar@/^5mm/[r]^{\frac{\lambda-5}{2}} \ar@/_5mm/@{<-}[r]_{\frac{\lambda+5}{2}} &
%     {-4} \ar@/^5mm/[r]^{\frac{\lambda-3}{2}} \ar@/_5mm/@{<-}[r]_{\frac{\lambda+3}{2}} &
%     {-2} \ar@/^5mm/[r]^{\frac{\lambda-1}{2}} \ar@/_5mm/@{<-}[r]_{\frac{\lambda+1}{2}} &
%     {0}  \ar@/^5mm/[r]^{\frac{\lambda+1}{2}} \ar@/_5mm/@{<-}[r]_{\frac{\lambda-1}{2}} &
%     {2}  \ar@/^5mm/[r]^{\frac{\lambda+3}{2}} \ar@/_5mm/@{<-}[r]_{\frac{\lambda-3}{2}} &
%     {4}  \ar@/^5mm/[r]^{\frac{\lambda+5}{2}} \ar@/_5mm/@{<-}[r]_{\frac{\lambda-5}{2}} &
%     {6}  \ar@/^5mm/[r]^{\frac{\lambda+7}{2}} \ar@/_5mm/@{<-}[r]_{\frac{\lambda-7}{2}} & **[r] \cdots
%     & \shortstack{$E$\\ \vspace{1mm} \\ $H$ \\ \vspace{1mm} \\$F$}\\
%    }\]

 It is easy to check that these maps satisfy $[E,F]=H$, $[H,E]=2E$, and $[H,F]=-2F$

 \begin{exercise}
   Do the case of $n$ odd.
 \end{exercise}

 Problem: These may not be irreducible, and we want to decompose them into irreducible
 representations. The only way they can fail to be irreducible if if $Ev_n=0$ of
 $Fv_n=0$ for some $n$ (otherwise, from any vector, you can generate the whole space).
 The only ways that can happen is if
 \[\begin{tabular}{l}
   $n$ even: $\lambda$ an odd integer\\
   $n$ odd: $\lambda$ an even integer.\\
 \end{tabular}\]
 What happens in these cases? The easiest thing is probably just to write out an
 example.
 \begin{example}
   Take $n$ even, and $\lambda=3$, so we have
 \[
  \begin{xy}<3em,0em>:
   (0,0) *!R{\cdots} *++{\,}; p+(.5,0) *{\usebox{\upright}} *{\usebox{\downleft}};
   p+(.5,0) *{-6};       p+(.5,0) *{\usebox{\upright}} *{\usebox{\downleft}};
   p+(.5,0) *{-4};       p+(.5,0) *{\usebox{\upright}} *{\usebox{\downleft}};
   p+(.5,0) *{-2};       p+(.5,0) *{\usebox{\upright}} *{\usebox{\downleft}};
   p+(.5,0) *{0};         p+(.5,0) *{\usebox{\upright}} *{\usebox{\downleft}};
   p+(.5,0) *{2};       p+(.5,0) *{\usebox{\upright}} *{\usebox{\downleft}};
   p+(.5,0) *{4};       p+(.5,0) *{\usebox{\upright}} *{\usebox{\downleft}};
   p+(.5,0) *{6};       p+(.5,0) *{\usebox{\upright}} *{\usebox{\downleft}};
   p+(.5,0) *!L{\cdots} *++{\,};
   p+(.7,0) *{\shortstack{$E$\\ \vspace{.25em} \\ $H$ \\ \vspace{.25em} \\$F$}};
   (.5,.55) *{{}^{-2}};
   p+(1,0) *{{}^{-1}};
   p+(1,0) *{{}^0};
   p+(1,0) *{{}^1};
   p+(1,0) *{{}^2};
   p+(1,0) *{{}^3};
   p+(1,0) *{{}^4};
   p+(1,0) *{{}^5};
   (.5,-.65) *{{}^5};
   p+(1,0) *{{}^4};
   p+(1,0) *{{}^3};
   p+(1,0) *{{}^2};
   p+(1,0) *{{}^1};
   p+(1,0) *{{}^0};
   p+(1,0) *{{}^{-1}};
   p+(1,0) *{{}^{-2}};
   (0,1);(0,-1) **\crv{(3.3,1)&(3.3,-1)};
   (8,1);(8,-1) **\crv{(4.7,1)&(4.7,-1)};
 \end{xy}
 \]
%    \[\xymatrix @!0 @R=12mm @C=12mm{
%     {} \POS[]; [dd]**\crv{<4cm,0cm>&<4cm,-24mm>}
%     & & & & & & & & {} \POS[]; [dd]**\crv{<5.7cm,0cm>&<5.7cm,-24mm>}\\
%     **[l] \cdots \ar@/^5mm/[r]^{-2} \ar@/_5mm/@{<-}[r]_{5} &
%     {-6} \ar@/^5mm/[r]^{-1} \ar@/_5mm/@{<-}[r]_{4} &
%     {-4} \ar@/^5mm/[r]^{0} \ar@/_5mm/@{<-}[r]_{3} &
%     {-2} \ar@/^5mm/[r]^{1} \ar@/_5mm/@{<-}[r]_{2} &
%     {0}  \ar@/^5mm/[r]^{2} \ar@/_5mm/@{<-}[r]_{1} &
%     {2}  \ar@/^5mm/[r]^{3} \ar@/_5mm/@{<-}[r]_{0} &
%     {4}  \ar@/^5mm/[r]^{4} \ar@/_5mm/@{<-}[r]_{-1} &
%     {6}  \ar@/^5mm/[r]^{5} \ar@/_5mm/@{<-}[r]_{-2} & **[r] \cdots
%     & \shortstack{$E$\\ \vspace{1mm} \\ $H$ \\ \vspace{1mm} \\$F$}\\
%     {} & & & & & & & & {}\\
%    }\]
   You can just see what the irreducible subrepresentations are ... they are shown in
   the picture. So $V$ has two irreducible subrepresentations $V_-$ and $V_+$, and
   $V/(V_-\oplus V_+)$ is an irreducible 3 dimensional representation.
 \end{example}
 \begin{example}
   If $n$ is even, but $\lambda$ is negative, say $\lambda=-3$, we get
 \[
  \begin{xy}<3em,0em>:
   (0,0) *!R{\cdots} *++{\,}; p+(.5,0) *{\usebox{\upright}} *{\usebox{\downleft}};
   p+(.5,0) *{-6};       p+(.5,0) *{\usebox{\upright}} *{\usebox{\downleft}};
   p+(.5,0) *{-4};       p+(.5,0) *{\usebox{\upright}} *{\usebox{\downleft}};
   p+(.5,0) *{-2};       p+(.5,0) *{\usebox{\upright}} *{\usebox{\downleft}};
   p+(.5,0) *{0};         p+(.5,0) *{\usebox{\upright}} *{\usebox{\downleft}};
   p+(.5,0) *{2};       p+(.5,0) *{\usebox{\upright}} *{\usebox{\downleft}};
   p+(.5,0) *{4};       p+(.5,0) *{\usebox{\upright}} *{\usebox{\downleft}};
   p+(.5,0) *{6};       p+(.5,0) *{\usebox{\upright}} *{\usebox{\downleft}};
   p+(.5,0) *!L{\cdots} *++{\,};
   p+(.7,0) *{\shortstack{$E$\\ \vspace{.25em} \\ $H$ \\ \vspace{.25em} \\$F$}};
   (.5,.55) *{{}^{-5}};
   p+(1,0) *{{}^{-4}};
   p+(1,0) *{{}^{-3}};
   p+(1,0) *{{}^{-2}};
   p+(1,0) *{{}^{-1}};
   p+(1,0) *{{}^0};
   p+(1,0) *{{}^1};
   p+(1,0) *{{}^2};
   (.5,-.65) *{{}^2};
   p+(1,0) *{{}^1};
   p+(1,0) *{{}^0};
   p+(1,0) *{{}^{-1}};
   p+(1,0) *{{}^{-2}};
   p+(1,0) *{{}^{-3}};
   p+(1,0) *{{}^{-4}};
   p+(1,0) *{{}^{-5}};
   (4,1);(4,1) **\crv{(2.5,1)&(2.2,-1.1)&(5.8,-1.1)&(5.5,1)};
 \end{xy}
 \]
%    \[\xymatrix @!0 @R=12mm @C=12mm{
%     & & & &
%     {\!\!\!-\!\!\!} \POS[]; [] **\crv{<3cm,0cm>&<3cm,-22mm>&<5cm,-24mm>&<6.5cm,-22mm>&<6.5cm,0mm>}
%     & & & & \\
%     **[l] \cdots \ar@/^5mm/[r]^{-4} \ar@/_5mm/@{<-}[r]_{2} &
%     {-6} \ar@/^5mm/[r]^{-3} \ar@/_5mm/@{<-}[r]_{1} &
%     {-4} \ar@/^5mm/[r]^{-1} \ar@/_5mm/@{<-}[r]_{0} &
%     {-2} \ar@/^5mm/[r]^{-2} \ar@/_5mm/@{<-}[r]_{-1} &
%     {0}  \ar@/^5mm/[r]^{-1} \ar@/_5mm/@{<-}[r]_{-2} &
%     {2}  \ar@/^5mm/[r]^{0} \ar@/_5mm/@{<-}[r]_{-3} &
%     {4}  \ar@/^5mm/[r]^{1} \ar@/_5mm/@{<-}[r]_{-4} &
%     {6}  \ar@/^5mm/[r]^{2} \ar@/_5mm/@{<-}[r]_{-5} & **[r] \cdots
%     & \shortstack{$E$\\ \vspace{1mm} \\ $H$ \\ \vspace{1mm} \\$F$}\\
%     & & & & & & & & \\
%    }\]
   Here we have an irreducible finite dimensional representation. If you quotient out
   by that subrepresentation, you get $V_+\oplus V_-$.
 \end{example}
 \begin{exercise}
   Show that for $n$ odd, and $\lambda=0$, $V=V_+\oplus V_-$.
 \end{exercise}
 So we have a complete list of all irreducible admissible representations:
 \begin{enumerate}
   \item if $\lambda\not\in \ZZ$, you get one representation (remember $\lambda\equiv
   -\lambda$). This is the bi-infinite case.

   \item Finite dimensional representation for each $n\ge 1$ ($\lambda=\pm n$)

   \item Discrete series for each $\lambda\in \ZZ\smallsetminus \{0\}$, which is the
   half infinite case: you get a lowest weight when $\lambda< 0$ and a highest weight
   when $\lambda>0$.

   \item two ``limits of discrete series'' where $n$ is odd and $\lambda=0$.
 \end{enumerate}
 Which of these can be made into \emph{unitary} representations? $H^\dag = -H$,
 $E^\dag =F$, and $F^\dag = E$. If we have a hermitian inner product $(\ ,\,)$, we see
 that
 \begin{align*}
 (v_{j+2},v_{j+2}) &= \frac{2}{\lambda + j+1} (Ev_j,v_{j+2})\\
            &= \frac{2}{\lambda + j+1}(v_j,-Fv_{j+2}) \\
            &= - \frac{2}{\lambda + j+1} \frac{\overline{\lambda - j-1}}{2} (v_j,v_j)
            >0
 \end{align*}
 where we fix the sign errors. So we want $-\frac{\overline{\lambda-1-j}}{\lambda+j+1}$
 to be real and positive whenever $j,j+2$ are non-zero eigenvectors. So
 \[
    -(\lambda-1-j)(\lambda+1+j) = -\lambda^2 + (j+1)^2
 \]
 should be positive for all $j$. Conversely, when you have this, blah.

 This condition is satisfied in the following cases:
 \begin{enumerate}
   \item $\lambda^2\le 0$. These representations are called PRINCIPAL SERIES
   representations. These are all irreducible \emph{except} when $\lambda=0$ and $n$
   is odd, in which case it is the sum of two limits of discrete series representations

   \item $0< \lambda < 1$ and $j$ even. These are called COMPLEMENTARY SERIES. They
   are annoying, and you spend a lot of time trying to show that they don't occur.

   \item $\lambda^2 =n^2$ for $n\ge 1$ (for some of the irreducible pieces).

   If $\lambda=1$, we get
 \[
  \begin{xy}<3em,0em>:
   (0,0) *!R{\cdots} *++{\,}; p+(.5,0) *{\usebox{\upright}} *{\usebox{\downleft}};
   p+(.5,0) *{-6};       p+(.5,0) *{\usebox{\upright}} *{\usebox{\downleft}};
   p+(.5,0) *{-4};       p+(.5,0) *{\usebox{\upright}} *{\usebox{\downleft}};
   p+(.5,0) *{-2};       p+(.5,0) *{\usebox{\upright}} *{\usebox{\downleft}};
   p+(.5,0) *{0};         p+(.5,0) *{\usebox{\upright}} *{\usebox{\downleft}};
   p+(.5,0) *{2};       p+(.5,0) *{\usebox{\upright}} *{\usebox{\downleft}};
   p+(.5,0) *{4};       p+(.5,0) *{\usebox{\upright}} *{\usebox{\downleft}};
   p+(.5,0) *{6};       p+(.5,0) *{\usebox{\upright}} *{\usebox{\downleft}};
   p+(.5,0) *!L{\cdots} *++{\,};
   p+(.7,0) *{\shortstack{$E$\\ \vspace{.25em} \\ $H$ \\ \vspace{.25em} \\$F$}};
   (.5,.55) *{{}^{-3}};
   p+(1,0) *{{}^{-2}};
   p+(1,0) *{{}^{-1}};
   p+(1,0) *{{}^0};
   p+(1,0) *{{}^1};
   p+(1,0) *{{}^2};
   p+(1,0) *{{}^3};
   p+(1,0) *{{}^4};
   (.5,-.65) *{{}^4};
   p+(1,0) *{{}^3};
   p+(1,0) *{{}^2};
   p+(1,0) *{{}^1};
   p+(1,0) *{{}^0};
   p+(1,0) *{{}^{-1}};
   p+(1,0) *{{}^{-2}};
   p+(1,0) *{{}^{-3}};
   (0,1);(0,-1) **\crv{(4.6,1.2)&(4.6,-1.2)};
   (8,1);(8,-1) **\crv{(3.4,1.2)&(3.4,-1.2)};
 \end{xy}
 \]
%    \[\xymatrix @!0 @R=12mm @C=12mm{
%     {} \POS[]; [dd]**\crv{<55mm,3mm>&<55mm,-27mm>}
%     & & & & & & & & {} \POS[]; [dd]**\crv{<41mm,3mm>&<41mm,-27mm>}\\
%     **[l] \cdots \ar@/^5mm/[r]^{-3} \ar@/_5mm/@{<-}[r]_{4} &
%     {-6} \ar@/^5mm/[r]^{-2} \ar@/_5mm/@{<-}[r]_{3} &
%     {-4} \ar@/^5mm/[r]^{-1} \ar@/_5mm/@{<-}[r]_{2} &
%     {-2} \ar@/^5mm/[r]^{0} \ar@/_5mm/@{<-}[r]_{1} &
%     {0}  \ar@/^5mm/[r]^{1} \ar@/_5mm/@{<-}[r]_{0} &
%     {2}  \ar@/^5mm/[r]^{2} \ar@/_5mm/@{<-}[r]_{-1} &
%     {4}  \ar@/^5mm/[r]^{3} \ar@/_5mm/@{<-}[r]_{-2} &
%     {6}  \ar@/^5mm/[r]^{4} \ar@/_5mm/@{<-}[r]_{-3} & **[r] \cdots
%     & \shortstack{$E$\\ \vspace{1mm} \\ $H$ \\ \vspace{1mm} \\$F$}\\
%     {} & & & & & & & & {}\\
%    }\]
   We see that we get two discrete series and a 1 dimensional representation, all of
   which are unitary

   For $\lambda=2$ (this is the more generic one), we have
 \[
  \begin{xy}<3em,0em>:
   (0,0) *!R{\cdots} *++{\,}; p+(.5,0) *{\usebox{\upright}} *{\usebox{\downleft}};
   p+(.5,0) *{-5};       p+(.5,0) *{\usebox{\upright}} *{\usebox{\downleft}};
   p+(.5,0) *{-3};       p+(.5,0) *{\usebox{\upright}} *{\usebox{\downleft}};
   p+(.5,0) *{-1};       p+(.5,0) *{\usebox{\upright}} *{\usebox{\downleft}};
   p+(.5,0) *{1};       p+(.5,0) *{\usebox{\upright}} *{\usebox{\downleft}};
   p+(.5,0) *{3};       p+(.5,0) *{\usebox{\upright}} *{\usebox{\downleft}};
   p+(.5,0) *{5};       p+(.5,0) *{\usebox{\upright}} *{\usebox{\downleft}};
   p+(.5,0) *!L{\cdots} *++{\,};
   p+(.7,0) *{\shortstack{$E$\\ \vspace{.25em} \\ $H$ \\ \vspace{.25em} \\$F$}};
   (.5,.55) *{{}^{-2}};
   p+(1,0) *{{}^{-1}};
   p+(1,0) *{{}^0};
   p+(1,0) *{{}^1};
   p+(1,0) *{{}^2};
   p+(1,0) *{{}^3};
   p+(1,0) *{{}^4};
   (.5,-.65) *{{}^4};
   p+(1,0) *{{}^3};
   p+(1,0) *{{}^2};
   p+(1,0) *{{}^1};
   p+(1,0) *{{}^0};
   p+(1,0) *{{}^{-1}};
   p+(1,0) *{{}^{-2}};
   (0,1);(0,-1) **\crv{(3.2,1)&(3.2,-1)};
   (7,1);(7,-1) **\crv{(3.8,1)&(3.8,-1)};
 \end{xy}
 \]
%    \[\xymatrix @!0 @R=12mm @C=12mm{
%     {} \POS[]; [dd]**\crv{<4cm,0cm>&<4cm,-24mm>}
%     & & & & & & & {} \POS[]; [dd]**\crv{<45mm,0cm>&<45mm,-24mm>}\\
%     **[l] \cdots \ar@/^5mm/[r]^{-2} \ar@/_5mm/@{<-}[r]_{4} &
%     {-5} \ar@/^5mm/[r]^{-1} \ar@/_5mm/@{<-}[r]_{3} &
%     {-3} \ar@/^5mm/[r]^{0} \ar@/_5mm/@{<-}[r]_{2} &
%     {-1} \ar@/^5mm/[r]^{1} \ar@/_5mm/@{<-}[r]_{1} &
%     {1}  \ar@/^5mm/[r]^{2} \ar@/_5mm/@{<-}[r]_{0} &
%     {3}  \ar@/^5mm/[r]^{3} \ar@/_5mm/@{<-}[r]_{-1} &
%     {5}  \ar@/^5mm/[r]^{4} \ar@/_5mm/@{<-}[r]_{-2} & **[r] \cdots
%     & \shortstack{$E$\\ \vspace{1mm} \\ $H$ \\ \vspace{1mm} \\$F$}\\
%     {} & & & & & & & {}\\
%    }\]
%    \mpar{What is going on here? $\lambda\neq n^2$}

   The middle representation (where $(j+1)^2<\lambda^2=4$ is NOT unitary, which we
   already knew. So the DISCRETE SERIES representations ARE unitary, and the FINITE
   dimensional representations of dimension greater than or equal to 2 are NOT.
 \end{enumerate}

 Summary: the irreducible unitary representations of $SL_2(\RR)$ are given by
 \begin{enumerate}
   \item the 1 dimensional representation
   \item Discrete series representations for any $\lambda\in \ZZ\smallsetminus \{0\}$
   \item Two limit of discrete series representations for $\lambda=0$
   \item Two series of principal series representations:
   \[\begin{tabular}{l}
     $j$ even: $\lambda\in i\RR$, $\lambda \ge 0$\\
     $j$ odd: $\lambda \in i\RR$, $\lambda > 0$
   \end{tabular}\]
   \item Complementary series: parameterized by $\lambda$, with $0< \lambda < 1$.
 \end{enumerate}

 The nice stuff that happened for $SL_2(\RR)$ breaks down for more complicated Lie
 groups.

 Representations of finite covers of $SL_2(\RR)$ are similar, except $j$ need not be
 integral. For example, for the double cover $\widehat{SL_2(\RR)} = Mp_2(\RR)$, $2j\in
 \ZZ$.
 \begin{exercise}
   Find the irreducible unitary representations of $Mp_2(\RR)$.
 \end{exercise}

 \index{Borcherds, Richard E.|)}
}    % Sevak Mkrtchyan, sevak@math
 {\ifthenelse{\boolean{proofmode}}{}{%
    \Closesolutionfile{exSolutions}
    \sektion{Solutions to (some) Exercises}
    \input{ExerciseSolutions}}}
 {\newpage \gdef\sectitle{References}
 \addcontentsline{toc}{section}{References}
  \nocite{Adams:LELG,
          Adams:LLG,
          Borel:LAG,
          CSM:LLGLA,
          Fuchs,
          FulHar,
          Hatcher,
          Helgason,
          Humphreys:LART,
          Kac:IDLA,
          Knapp:LGBI,
          Knutson:261A,
          Knutson:261B,
          Lee:ISM,
          Montgomery,
          OnishchikVinberg,
          Serre:CSLA,
          Warner}
  \bibliography{LieGroupsBib}
  \bibliographystyle{alpha}
  }
 {\newcommand\indexpreamble{%
  \textbf{Bold} page numbers indicate that the index entry was defined, used
  in a theorem, or proven on that page. \textit{Italic} page numbers indicate that the index
  entry was exemplified or used in an example on that page. If the index entry is a
  result, then the page number is \textbf{bold} only for the pages on which the result
  is proven.
 \bigskip}\printindex
 }

 \anton{Other stuff you should know: Bruhat decomposition, Harish-Chandra's Theorem
 (lec 19-ish), Levi decomposition}

\end{document}
