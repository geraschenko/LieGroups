 \stepcounter{lecture}
 \setcounter{lecture}{29}
 \sektion{Lecture 29}

 Split form of Lie algebra (we did this for $A_n$, $D_n$, $E_6$, $E_7$, $E_8$): $A=\bigoplus
 \hat e^\alpha \oplus L$. Compact form $A^++iA^-$, where $A^\pm$ eigenspaces of
 $\w:\hat e^\alpha\mapsto (-1)^{\alpha^2/2}\hat e^{-\alpha}$.

 We talked about other involutions of the compact form. You get all the other forms
 this way.

 The idea now is to find ALL real simple Lie algebras by listing all involutions of
 the compact form. We will construct all of them, but we won't prove that we have all
 of them.

 We'll use Kac's method for classifying all automorphisms of order $N$ of a compact
 Lie algebra (and we'll only use the case $N=2$). First let's look at inner
 automorphisms. Write down the AFFINE Dynkin diagram
 \[\begin{xy}
   (0,0) *+!U{2} *\cir<2pt>{};
   p-(1,0) *+!U{4} *\cir<2pt>{} **@{-};
   p-(1,0) *+!U{6} *\cir<2pt>{} **@{-};
       p+(0,+1) *+!L{3} *\cir<2pt>{} **@{-},
   p-(1,0) *+!U{5} *\cir<2pt>{} **@{-};
   p-(1,0) *+!U{4} *\cir<2pt>{} **@{-};
  p-(1,0) *+!U{3} *\cir<2pt>{} **@{-};
   p-(1,0) *+!U{2} *\cir<2pt>{} **@{-};
   p-(1,0) *+!U{1} *++!R{-\text{highest weight} =} *\cir<2pt>{} **@{-};
 \end{xy}\]
 Choose $n_i$ with $\sum n_im_i=N$ where the $m_i$ are the numbers on the diagram. We
 have an automorphism $e^{\alpha_j}\mapsto e^{2\pi i n_j/N}e^{\alpha_j}$ induces an
 automorphism of order dividing $N$. This is obvious. The point of Kac's theorem is
 that all inner automorphisms of order dividing $N$ are obtained this way and are
 conjugate if and only if they are conjugate by an automorphism of the Dynkin diagram.
 We won't actually prove Kac's theorem because we just want to get a bunch of
 examples. See \cite{Kac:IDLA} or \cite{Helgason}.

 \begin{example}
   Real forms of $E_8$. We've already found three, and it took us a long time. We can
   now do it fast. We need to solve $\sum n_i m_i =2$ where $n_i\ge 0$; there are only
   a few possibilities:
   \[
   \begin{tabular}{clcl}
     $\sum n_im_i=2$ & \# of ways & how to do it & \txt{maximal compact\\ subgroup $K$}\\
     $2\times 1$ & one way &
        $\begin{xy}<1.25em,0em>:
         (0,0) *\cir<2pt>{};
         p-(1,0) *\cir<2pt>{} **@{-};
         p-(1,0) *\cir<2pt>{} **@{-};
             p+(0,+1) *\cir<2pt>{} **@{-},
         p-(1,0) *\cir<2pt>{} **@{-};
         p-(1,0) *\cir<2pt>{} **@{-};
         p-(1,0) *\cir<2pt>{} **@{-};
         p-(1,0) *\cir<2pt>{} **@{-};
         p-(1,0) *{\times} *\cir<2pt>{} **@{-};
        \end{xy}$ & $E_8$ (compact form)\\
     $1\times 2$ & two ways&
        $\begin{xy}<1.25em,0em>:
         (0,0) *\cir<2pt>{};
         p-(1,0) *\cir<2pt>{} **@{-};
         p-(1,0) *\cir<2pt>{} **@{-};
             p+(0,+1) *\cir<2pt>{} **@{-},
         p-(1,0) *\cir<2pt>{} **@{-};
         p-(1,0) *\cir<2pt>{} **@{-};
         p-(1,0) *\cir<2pt>{} **@{-};
         p-(1,0) *{\times} *\cir<2pt>{} **@{-};
         p-(1,0) *\cir<2pt>{} **@{-};
        \end{xy}$ & $A_1 E_7$\\
      & &
        $\begin{xy}<1.25em,0em>:
         (0,0) *{\times} *\cir<2pt>{};
         p-(1,0) *\cir<2pt>{} **@{-};
         p-(1,0) *\cir<2pt>{} **@{-};
             p+(0,+1) *\cir<2pt>{} **@{-},
         p-(1,0) *\cir<2pt>{} **@{-};
         p-(1,0) *\cir<2pt>{} **@{-};
         p-(1,0) *\cir<2pt>{} **@{-};
         p-(1,0) *\cir<2pt>{} **@{-};
         p-(1,0) *\cir<2pt>{} **@{-};
        \end{xy}$ & $D_8$ (split form)\\
     $1\times 1+1\times 1$ & no ways
   \end{tabular}
   \]
%   $2\times 1=2$, $1\times 2=2$, and $1\times 1+1\times 1=2$.
%   There is only one node of weight 1, and two nodes of weight 2, so we can do the
%   first way one way, the second way two ways, and we can't do the third way. These
%   correspond to the three things we found last time
%
%   pictures with comments
   The points NOT crossed off form the Dynkin diagram of the maximal compact subgroup.
   Thus, by just looking at the diagram, we can see what all the real forms are!
 \end{example}
 \begin{example}
   Let's do $E_7$. Write down the affine diagram:
   \[
   \begin{xy}
      (0,0) *+!D{1} *\cir<2pt>{};
      p+(1,0) *+!D{2} *\cir<2pt>{} **@{-};
      p+(1,0) *+!D{3} *\cir<2pt>{} **@{-};
      p+(1,0) *+!D{4} *\cir<2pt>{} **@{-};
           p+(0,-1) *+!L{2} *\cir<2pt>{} **@{-},
      p+(1,0) *+!D{3} *\cir<2pt>{} **@{-};
      p+(1,0) *+!D{2} *\cir<2pt>{} **@{-};
      p+(1,0) *+!D{1} *\cir<2pt>{} **@{-};
    \end{xy}
   \]
   We get the possibilities
    \[
   \begin{tabular}{clcl}
     $\sum n_im_i=2$ & \# of ways & how to do it & \txt{maximal compact\\ subgroup $K$}\\
     $2\times 1$ & one way* &
      \begin{xy}<1.25em,0em>:
         (0,0) *{\times} *\cir<2pt>{};
         p+(1,0) *\cir<2pt>{} **@{-};
         p+(1,0) *\cir<2pt>{} **@{-};
         p+(1,0) *\cir<2pt>{} **@{-};
              p+(0,-1) *\cir<2pt>{} **@{-},
         p+(1,0) *\cir<2pt>{} **@{-};
         p+(1,0) *\cir<2pt>{} **@{-};
         p+(1,0) *\cir<2pt>{} **@{-};
       \end{xy} & $E_7$ (compact form)\\
     $1\times 2$ & two ways*&
      \begin{xy}<1.25em,0em>:
         (0,0) *\cir<2pt>{};
         p+(1,0) *{\times} *\cir<2pt>{} **@{-};
         p+(1,0) *\cir<2pt>{} **@{-};
         p+(1,0) *\cir<2pt>{} **@{-};
              p+(0,-1) *\cir<2pt>{} **@{-},
         p+(1,0) *\cir<2pt>{} **@{-};
         p+(1,0) *\cir<2pt>{} **@{-};
         p+(1,0) *\cir<2pt>{} **@{-};
       \end{xy} & $A_1 D_6$\\
     & &
      \begin{xy}<1.25em,0em>:
         (0,0) *\cir<2pt>{};
         p+(1,0) *\cir<2pt>{} **@{-};
         p+(1,0) *\cir<2pt>{} **@{-};
         p+(1,0) *\cir<2pt>{} **@{-};
              p+(0,-1) *{\times} *\cir<2pt>{} **@{-},
         p+(1,0) *\cir<2pt>{} **@{-};
         p+(1,0) *\cir<2pt>{} **@{-};
         p+(1,0) *\cir<2pt>{} **@{-};
       \end{xy} & $A_7$ (split form)**\\
     $1\times 1+1\times 1$ & one way &
      \begin{xy}<1.25em,0em>:
         (0,0) *{\times} *\cir<2pt>{};
         p+(1,0) *\cir<2pt>{} **@{-};
         p+(1,0) *\cir<2pt>{} **@{-};
         p+(1,0) *\cir<2pt>{} **@{-};
              p+(0,-1) *\cir<2pt>{} **@{-},
         p+(1,0) *\cir<2pt>{} **@{-};
         p+(1,0) *\cir<2pt>{} **@{-};
         p+(1,0) *{\times} *\cir<2pt>{} **@{-};
       \end{xy} & $E_6\oplus \RR$ \ ***\\
   \end{tabular}
   \]
   (*) The number of ways is counted up to automorphisms of the diagram.\\
   (**) In the split real form, the maximal compact subgroup has dimension equal to
        half the number of roots. The roots of $A_7$ look like $\e_i-\e_j$ for $i,j\le
        8$ and $i\neq j$, so the dimension is $8\cdot 7 + 7 = 56 = \frac{112}{2}$.\\
%   We can delete one node of weight 1, giving the compact form (the two ways are
%   conjugate by an automorphism of the affine Dynkin diagram):
%
%   picture
%
%   Next, you could delete one node of weight 2 ... three ways, but two are conjugate:
%
%   two pictures
%
%   The split form is something ... the maximal compact of the split for has dimension
%   half the number of roots.
%
%   Finally, you could delete two nodes of weight 1:
%
%   picture
%
   (***) The maximal compact subgroup is $E_6\oplus \RR$ because the fixed subalgebra
   contains the whole Cartan subalgebra, and the $E_6$ only accounts for $6$ of the $7$
   dimensions. You can use this to construct some interesting representations of $E_6$
   (the minuscule ones). How does the algebra $E_7$ decompose as a representation of
   the algebra $E_6\oplus \RR$?

   We can decompose it according to the eigenvalues of $\RR$. The $E_6\oplus \RR$ is
   the zero eigenvalue of $\RR$ [why?], and the rest is 54 dimensional. The easy way to see
   the decomposition is to look at the roots. Remember when we computed the Weyl group
   we looked for vectors like
   \[\begin{xy}
     (0,0) *\cir<2pt>{}; (1,0) *\cir<2pt>{} **@{-}; (2,0) *\cir<2pt>{} **@{.}
   \end{xy} \qquad\qquad \text{or} \qquad\qquad
   \begin{xy}
     (0,0) *\cir<2pt>{}; (1,0) *\cir<2pt>{} **@{.}; (2,0) *\cir<2pt>{} **@{-}
   \end{xy}\]
   The 27 possibilities (for each) form the weights of a 27 dimensional representation
   of $E_6$. The orthogonal complement of the two nodes is an $E_6$ root system whose
   Weyl group acts transitively on these 27 vectors (we showed that these form a
   single orbit, remember?). Vectors of the $E_7$ root system are the vectors of the
   $E_6$ root system plus these 27 vectors plus the other 27 vectors. This splits up the
   $E_7$ explicitly. The two 27s form single orbits, so they are irreducible. Thus,
   $E_7\cong E_6\oplus \RR\oplus 27\oplus 27$, and the 27s are minuscule.
 \end{example}
 Let $K$ be a maximal compact subgroup, with Lie algebra $\RR + E_6$. The factor of
 $\RR$ means that $K$ has an $S^1$ in its center. Now look at the space $G/K$, where
 $G$ is the Lie group of type $E_7$, and $K$ is the maximal compact subgroup. It is a
 \emph{Hermitian symmetric space}\index{symmetric space|idxbf}. Symmetric space means
 that it is a (simply connected) Riemannian manifold $M$ such that for each point
 $p\in M$, there is an automorphism fixing $p$ and acting as $-1$ on the tangent
 space. This looks weird, but it turns out that all kinds of nice objects you know
 about are symmetric spaces. Typical examples you may have seen: spheres $S^n$,
 hyperbolic space $\HH^n$, and Euclidean space $\RR^n$. Roughly speaking, symmetric
 spaces have nice properties of these spaces. Cartan\index{Cartan} classified all
 symmetric spaces: they are non-compact simple Lie groups modulo the maximal compact
 subgroup (more or less ... depending on simply connectedness hypotheses 'n such).
 Historically, Cartan classified simple Lie groups, and then later classified
 symmetric spaces, and was surprised to find the same result. Hermitian symmetric
 spaces are just symmetric spaces with a complex structure. A standard example of this
 is the upper half plane $\{x+iy|y>0\}$. It is acted on by $SL_2(\RR)$, which acts by
 $\matrix abcd \tau = \frac{a\tau + b}{c\tau + d}$.

 Let's go back to this $G/K$ and try to explain why we get a Hermitian symmetric space
 from it. We'll be rather sketchy here. First of all, to make it a symmetric space, we
 have to find a nice invariant Riemannian metric on it. It is sufficient to find a
 positive definite bilinear form on the tangent space at $p$ which is invariant under
 $K$ ... then you can translate it around. We can do this as $K$ is compact (so you
 have the averaging trick). Why is it Hermitian? We'll show that there is an almost
 complex structure. We have $S^1$ acting on the tangent space of each point because we
 have an $S^1$ in the center of the stabilizer of any given point. Identify this $S^1$
 with complex numbers of absolute value 1. This gives an invariant almost complex
 structure on $G/K$. That is, each tangent space is a complex vector space. Almost
 complex structures don't always come from complex structures, but this one does (it
 is integrable). Notice that it is a little unexpected that $G/K$ has a complex
 structure ($G$ and $K$ are odd dimensional in the case of $G=E_7$, $K=E_6\oplus \RR$,
 so they have no hope of having a complex structure).

 \begin{example}
   Let's look at $E_6$, with affine Dynkin diagram
   \[\begin{xy}
   (0,0) *+!D{1} *\cir<2pt>{};
   (1,0) *+!D{2} *\cir<2pt>{} **@{-};
   p+(1,0)="y" *+!D{3} *\cir<2pt>{} **@{-};
       p+(0,-1) *+!L{2} *\cir<2pt>{} **@{-};
       p+(0,-1) *+!L{1} *\cir<2pt>{} **@{-};
   "y" *{\hspace{4pt}};p+(1,0) *+!D{2} *\cir<2pt>{} **@{-};
   p+(1,0) *+!D{1} *\cir<2pt>{} **@{-};
  \end{xy}\]
  We get the possibilities
    \[
   \begin{tabular}{clcl}
     $\sum n_im_i=2$ & \# of ways & how to do it & \txt{maximal compact\\ subgroup $K$}\\
     $2\times 1$ & one way &
       \begin{xy}<1.75em,0em>:
        (0,0) *\cir<2pt>{};
        p+(1,0) *\cir<2pt>{} **@{-};
        p+(1,0)="y" *\cir<2pt>{} **@{-};
            p+(0,-1) *\cir<2pt>{} **@{-};
            p+(0,-1) *{\times} *\cir<2pt>{} **@{-};
        "y" *{\hspace{4pt}};p+(1,0) *\cir<2pt>{} **@{-};
        p+(1,0) *\cir<2pt>{} **@{-};
       \end{xy} & $E_6$ (compact form)\\
     $1\times 2$ & one way&
       \begin{xy}<1.75em,0em>:
        (0,0) *\cir<2pt>{};
        p+(1,0) *\cir<2pt>{} **@{-};
        p+(1,0)="y" *\cir<2pt>{} **@{-};
            p+(0,-1) *{\times} *\cir<2pt>{} **@{-};
            p+(0,-1) *\cir<2pt>{} **@{-};
        "y" *{\hspace{4pt}};p+(1,0) *\cir<2pt>{} **@{-};
        p+(1,0) *\cir<2pt>{} **@{-};
       \end{xy} & $A_1A_5$\\
     $1\times 1+1\times 1$ & one way &
       \begin{xy}<1.75em,0em>:
        (0,0) *{\times} *\cir<2pt>{};
        p+(1,0) *\cir<2pt>{} **@{-};
        p+(1,0)="y" *\cir<2pt>{} **@{-};
            p+(0,-1) *\cir<2pt>{} **@{-};
            p+(0,-1) *{\times} *\cir<2pt>{} **@{-};
        "y" *{\hspace{4pt}};p+(1,0) *\cir<2pt>{} **@{-};
        p+(1,0) *\cir<2pt>{} **@{-};
       \end{xy} & $D_5\oplus \RR$\\
   \end{tabular}
   \]
%   We can delete one point of weight 1, which gives the compact form
%
%   pictures
%
%   or one point of weight 2, which is blah. Or we can delete two points of weight 1:
%
%   picture
%
   In the last one, the maximal compact subalgebra is $D_5\oplus \RR$. Just as before,
   we get a Hermitian symmetric space. Let's compute its dimension (over $\CC$). The
   dimension will be the dimension of $E_6$ minus the dimension of $D_5\oplus\RR$, all
   divided by 2 (because we want complex dimension), which is $(78-46)/2=16$.

   So we have found two non-compact simply connected Hermitian symmetric spaces of
   dimensions 16 and 27. These are the only ``exceptional'' cases; all the others fall
   into infinite families!

   There are also some OUTER automorphisms of $E_6$ coming from the diagram
   automorphism
   \[\begin{xy}
     (0,0)="1" *\cir<2pt>{};
     (1,0)="2"  *\cir<2pt>{} **@{-};
     p+(1,0) *\cir<2pt>{} **@{-};
     p+(0,-1) *\cir<2pt>{} **@{-},
     p+(1,0)="22" *\cir<2pt>{} **@{-};
     p+(1,0)="11" *\cir<2pt>{} **@{-};
     "1" *+{\ };"11" *+{\ } **\crv{(2,3.3)} ?<*@{<} ?>*@{>} ?(.5)*+!U{\sigma},
     "2" *+{\ };"22" *+{\ } **\crv{(2,2.6)} ?<*@{<} ?>*@{>},
   \end{xy}\qquad \qquad \longrightarrow \qquad \qquad
   \begin{xy}
      (0,2) *\cir<2pt>{};
      p+(0,-1)  *\cir<2pt>{} **@{-};
      p+(0,-1)="x" *\cir<2pt>{} **@{=} ?*@{>};
      "x" *+<4pt>{\ };"x"+(0,-1) *\cir<2pt>{} **@{-};
   \end{xy}
   \]
   The fixed point subalgebra has Dynkin diagram
   obtained by folding the $E_6$ on itself. This is the $F_4$ Dynkin diagram. The
   fixed points of $E_6$ under the diagram automorphism is an $F_4$ Lie algebra. So we
   get a real form of $E_6$ with maximal compact subgroup $F_4$. This is probably the
   easiest way to construct $F_4$, by the way. Moreover, we can decompose $E_6$ as a
   representation of $F_4$. $\dim E_6=78$ and $\dim F_4=52$, so $E_6=F_4\oplus 26$,
   where $26$ turns out to be irreducible (the smallest non-trivial representation of
   $F_4$ ... the only one anybody actually works with). The roots of $F_4$ look like
   $(\dots, \pm 1, \pm 1\dots)$ (24 of these) and $(\pm \half \dots \pm \half)$ (16 of
   these), and $(\dots, \pm 1\dots)$ (8 of them) ... the last two types are in the
   same orbit of the Weyl group.

   The 26 dimensional representation has the following character: it has all norm 1
   roots with multiplicity 1 and 0 with multiplicity 2 (note that this is not
   minuscule).

   There is one other real form of $E_6$. To get at it, we have to talk about Kac's
   description of non-inner automorphisms of order $N$. The non-inner automorphisms
   all turn out to be related to diagram automorphisms. Choose a diagram automorphism
   of order $r$, which divides $N$. Let's take the standard thing on $E_6$. Fold the
   diagram (take the fixed points), and form a TWISTED affine Dynkin diagram (note
   that the arrow goes the wrong way from the affine $F_4$)
   \[
      \begin{xy}
      (0,-1.4)="1" *+!L{1} *\cir<2pt>{};
      p+(0,.7)="2" *+!L{2} *\cir<2pt>{} **@{-};
      p+(0,.7)="y" *+!R{3} *\cir<2pt>{} **@{-};
          p+(.7,0) *+!D{2} *\cir<2pt>{} **@{-};
          p+(.7,0) *+!D{1} *\cir<2pt>{} **@{-};
      "y" *=<4pt,4pt>{};p+(0,.7)="22" *+!L{2} *\cir<2pt>{} **@{-};
      p+(0,.7)="11" *+!L{1} *\cir<2pt>{} **@{-};
       "1" *+{\ };"11" *+{\ } **\crv{"y"+(-1.6,0)} ?<*@{<} ?>*@{>} ?(.5)*+!R{r},
       "2" *+{\ };"22" *+{\ } **\crv{"y"+(-1.2,0)} ?<*@{<} ?>*@{>},
      %%%%%%%%%%%
      (3,.5) *+!D{1} *\cir<2pt>{};
      p+(1,0) *+!D{2} *\cir<2pt>{} **@{-};
      p+(1,0)="x" *+!D{3} *\cir<2pt>{} **@{=} ?*@{>};
      "x" *{\hspace{4pt}};p+(1,0) *+!D{2} *\cir<2pt>{} **@{-};
      p+(1,0) *+!D{1} *\cir<2pt>{} *+<2.5em>!L{\text{Twisted Affine }F_4} **@{-};
      %%%%%%%%%%%
      (3,-1) *+!D{1} *+!R{\biggr(} *\cir<2pt>{};
      p+(1,0) *+!D{2} *\cir<2pt>{} **@{-};
      p+(1,0) *+!D{3} *\cir<2pt>{} **@{-};
      p+(1,0)="x" *+!D{4} *\cir<2pt>{} **@{=} ?*@{>};
      "x" *{\hspace{4pt}};p+(1,0) *+!D{2} *+<2.5em>!L{\text{Affine }F_4 \biggr)} *\cir<2pt>{} **@{-};
      %%%%%%%%%%%
      \ar (1.7,.1);(2.5,.3)
     \end{xy}
   \]
   There are also numbers on the twisted diagram, but nevermind them. Find $n_i$ so
   that $r\sum n_i m_i=N$. This is Kac's general rule. We'll only use the case $N=2$.

   If $r>1$, the only possibility is $r=2$ and one $n_1$ is 1 and the corresponding
   $m_i$ is 1. So we just have to find points of weight 1 in the twisted affine Dynkin
   diagram. There are just two ways of doing this in the case of $E_6$
   \[
    \begin{xy}
      (0,0) *\cir<2pt>{};
      p+(1,0) *\cir<2pt>{} **@{-};
      p+(1,0)="x" *\cir<2pt>{} **@{=} ?*@{>};
      "x" *{\hspace{4pt}};p+(1,0) *\cir<2pt>{} **@{-};
      p+(1,0) *{\times} *\cir<2pt>{} **@{-};
    \end{xy}\qquad\qquad \text{and} \qquad\qquad
    \begin{xy}
      (0,0) *{\times} *\cir<2pt>{};
      p+(1,0) *\cir<2pt>{} **@{-};
      p+(1,0)="x" *\cir<2pt>{} **@{=} ?*@{>};
      "x" *{\hspace{4pt}};p+(1,0) *\cir<2pt>{} **@{-};
      p+(1,0) *\cir<2pt>{} **@{-};
    \end{xy}
   \]
   one of these gives us $F_4$, and the other has maximal compact subalgebra $C_4$,
   which is the split form since $\dim C_4=\#\text{roots of }F_4/2 =24$.
 \end{example}

 \begin{example}
   $F_4$. The affine Dynkin is
    \begin{xy}
      (0,0) *{};
      p+(0,.1) *+!D{1} *\cir<2pt>{};
      p+(1,0) *+!D{2} *\cir<2pt>{} **@{-};
      p+(1,0) *+!D{3} *\cir<2pt>{} **@{-};
      p+(1,0)="x" *+!D{4} *\cir<2pt>{} **@{=} ?*@{>};
      "x" *{\hspace{4pt}};p+(1,0) *+!D{2} *\cir<2pt>{} **@{-};
    \end{xy}
   We can cross out one node of weight 1, giving the compact form (split form), or a
   node of weight 2 (in two ways), giving maximal compacts $A_1C_3$ or $B_4$. This
   gives us three real forms.
 \end{example}
 \begin{example}
   $G_2$. We can actually draw this root system ... UCB won't supply me with a
   four dimensional board. The construction is to take the $D_4$ algebra and look at
   the fixed points of:
   \[
   \begin{xy}
     (0,0) *\cir<2pt>{};
     a(60)="1" *\cir<2pt>{} **@{-},
     a(180)="2" *\cir<2pt>{} **@{-},
     a(-60)="3" *\cir<2pt>{} **@{-},
     \ar@/_2ex/ "1" *+{\ };"2" *+{\ }
     \ar@/_2ex/ "2" *+{\ };"3" *+{\ }
     \ar@/_2ex/_{\rho} "3" *+{\ };"1" *+{\ }
   \end{xy}
   \]
   We want to find the fixed point subalgebra.

   Fixed points on Cartan subalgebra: $\rho$ fixes a two dimensional space,
   and has 1 dimensional eigenspaces corresponding to $\w$ and $\bar \w$, where
   $\w^3=1$. The 2 dimensional space will be the Cartan subalgebra of $G_2$.

   Positive roots of $D_4$ as linear combinations of simple roots (not fundamental weights):
   \[\def\myDfour#1#2#3#4{
                 \begin{xy}
                   (0,0) *+{#1};
                   (-1,0) *+{#2} **@{-},
                    a(60) *+{#3} **@{-},
                    a(-60) *+{#4} **@{-},
                 \end{xy}}
     \def\myframecurve{2em}
    \begin{xy}
      (0,6) *++{\myDfour 0100 \qquad
               \myDfour 0010 \qquad
               \myDfour 0001} *\frm<\myframecurve>{-};
      (6,6) *++{\myDfour 1000} *\frm<\myframecurve>{-};
      (0,3) *++{\myDfour 1100 \qquad
               \myDfour 1010 \qquad
               \myDfour 1001} *\frm<\myframecurve>{-};
      (6,3) *++{\myDfour 1111} *\frm<\myframecurve>{-};
      (0,0) *++{\myDfour 1110 \qquad
               \myDfour 1011 \qquad
               \myDfour 1101} *\frm<\myframecurve>{-};
      (6,0) *++{\myDfour 2111} *\frm<\myframecurve>{-};
      (0,-1.8) *=<20em,0em>\frm{_\}} *+!U{\text{projections of norm }2/3};
      (6,-1.8) *=<6.25em,0em>\frm{_\}} *+!U{\text{projections of norm }2};
    \end{xy}
   \]
   There are six orbits under $\rho$, grouped above. It obviously acts on the negative
   roots in exactly the same way. What we have is a root system with six roots of norm
   2 and six roots of norm $2/3$. Thus, the root system is $G_2$:
   \[\begin{xy}
     (0,0) *+!DR{2} *{\bullet};
     a(0) *++!L{1} *{\bullet};
     a(60) *+!DL{1} *{\bullet};
     a(120) *+!DR{1} *{\bullet};
     a(180) *++!R{1} *{\bullet};
     a(240) *+!UR{1} *{\bullet};
     a(300) *+!UL{1} *{\bullet};
     a(60)+a(120) *+!D{1} *{\bullet};
     a(180)+a(240) *+!R{1} **@{-} *{\bullet};
     a(300)+a(0) *+!L{1} **@{-} *{\bullet};
     a(60)+a(120) **@{-};
     a(60)+a(0) *+!L{1} *{\bullet};
     a(180)+a(120) *+!R{1} **@{-} *{\bullet};
     a(300)+a(240) *+!U{1} **@{-} *{\bullet};
     a(60)+a(0) **@{-};
   \end{xy}\]
   One of the only root systems to appear on a country's national flag. Now let's work
   out the real forms. Look at the affine:
   \begin{xy}
   (0,0) *{};
   p+(0,.05) *+!D{1} *\cir<2pt>{};
   p+(1,0)="1" *+!D{2} *\cir<2pt>{} **@{-};
   p+(1,0)="2" *+!D{3} *\cir<2pt>{} **@{-} ?*@{>},
   \ar@{-} "1" *{\hspace{3pt}};"2" *{\hspace{3pt}} <1.5pt>
   \ar@{-} "1" *{\hspace{3pt}};"2" *{\hspace{3pt}} <-1.5pt>
   \end{xy}.
   we can delete the node of weight 1, giving the compact form:
   \begin{xy}
   (0,0) *{};
   p+(0,.05) *{\times} *\cir<2pt>{};
   p+(1,0)="1" *\cir<2pt>{} **@{-};
   p+(1,0)="2" *\cir<2pt>{} **@{-} ?*@{>},
   \ar@{-} "1" *{\hspace{3pt}};"2" *{\hspace{3pt}} <1.5pt>
   \ar@{-} "1" *{\hspace{3pt}};"2" *{\hspace{3pt}} <-1.5pt>
   \end{xy}
   . We can delete the node
   of weight 2, giving $A_1A_1$ as the compact subalgebra:
   \begin{xy}
   (0,0) *{};
   p+(0,.05) *\cir<2pt>{};
   p+(1,0)="1" *{\times} *\cir<2pt>{} **@{-};
   p+(1,0)="2" *\cir<2pt>{} **@{-} ?*@{>},
   \ar@{-} "1" *{\hspace{3pt}};"2" *{\hspace{3pt}} <1.5pt>
   \ar@{-} "1" *{\hspace{3pt}};"2" *{\hspace{3pt}} <-1.5pt>
   \end{xy}
   ... this must be the split
   form because there is nothing else the split form can be.

   Let's say some more about the split form. What does the Lie algebra of $G_2$ look
   like as a representation of the maximal compact subalgebra $A_1 \times A_1$? In
   this case, it is small enough that we can just draw a picture:
   \[\begin{xy}
     (0,0) *{2};
     a(0) *++!L{1} *{\bullet};
     a(60) *+!DL{1} *{\bullet};
     a(120) *+!DR{1} *{\bullet};
     a(180) *++!R{1} *{\bullet};
     a(240) *+!UR{1} *{\bullet};
     a(300) *+!UL{1} *{\bullet};
     a(60)+a(120) *+!D{1} *{\bullet};
     a(180)+a(240) *+!R{1} **@{.} *{\bullet};
     a(300)+a(0) *+!L{1} **@{.} *{\bullet};
     a(60)+a(120) **@{.};
     a(60)+a(0) *+!L{1} *{\bullet};
     a(180)+a(120) *+!R{1} **@{.} *{\bullet};
     a(300)+a(240) *+!U{1} **@{.} *{\bullet};
     a(60)+a(0) **@{.};
     (0,0) *=<8.75em,1.5em>\frm<8pt>{-} *=<1.5em,11.25em>\frm<8pt>{-};
   \end{xy} \qquad \qquad \longrightarrow \qquad \qquad
   \begin{xy}
     (0,0);
     a(60) *+!D{1} *{\bullet};
     a(120) *+!D{1} *{\bullet};
     a(240) *+!U{1} *{\bullet};
     a(300) *+!U{1} *{\bullet};
     a(60)+a(120);
     a(180)+a(240) *+!U{1} *{\bullet} **@{.};
     a(300)+a(0) *+!U{1} *{\bullet} **@{.};
     a(60)+a(120) **@{.};
     a(60)+a(0) *+!D{1} *{\bullet};
     a(180)+a(120) *+!D{1} *{\bullet} **@{.};
     a(300)+a(240) **@{.};
     a(60)+a(0) **@{.};
     (0,\halfrootthree) *=<11.25em,2.2em>\frm<8pt>{-};
     (1.5,0) *=<2em,7.5em>\frm<8pt>{-};
   \end{xy}\]
   We have two orthogonal $A_1$s, and we have leftover the stuff on the right. This
   thing on the right is a tensor product of the 4 dimensional irreducible
   representation of the horizontal and the 2 dimensional of the vertical. Thus, $G_2=
   3\times 1 + 1\otimes 3 + 4\otimes 2$ as irreducible representations of
   $A_1^{(\mathrm{horizontal})} \otimes A_1^{(\mathrm{vertical})}$.

   Let's use this to determine exactly what the maximal compact subgroup is. It is a
   quotient of the simply connected compact group $SU(2)\times SU(2)$, with Lie
   algebra $A_1\times A_1$. Just as for $E_8$, we need to identify which elements of
   the center act trivially on $G_2$. The center is $\ZZ/2\times \ZZ/2$. Since we've
   decomposed $G_2$, we can compute this easily. A non-trivial element of the center
   of $SU(2)$ acts as 1 (on odd dimensional representations) or $-1$ (on even
   dimensional representations). So the element $z\times z\in SU(2)\times SU(2)$ acts
   trivially on $3\otimes 1 + 1\otimes 3 + 4\times 2$. Thus the maximal compact
   subgroup of the non-compact simple $G_2$ is $SU(2)\times SU(2)/(z\times z)\cong
   SO_4(\RR)$, where $z$ is the non-trivial element of $\ZZ/2$.
 \end{example}

 So we have constructed $3+4+5+3+2$ (from $E_8$, $E_7$, $E_6$, $F_4$, $G_2$) real
 forms of exceptional simple Lie groups.

 There are another 5 exceptional real Lie groups: Take COMPLEX groups $E_8(\CC)$,
 $E_7(\CC)$, $E_6(\CC)$, $F_4(\CC)$, and $G_2(\CC)$, and consider them as REAL. These
 give simple real Lie groups of dimensions $248\times 2$, $133\times 2$, $78\times 2$,
 $52 \times 2$, and $14\times 2$.
