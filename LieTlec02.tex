 \stepcounter{lecture}
 \setcounter{lecture}{2}
 \sektion{Lecture 2}

 Last time we talked about Lie groups, Lie algebras, and gave examples. Recall that
 $M\subseteq L$ is a Lie subalgebra if $[M,M]\subseteq M$. We say that $M$ is a
 \emph{Lie ideal}\index{Lie ideal} if $[M,L]\subseteq M$.

 \begin{claim}
   If $M$ is an ideal, then $L/M$ has the structure of a Lie algebra such that the
   canonical projection is a morphism of Lie algebras.
 \end{claim}
 \begin{proof}
   Take $l_1,l_2\in L$, check that $[l_1+M,l_2+M]\subseteq [l_1,l_2]+M$.
 \end{proof}

 \begin{claim} For $\phi:L_1\to L_2$ a Lie algebra homomorphism,
   \begin{enumerate}
   \item $\ker \phi \subseteq L_1$ is an ideal,
   \item $\im \phi\subseteq L_2$ is a Lie subalgebra,
   \item $L_1/\ker\phi \cong \im\phi$ as Lie algebras.
   \end{enumerate}
 \end{claim}
 \begin{exercise}
   Prove this claim.
 \end{exercise}

 \subsektion{Tangent Lie algebras to Lie groups} Let's recall some differential
 geometry. You can look at \cite{Lee:ISM} as a reference. If $f:M\to N$ is a
 differentiable map, then $df:TM\to TN$ is the derivative. If $G$ is a group, then we
 have the maps $l_g:x\mapsto gx$ and $r_g:x\mapsto xg$. Recall that a smooth vector
 field is a smooth section of the tangent bundle $TM\to M$.
 \begin{definition}
   A vector field $X$ is \emph{left invariant} if $(dl_g)\circ X=X\circ l_g$ for all
   $g\in G$. The set of left invariant vector fields is called $\mathrm{Vect}_L(G)$.
   \[\xymatrix{
    TG \ar[r]^{dl_g} & TG\\
    G\ar[u]^X \ar[r]^{l_g} & G\ar[u]_X
   }\]
 \end{definition}

 \begin{proposition}
   $\mathrm{Vect}_L(G)\subseteq \mathrm{Vect}(G)$ is a Lie subalgebra.
 \end{proposition}
 \begin{proof}
   We get an induced map $l_g^*:C^\infty(G)\to C^\infty(G)$, and $X$ is left invariant
   if and only if $L_X$ commutes with $l_G^*$. Then\\
    $X,Y$ left invariant $\Longleftrightarrow [L_X,L_Y]$ invariant $\Longleftrightarrow [X,Y]$
    left invariant.
 \end{proof}
 All the same stuff works for right invariant vector fields $\mathrm{Vect}_R(G)$.

 \begin{definition}
   $\g = \mathrm{Vect}_L(G)$ is the tangent Lie algebra of $G$. \index{Lie algebra!of
   a Lie group}
 \end{definition}

 \begin{proposition}
   There are vector space isomorphisms $\mathrm{Vect}_L(G)\simeq T_eG$ and
   $\mathrm{Vect}_R(G)\simeq T_eG$. Moreover, the Lie algebra structures on $T_eG$
   induced by these isomorphisms agree.
 \end{proposition}
 Note that it follows that $\dim \g=\dim G$.
 \begin{proof}
   \anton{I think this proof could be improved ... it is incomplete as is}
   Recall fibre bundles. $dl_g:T_eG\xrightarrow{\sim} T_gG$, so $TG\simeq T_e\times
   G$. $X$ is a section of $TG$, so it can be thought of as
   $X:G\to T_eG$, in which case the left invariant fields are exactly those which are
   constant maps, but the set of constants maps to $T_eG$ is isomorphic to $T_eG$.
 \end{proof}

 If $G$ is an $n$ dimensional $C^\w$ Lie group, then $\g$ is an $n$ dimensional Lie algebra.
 If we take local coordinates near $e\in G$ to be $x^1,\dots, x^n:U_e\to \RR^n$ with
 $m:\RR^n\times \RR^n\to \RR^n$ the multiplication (defined near 0). We have a power
 series for $m$ near 0,
 \[
    m(x,y) = Ax+By+\alpha_2(x,y)+\alpha_3(x,y)+\cdots
 \]
 where $A,B:\RR^n\to \RR^n$ are linear, $\alpha_i$ is degree $i$. Then we can consider
 the condition that $m$ be associative (only to degree 3): $m(x,m(y,z))=m(m(x,y),z)$.
 \begin{align*}
   m(x,m(y,z)) &= Ax+Bm(y,z)+\alpha_2(x,m(y,z))+\alpha_3(x,m(y,z))+\cdots\\
        &=
        Ax+B(Ay+Bz+\alpha_2(y,z)+\alpha_3(y,z)))\\
        &\qquad +\alpha_2(x,Ay+Bz+\alpha_2(y,z))+\alpha_3(x,Ay+Bz)\\
   m(m(x,y),z) &= \anton{\text{compute this stuff}}
 \end{align*}
 Comparing first order terms (remember that $A,B$ must be non-singular), we can get that
 $A=B=I_n$. From the second order term, we can get that $\alpha_2$ is bilinear!
 Changing coordinates ($\phi(x)=x+\phi_2(x)+\phi_3(x)+\cdots$, with
 $\phi^{-1}(x)=x-\phi_2(x)+\tilde\phi_3(x)+\cdots$), we use the fact that
 $m_\phi(x,y)=\phi^{-1}m(\phi x,\phi y)$ is the new multiplication, we have
 \[
    m_\phi(x,y) = x+y + (\underbrace{\phi_2 (x) +\phi_2 (y)+\phi_2 (x+y)}_{\text{can
    be any symm form}}) + \alpha_2(x,y)+ \cdots
 \]
 so we can tweak the coordinates to make $\alpha_2$ skew-symmetric. Looking at order 3, we
 have
 \begin{equation}\label{lec02Eq:1}
  \alpha_2(x,\alpha_2(y,z))+\alpha_3(x,y+z) =
  \alpha_2(\alpha_2(x,y),z)+\alpha_3(x+y,z)
 \end{equation}
 \begin{exercise}
   Prove that this implies the Jacobi identity for $\alpha_2$. (hint: skew-symmetrize
   equation \ref{lec02Eq:1})
 \end{exercise}
 Remarkably, the Jacobi identity is the only obstruction to associativity; all
 other coefficients can be eliminated by coordinate changes.

 \begin{example}
   Let $G$ be the set of matrices of the form $\matrix{a}{b}{0}{a^{-1}}$ for $a,b$
   real, $a>0$. Use coordinates $x,y$ where $e^x=a$, $y=b$, then
   \begin{align*}
      m((x,y),(x',y')) &= (x+x',e^xy'+ye^{-x'})\\
              &= (x+x',y+y'+ (\underbrace{xy'-x'y}_{skew})+ \cdots).
   \end{align*}
   The second order term is skew symmetric, so these are good coordinates. There are
   $H,E\in T_eG$ corresponding to $x$ and $y$ respectively so that
   $[H,E]=E$\footnote{what does this part mean?}.
 \end{example}

 \begin{exercise}
    Think about this. If $a,b$ commute, then $e^ae^b=e^{a+b}$. If
    they do not commute, then $e^ae^b=e^{f(a,b)}$. Compute $f(a,b)$ to order 3.
 \end{exercise}
