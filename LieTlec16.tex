 \stepcounter{lecture}
 \setcounter{lecture}{16}
 \sektion{Lecture 16 - Serre's Theorem}\index{Serre's Theorem|(idxbf}
 \newcommand\jj{\mathfrak{j}}

 Start with a semisimple Lie algebra $\g$ over an algebraically closed field $k$ of
 characteristic zero, with Cartan subalgebra $\h\subset \g$. Then we have the root
 system $\Delta\subseteq \h^*$, with a fixed set of simple roots $\Pi =
 \{\alpha_1,\dots, \alpha_n\}$. We have a copy of $\sl_2$---generated by
 $X_i$, $Y_i$, and $H_i$---associated to each simple root.

 The \emph{Cartan matrix}\index{Cartan!matrix|idxbf} $(a_{ij})$ of $\g$ is given by
 $a_{ij} = \langle \check\alpha_i,\alpha_j\rangle = \alpha_j(H_i) =
 \frac{2(\alpha_i,\alpha_j)}{(\alpha_i,\alpha_i)}$. From the definition of coroots and
 from properties of simple roots, we know that $a_{ij} \in \ZZ_{\le 0}$ for $i\neq j$,
 that $a_{ii}=2$, and that $a_{ij}=0$ implies $a_{ji}=0$.

 \begin{claim}
   The following relations (called \emph{Serre relations}\index{Serre
   relations|idxbf}\footnote{Serre called them Weyl relations.}) are satisfied
   in $\g$. \hypertarget{Serre}
   {\begin{gather*}
     \begin{array}{lr}
       [H_i, X_j]=a_{ij} X_j & \text{\normalfont \quad(a)} \\
       {[H_i,Y_j]=-a_{ij}Y_j} & \text{\normalfont \quad(b)}
     \end{array}\qquad\qquad
     \begin{array}{lr}
       {[H_i, H_j] = 0} & \text{\normalfont \quad(c)}\\
       {[X_i,Y_j]=\delta_{ij}H_i} & \text{\normalfont \quad(d)}
     \end{array} \tag{Ser1}\\
     \begin{array}{l}
       \theta_{ij}^+ := (ad_{X_i})^{1-a_{ij}}X_j = 0 \\
       \theta_{ij}^- := (ad_{Y_i})^{1-a_{ij}}Y_j = 0
     \end{array}\text{, for }i\neq j. \tag{Ser2}
   \end{gather*}}
 \end{claim}
 \begin{proof}
   (Ser1a), (Ser1b), and (Ser1c) are immediate because $X_i\in \g_{\alpha_i}$, $Y_i\in
   \g_{-\alpha_i}$, and $H_i=[X_i,Y_i]\in \h$. To show (Ser1d), we need to show that
   $[X_i,Y_i]=0$ for $i\neq j$. This is because $[X_i,Y_j]\in \g_{\alpha_i-\alpha_j}$,
   which is not in $\Delta$ because every element of $\Delta$ is a non-negative or
   non-positive combination of the $\alpha_i$.

   Since $ad_{X_i}(Y_j)=0$, we get that $Y_j$ is a highest vector for the $\sl(2)$
   generated by $X_i$, $Y_i$, and $H_i$. We also have that $ad_{H_i}(Y_j)=
   -a_{ij}Y_j$. Thus, the $\alpha_i$-string\index{alpha-string@$\alpha$-string|idxit}
   through $Y_j$ is spanned by $Y_j$, $ad_{Y_i}Y_j$, \dots, $ad_{Y_i}^{-a_{ij}}Y_j$.
   In particular, $\theta^-_{ij}=ad_{Y_i}^{1-a_{ij}} Y_j=0$. Similarly,
   $\theta^+_{ij}=0$, so the relations (Ser2) hold.
 \end{proof}
 So far, all we know is that any Lie algebra with root system $\Delta$ satisfies these
 relations. We have yet to show that such an algebra exists, that it is unique, and
 that these relations define it.

% \begin{example}
%   $\g = \sl(n+1)$. We have $X_i = E_{i,i+1}, Y_i=E_{i+1,i}, H_i = [X_i,Y_i] = E_{ii} -
%   E_{i+1,i+1}$.
%
%  Let's check that the second group of relations hold. We want to show that
%  $(ad_{X_i})^{1-a_{ij}}X_j=0$. What is $a_{ij}$? We know that
%  $(ad_{X_i})^{1-a_{i,i+1}}X_{i+1} = 0$ by a calculation ... just do it.
% \end{example}

 \begin{theorem}[Serre's Theorem]\label{lec16T:Serre} Let $\Delta$ be a root system,
   with a fixed set of simple roots $\Pi=\{\alpha_1,\dots, \alpha_n\}$, yielding the
   Cartan matrix $a_{ij}=\frac{2(\alpha_i,\alpha_j)}{(\alpha_i,\alpha_i)}$. Let $\g$
   be the Lie algebra generated by $H_i, X_i,Y_i$ for $1\le i\le n$, with relations
   \textrm{(Ser1)} and \textrm{(Ser2)}. Then $\g$ is a finite dimensional semisimple Lie algebra with a
   Cartan subalgebra spanned by $H_1,\dots, H_n$, and with root system $\Delta$.
 \end{theorem}
 \begin{remark}\label{lec16Rmk:freelie}
   In order to talk about a Lie algebra given by certain generators and relations, it
   is necessary to understand the notion of a \emph{free Lie algebra}\index{Lie
   algebra!free} $L(X)$ on a set of generators $X$, which is non-trivial (because of
   the Jacobi identity). We define $L(X)$ as the Lie subalgebra of the tensor algebra
   $T(X)$ generated by the set $X$. This algebra has the universal property that for
   any Lie algebra $L'$ and for any function $f:X\to L'$, there is a \emph{unique}
   extension of $f$ to a Lie algebra homomorphism $\tilde f:L(X)\to L'$ (to
   prove this, one needs the PBW theorem\index{PBW|idxit}).

   To impose a set of relations $R$, quotient $L(X)$ by the smallest ideal containing
   $R$. The resulting Lie algebra $L(X,R)$ has the universal property that for any Lie
   algebra $L'$ and for any function $f:X\to L'$ such that the image satisfies the
   relations $R$, there is a \emph{unique} extension of $f$ to a Lie algebra
   homomorphism $\tilde f:L(X,R)\to L'$.
 \end{remark}
 \begin{remark}
   Serre's Theorem proves that for any root system $\Delta$ there is a finite
   dimensional semisimple Lie algebra $\g$ with root system $\Delta$. But since any
   other Lie algebra $\g'$ with root system $\Delta$ satisfies (Ser1) and (Ser2), and
   since $\g$ is the universal Lie algebra satisfying these relations, we get an
   induced Lie algebra homomorphism $\phi:\g\to \g'$. This homomorphism is surjective
   because $\g'$ is spanned by $\phi(X_i),\phi(Y_i)$, and $\phi(H_i)$. Moreover, both
   $\dim \g$ and $\dim \g'$ must be equal to $|\Delta|+$rank$(\Delta)$, so $\phi$ must
   be an isomorphism. Therefore, we get uniqueness of $\g$.
 \end{remark}

 \begin{proof}[Proof of Serre's Theorem]{\ }

    \underline{Step 1. Decompose $\tilde\g$}: Consider the free Lie algebra with
    generators $X_i$, $Y_i$, $H_i$ for $1\le i\le n$ and impose the relations (Ser1).
    Call the result $\tilde \g$. Let $\h$ be the abelian Lie subalgebra generated by
    $H_1,\dots, H_n$, and let $\tilde \n^+$ (resp.\ $\tilde \n^-$) be the Lie
    subalgebra generated by the $X_i$ (resp.\ $Y_i$). The goal is to show that $\tilde
    \g = \tilde \n^- \oplus \h \oplus \tilde \n^+$ as a vector space.

%    , where $\h$ is the abelian Lie algebra generated by $H_1,\dots,
%    H_n$, and $\tilde \n^{\pm}$ are free Lie algebras generated by the $X_i$ (resp.\
%    $Y_i$). They should be free because there are no relations among the $X_i$'s, but
%    it is not so easy.
%%%%%%%%%%%%%%%%%%%%%%%%%%%%%%%%%%%%%%%%%%%%%%%%%%%%%%%%%%%%%%%%%%%%%%%%%%%%%%%%%%%%

    There is a standard trick for doing such things. It is easy to see from
    \hyperlink{Serre}{(Ser1)} that $U\tilde \g = U\tilde \n^- \cdot U\h\cdot U\tilde
    \n^+$.\footnote{By $U\tilde\n^-\cdot U\h\cdot U\tilde\n^+$, we mean the set of
    linear combinations of terms of the form $y\cdot h\cdot x$, where $y\in
    U\tilde\n^-$, $h\in U\h$, and $x\in U\tilde\n^+$.} Let $T(X)$ be the tensor
    algebra on the $X_i$, let $T(Y)$ be the tensor algebra on the $Y_i$, and let $S\h$
    be the symmetric algebra on the $H_i$. We define a representation $U\tilde \g\to
    \End\bigl(T(Y) \otimes S\h\otimes T(X)\bigr)$. For $a\in T(Y)$, $b\in S\h$, and
    $c\in T(X)$, define
    \begin{align*}
      X_i(1\otimes 1\otimes c) &= 1\otimes 1\otimes X_ic,\\
      H_i(1\otimes b\otimes c) &= 1\otimes H_ib\otimes c,\text{ and}\\
      Y_i(a\otimes b\otimes c) &= (Y_ia)\otimes b\otimes c.
    \end{align*}
    Then extend inductively by
    \begin{align*}
      H_i(Y_ja\otimes b\otimes c) &= Y_jH_i(a\otimes b\otimes c) - a_{ij}Y_j(a\otimes
              b\otimes c)\\
      X_i(1\otimes H_jb\otimes c) &= H_jX_i(1\otimes b\otimes c) - a_{ji}X_i(1\otimes
              b\otimes c)\\
      X_i(Y_ja\otimes b\otimes c) &= Y_jX_i(a\otimes b\otimes c) +
              \delta_{ij}H_i(a\otimes b\otimes c).
    \end{align*}
    \begin{exercise}
      Check that this is a representation.
      \begin{solution}
        It is enough to check that the proposed endomorphisms of $T(Y)\otimes
        S\h\otimes T(X)$ satisfy \hyperlink{Serre}{(Ser1)}. Then the universal
        property $\tilde \g$ (from Remark \ref{lec16Rmk:freelie}) and the universal
        property of $U\tilde \g$ (from Proposition \ref{lec07P:Ug}) tell us exactly
        that there is a unique algebra homomorphism $U\tilde\g \to
        \End\bigl(T(Y)\otimes S\h\otimes T(X)\bigr)$ such that $X_i$, $Y_i$, and $H_i$
        act as described. We get (Ser1a), (Ser1b), and (Ser1d) by construction. We
        need only check that $H_iH_j$ acts in the same way as $H_jH_i$. It is clear
        that $H_iH_j(1\otimes b\otimes c)=H_jH_i(1\otimes b\otimes c)$. Now we induct
        on the degree of $a$.
        \begin{align*}
          H_iH_j(Y_ka\otimes b\otimes c)&= (H_iY_kH_j - a_{jk}H_iY_k)(a\otimes b\otimes
                    c) & \text{(Ser1b)}\\
                &= (Y_kH_iH_j - a_{ik}Y_kH_j \\
                &\phantom{= (H_iY_kH_j }\; - a_{jk}Y_kH_i +
                    a_{jk}a_{ik}Y_k)(a\otimes b\otimes c) &\text{(Ser1b)}\\
                &= H_jH_i(Y_ka\otimes b \otimes c) &\llap{($i$, $j$ symmetric)}
        \end{align*}
        This shows that the representation is well defined.
      \end{solution}
    \end{exercise}
    Observe that the canonical (graded vector space) homomorphism $T(Y)\otimes
    S\h\otimes T(X)\to U\tilde\n^-\cdot U\h\cdot U\tilde\n^+ = U\tilde \g$ is the
    inverse of the map $w\mapsto w(1\otimes 1\otimes 1)$, so $U\tilde\g \simeq
    T(Y)\otimes S\h\otimes T(X)$ as graded vector spaces.\footnote{$U\tilde\g$ is
    graded as a vector space, but only \emph{filtered} as an algebra.} Looking at the
    degree 1 parts, we get the vector space isomorphism
    $\tilde\g\simeq\tilde\n^-\oplus \h\oplus\tilde\n^+$.

%%%%%%%%%%%%%%%%%%%%%%%%%%%%%%%%%%%%%%%%%%%%%%%%%%%%%%%%%%%%%%%%%%%%%%%%%%%%%%%%%%%%
%{
%    There is a standard trick for doing such things. It is easy to see from the
%    relations (Ser1) that $U\tilde\g = U(\tilde \n^-)U(\h)U(\tilde \n^+)$. In fact, we
%    want $U(\tilde \g) =\underbrace{U(\tilde\n^-)}_{T(\tilde\n^-)}\otimes
%    \underbrace{U(\h)}_{S(\h)}\otimes \underbrace{U(\tilde\n^+)}_{T(\tilde\n^+)}$. We
%    construct a representation $U\tilde\g\to \End\bigl(T(\tilde \n^-)\otimes
%    S(\h)\bigr)$.
%
%    For $b\in S(\h)$, define $Y_i(1\otimes b)=Y_i\otimes b$, $H_i(1\otimes b)=1\otimes
%    H_ib$, and $X_i(1\otimes b)=0$. Then for $a\in T(\tilde n^-)$, define
%    \begin{align*}
%    Y_i (Y_ja\otimes b) &= (Y_iY_ja)\otimes b\\
%    H_i(Y_ja\otimes b)  &= Y_j(H_i(a\otimes b)) - a_{ij}Y_j(a\otimes b)\\
%    X_i(Y_ja\otimes b)  &= Y_j(X_i(a\otimes b)) + \delta_{ij} H_i(a\otimes b)
%    \end{align*}
%    \begin{exercise}
%      \mpar[\noindent \raggedright What we did in class \hfill \smash{\rule[-1.85in]{.6pt}{4.2in}}]{}
%      Check that this is a representation, and that $1\otimes 1$ generates the
%      representation.
%      \begin{solution}\anton{I can do this, but it is long}
%        It is clear that $1\otimes 1$ generates. To check that this is a
%        representation, we must check that the relations \hyperlink{Serre}{(Ser1)}
%        hold in the image. We get (Ser1b) and (Ser1d) immediately, and (Ser1c) holds
%        because the second factor is the \emph{symmetric} algebra on $\h$. It remains
%        to show (Ser1a), which is painful. First we check that
%        \begin{align*}
%          X_iH_j(1\otimes b) &= X_i(1\otimes H_jb) = 0\\
%                &= H_jX_i(1\otimes b) - a_{ji}X_i(1\otimes b).
%        \end{align*}
%        Now we induct on the degree of $a$,
%        {\newcommand{\ab}{\mbox{\relsize{-1} $(a\otimes b)$}}
%         \newcommand{\uthinsp}{\kern .055 em }
%        \begin{align*}%\hspace*{-2em}
%          \!\! X_iH_j(Y_ka\otimes b) &= ( X_iY_kH_j - a_{jk}X_iY_k )\ab & \text{(Ser1b)}\\
%            &= (Y_kX_iH_j+\delta_{ik}H_iH_j - a_{jk}Y_kX_i - \delta_{ik}a_{jk}H_i )\ab &
%              \text{(Ser1d)}\\
%            &= ( \underline{\underline{Y_kH_jX_i}}
%                 - a_{ji}Y_kX_i + &\llap{(induction)}\\
%            &\phantom{= (Y_kX_iH_j}\;
%                 +\underbrace{\delta_{ik}H_iH_j}\uthinsp
%                 \underline{-\: a_{jk}Y_kX_i}\uthinsp
%                 \underbracket{-\: \delta_{ik}a_{jk}H_i})\ab\\
%      \!\!(H_jX_i-a_{ji}X_i\rlap{$)(Y_k a\otimes b)$}\\
%            &= (H_jY_kX_i + \delta_{ik}H_jH_i - a_{ji}Y_kX_i -\delta_{ik}a_{ji}H_i)\ab
%              & \text{(Ser1d)}\\
%            &= ( \underline{\underline{Y_kH_jX_i}}\:
%                 \underline{-\: a_{jk}Y_kX_i} &\text{(Ser1b)}\\
%            &\phantom{= (H_jY_kX_i}\;
%                 +\underbrace{\delta_{ik}H_jH_i}\uthinsp
%                 -\: a_{ji}Y_kX_i\uthinsp
%                 \underbracket{-\:\delta_{ik}a_{ji}H_i}
%            )\ab
%        \end{align*}}
%        Note that $\delta_{ik}a_{jk}=\delta_{ik}a_{ji}$.
%      \end{solution}
%    \end{exercise}
%    So we get an isomorphism of $U(\tilde \n^-)$ and $T(\tilde \n^-)$. We define
%    $U(\tilde \n^- + \h) \twoheadrightarrow T(\tilde \n^-)\otimes S(\h)$ given by
%    $x\mapsto x(1\otimes 1)$. We do the same thing for $\tilde \n^+$.\mpar{\vspace{2cm}\anton{I don't see how this part works.}}
%   \begin{gather*}
%    0\to U(\tilde \n^+) \to  U(\tilde \g) \to T(\tilde \n^-)\otimes S(\h)\to 0\\
%    0\to U(\tilde \n^-) \to  U(\tilde \g) \to T(\tilde \n^+)\otimes S(\h)\to 0
%   \end{gather*}
%  Hence $U(\tilde \g) \simeq T(\tilde \n^-)\otimes S(\h)\otimes T(\tilde \n^+)$. So
%  $\tilde \g = \tilde \n^-\oplus \h\oplus \tilde \n^+$.
%}
%%%%%%%%%%%%%%%%%%%%%%%%%%%%%%%%%%%%%%%%%%%%%%%%%%%%%%%%%%%%%%%%%%%%%%%%%%%%%%%%%%%%%%%%

  \underline{Step 2. Construct $\g$}: We have that $\theta_{ij}^\pm \in \tilde \n^\pm$. Let
  $\jj^+$ (resp.\ $\jj^-$) be the ideal in $\tilde\n^+$ (resp.\
  $\tilde\n^-$) generated by the set $\{\theta_{ij}^+\}$ (resp. $\{\theta_{ij}^-\}$).
  \begin{exercise}
    Check that
    \[
        [Y_k,\theta_{ij}^+]=0 \text{\qquad and \qquad} [H_k,\theta_{ij}^+]=c_{kij}\theta_{ij}^+
    \]
    for some constants $c_{kij}$. Therefore, $\jj^\pm$ are ideals in $\tilde \g$.
    \begin{solution}
      It is easy to check by induction that in $U\tilde \g$,
      \begin{align*}
        H_k X_i^r &= X_i^r H_k + ra_{ki}X_i^r\text{, and}\\
        Y_kX_i^r &= X_i^rY_k - r\delta_{ik}\bigl(X_i^{r-1}H_i +(r-1)X_i^{r-1}\bigr).
      \end{align*}
      Since $ad$ is a representation, it follows that
      \begin{align*}
        [H_k,\theta_{ij}^+] &= ad_{H_k}ad_{X_i}^{1-a_{ij}}X_j\\
            &= ad_{X_i}^{1-a_{ij}} ad_{H_k}X_j +
            (1-a_{ij})a_{ki}ad_{X_i}^{1-a_{ij}}X_j\\
            &= (a_{kj}+a_{ki}-a_{ij}a_{ki})\theta_{ij}^+\\
        [Y_k,\theta_{ij}^+] &= ad_{X_i}^{1-a_{ij}}[Y_k,X_j] = 0&\text{(if $k\neq
                j$)}\\
        [Y_j,\theta_{ij}^+] &=
            ad_{X_i}^{1-a_{ij}}\overbrace{[Y_j,X_j]}^{-H_j}
            -(1-a_{ij})ad_{X_i}^{-a_{ij}}\overbrace{[H_i,X_j]}^{a_{ij}X_j} \\
            &\phantom{=ad_{X_i}^{1-a_{ij}}[Y_j,X_j]}
                +(1-a_{ij})a_{ij}ad_{X_i}^{-a_{ij}}X_j\\
            &=a_{ji}ad_{X_i}^{-a_{ij}}X_i
      \end{align*}
      which is zero if $a_{ij}=a_{ji}=0$, and is zero if $a_{ij}<0$.
    \end{solution}
  \end{exercise}
  Now define $\n^+=\tilde\n^+/\jj^+$, $\n^-=\tilde\n^-/\jj^-$, and $\g =
  \tilde\g/(\jj^++\jj^-) = \n^-\oplus \h\oplus \n^+$. From relations
  \hyperlink{Serre}{(Ser1)}, we know that $\h$ acts diagonalizably on $\n^+$, $\n^-$,
  and $\h$, so we get the decomposition $\g = \h\oplus \bigoplus_{\alpha\in \h^*}
  \g_\alpha$, where $\g_\alpha = \{x\in \g| [h,x]=\alpha(h)x\}$. Note that each
  $\g_\alpha$ is either in $\n^+$ or in $\n^-$.

  Define $R$ as the set of non-zero $\alpha\in \h^*$ such that $\g_\alpha\neq 0$. We
  know that $\pm \alpha_1,\dots,\pm \alpha_n\in R$ because $X_i\in \g_{\alpha_i}$ and
  $Y_i\in \g_{-\alpha_i}$. Since each $\g_\alpha$ is either in $\n^+$ or in $\n^-$,
  $\alpha$ must be a non-negative or a non-positive combination of the $\alpha_i$
  (recalling that $[\g_\alpha,\g_\beta]\subseteq \g_{\alpha+\beta}$). This gives us
  the decomposition $R = R^+\amalg R^-$.

  Since $\g$ is generated by the $X_i$ and $Y_i$, the relation
  $[\g_\alpha,\g_\beta]\subseteq \g_{\alpha+\beta}$ tells us that $R$ is contained in
  the lattice $\sum_{i=1}^n \ZZ\alpha_i$. Since $\n^+=\bigoplus_{\alpha\in
  R^+}\g_\alpha$ is a quotient of $\tilde\n^+$, it is generated as Lie algebra by the
  $X_i$. Together with the relation $[\g_\alpha,\g_{\beta}]\subseteq
  \g_{\alpha+\beta}$ and the linear independence of the $\alpha_i$, this tells us that
  $\g_{\alpha_i}$ is one dimensional, spanned by $X_i$, and that
  $\g_{n\alpha_i}=[\g_{\alpha_i},\g_{(n-1)\alpha_i}]=0$ for $n>1$.

 \underline{Step 3. $R$ is $\weyl$-invariant}: Let $\weyl$ be the Weyl group of the
 root system $\Delta$, generated by the simple reflections $r_i:\lambda\mapsto
 \lambda-\lambda(H_i)\alpha_i$. We would like to show that $R$ is invariant under the
 action of $\weyl$. To do this, we need to make sense of the element $s_i =
 \exp(ad_{X_i})\exp(-ad_{Y_i})\exp(ad_{X_i})\in \aut \g$.

 The main idea is that \hyperlink{Serre}{(Ser1)} and
 \hyperlink{Serre}{(Ser2)} imply that $ad_{X_i},ad_{Y_i}$ are locally nilpotent
 operators on $\g$.\footnote{An operator $A$ on $V$ is \emph{locally nilpotent} if for
 any vector $v\in V$, there is some $n(v)$ such that $A^{n(v)}v=0$.} The Serre
 relations say that $ad_{X_i}$ and $ad_{Y_i}$ are nilpotent on generators, and then
 the Jacobi identity implies that they are locally nilpotent. Thus,
 $s_i=\exp(ad_{X_i})\exp(-ad_{Y_i})\exp(ad_{X_i})$ is a well-defined automorphism of
 $\g$ because each power series is (locally) finite.

 As in Exercise \ref{lec14Ex:Weyl}, we get $s_i(\h)\subseteq \h$ and
 \begin{equation}\label{lec16dag}
  \lambda\bigl( s_i(h)\bigr) = \langle \lambda, s_i(h) \rangle =\langle r_i(\lambda),h\rangle
  = (r_i \lambda)(h)
 \end{equation}
 for any $h\in \h$ and any $\lambda\in \h^*$.

 Now we are ready to show that $R$ is $\weyl$-invariant. If $\alpha\in R$, with $X\in
 \g_\alpha$, then we will show that $s_i^{-1}X$ is a root vector for $r_i\alpha$. For
 $h\in \h$, we have
 \begin{align*}
   [h,s_i^{-1}X] &= s_i^{-1}([s_i h,X]) = s_i^{-1}\bigl( \alpha(s_i h)X\bigr) \\
        &= \alpha(s_ih)\,s_i^{-1}X = (r_i\alpha)(h)\,s_i^{-1}X, &\text{(by \ref{lec16dag})}
 \end{align*}
 so $r_i\alpha\in R$. So $\weyl$ preserves $R$.

 On the other hand, we know that $\pm \alpha_i \subseteq R$ from the end of Step 2, so
 we get $\Delta\subseteq R$. Moreover, for any $\alpha\in \Delta$, we have that $\dim
 \g_\alpha = 1$ because we can choose $w=r_{i_1}\cdots r_{i_k}$ and $s=s_{i_1}\cdots
 s_{i_k}$ so that $\alpha = w(\alpha_i)$ for some $i$; then $\g_\alpha =
 s(\g_{\alpha_i})$ has dimension one by the last sentence of Step 2.

 \underline{Step 4. Prove that $\Delta=R$}: Let $\lambda\in R\smallsetminus \Delta$.
 Then $\lambda$ is not proportional to any $\alpha\in \Delta$. One can find some $h$
 in the real span of the $H_i$ such that $\langle \lambda, h\rangle =0$ and $\langle
 \alpha, h\rangle \neq 0$ for all $\alpha\in \Delta$. This decomposes $\Delta$ as
 $\Delta^{+'}\coprod \Delta^{-'}$, and gives a new basis of simple roots
 $\{\beta_1,\dots, \beta_n\} = \Pi'\subseteq \Delta^{+'}$. By Proposition
 \ref{lec14P:tansitive}, $\weyl$ acts transitively on the sets of simple roots, so we can
 find some $w\in \weyl$ such that $w(\alpha_i)=\beta_i$ (after permutation of the
 $\beta_i$, if necessary). Then look at $w^{-1}(\lambda)\in R$.

 By construction $\lambda$ is neither in the non-negative span nor the non-positive
 span of the $\beta_i$, so $w^{-1}(\lambda)$ is neither in the non-negative nor the
 non-positive span of the $\alpha_i$. But we had the decomposition $R = R^+ \coprod
 R^-$ from Step 2, so this is a contradiction. Hence $\Delta = R$.

 \underline{Step 5. Check that $\g$ is semisimple}: It is enough to show that $\h$ has
 no nontrivial abelian ideals. We already know that $\g = \h\oplus
 \bigoplus_{\alpha\in \Delta} \g_\alpha$ and that each $\g_\alpha$ is 1 dimensional.
 In particular, $\g$ is \emph{finite dimensional}. We also know that the Serre
 relations hold. Notice that for any ideal $\a$, $ad_\h$-invariance implies that $\a =
 \h'\oplus_{\alpha \in S} \g_\alpha$ for some subspace $\h'\subseteq \h$ and some
 subset $S\subseteq \Delta$. If $\g_\alpha\subseteq \a$, then $X_\alpha\in \a$, so
 $[X_\alpha,Y_\alpha]=H_\alpha\in \a$ ($\a$ is an ideal), and
 $[Y_\alpha,H_\alpha]=2Y_\alpha\in \a$. Thus, we have the whole $\sl(2)$ in $\a$, so
 it cannot be abelian. So $\a = \h'\subseteq \h$. Take a nonzero element $h\in \h'$.
 Since $\{\alpha_1,\dots, \alpha_n\}$ spans $\h^*$, there is some $\alpha_i$ with
 $\alpha_i(h)\neq 0$, then $[h,X_i]=\alpha_i(h)X_i\in \a$, contradicting $\a\subseteq
 \h$.
 \end{proof}

 In the non-exceptional cases, we have nice geometric descriptions of these Lie
 algebras. Next time, we will explicitly construct the exceptional Lie algebras.

 \index{Serre's Theorem|)idxbf}
