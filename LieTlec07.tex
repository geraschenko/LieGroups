 \stepcounter{lecture}
 \setcounter{lecture}{7}
 \sektion{Lecture 7}

 Last time we talked about Hopf algebras. Our basic examples were $\CC[\Gamma]$ and
 $C(\Gamma)=\CC[\Gamma]^*$. Also, for a vector space $V$, $T(V)$ is a Hopf algebra.
 Then $S(V)=T(V)/\langle x\otimes y-y\otimes x|x,y\in V\rangle$. And we also have $U\g
 = T\g/\langle x\otimes y-y\otimes x-[x,y]|x,y\in \g\rangle$.

 Today we'll talk about the universal enveloping algebra. Later, we'll talk about
 deformations of associative algebras because that is where recent progress in
 representation theory has been.

 \subsektion{Universality of \texorpdfstring{$U\g$}{Ug}} We have that
 $\g\hookrightarrow T\g\to U\g$. And $\sigma :\g\hookrightarrow U\g$ canonical
 embedding (of vector spaces and Lie algebras). Let $A$ be an associative algebra with
 $\tau:\g\to L(A)=\{A|[a,b]=ab-ba\}$ a Lie algebra homomorphism such that
 $\tau([x,y])=\tau(x)\tau(y)-\tau(y)\tau(x)$.
 \begin{proposition}\label{lec07P:Ug}
   For any such $\tau$, there is a unique $\tau':U\g\to A$ homomorphism of associative
   algebras which extends $\tau$:
   \[\xymatrix{
    U\g \ar[r]^{\tau'} & A\\
    \g\ar[u]_\sigma \ar[ur]_\tau
   }\]
 \end{proposition}
 \begin{proof}
   Because $T(V)$ is generated (freely) by $1$ and $V$, $U\g$ is generated by $1$ and
   the elements of $\g$. Choose a basis $e_1,\dots,e_n$ of $\g$. Then we have that
   $\tau(e_i)\tau(e_j)-\tau(e_j)\tau(e_i)=\sum_k c^k_{ij} \tau(e_k)$. The   elements
   $e_{i_1}\cdots e_{i_k}$ (this is a product) span $U\g$ for indices $i_j$. From the
   commutativity of the diagram, $\tau'(e_i)=\tau(e_i)$. Since $\tau'$ is a
   homomorphism of associative algebras, we have that $\tau'(e_{i_1}\cdots
   e_{i_k})=\tau'(e_{i_1})\cdots \tau'(e_{i_k})$, so $\tau'$ is determined by $\tau$
   uniquely: $\tau'(e_{i_1}\cdots e_{i_k})=\tau(e_{i_1})\cdots \tau(e_{i_k})$. We have
   to check that the ideal we mod out by is in the kernel. But that ideal is in the
   kernel because $\tau$ is a mapping of Lie algebras.
 \end{proof}
 \begin{definition}\index{representations} A linear representation
 of $\g$ in $V$ is a pair $(V, \phi: \g\to \End(V))$, where $\phi$ is a Lie algebra
 homomorphism. If $A$ is an associative algebra, then  $(V, \phi:A\to \End(V))$ a
 linear representation of $A$ in $V$.
 \end{definition}
 \begin{corollary} There is a bijection between representations of $\g$ (as a Lie
    algebra) and representations of $U\g$ (as an associative algebra).
 \end{corollary}
 \begin{proof} $(\Rightarrow)$ By the universality, $A=End(V)$, $\tau=\phi$.
$(\Leftarrow) \g\subset L(U\g)$ is a Lie subalgebra.
 \end{proof}
 \begin{example}[Adjoint representation]\label{lec07eg:adjoint}
  \index{adjoint representation|idxbfit}
  $ad:\g\to \End \g$ given by $x:y\mapsto~[x,y]$. This is also a representation of
  $U\g$. Let $e_1,\dots, e_n$ be a basis in $\g$. Then we have that $ad_{e_i}(e_j) =
  [e_i,e_j] = \sum_k c^k_{ij} e_k$, so the matrix representing the adjoint action of
  the element $e_i$ is the matrix $(ad_{e_i})_{jk}=(c^k_{ij})$ of structural
  constants. You can check that $ad_{[e_i,e_j]} = (ad_{e_i})(ad_{e_j})
  -(ad_{e_j})(ad_{e_i})$ is same as the Jacobi identity for the $c^k_{ij}$. We get
  $ad:U\g\to \End(\g)$ by defining it on the monomials $e_{i_1}\cdots e_{i_k}$ as
  $ad_{e_{i_1}\cdots e_{i_k}}= (ad_{e_{i_1}})\cdots (ad_{e_{i_k}})$ (the product of
  matrices).
 \end{example}

 Let's look at some other properties of $U\g$.
\subsektion{Gradation in \texorpdfstring{$U\g$}{Ug}}
 Recall that $V$ is a \emph{$\ZZ_+$-graded vector space} if $V=\oplus_{n=0}^\infty
 V_n$. A linear map $f:V\to W$ between graded vector spaces is
 \emph{grading-preserving} if $f(V_n)\subseteq W_n$. If we have a tensor product
 $V\otimes W$ of graded vector spaces, it has a natural grading given by $(V\otimes
 W)_{n} = \oplus_{i=0}^n V_i\otimes W_{n-i}$. The ``geometric meaning'' of this is
 that there is a linear action of $\CC$ on $V$ such that $V_n=\{x|t(x)=t^n\cdot
 x\text{ for all } t\in \CC\}$. A graded morphism is a linear map respecting this
 action, and the tensor product has the diagonal action of $\CC$, given by $t(x\otimes
 y) = t(x)\otimes t(y)$.
\begin{example} If $V=\CC[x]$, $\frac{d}{dx}$ is not grading preserving, $x\frac{d}{dx}$ is.
\end{example}
We say that $(V,[\ ,\,])$ is a \emph{$\ZZ_+$-graded Lie algebra} if $[\ ,\,]:V\otimes
V\to V$ is grading-preserving.
\begin{example}
  Let $V$ be the space of polynomial vector fields on
  $\CC=Span(z^n\frac{d}{dz})_{n\geq 0}$. Then $V_n=\CC z^n\frac{d}{dz}$.
\end{example}
 An associative algebra $(V,m:V\otimes V\to V)$ is \emph{$\ZZ_+$-graded} if $m$ is grading-preserving.
\begin{example}
   \begin{itemize}
   \item[]

   \item[(1)] $V=\CC[x]$, where the action of $\CC$ is given by $x\mapsto tx$.

   \item[(2)] $V=\CC[x_1,\dots, x_n]$ where the degree of each variable is 1 ... this
   is the $n$-th tensor power of the previous example.

   \item[(3)] Lie algebra:
   $\mathrm{Vect}(\CC) = \{\sum_{n\ge 0} a_n x^{n+1}\der{}{x}\}$ with
   $\mathrm{Vect}_n(\CC) = \CC x^{n+1}\der{}{x}$, $deg(x)=1$.  You can embed
   $\mathrm{Vect}(\CC)$ into polynomial vector fields on $S^1$ (Virasoro algebra).

   \item[(4)] T(V) is a $\ZZ_+$-graded associative algebra, as is $S(V)$. However,
   $U\g$ is not because we have modded out by a non-homogeneous ideal. But the ideal
   is not so bad. $U\g$ is a \emph{$\ZZ_+$-filtered algebra}:
   \end{itemize}
 \end{example}

 \subsektion{Filtered spaces and algebras}\
 \begin{definition}
   $V$ is a \emph{filtered space}\index{filtered space} if it has an increasing
   filtration
   \[
    V_0\subset V_1\subset V_2 \subset \cdots \subset V
   \]
   such that $V=\bigcup V_i$, and $V_n=$ is a subspace of dimension less than or equal
   to $n$. $f:V\to W$ is a \emph{morphism of filtered vector spaces} if
   $f(V_n)\subseteq W_n$.
 \end{definition}
 We can define filtered Lie algebras and associative algebras as such that the
 bracket/multiplication are filtered maps.

 There is a functor from filtered vector spaces to graded associative algebras $Gr:V\to Gr(V)$, where $Gr(V)= V_0\oplus V_1/V_0\oplus V_2/V_1\cdots$. If $f:V\to W$ is filtration preserving, it induces a map $Gr(f):Gr(V)\to Gr(W)$ functorially such that this diagram commutes:
 \[\xymatrix{
 V\ar[r]^f \ar[d]_{Gr} & W\ar[d]^{Gr}\\
 Gr(V)\ar[r]^{Gr(f)} & Gr(W)
 }\]

 Let $A$ be an associative filtered algebra (i.e.\ $A_iA_j\subseteq A_{i+j}$) such
that for all $a\in A_i, b\in A_j$,  $ab-ba \in A_{i+j-1}$.
\begin{proposition} For such an $A$,
  \begin{itemize}
  \item[(1)] $Gr(A)$ has a natural structure of an associative, commutative algebra
  (that is, the multiplication in $A$ defines an associative, commutative
  multiplication in $Gr(A)$).
  \item[(2)] For $a\in A_{i+1}, b\in A_{j+1}$, the operation $\{aA_i,bA_j\}=
  aA_ibA_j-bA_jaA_i \mod A_{i+j}$ is a lie bracket on $Gr(A)$.
  \item[(3)] $\{x,yz\} = \{x,y\}z+y\{x,z\}$.
  \end{itemize}
\end{proposition}
  \begin{proof}
    Exercise$_1$. You need to show that the given bracket is well defined, and then do a little dance, keeping track of which graded component you are in.
  \end{proof}

  \begin{definition}
    A commutative associative algebra $B$ is called a \emph{Poisson algebra} if $B$ is also a Lie algebra with lie bracket $\{\ ,\,\}$ (called a Poisson bracket) such that $\{x,yz\} = \{x,y\}z+y\{x,z\}$ (the bracket is a derivation).
  \end{definition}
  \begin{example}
    Let $(M,\w)$ be a symplectic manifold (i.e.\ $\w$ is a closed non-degenerate
    2-form on $M$), then functions on $M$ form a Poisson algebra. We could have
    $M=\RR^{2n}$ with coordinates $p_1,\dots, p_n,q_1,\dots, q_n$, and $\w = \sum_i
    dp_i\wedge dq_i$. Then the multiplication and addition on $C^\infty(M)$ is the
    usual one, and we can define $\{f,g\} = \sum_{ij} p^{ij}
    \pder{f}{x^i}\pder{g}{x^j}$, where $\w = \sum \w_{ij}dx^i\wedge dx^j$ and
    $(p^{ij})$ is the inverse matrix to $(\w_{ij})$. You can check that this is a
    Poisson bracket.
  \end{example}

  Let's look at $U\g = \langle 1,e_i|e_ie_j-e_je_i = \sum_k c^k_{ij} e_k\rangle$. Then
  $U\g$ is filtered, with $(U\g)_n = Span\{e_{i_1}\cdots e_{i_k}|k\le n\}$. We have
  the obvious inclusion $(U\g)_n\subseteq (U\g)_{n+1}$ and $(U\g)_0=\CC\cdot 1$.
  \begin{proposition}
    \begin{itemize}
    \item[]
    \item[(1)] $U\g$ is a filtered algebra (i.e.\ $(U\g)_r(U\g)_s\subseteq
    (U\g)_{r+s}$)
    \item[(2)] $[(U\g)_r,(U\g)_s]\subseteq (U\g)_{r+s-1}$.
    \end{itemize}
  \end{proposition}
  \begin{proof}
    1) obvious. 2) Exercise$_2$ (almost obvious).
  \end{proof}

  Now we can consider $Gr(U\g) = \CC\cdot 1\oplus (\bigoplus_{r\ge 1}
  (U\g)_r/(U\g)_{r-1})$
  \begin{claim}
    $(U\g)_r/(U\g)_{r-1} \simeq S^r(\g) = $ symmetric elements of $(\CC[e_1,\dots, e_n])_r$.
  \end{claim}
  \begin{proof}
    Exercise$_3$.
  \end{proof}
  So $Gr(U\g)\simeq S(\g)$ as a commutative algebra.

 $S(\g) \cong $ Polynomial functions on $\g^* = \hom_\CC(\g,\CC)$.

 Consider $C^\infty(M)$. How can we construct a bracket $\{\ ,\,\}$ which satisfies
 Liebniz (i.e.\ $\{f,g_1g_2\}=\{f,g_1\}g_2+\{f,g_2\}g_1$). We expect that $\{f,g\}(x)=
 p^{ij}(x)\pder{f}{x^i}\pder{g}{x^j} = \langle p(x),df(x)\wedge dg(x)\rangle$. Such a
 $p$ is called a bivector field (it is a section of the bundle $TM\wedge TM\to M$). So
 a Poisson structure on $C^\infty(M)$ is the same as a bivector field $p$ on $M$
 satisfying the Jacobi identity. You can check that the Jacobi identity is some
 bilinear identity on $p^{ij}$ which follows from the Jacobi identity on $\{\ ,\,\}$.
 This is equivalent to the Schouten identity, which says that the Schouten bracket of
 some things vanishes [There should be a reference here]. This is more general than
 the symplectic case because $p^{ij}$ can be degenerate.

 Let $\g$ have the basis $e_1,\dots, e_n$ and corresponding coordinate functions
 $x^1,\dots, x^n$. On $\g^*$, we have that dual basis $e^1,\dots, e^n$ (you can
 identify these with the coordinates $x^1,\dots, x^n$), and coordinates $x_1,\dots,
 x_n$ (which you can identify with the $e_i$). The bracket on polynomial functions on
 $\g^*$ is given by
 \[
    \{p,q\} = \sum c_{ij}^k x_k \pder{p}{x_i}\pder{q}{x_j}.
 \]
 This is a Lie bracket and clearly acts by derivations.

 Next we will study the following. If you have polynomials $p,q$ on $\g^*$, you can
 try to construct an associative product $p\ast_t q = pq+tm_1(p,q)+\cdots$. We will
 discuss deformations of commutative algebras. The main example will be the universal
 enveloping algebra as a deformation of polynomial functions on $\g^*$.
