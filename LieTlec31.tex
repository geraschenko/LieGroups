 \stepcounter{lecture}
 \setcounter{lecture}{31}
 \sektion{Lecture 31 - Unitary representations of \texorpdfstring{$SL_2(\RR)$}{SL(2,R)}}

 Last lecture, we found the finite dimensional (non-unitary) representations of
 $SL_2(\RR)$.

 \subsektion{Background about infinite dimensional representations} (of a Lie group
 $G$) What is an finite dimensional representation?
 \begin{itemize}
   \item[1st guess] Banach space acted on by $G$?

   This is no good for some reasons: Look at the action of $G$ on the functions on $G$
   (by left translation). We could use $L^2$ functions, or $L^1$ or $L^p$. These are
   completely different Banach spaces, but they are essentially the same
   representation.

   \item[2nd guess] Hilbert space acted on by $G$? This is sort of okay.

   The problem is that finite dimensional representations of $SL_2(\RR)$ are NOT
   Hilbert space representations, so we are throwing away some interesting
   representations.

   \item[Solution] (Harish-Chandra) Take $\g$ to be the Lie algebra of $G$, and let
   $K$ be the maximal compact subgroup. If $V$ is an infinite dimensional
   representation of $G$, there is no reason why $\g$ should act on $V$.

   The simplest example fails. Let $\RR$ act on $L^2(\RR)$ by left translation. Then
   the Lie algebra is generated by $\der{}{x}$ (or $i\der{}{x}$) acting on $L^2(\RR)$,
   but $\der{}{x}$ of an $L^2$ function is not in $L^2$ in general.

   Let $V$ be a Hilbert space. Set $V_\w$ to be the $K$-finite vectors of $V$, which are
   the vectors contained in a finite dimensional representation of $K$. The point is
   that $K$ is compact, so $V$ splits into a Hilbert space direct sum finite dimensional
   representations of $K$, at least if $V$ is a Hilbert space. Then $V_\w$ is a
   representation of the Lie algebra $\g$, not a representation of $G$. $V_\w$ is a
   representation of the group $K$. It is a $(\g,K)$-module, which means that it is
   acted on by $\g$ and $K$ in a ``compatible'' way, where compatible means that
   \begin{enumerate}
     \item they give the same representations of the Lie algebra of $K$.
     \item $k(u)v = k(u(k^{-1} v))$ for $k\in K$, $u\in \g$, and $v\in V$.
   \end{enumerate}
   The $K$-finite vectors of an irreducible unitary representation of $G$ is
   ADMISSIBLE, which means that every representation of $K$ only occurs a
   \emph{finite} number of times. The GOOD category of representations is the
   representations of admissible $(\g,K)$-modules. It turns out that this is a really
   well behaved category.
 \end{itemize}

 We want to find the unitary irreducible representations of $G$. We will do this in
 several steps:
 \begin{enumerate}
   \item Classify all irreducible admissible representations of $G$. This was solved
   by Langlands, Harish-Chandra et.\ al.

   \item Find which have hermitian inner products $(\ ,\,)$. This is easy.

   \item Find which ones are positive definite. This is VERY HARD. We'll only do this
   for the simplest case: $SL_2(\RR)$.
 \end{enumerate}

 \subsektion{The group \texorpdfstring{$SL_2(\RR)$}{SL(2,R)}} We found some generators (in $Lie(SL_2(\RR))\otimes
 \CC$ last time: $E$, $F$, $H$, with $[H,E]=2E$, $[H,F]=-2F$, and $[E,F]=H$. We have
 that $H = -i \matrix 01{-1}0$, $E= \half \matrix 1ii{-1}$, and $F=\half \matrix
 1{-i}{-i}{-1}$. Why not use the old $\matrix 100{-1}$, $\matrix 0100$, and $\matrix
 0010$?

 Because $SL_2(\RR)$ has two different classes of Cartan subgroup: $\matrix
 a00{a^{-1}}$, spanned by $\matrix 100{-1}$, and $\matrix {\cos\theta}{\sin
 \theta}{-\sin\theta}{\cos\theta}$, spanned by $\matrix 01{-1}0$, and the second one
 is COMPACT. The point is that non-compact (abelian) groups need not have eigenvectors
 on infinite dimensional spaces. An eigenvector is the same as a weight space. The
 first thing you do is split it into weight spaces, and if your Cartan subgroup is not
 compact, you can't get started. We work with the compact subalgebra so that the
 weight spaces exist.

 Given the representation $V$, we can write it as some direct sum of eigenspaces of
 $H$, as the Lie group $H$ generates is compact (isomorphic to $S^1$). In the
 finite dimensional case, we found a HIGHEST weight, which gave us complete control
 over the representation. The trouble is that in infinite dimensions, there is no
 reason for the highest weight to exist, and in general they don't. The highest weight
 requires a finite number of eigenvalues.

 A good substituted for the highest weight vector: Look at the Casimir operator $\W =
 2EF+2FE + H^2+1$. The key point is that $\W$ is in the center of the universal
 enveloping algebra. As $V$ is assumed admissible, we can conclude that $\W$ has
 eigenvectors (because we can find a finite dimensional space acted on by $\W$). As
 $V$ is irreducible and $\W$ commutes with $G$, all of $V$ is an eigenspace of $\W$.
 We'll see that this gives us about as much information as a highest weight vector.

 Let the eigenvalue of $\W$ on $V$ be $\lambda^2$ (the square will make the
 interesting representations have integral $\lambda$; the $+1$ in $\W$ is for the same
 reason).

 Suppose $v\in V_n$, where $V_n$ is the space of vectors where $H$ has eigenvalue $n$.
 In the finite dimensional case, we looked at $Ev$, and saw that $HEv=(n+2)Ev$. What
 is $FEv$? If $v$ was a highest weight vector, we could control this. Notice that
 $\W=4FE + H^2 + 2H +1$ (using $[E,F]=H$), and $\W v=\lambda^2 v$. This says that
 $4FEv + n^2 v + 2nv + v = \lambda^2 v$. This shows that $FEv$ is a multiple of $v$.

 Now we can draw a picture of what the representation looks like:
 \newsavebox{\upright}\savebox{\upright}{\raisebox{2ex}{\xymatrix @!0 @C=3em {\ar@/^1em/[r] & }}}
 \newsavebox{\downleft}\savebox{\downleft}{\raisebox{-2ex}{\xymatrix @!0 @C=3em {\ar@/_1em/@{<-}[r] & }}}
 \newsavebox{\uprights}\savebox{\uprights}{\raisebox{2ex}{\xymatrix @!0 @C=3em {\ar@/^1em/@{.>}[r] & }}}
 \newsavebox{\downlefts}\savebox{\downlefts}{\raisebox{-2ex}{\xymatrix @!0 @C=3em {\ar@/_1em/@{<.}[r] & }}}
 \newsavebox{\pright}\savebox{\pright}{\raisebox{1ex}{\xymatrix @!0 @C=2.5em {\ar@/^.5em/[r] & }}}
 \newsavebox{\nleft}\savebox{\nleft}{\raisebox{-1ex}{\xymatrix @!0 @C=2.5em {\ar@/_.5em/@{<-}[r] & }}}
% \newsavebox{\prights}\savebox{\prights}{\raisebox{1ex}{\xymatrix @!0 @C=2.5em {\ar@/^.5em/@{.>}[r] & }}}
% \newsavebox{\nlefts}\savebox{\nlefts}{\raisebox{-1ex}{\xymatrix @!0 @C=2.5em {\ar@/_.5em/@{<-}[r] & }}}
 \[
  \begin{xy}<3.5em,0em>:
   (0,0) *!R{\cdots} *++{\,}; p+(.5,0) *{\usebox{\uprights}} *{\usebox{\downleft}};
   p+(.5,0) *{v_{n-4}};       p+(.5,0) *{\usebox{\uprights}} *{\usebox{\downleft}};
   p+(.5,0) *{v_{n-2}};       p+(.5,0) *{\usebox{\uprights}} *{\usebox{\downleft}};
   p+(.5,0) *{v_{n}};         p+(.5,0) *{\usebox{\upright}} *{\usebox{\downleft}};
     p+(0,-.4) *+!U{\mbox{\scriptsize $\left(\frac{n^2+2n+1-\lambda^2}{4}\right)$}},
   p+(.5,0) *{v_{n+2}};       p+(.5,0) *{\usebox{\upright}} *{\usebox{\downlefts}};
   p+(.5,0) *{v_{n+4}};       p+(.5,0) *{\usebox{\upright}} *{\usebox{\downlefts}};
   p+(.5,0) *!L{\cdots} *++{\,};
   p+(.7,0) *{\shortstack{$E$\\ \vspace{.25em} \\ $H$ \\ \vspace{.25em} \\$F$}}
 \end{xy}
 \]
% \[\xymatrix @!0 @C=14mm{
%  **[l] \cdots \ar@{.>}@/^5mm/[r] \ar@/_5mm/@{<-}[r] &
%  v_{n-4} \ar@{.>}@/^5mm/[r] \ar@/_5mm/@{<-}[r] &
%  v_{n-2} \ar@{.>}@/^5mm/[r] \ar@/_5mm/@{<-}[r] &
%  v_{n}  \ar@/^5mm/[r] \ar@/_5mm/@{<-}[r]_{\left(\frac{n^2+2n+1-\lambda^2}{4}\right)} &
%  v_{n+2}  \ar@/^5mm/[r] \ar@/_5mm/@{<.}[r] &
%  v_{n+4}  \ar@/^5mm/[r] \ar@/_5mm/@{<.}[r] &
%  **[r] \cdots
%  & \shortstack{$E$\\ \vspace{1mm} \\ $H$ \\ \vspace{1mm} \\$F$}
% }\]
 Thus, $V_\w$ is spanned by $V_{n+2k}$, where $k$ is an integer. The non-zero elements
 among the $V_{n+2k}$ are linearly independent as they have different eigenvalues. The
 only question remaining is whether any of the $V_{n+2k}$ vanish.

 There are four possible shapes for an irreducible representation
 \begin{itemize}
   \item infinite in both directions:
    $\begin{xy}
       (0,0) *!R{\cdots} *++{\,};p+(.5,0) *{\usebox{\pright}} *{\usebox{\nleft}};
       p+(.5,0) *{\udot};        p+(.5,0) *{\usebox{\pright}} *{\usebox{\nleft}};
       p+(.5,0) *{\udot};        p+(.5,0) *{\usebox{\pright}} *{\usebox{\nleft}};
       p+(.5,0) *{\udot};        p+(.5,0) *{\usebox{\pright}} *{\usebox{\nleft}};
       p+(.5,0) *{\udot};        p+(.5,0) *{\usebox{\pright}} *{\usebox{\nleft}};
       p+(.5,0) *!L{\cdots} *++{\,};
       p+(.7,0) *{\shortstack{$E$\\ $H$ \\$F$}}
    \end{xy}$

%    $\xymatrix @!0 @C=10mm{
%     **[l] \cdots \ar@/^3mm/[r] \ar@/_3mm/@{<-}[r] &
%     \udot \ar@/^3mm/[r] \ar@/_3mm/@{<-}[r] &
%     \udot \ar@/^3mm/[r] \ar@/_3mm/@{<-}[r] &
%     \udot \ar@/^3mm/[r] \ar@/_3mm/@{<-}[r] &
%     \udot \ar@/^3mm/[r] \ar@/_3mm/@{<-}[r] & **[r] \cdots
%     & \shortstack{$E$\\ $H$ \\$F$}
%    }$
%    \mpar{I don't know if these pictures are really needed.}

   \item a lowest weight, and infinite in the other direction:
   \[\begin{xy}
       (0,0) *!R{\cdots} *++{\,};p+(.5,0) *{\usebox{\pright}} *{\usebox{\nleft}};
       p+(.5,0) *{\udot};        p+(.5,0) *{\usebox{\pright}} *{\usebox{\nleft}};
       p+(.5,0) *{\udot};        p+(.5,0) *{\usebox{\pright}} *{\usebox{\nleft}};
       p+(.5,0) *{\udot};        p+(.5,0) *{\usebox{\pright}} *{\usebox{\nleft}};
       p+(.5,0) *{\udot};        p+(.5,0) *{\usebox{\pright}} *{\usebox{\nleft}};
       p+(.5,0) *{\udot};
       p+(.7,0) *{\shortstack{$E$\\ $H$ \\$F$}}
    \end{xy}\]

%    \[\xymatrix @!0 @C=12mm{
%     **[l] \cdots \ar@/^5mm/[r] \ar@/_5mm/@{<-}[r] &
%     \udot \ar@/^5mm/[r] \ar@/_5mm/@{<-}[r] &
%     \udot \ar@/^5mm/[r] \ar@/_5mm/@{<-}[r] &
%     \udot \ar@/^5mm/[r] \ar@/_5mm/@{<-}[r] &
%     \udot \ar@/^5mm/[r] \ar@/_5mm/@{<-}[r] & \udot
%     & \shortstack{$E$\\ \vspace{1mm} \\ $H$ \\ \vspace{1mm} \\$F$}
%    }\]

   \item a highest weight, and infinite in the other direction:
    \[\begin{xy}
       (0,0)    *{\udot} *++{\,};p+(.5,0) *{\usebox{\pright}} *{\usebox{\nleft}};
       p+(.5,0) *{\udot};        p+(.5,0) *{\usebox{\pright}} *{\usebox{\nleft}};
       p+(.5,0) *{\udot};        p+(.5,0) *{\usebox{\pright}} *{\usebox{\nleft}};
       p+(.5,0) *{\udot};        p+(.5,0) *{\usebox{\pright}} *{\usebox{\nleft}};
       p+(.5,0) *{\udot};        p+(.5,0) *{\usebox{\pright}} *{\usebox{\nleft}};
       p+(.5,0) *!L{\cdots} *++{\,};
       p+(.7,0) *{\shortstack{$E$\\ $H$ \\$F$}}
    \end{xy}\]

%    \[\xymatrix @!0 @C=12mm{
%     \udot \ar@/^5mm/[r] \ar@/_5mm/@{<-}[r] &
%     \udot \ar@/^5mm/[r] \ar@/_5mm/@{<-}[r] &
%     \udot \ar@/^5mm/[r] \ar@/_5mm/@{<-}[r] &
%     \udot \ar@/^5mm/[r] \ar@/_5mm/@{<-}[r] &
%     \udot \ar@/^5mm/[r] \ar@/_5mm/@{<-}[r] & **[r] \cdots
%     & \shortstack{$E$\\ \vspace{1mm} \\ $H$ \\ \vspace{1mm} \\$F$}
%    }\]

   \item we have a highest weight and a lowest weight, in which case it is
    finite dimensional
    $\begin{xy}
       (0,0)    *{\udot} *++{\,};p+(.5,0) *{\usebox{\pright}} *{\usebox{\nleft}};
       p+(.5,0) *{\udot};        p+(.5,0) *{\usebox{\pright}} *{\usebox{\nleft}};
       p+(.5,0) *!L{\cdots};
       p+(.9,0) *{\usebox{\pright}} *{\usebox{\nleft}};
       p+(.5,0) *{\udot};        p+(.5,0) *{\usebox{\pright}} *{\usebox{\nleft}};
       p+(.5,0) *{\udot} *++{\,};
       p+(.7,0) *{\shortstack{$E$\\ $H$ \\$F$}}
    \end{xy}$

%    \[\xymatrix @!0 @C=12mm{
%     \udot \ar@/^5mm/[r] \ar@/_5mm/@{<-}[r] &
%     \udot \ar@/^5mm/[r] \ar@/_5mm/@{<-}[r] &
%     \cdots \ar@/^5mm/[r] \ar@/_5mm/@{<-}[r] &
%     \udot \ar@/^5mm/[r] \ar@/_5mm/@{<-}[r] &
%     \udot
%     & \shortstack{$E$\\ \vspace{1mm} \\ $H$ \\ \vspace{1mm} \\$F$}
%    }\]

 \end{itemize}
 We'll see that all these show up. We also see that an irreducible representation is
 completely determined once we know $\lambda$ and some $n$ for which $V_n\neq 0$. The
 remaining question is to construct representations with all possible values of
 $\lambda\in \CC$ and $n\in \ZZ$. $n$ is an integer because it must be a
 representations of the circle.

 If $n$ is even, we have
 \savebox{\upright}{\raisebox{2ex}{\xymatrix @!0 @C=2.5em {\ar@/^1em/[r] & }}}
 \savebox{\downleft}{\raisebox{-2ex}{\xymatrix @!0 @C=2.5em {\ar@/_1em/@{<-}[r] & }}}
 \[
  \begin{xy}<3em,0em>:
   (0,0) *!R{\cdots} *++{\,}; p+(.5,0) *{\usebox{\upright}} *{\usebox{\downleft}};
   p+(.5,0) *{-6};       p+(.5,0) *{\usebox{\upright}} *{\usebox{\downleft}};
   p+(.5,0) *{-4};       p+(.5,0) *{\usebox{\upright}} *{\usebox{\downleft}};
   p+(.5,0) *{-2};       p+(.5,0) *{\usebox{\upright}} *{\usebox{\downleft}};
   p+(.5,0) *{0};         p+(.5,0) *{\usebox{\upright}} *{\usebox{\downleft}};
   p+(.5,0) *{2};       p+(.5,0) *{\usebox{\upright}} *{\usebox{\downleft}};
   p+(.5,0) *{4};       p+(.5,0) *{\usebox{\upright}} *{\usebox{\downleft}};
   p+(.5,0) *{6};       p+(.5,0) *{\usebox{\upright}} *{\usebox{\downleft}};
   p+(.5,0) *!L{\cdots} *++{\,};
   p+(.7,0) *{\shortstack{$E$\\ \vspace{.25em} \\ $H$ \\ \vspace{.25em} \\$F$}};
   (.5,.65) *{{}^\frac{\lambda-7}{2}};
   p+(1,0) *{{}^\frac{\lambda-5}{2}};
   p+(1,0) *{{}^\frac{\lambda-3}{2}};
   p+(1,0) *{{}^\frac{\lambda-1}{2}};
   p+(1,0) *{{}^\frac{\lambda+1}{2}};
   p+(1,0) *{{}^\frac{\lambda+3}{2}};
   p+(1,0) *{{}^\frac{\lambda+5}{2}};
   p+(1,0) *{{}^\frac{\lambda+7}{2}};
   (.5,-.65) *{{}^\frac{\lambda+7}{2}};
   p+(1,0) *{{}^\frac{\lambda+5}{2}};
   p+(1,0) *{{}^\frac{\lambda+3}{2}};
   p+(1,0) *{{}^\frac{\lambda+1}{2}};
   p+(1,0) *{{}^\frac{\lambda-1}{2}};
   p+(1,0) *{{}^\frac{\lambda-3}{2}};
   p+(1,0) *{{}^\frac{\lambda-5}{2}};
   p+(1,0) *{{}^\frac{\lambda-7}{2}};
 \end{xy}
 \]
%    \[\xymatrix @!0 @C=12mm{
%     **[l] \cdots \ar@/^5mm/[r]^{\frac{\lambda-7}{2}} \ar@/_5mm/@{<-}[r]_{\frac{\lambda+7}{2}} &
%     {-6} \ar@/^5mm/[r]^{\frac{\lambda-5}{2}} \ar@/_5mm/@{<-}[r]_{\frac{\lambda+5}{2}} &
%     {-4} \ar@/^5mm/[r]^{\frac{\lambda-3}{2}} \ar@/_5mm/@{<-}[r]_{\frac{\lambda+3}{2}} &
%     {-2} \ar@/^5mm/[r]^{\frac{\lambda-1}{2}} \ar@/_5mm/@{<-}[r]_{\frac{\lambda+1}{2}} &
%     {0}  \ar@/^5mm/[r]^{\frac{\lambda+1}{2}} \ar@/_5mm/@{<-}[r]_{\frac{\lambda-1}{2}} &
%     {2}  \ar@/^5mm/[r]^{\frac{\lambda+3}{2}} \ar@/_5mm/@{<-}[r]_{\frac{\lambda-3}{2}} &
%     {4}  \ar@/^5mm/[r]^{\frac{\lambda+5}{2}} \ar@/_5mm/@{<-}[r]_{\frac{\lambda-5}{2}} &
%     {6}  \ar@/^5mm/[r]^{\frac{\lambda+7}{2}} \ar@/_5mm/@{<-}[r]_{\frac{\lambda-7}{2}} & **[r] \cdots
%     & \shortstack{$E$\\ \vspace{1mm} \\ $H$ \\ \vspace{1mm} \\$F$}\\
%    }\]

 It is easy to check that these maps satisfy $[E,F]=H$, $[H,E]=2E$, and $[H,F]=-2F$

 \begin{exercise}
   Do the case of $n$ odd.
 \end{exercise}

 Problem: These may not be irreducible, and we want to decompose them into irreducible
 representations. The only way they can fail to be irreducible if if $Ev_n=0$ of
 $Fv_n=0$ for some $n$ (otherwise, from any vector, you can generate the whole space).
 The only ways that can happen is if
 \[\begin{tabular}{l}
   $n$ even: $\lambda$ an odd integer\\
   $n$ odd: $\lambda$ an even integer.\\
 \end{tabular}\]
 What happens in these cases? The easiest thing is probably just to write out an
 example.
 \begin{example}
   Take $n$ even, and $\lambda=3$, so we have
 \[
  \begin{xy}<3em,0em>:
   (0,0) *!R{\cdots} *++{\,}; p+(.5,0) *{\usebox{\upright}} *{\usebox{\downleft}};
   p+(.5,0) *{-6};       p+(.5,0) *{\usebox{\upright}} *{\usebox{\downleft}};
   p+(.5,0) *{-4};       p+(.5,0) *{\usebox{\upright}} *{\usebox{\downleft}};
   p+(.5,0) *{-2};       p+(.5,0) *{\usebox{\upright}} *{\usebox{\downleft}};
   p+(.5,0) *{0};         p+(.5,0) *{\usebox{\upright}} *{\usebox{\downleft}};
   p+(.5,0) *{2};       p+(.5,0) *{\usebox{\upright}} *{\usebox{\downleft}};
   p+(.5,0) *{4};       p+(.5,0) *{\usebox{\upright}} *{\usebox{\downleft}};
   p+(.5,0) *{6};       p+(.5,0) *{\usebox{\upright}} *{\usebox{\downleft}};
   p+(.5,0) *!L{\cdots} *++{\,};
   p+(.7,0) *{\shortstack{$E$\\ \vspace{.25em} \\ $H$ \\ \vspace{.25em} \\$F$}};
   (.5,.55) *{{}^{-2}};
   p+(1,0) *{{}^{-1}};
   p+(1,0) *{{}^0};
   p+(1,0) *{{}^1};
   p+(1,0) *{{}^2};
   p+(1,0) *{{}^3};
   p+(1,0) *{{}^4};
   p+(1,0) *{{}^5};
   (.5,-.65) *{{}^5};
   p+(1,0) *{{}^4};
   p+(1,0) *{{}^3};
   p+(1,0) *{{}^2};
   p+(1,0) *{{}^1};
   p+(1,0) *{{}^0};
   p+(1,0) *{{}^{-1}};
   p+(1,0) *{{}^{-2}};
   (0,1);(0,-1) **\crv{(3.3,1)&(3.3,-1)};
   (8,1);(8,-1) **\crv{(4.7,1)&(4.7,-1)};
 \end{xy}
 \]
%    \[\xymatrix @!0 @R=12mm @C=12mm{
%     {} \POS[]; [dd]**\crv{<4cm,0cm>&<4cm,-24mm>}
%     & & & & & & & & {} \POS[]; [dd]**\crv{<5.7cm,0cm>&<5.7cm,-24mm>}\\
%     **[l] \cdots \ar@/^5mm/[r]^{-2} \ar@/_5mm/@{<-}[r]_{5} &
%     {-6} \ar@/^5mm/[r]^{-1} \ar@/_5mm/@{<-}[r]_{4} &
%     {-4} \ar@/^5mm/[r]^{0} \ar@/_5mm/@{<-}[r]_{3} &
%     {-2} \ar@/^5mm/[r]^{1} \ar@/_5mm/@{<-}[r]_{2} &
%     {0}  \ar@/^5mm/[r]^{2} \ar@/_5mm/@{<-}[r]_{1} &
%     {2}  \ar@/^5mm/[r]^{3} \ar@/_5mm/@{<-}[r]_{0} &
%     {4}  \ar@/^5mm/[r]^{4} \ar@/_5mm/@{<-}[r]_{-1} &
%     {6}  \ar@/^5mm/[r]^{5} \ar@/_5mm/@{<-}[r]_{-2} & **[r] \cdots
%     & \shortstack{$E$\\ \vspace{1mm} \\ $H$ \\ \vspace{1mm} \\$F$}\\
%     {} & & & & & & & & {}\\
%    }\]
   You can just see what the irreducible subrepresentations are ... they are shown in
   the picture. So $V$ has two irreducible subrepresentations $V_-$ and $V_+$, and
   $V/(V_-\oplus V_+)$ is an irreducible 3 dimensional representation.
 \end{example}
 \begin{example}
   If $n$ is even, but $\lambda$ is negative, say $\lambda=-3$, we get
 \[
  \begin{xy}<3em,0em>:
   (0,0) *!R{\cdots} *++{\,}; p+(.5,0) *{\usebox{\upright}} *{\usebox{\downleft}};
   p+(.5,0) *{-6};       p+(.5,0) *{\usebox{\upright}} *{\usebox{\downleft}};
   p+(.5,0) *{-4};       p+(.5,0) *{\usebox{\upright}} *{\usebox{\downleft}};
   p+(.5,0) *{-2};       p+(.5,0) *{\usebox{\upright}} *{\usebox{\downleft}};
   p+(.5,0) *{0};         p+(.5,0) *{\usebox{\upright}} *{\usebox{\downleft}};
   p+(.5,0) *{2};       p+(.5,0) *{\usebox{\upright}} *{\usebox{\downleft}};
   p+(.5,0) *{4};       p+(.5,0) *{\usebox{\upright}} *{\usebox{\downleft}};
   p+(.5,0) *{6};       p+(.5,0) *{\usebox{\upright}} *{\usebox{\downleft}};
   p+(.5,0) *!L{\cdots} *++{\,};
   p+(.7,0) *{\shortstack{$E$\\ \vspace{.25em} \\ $H$ \\ \vspace{.25em} \\$F$}};
   (.5,.55) *{{}^{-5}};
   p+(1,0) *{{}^{-4}};
   p+(1,0) *{{}^{-3}};
   p+(1,0) *{{}^{-2}};
   p+(1,0) *{{}^{-1}};
   p+(1,0) *{{}^0};
   p+(1,0) *{{}^1};
   p+(1,0) *{{}^2};
   (.5,-.65) *{{}^2};
   p+(1,0) *{{}^1};
   p+(1,0) *{{}^0};
   p+(1,0) *{{}^{-1}};
   p+(1,0) *{{}^{-2}};
   p+(1,0) *{{}^{-3}};
   p+(1,0) *{{}^{-4}};
   p+(1,0) *{{}^{-5}};
   (4,1);(4,1) **\crv{(2.5,1)&(2.2,-1.1)&(5.8,-1.1)&(5.5,1)};
 \end{xy}
 \]
%    \[\xymatrix @!0 @R=12mm @C=12mm{
%     & & & &
%     {\!\!\!-\!\!\!} \POS[]; [] **\crv{<3cm,0cm>&<3cm,-22mm>&<5cm,-24mm>&<6.5cm,-22mm>&<6.5cm,0mm>}
%     & & & & \\
%     **[l] \cdots \ar@/^5mm/[r]^{-4} \ar@/_5mm/@{<-}[r]_{2} &
%     {-6} \ar@/^5mm/[r]^{-3} \ar@/_5mm/@{<-}[r]_{1} &
%     {-4} \ar@/^5mm/[r]^{-1} \ar@/_5mm/@{<-}[r]_{0} &
%     {-2} \ar@/^5mm/[r]^{-2} \ar@/_5mm/@{<-}[r]_{-1} &
%     {0}  \ar@/^5mm/[r]^{-1} \ar@/_5mm/@{<-}[r]_{-2} &
%     {2}  \ar@/^5mm/[r]^{0} \ar@/_5mm/@{<-}[r]_{-3} &
%     {4}  \ar@/^5mm/[r]^{1} \ar@/_5mm/@{<-}[r]_{-4} &
%     {6}  \ar@/^5mm/[r]^{2} \ar@/_5mm/@{<-}[r]_{-5} & **[r] \cdots
%     & \shortstack{$E$\\ \vspace{1mm} \\ $H$ \\ \vspace{1mm} \\$F$}\\
%     & & & & & & & & \\
%    }\]
   Here we have an irreducible finite dimensional representation. If you quotient out
   by that subrepresentation, you get $V_+\oplus V_-$.
 \end{example}
 \begin{exercise}
   Show that for $n$ odd, and $\lambda=0$, $V=V_+\oplus V_-$.
 \end{exercise}
 So we have a complete list of all irreducible admissible representations:
 \begin{enumerate}
   \item if $\lambda\not\in \ZZ$, you get one representation (remember $\lambda\equiv
   -\lambda$). This is the bi-infinite case.

   \item Finite dimensional representation for each $n\ge 1$ ($\lambda=\pm n$)

   \item Discrete series for each $\lambda\in \ZZ\smallsetminus \{0\}$, which is the
   half infinite case: you get a lowest weight when $\lambda< 0$ and a highest weight
   when $\lambda>0$.

   \item two ``limits of discrete series'' where $n$ is odd and $\lambda=0$.
 \end{enumerate}
 Which of these can be made into \emph{unitary} representations? $H^\dag = -H$,
 $E^\dag =F$, and $F^\dag = E$. If we have a hermitian inner product $(\ ,\,)$, we see
 that
 \begin{align*}
 (v_{j+2},v_{j+2}) &= \frac{2}{\lambda + j+1} (Ev_j,v_{j+2})\\
            &= \frac{2}{\lambda + j+1}(v_j,-Fv_{j+2}) \\
            &= - \frac{2}{\lambda + j+1} \frac{\overline{\lambda - j-1}}{2} (v_j,v_j)
            >0
 \end{align*}
 where we fix the sign errors. So we want $-\frac{\overline{\lambda-1-j}}{\lambda+j+1}$
 to be real and positive whenever $j,j+2$ are non-zero eigenvectors. So
 \[
    -(\lambda-1-j)(\lambda+1+j) = -\lambda^2 + (j+1)^2
 \]
 should be positive for all $j$. Conversely, when you have this, blah.

 This condition is satisfied in the following cases:
 \begin{enumerate}
   \item $\lambda^2\le 0$. These representations are called PRINCIPAL SERIES
   representations. These are all irreducible \emph{except} when $\lambda=0$ and $n$
   is odd, in which case it is the sum of two limits of discrete series representations

   \item $0< \lambda < 1$ and $j$ even. These are called COMPLEMENTARY SERIES. They
   are annoying, and you spend a lot of time trying to show that they don't occur.

   \item $\lambda^2 =n^2$ for $n\ge 1$ (for some of the irreducible pieces).

   If $\lambda=1$, we get
 \[
  \begin{xy}<3em,0em>:
   (0,0) *!R{\cdots} *++{\,}; p+(.5,0) *{\usebox{\upright}} *{\usebox{\downleft}};
   p+(.5,0) *{-6};       p+(.5,0) *{\usebox{\upright}} *{\usebox{\downleft}};
   p+(.5,0) *{-4};       p+(.5,0) *{\usebox{\upright}} *{\usebox{\downleft}};
   p+(.5,0) *{-2};       p+(.5,0) *{\usebox{\upright}} *{\usebox{\downleft}};
   p+(.5,0) *{0};         p+(.5,0) *{\usebox{\upright}} *{\usebox{\downleft}};
   p+(.5,0) *{2};       p+(.5,0) *{\usebox{\upright}} *{\usebox{\downleft}};
   p+(.5,0) *{4};       p+(.5,0) *{\usebox{\upright}} *{\usebox{\downleft}};
   p+(.5,0) *{6};       p+(.5,0) *{\usebox{\upright}} *{\usebox{\downleft}};
   p+(.5,0) *!L{\cdots} *++{\,};
   p+(.7,0) *{\shortstack{$E$\\ \vspace{.25em} \\ $H$ \\ \vspace{.25em} \\$F$}};
   (.5,.55) *{{}^{-3}};
   p+(1,0) *{{}^{-2}};
   p+(1,0) *{{}^{-1}};
   p+(1,0) *{{}^0};
   p+(1,0) *{{}^1};
   p+(1,0) *{{}^2};
   p+(1,0) *{{}^3};
   p+(1,0) *{{}^4};
   (.5,-.65) *{{}^4};
   p+(1,0) *{{}^3};
   p+(1,0) *{{}^2};
   p+(1,0) *{{}^1};
   p+(1,0) *{{}^0};
   p+(1,0) *{{}^{-1}};
   p+(1,0) *{{}^{-2}};
   p+(1,0) *{{}^{-3}};
   (0,1);(0,-1) **\crv{(4.6,1.2)&(4.6,-1.2)};
   (8,1);(8,-1) **\crv{(3.4,1.2)&(3.4,-1.2)};
 \end{xy}
 \]
%    \[\xymatrix @!0 @R=12mm @C=12mm{
%     {} \POS[]; [dd]**\crv{<55mm,3mm>&<55mm,-27mm>}
%     & & & & & & & & {} \POS[]; [dd]**\crv{<41mm,3mm>&<41mm,-27mm>}\\
%     **[l] \cdots \ar@/^5mm/[r]^{-3} \ar@/_5mm/@{<-}[r]_{4} &
%     {-6} \ar@/^5mm/[r]^{-2} \ar@/_5mm/@{<-}[r]_{3} &
%     {-4} \ar@/^5mm/[r]^{-1} \ar@/_5mm/@{<-}[r]_{2} &
%     {-2} \ar@/^5mm/[r]^{0} \ar@/_5mm/@{<-}[r]_{1} &
%     {0}  \ar@/^5mm/[r]^{1} \ar@/_5mm/@{<-}[r]_{0} &
%     {2}  \ar@/^5mm/[r]^{2} \ar@/_5mm/@{<-}[r]_{-1} &
%     {4}  \ar@/^5mm/[r]^{3} \ar@/_5mm/@{<-}[r]_{-2} &
%     {6}  \ar@/^5mm/[r]^{4} \ar@/_5mm/@{<-}[r]_{-3} & **[r] \cdots
%     & \shortstack{$E$\\ \vspace{1mm} \\ $H$ \\ \vspace{1mm} \\$F$}\\
%     {} & & & & & & & & {}\\
%    }\]
   We see that we get two discrete series and a 1 dimensional representation, all of
   which are unitary

   For $\lambda=2$ (this is the more generic one), we have
 \[
  \begin{xy}<3em,0em>:
   (0,0) *!R{\cdots} *++{\,}; p+(.5,0) *{\usebox{\upright}} *{\usebox{\downleft}};
   p+(.5,0) *{-5};       p+(.5,0) *{\usebox{\upright}} *{\usebox{\downleft}};
   p+(.5,0) *{-3};       p+(.5,0) *{\usebox{\upright}} *{\usebox{\downleft}};
   p+(.5,0) *{-1};       p+(.5,0) *{\usebox{\upright}} *{\usebox{\downleft}};
   p+(.5,0) *{1};       p+(.5,0) *{\usebox{\upright}} *{\usebox{\downleft}};
   p+(.5,0) *{3};       p+(.5,0) *{\usebox{\upright}} *{\usebox{\downleft}};
   p+(.5,0) *{5};       p+(.5,0) *{\usebox{\upright}} *{\usebox{\downleft}};
   p+(.5,0) *!L{\cdots} *++{\,};
   p+(.7,0) *{\shortstack{$E$\\ \vspace{.25em} \\ $H$ \\ \vspace{.25em} \\$F$}};
   (.5,.55) *{{}^{-2}};
   p+(1,0) *{{}^{-1}};
   p+(1,0) *{{}^0};
   p+(1,0) *{{}^1};
   p+(1,0) *{{}^2};
   p+(1,0) *{{}^3};
   p+(1,0) *{{}^4};
   (.5,-.65) *{{}^4};
   p+(1,0) *{{}^3};
   p+(1,0) *{{}^2};
   p+(1,0) *{{}^1};
   p+(1,0) *{{}^0};
   p+(1,0) *{{}^{-1}};
   p+(1,0) *{{}^{-2}};
   (0,1);(0,-1) **\crv{(3.2,1)&(3.2,-1)};
   (7,1);(7,-1) **\crv{(3.8,1)&(3.8,-1)};
 \end{xy}
 \]
%    \[\xymatrix @!0 @R=12mm @C=12mm{
%     {} \POS[]; [dd]**\crv{<4cm,0cm>&<4cm,-24mm>}
%     & & & & & & & {} \POS[]; [dd]**\crv{<45mm,0cm>&<45mm,-24mm>}\\
%     **[l] \cdots \ar@/^5mm/[r]^{-2} \ar@/_5mm/@{<-}[r]_{4} &
%     {-5} \ar@/^5mm/[r]^{-1} \ar@/_5mm/@{<-}[r]_{3} &
%     {-3} \ar@/^5mm/[r]^{0} \ar@/_5mm/@{<-}[r]_{2} &
%     {-1} \ar@/^5mm/[r]^{1} \ar@/_5mm/@{<-}[r]_{1} &
%     {1}  \ar@/^5mm/[r]^{2} \ar@/_5mm/@{<-}[r]_{0} &
%     {3}  \ar@/^5mm/[r]^{3} \ar@/_5mm/@{<-}[r]_{-1} &
%     {5}  \ar@/^5mm/[r]^{4} \ar@/_5mm/@{<-}[r]_{-2} & **[r] \cdots
%     & \shortstack{$E$\\ \vspace{1mm} \\ $H$ \\ \vspace{1mm} \\$F$}\\
%     {} & & & & & & & {}\\
%    }\]
%    \mpar{What is going on here? $\lambda\neq n^2$}

   The middle representation (where $(j+1)^2<\lambda^2=4$ is NOT unitary, which we
   already knew. So the DISCRETE SERIES representations ARE unitary, and the FINITE
   dimensional representations of dimension greater than or equal to 2 are NOT.
 \end{enumerate}

 Summary: the irreducible unitary representations of $SL_2(\RR)$ are given by
 \begin{enumerate}
   \item the 1 dimensional representation
   \item Discrete series representations for any $\lambda\in \ZZ\smallsetminus \{0\}$
   \item Two limit of discrete series representations for $\lambda=0$
   \item Two series of principal series representations:
   \[\begin{tabular}{l}
     $j$ even: $\lambda\in i\RR$, $\lambda \ge 0$\\
     $j$ odd: $\lambda \in i\RR$, $\lambda > 0$
   \end{tabular}\]
   \item Complementary series: parameterized by $\lambda$, with $0< \lambda < 1$.
 \end{enumerate}

 The nice stuff that happened for $SL_2(\RR)$ breaks down for more complicated Lie
 groups.

 Representations of finite covers of $SL_2(\RR)$ are similar, except $j$ need not be
 integral. For example, for the double cover $\widehat{SL_2(\RR)} = Mp_2(\RR)$, $2j\in
 \ZZ$.
 \begin{exercise}
   Find the irreducible unitary representations of $Mp_2(\RR)$.
 \end{exercise}

 \index{Borcherds, Richard E.|)}
