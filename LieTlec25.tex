 \stepcounter{lecture}
 \setcounter{lecture}{25}
 \sektion{Lecture 25 - \texorpdfstring{$E_8$}{E8}}

In this lecture we use a vector notation in which powers represent repetitions:  so
$(1^8)=(1,1,1,1,1,1,1,1)$ and $(\pm \half^2 , 0^6)=(\pm \half, \pm \half,
0,0,0,0,0,0)$.

  Recall that $E_8$ has the Dynkin diagram
 \[\begin{xy}
   (0,0) *+!U{e_1-e_2} *\cir<2pt>{};
   (1,0) *+!D{e_2-e_3} *\cir<2pt>{} **@{-};
   p+(1,0) *+!U{e_3-e_4} *\cir<2pt>{} **@{-};
   p+(1,0) *+!D{e_4-e_5} *\cir<2pt>{} **@{-};
   p+(1,0) *+!U{e_5-e_6} *\cir<2pt>{} **@{-};
   p+(0,1) *+!D{(-\half^5 , \, \half^3)} *\cir<2pt>{} **@{-},
   p+(1,0) *+!D{e_6-e_7} *\cir<2pt>{} **@{-};
   p+(1,0) *+!U{e_7-e_8} *\cir<2pt>{} **@{-};
 \end{xy} \]
  where each vertex is a root $r$ with $(r,r)=2$; $(r,s)=0$ when $r$ and $s$ are
  not joined, and $(r,s)=-1$ when $r$ and $s$ are joined. We choose an orthonormal basis
   $e_1,\dots, e_8$, in which the roots are as given.

  We want to figure out what the root lattice $L$ of $E_8$ is (this is the lattice
  generated by the roots). If you take $\{e_i-e_{i+1}\} \cup (-1^5 , \, 1^3)$ (all the
  $A_7$ vectors plus twice the strange vector), they generate the $D_8$ lattice
  $=\{(x_1,\dots, x_8)|x_i\in \ZZ, \quad \sum x_i \text{ even}\}$. So the $E_8$
  lattice consists of two cosets of this lattice, where the other coset is
  $\{(x_1,\dots, x_8)|x_i\in \ZZ+\frac{1}{2},\quad \sum x_i\text{ odd}\}$.

  Alternative version: If you reflect this lattice through the hyperplane $e_1^\perp$,
  then you get the same thing except that $\sum x_i$ is always even.  We will freely
  use both characterizations, depending on which is more convenient for the
  calculation at hand.

  We should also work out the weight lattice, which is the vectors $s$ such that
  $(r,r)/2$ divides $(r,s)$ for all roots $r$. Notice that the weight lattice of $E_8$
  is contained in the weight lattice of $D_8$, which is the union of four cosets of
  $D_8$:  $D_8$, $D_8+(1 , \, 0^7)$, $D_8+(\half^8)$ and $ D_8+(-\half, \, \half^7)$.
  Which of these have integral inner product with the vector $(-\half^5 , \,
  \half^3)$? They are the first and the last, so the weight lattice of $E_8$ is
  $D_8\cup D_8+(-\half, \half^7)$, which is equal to the root lattice of $E_8$.

  In other words, the $E_8$ lattice $L$ is UNIMODULAR (equal to its dual $L'$), where
  the dual is the lattice of vectors having integral inner product with all lattice
  vectors.  This is also true of $G_2$ and $F_4$, but is not in general true of Lie
  algebra lattices.

  The $E_8$ lattice is EVEN, which means that the inner product of any vector with
  itself is always even.

  Even unimodular lattices in $\RR^n$ only exist if $8|n$ (this $8$ is the same $8$
 that shows up in the periodicity of Clifford groups). The $E_8$ lattice is the only
 example in dimension equal to $8$ (up to isomorphism, of course). There are two in
 dimension 16 (one of which is $L\oplus L$, the other is $D_{16}\cup$ some coset).
 There are 24 in dimension 24, which are the Niemeier lattices. In 32 dimensions,
 there are more than a billion!

 The Weyl group of $E_8$ is generated by the reflections through $s^\perp$ where $s
 \in L$ and $(s,s)=2$ (these are called roots). First, let's find all the roots:
 $(x_1,\dots,x_8)$ such that $\sum x_i^2=2$ with $x_i\in \ZZ$ or $\ZZ+\half$ and $\sum
 x_i$ even. If $x_i \in \ZZ$, obviously the only solutions are permutations of $(\pm
 1, \pm 1, 0^6)$, of which there are $\binom{8}{2}\times 2^2=112$ choices. In the
 $\ZZ+\half$ case, you can choose the first 7 places to be $\pm \half$, and the last
 coordinate is forced, so there are $2^7$ choices. Thus, you get $240$ roots.

 Let's find the orbits of the roots under the action of the Weyl group. We don't yet
 know what the Weyl group looks like, but we can find a large subgroup that is easy to
 work with. Let's use the Weyl group of $D_8$, which consists of the following: we can
 apply all permutations of the coordinates, or we can change the sign of an even
 number of coordinates (e.g., reflection in $(1 , {-1} , 0^6)$ swaps the first two
 coordinates, and reflection in $(1 , \, {-1} , \, 0^6)$ followed by reflection in $(1
 ,  1, 0^6)$ changes the sign of the first two coordinates.)

 Notice that under the Weyl group of $D_8$, the roots form two orbits: the set which
 is all permutations of $(\pm 1^2 , 0^6)$, and the set $(\pm \half^8)$. Do these
 become the same orbit under the Weyl group of $E_8$? Yes; to show this, we just need
 one element of the Weyl group of $E_8$ taking some element of the first orbit to the
 second orbit. Take reflection in $(\half^8)^\perp$ and apply it to $(1^2 , 0^6)$: you
 get $(\half^2 , -\half^6)$, which is in the second orbit. So there is just one orbit
 of roots under the Weyl group.

 What do orbits of $W(E_8)$ on other vectors look like? We're interested in this
 because we might want to do representation theory. The character of a representation
 is a map from weights to integers, which is $W(E_8)$-invariant. Let's look at vectors
 of norm $4$ for example. So $\sum x_i^2=4$, $\sum x_i$ even, and $x_i\in \ZZ$ or
 $x_i\in \ZZ+\half$. There are $8\times 2$ possibilities which are permutations of
 $(\pm 2 ,  0^7)$. There are $\binom 84 \times 2^4$ permutations of $(\pm 1^4 ,
 0^4)$, and there are $8\times 2^7$ permutations of  $(\pm \frac{3}{2}, \pm \half^7)$.
 So there are a total of $240\times 9$ of these vectors. There are 3 orbits under
 $W(D_8)$, and as before, they are all one orbit under the action of $W(E_8)$. Just
 reflect $(2 , \, 0^7)$ and $(1^3 , -1, 0^4)$ through $(\half^8)$.

 \begin{exercise}
   Show that the number of norm 6 vectors is $240\times 28$, and they form one orbit
 \end{exercise}
(If you've seen a course on modular forms, you'll know that the number of vectors of
norm
 $2n$ is given by $240\times \sum_{d|n} d^3$. If you let call these $c_n$, then
 $\sum c_n q^n$ is a modular form of level 1 ($E_8$ even, unimodular), weight 4 ($\dim
 E_8/2$).)

 For norm $8$ there are two orbits, because you have vectors that are twice a norm 2
 vector, and vectors that aren't.  As the norm gets bigger, you'll get a large number
 of orbits.

 What is the order of the Weyl group of $E_8$? We'll do this by 4 different methods,
 which illustrate the different techniques for this kind of thing:
 \begin{itemize}
 \item[(1)] This is a good one as a mnemonic. The order of $E_8$ is given by
 \begin{align*}
   |W(E_8)| &= 8! \times \prod\left(
        \genfrac{}{}{0em}{}{\text{numbers on the}}{\text{affine $E_8$
        diagram\footnotemark}}
    \right) \times \frac{\text{Weight lattice of $E_8$}}{\text{Root lattice of $E_8$}}\\
    &= 8! \times
        \biggr( \begin{xy}
       <1em,0em>:
       (0,0) *+!U{1} *\cir<2pt>{};
       (1,0) *+!U{2} *\cir<2pt>{} **@{-};
       p+(1,0) *+!U{3} *\cir<2pt>{} **@{-};
       p+(1,0) *+!U{4} *\cir<2pt>{} **@{-};
       p+(1,0) *+!U{5} *\cir<2pt>{} **@{-};
       p+(1,0) *+!U{6} *\cir<2pt>{} **@{-};
       p+(0,1) *+!R{3} *\cir<2pt>{} **@{-},
       p+(1,0) *+!U{4} *\cir<2pt>{} **@{-};
       p+(1,0) *+!U{2} *\cir<2pt>{} **@{-};
       \end{xy}\biggr)
       \times 1\\
    &= 2^{14} \times 3^5\times 5^2\times 7
 \end{align*}
 \footnotetext{These are the numbers giving highest root.}

 We can do the same thing for any other Lie algebra, for example,
 \begin{align*}
   |W(F_4)| &= 4!\times (
   \begin{xy}
   <2.25em,0em>:
   (0,0) *+!D{1} *\cir<2pt>{};
   (1,0) *+!D{2} *\cir<2pt>{} **@{-};
   p+(1,0) *+!D{3} *\cir<2pt>{} **@{-};
   p+(1,0)="x" *+!D{4} *\cir<2pt>{} **@{=} ?*@{>};
   "x" *{\hspace{4pt}};p+(1,0) *+!D{2} *\cir<2pt>{} **@{-};
   \end{xy}
   ) \times 1\\
   &=2^7 \times 3^2
 \end{align*}

 \item[(2)] The order of a reflection group is equal to the products of degrees of the
 fundamental invariants. For $E_8$, the fundamental invariants are of degrees
 2,8,12,14,18,20,24,30 (primes $+1$).

 \item[(3)] This one is actually an honest method (without quoting weird facts). The
 only fact we will use is the following: suppose $G$ acts transitively on a set $X$
 with $H=$ the group fixing some point; then $|G|=|H|\cdot |X|$.

 This is a general purpose method for working out the orders of groups. First, we need
 a set acted on by the Weyl group of $E_8$. Let's take the root vectors (vectors of
 norm 2). This set has 240 elements, and the Weyl group of $E_8$ acts transitively on it.
 So $|W(E_8)|=240\times |$subgroup fixing $(1 , -1 , 0^6)|$. But what is the order
 of this subgroup (call it $G_1$)? Let's find a set acted on by this group. It acts on
 the  set of norm 2 vectors, but the action is NOT transitive. What are the orbits?
 $G_1$ fixes $s=(1 , -1 , 0^6)$. For other roots $r$, $G_1$ obviously fixes
 $(r,s)$. So how many roots are there with a given inner product with $s$?
 \[\begin{array}{c|c|c}
   (s,r) & \text{number} & \text{choices}\\ \hline
   2 & 1 & s\\
   1 & 56 & (1 ,  0 , \pm 1^6), (0,-1,\pm 1^6), (\half,-\half, \half^6)\\
   0 & 126 & \\
   -1 & 56 & \\
   -2 & 1 & -s\\
 \end{array}\]
 So there are at least 5 orbits under $G_1$. In fact, each of these sets is a single
 orbit under $G_1$. How can we see this? Find a large subgroup of $G_1$. Take
 $W(D_6)$, which is all permutations of the last 6 coordinates and all even sign
 changes of the last 6 coordinates. It is generated by reflections associated to the roots
 orthogonal to $e_1$ and $e_2$ (those that start with two 0s). The three cases with
 inner product 1 are three orbits under $W(D_6)$. To see that there is a single orbit
 under $G_1$, we just need some reflections that mess up these orbits. If you take a vector
 $(\half,\half,\pm \half^6)$ and reflect norm $2$ vectors through it, you will get
 exactly $5$ orbits. So $G_1$ acts transitively on these orbits.

 We'll use the orbit of vectors $r$ with $(r,s)=-1$. Let $G_2$ be the vectors fixing
 $s$ and $r$:
 \begin{xy}
   (0,0) *+!D{s} *\cir<2pt>{};
   (1,0) *+!D{r} *\cir<2pt>{} **@{-};
 \end{xy}
 We have that $|G_1| = |G_2|\cdot 56$.

 Keep going ... it gets tedious, but here are the answers up to the last step:

 Our plan is to chose vectors acted on by $G_i$, fixed by $G_{i+1}$ which give us the
Dynkin diagram of $E_8$.  So the next step is to try to find vectors $t$ that give us
the picture
 \begin{xy}
   (0,0) *+!D{s} *\cir<2pt>{};
   (1,0) *+!D{r} *\cir<2pt>{} **@{-};
   (2,0) *+!D{t} *\cir<2pt>{} **@{.};
 \end{xy},
 i.e, they have inner product $-1$ with $r$ and $0$ with $s$. The possibilities for
 $t$ are $(-1, -1,0,0^5)$ (one of these), $(0,0,1, \pm 1, 0^4)$ and permutations of
 its last five coordinates (10 of these), and $(-\half,-\half,\half,\pm \half^5)$
 (there are 16 of these), so we get 27 total. Then we could check that they form one
 orbit, which is boring.

 Next find vectors which go next to $t$ in our picture:\\
  $\begin{xy}
   (0,0) *+!D{s} *\cir<2pt>{};
   (1,0) *+!D{r} *\cir<2pt>{} **@{-};
   p+(1,0) *+!D{t} *\cir<2pt>{} **@{-};
   p+(1,0) *\cir<2pt>{} **@{.};
 \end{xy}\ $,
 i.e., whose inner product is $-1$ with $t$ and zero with $r,s$. The possibilities are
 permutations of the last four coords of $(0,0,0,1,\pm 1, 0^3)$ (8 of these) and
 $(-\half,-\half,-\half,\half,\pm \half^4)$ (8 of these), so there are 16 total. Again
 check transitivity.

 Find a fifth vector; the possibilities are $(0^4,1,\pm 1, 0^2)$ and perms of the last
 three coords (6 of these), and $(-\half^4,\half, \pm \half^3)$ (4 of these) for a
 total of 10.

 For the sixth vector, we can have $(0^5, 1,\pm 1, 0)$ or $(0^5, 1, 0, \pm 1)$ (4
possibilites) or $(-\half^5,\half,\pm \half^2)$ (2 possibilities), so we get 6 total.

 NEXT CASE IS TRICKY: finding the seventh one, the possibilities are $(0^6,1,\pm 1)$
 (2 of these) and $((-\half)^6,\half,\half)$ (just 1). The proof of transitivity fails
 at this point. The group we're using by now doesn't even act transitively on the pair
 (you can't get between them by changing an even number of signs). What elements of
 $W(E_8)$ fix all of these first 6 points $\begin{xy}
   (0,0) *+!D{s} *\cir<2pt>{};
   (1,0) *+!D{r} *\cir<2pt>{} **@{-};
   p+(1,0) *+!D{t} *\cir<2pt>{} **@{-};
   p+(1,0) *\cir<2pt>{} **@{-};
   p+(1,0) *\cir<2pt>{} **@{-};
   p+(1,0) *\cir<2pt>{} **@{-};
 \end{xy}\ $
 ? We want to find roots perpendicular to all of these vectors, and the only
 possibility is $((\half)^8)$. How does reflection in this root act on the three vectors
 above? $(0^6, 1^2)\mapsto ((-\half)^6,\half^2)$ and $(0^6,1,-1)$ maps to
 itself. Is this last vector in the same orbit? In fact they are in different orbits.
 To see this, look for vectors
 \[\begin{xy}
   (0,0) *+!D{s} *\cir<2pt>{};
   (1,0) *+!D{r} *\cir<2pt>{} **@{-};
   p+(1,0) *+!D{t} *\cir<2pt>{} **@{-};
   p+(1,0) *\cir<2pt>{} **@{-};
   p+(1,0) *\cir<2pt>{} **@{-};
    p+(0,1) *+!D{?} *\cir<2pt>{} **@{.},
   p+(1,0) *\cir<2pt>{} **@{-};
   p+(1,0) *+!L{(0^6,1,\pm 1)} *\cir<2pt>{} **@{-};
 \end{xy}\]

 completing the $E_8$ diagram. In the $(0^6, 1,1)$ case, you can take the vector
 $((-\half)^5,\half,\half,-\half)$. But in the other case, you can show that there are no
 possibilities. So these really are different orbits.

 Use the orbit with 2 elements, and you get
 \[
    |W(E_8)| = 240\times \underbrace{56\times \overbrace{27 \times 16\times 10 \times 6\times 2\times
    1}^{\text{order of $W(E_6)$}}}_{\text{order of $W(E_7)$}}
 \]
 because the group fixing all 8 vectors must be trivial.
 You also get that
 \[
    |W(\text{``}E_5\text{''})| = 16\times \underbrace{10 \times \overbrace{6\times
            2\times 1}^{|W(A_2\times A_1)|}}_{|W(A_4)|}
 \]
 where ``$E_5$'' is the algebra with diagram
 $\begin{xy}<1.75em,0em>:
   (0,0) *\cir<2pt>{};
   (1,0) *\cir<2pt>{} **@{-};
    p+(0,1) *\cir<2pt>{} **@{-},
   p+(1,0) *\cir<2pt>{} **@{-};
   p+(1,0) *\cir<2pt>{} **@{-};
 \end{xy}$ (that is, $D_5$). Similarly, $E_4$ is $A_4$ and $E_3$ is $A_2\times A_1$.

 We got some other information. We found that the Weyl group of $E_8$ acts
 transitively on all the configurations

 \[\begin{array}{l}
    \begin{xy}
      (0,0) *\cir<2pt>{};
    \end{xy} \\
    \begin{xy}
      (0,0) *\cir<2pt>{}; p+(1,0) *\cir<2pt>{} **@{-};
    \end{xy} \\
    \begin{xy}
      (0,0) *\cir<2pt>{}; p+(1,0) *\cir<2pt>{} **@{-};
      p+(1,0) *\cir<2pt>{} **@{-};
    \end{xy} \\
    \begin{xy}
      (0,0) *\cir<2pt>{}; p+(1,0) *\cir<2pt>{} **@{-};
      p+(1,0) *\cir<2pt>{} **@{-}; p+(1,0) *\cir<2pt>{} **@{-};
    \end{xy} \\
    \begin{xy}
      (0,0) *\cir<2pt>{}; p+(1,0) *\cir<2pt>{} **@{-};
      p+(1,0) *\cir<2pt>{} **@{-}; p+(1,0) *\cir<2pt>{} **@{-};
      p+(1,0) *\cir<2pt>{} **@{-};
    \end{xy} \\
    \begin{xy}
      (0,0) *\cir<2pt>{}; p+(1,0) *\cir<2pt>{} **@{-};
      p+(1,0) *\cir<2pt>{} **@{-}; p+(1,0) *\cir<2pt>{} **@{-};
      p+(1,0) *\cir<2pt>{} **@{-}; p+(1,0) *\cir<2pt>{} **@{-};
    \end{xy} \\
 \end{array}\]
 but not on
 \[
     \begin{xy}
      (0,0) *\cir<2pt>{}; p+(1,0) *\cir<2pt>{} **@{-};
      p+(1,0) *\cir<2pt>{} **@{-}; p+(1,0) *\cir<2pt>{} **@{-};
      p+(1,0) *\cir<2pt>{} **@{-}; p+(1,0) *\cir<2pt>{} **@{-};
      p+(1,0) *\cir<2pt>{} **@{-};
    \end{xy}
 \]
 \item[(4)] We'll slip this in to next lecture
 \end{itemize}
Also, next time we'll construct the Lie algebra $E_8$.
