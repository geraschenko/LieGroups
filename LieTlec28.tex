 \stepcounter{lecture}
 \setcounter{lecture}{28}
 \sektion{Lecture 28}

 Last time, we constructed a Lie algebra of type $E_8$, which was $L\oplus \bigoplus
 \hat e^\alpha$, where $L$ is the root lattice and $\alpha^2=2$. This gives a double
 cover of the root lattice:
 \[
    1\to \pm 1\to \hat e^L\to e^L\to 1.
 \]
 We had a lift for $\w(\alpha)=-\alpha$, given by $\w(\hat
 e^\alpha)=(-1)^{(\alpha^2/2)}(\hat e^\alpha)^{-1}$. So $\w$ becomes an automorphism of
 order 2 on the Lie algebra. $e^\alpha\mapsto (-1)^{(\alpha,\beta)} e^\alpha$
 is also an automorphism of the Lie algebra.

 Suppose $\sigma$ is an automorphism of order 2 of the real Lie algebra $L=L^+ +L^-$
 (eigenspaces of $\sigma$). We saw that you can construct another real form given by $L^+
 +iL^-$. Thus, we have a map from conjugacy classes of automorphisms with $\sigma^2=1$
 to real forms of $L$. This is not in general in isomorphism.

 Today we'll construct some more real forms of $E_8$. $E_8$ has an invariant symmetric
 bilinear form $(e^\alpha,(e^\alpha)^{-1})=1$, $(\alpha,\beta)=(\beta,\alpha)$. The
 form is unique up to multiplication by a constant since $E_8$ is an irreducible
 representation of $E_8$. So the \emph{absolute value of the signature} is an
 invariant of the Lie algebra.

 For the split form of $E_8$, what is the signature of the invariant bilinear form
 (the split form is the one we just constructed)? On the Cartan subalgebra $L$, $(\
 ,\,)$ is positive definite, so we get $+8$ contribution to the signature. On
 $\{e^\alpha,(e^\alpha)^{-1}\}$, the form is $\matrix 0110$, so it has signature
 $0\cdot 120$. Thus, the signature is 8. So if we find any real form with a different
 signature, we'll have found a new Lie algebra.

 Let's first try involutions $e^\alpha\mapsto (-1)^{(\alpha,\beta)}e^\alpha$. But this
 doesn't change the signature. $L$ is still positive definite, and you still have
 $\matrix 0110$ or $\matrix 0{-1}{-1}0$ on the other parts. These Lie algebras
 actually turn out to be isomorphic to what we started with (though we haven't shown
 that they are isomorphic).

 Now try $\w:e^\alpha\mapsto (-1)^{\alpha^2/2}(e^\alpha)^{-1}$, $\alpha\mapsto
 -\alpha$. What is the signature of the form? Let's write down the $+$ and $-$
 eigenspaces of $\w$. The $+$ eigenspace will be spanned by $e^\alpha - e^{-\alpha}$,
 and these vectors have norm $-2$ and are orthogonal. The $-$ eigenspace will be
 spanned by $e^\alpha + e^{-\alpha}$ and $L$, which have norm 2 and are orthogonal,
 and $L$ is positive definite. What is the Lie algebra corresponding to the involution
 $\w$? It will be spanned by $e^\alpha - e^{-\alpha}$ where $\alpha^2=2$ (norm $-2$), and
 $i(e^\alpha + e^{-\alpha})$ (norm $-2$), and $iL$ (which is now negative definite).
 So the bilinear form is \emph{negative definite}, with signature $-248 (\neq \pm 8)$.

 With some more work, you can actually show that this is the Lie algebra of the
 \emph{compact} form of $E_8$. This is because the automorphism group of $E_8$ preserves the
 invariant bilinear form, so it is contained in $O_{0,248}(\RR)$, which is compact.

 Now let's look at involutions of the form $e^\alpha\mapsto
 (-1)^{(\alpha,\beta)}\w(e^\alpha)$. Notice that $\w$ commutes with $e^\alpha\mapsto
 (-1)^{(\alpha,\beta)}e^\alpha$. The $\beta$'s in $(\alpha,\beta)$ correspond to
 $L/2L$ modulo the action of the Weyl group $W(E_8)$. Remember this has three orbits,
 with 1 norm 0 vector, 120 norm 2 vectors, and 135 norm 4 vectors. The norm 0 vector
 gives us the compact form. Let's look at the other cases and see what we get.

 Suppose $V$ has a negative definite symmetric inner product $(\ ,\,)$, and suppose
 $\sigma$ is an involution of $V=V_+\oplus V_-$ (eigenspaces of $\sigma$). What is the
 signature of the invariant inner product on $V_+\oplus iV_-$? On $V_+$, it is
 negative definite, and on $iV_-$ it is positive definite. Thus, the signature is
 $\dim V_- - \dim V_+= -\mathrm{tr}(\sigma)$. So we want to work out the traces of these
 involutions.

 Given some $\beta \in L/2L$, what is $\mathrm{tr}(e^\alpha\mapsto
 (-1)^{(\alpha,\beta)}e^\alpha)$? If $\beta =0$, the traces is obviously 248 because
 we just have the identity map. If $\beta^2=2$, we need to figure how many roots have
 a given inner product with $\beta$. Recall that this was determined before:
 \begin{center}
 \begin{tabular}{|c|c|c|}
 \hline
 $(\alpha,\beta)$ & \# of roots $\alpha$ with given inner product & eigenvalue\\
 \hline
 2                & 1                                        & 1  \\
 1                & 56                                       & -1 \\
 0                & 126                                      & 1  \\
 -1               & 56                                       & -1 \\
 -2               & 1                                        & 1  \\
 \hline
 \end{tabular}
 \end{center}
 Thus, the trace is $1-56+126-56+1+8=24$ (the $8$
 is from the Cartan subalgebra). So the signature of the corresponding form on the Lie
 algebra is $-24$. We've found a third Lie algebra.

 If we also look at the case when $\beta^2=4$, what happens? How many $\alpha$ with
 $\alpha^2=2$ and with given $(\alpha,\beta)$ are there?  In this case,
 we have:
 \begin{center}
 \begin{tabular}{|c|c|c|}
 \hline
 $(\alpha,\beta)$ & \# of roots $\alpha$ with given inner product & eigenvalue\\
 \hline
 2                & 14                                       & 1  \\
 1                & 64                                       & -1 \\
 0                & 84                                       & 1  \\
 -1               & 64                                       & -1 \\
 -2               & 14                                       & 1  \\
 \hline
 \end{tabular}
 \end{center}
 The trace will be $14-64+84-64+14+8=-8$. This is just the split form again.

 Summary: We've found $3$ forms of $E_8$, corresponding to 3 classes in $L/2L$, with
 signatures 8, $-24$, $-248$, corresponding to involutions $e^\alpha\mapsto
 (-1)^{(\alpha,\beta)}e^{-\alpha}$ of the \emph{compact} form. If $L$ is the
 \emph{compact} form of a simple Lie algebra, then Cartan\index{Cartan} showed that
 the other forms correspond exactly to the conjugacy classes of involutions in the
 automorphism group of $L$ (this doesn't work if you don't start with the compact form
 --- so always start with the compact form).

 In fact, these three are the \emph{only} forms of $E_8$, but we won't prove that.

 \subsektion{Working with simple Lie groups}
 As an example of how to work with simple Lie groups, we will look at the
 general question: Given a simple Lie group, what is its homotopy type?
 Answer: $G$ has a unique conjugacy class of maximal compact subgroups $K$, and $G$ is
 homotopy equivalent to $K$.
 \begin{proof}[Proof for $GL_n(\RR)$]
   First pretend $GL_n(\RR)$ is simple, even though it isn't; whatever. There is an
   obvious compact subgroup: $O_n(\RR)$.  Suppose $K$ is \emph{any} compact subgroup of
   $GL_n(\RR)$. Choose any positive definite form $(\ ,\,)$ on $\RR^n$. This will
   probably not be invariant under $K$, but since $K$ is compact, we can average it
   over $K$ get one that is: define a new form
   $(a,b)_{\mathrm{new}} = \int_K (ka,kb)\, dk$. This gives an
   invariant positive definite bilinear form (since integral of something
   positive definite is
   positive definite). Thus, any compact subgroup preserves some positive definite
   form. But the subgroup fixing some positive definite bilinear form is conjugate to
   a subgroup of $O_n(\RR)$ (to see this, diagonalize the form). So $K$ is contained
   in a conjugate of $O_n(\RR)$.

   Next we want to show that $G=GL_n(\RR)$ is homotopy equivalent to $O_n(\RR)=K$. We
   will show that $G=KAN$, where $K$ is $O_n$, $A$ is all diagonal matrices with
   positive coefficients, and $N$ is matrices which are upper triangular with 1s on
   the diagonal. This is the \emph{Iwasawa decomposition}. In general, we get $K$ compact,
   $A$ semisimple abelian, and $N$ is unipotent. The proof of this you saw before was
   called the Grahm-Schmidt process for orthonormalizing a basis. Suppose $v_1,\dots,
   v_n$ is any basis for $\RR^n$.
   \begin{enumerate}
     \item Make it orthogonal by subtracting some stuff, you'll get $v_1$,
     $v_2-\ast v_1$, $v_3 - \ast v_2 - \ast v_1$, $\dots$.
     \item Normalize by multiplying each basis vector so that it has norm 1. Now we
     have an orthonormal basis.
   \end{enumerate}
   This is just another way to say that $GL_n$ can be written as $KAN$. Making things
   orthogonal is just multiplying by something in $N$, and normalizing is just
   multiplication by some diagonal matrix with positive entries. An orthonormal basis
   is an element of $O_n$. Tada! This decomposition is just a topological one, not
   a decomposition as groups. Uniqueness is easy to check.

   Now we can get at the homotopy type of $GL_n$. $N\cong \RR^{n(n-1)/2}$, and $A\cong
   (\RR^+)^n$, which are contractible. Thus, $GL_n(\RR)$ has the same homotopy type as
   $O_n(\RR)$, its maximal compact subgroup.
 \end{proof}
 If you wanted to know $\pi_1(GL_3(\RR))$, you could calculate $\pi_1(O_3(\RR))\cong
 \ZZ/2\ZZ$, so $GL_3(\RR)$ has a double cover. Nobody has shown you this double cover
 because it is \emph{not algebraic}.

 \begin{example}
  Let's go back to various forms of $E_8$ and figure out (guess) the fundamental
  groups. We need to know the maximal compact subgroups.

  \begin{enumerate}
  \item One of them is easy: the
  compact form is its own maximal compact subgroup. What is the fundamental group?
  Remember or quote the fact that for compact simple groups, $\pi_1\cong \frac{\text{weight
  lattice}}{\text{root lattice}}$, which is 1. So this form is simply connected.

  \item $\beta^2=2$ case (signature $-24$).
  Recall that there were 1, 56, 126, 56, and 1 roots $\alpha$
  with $(\alpha,\beta)=2,1,0,-1$, and -2 respectively,
  and there are another $8$ dimensions for the Cartan subalgebra.
   On the $1,126,1,8$ parts, the form is negative definite. The sum of these root
   spaces gives a Lie algebra of type $E_7 A_1$ with a negative definite bilinear form
   (the $126$ gives you the roots of an $E_7$, and the $1$s are the roots of an
   $A_1$). So it a reasonable guess that the maximal compact subgroup has something to
   do with $E_7A_1$. $E_7$ and $A_1$ are not simply connected: the compact form of
   $E_7$ has $\pi_1$ = $\ZZ/2$ and the compact form of $A_1$ also has $\pi_1 = \ZZ/2$.
   So the universal cover of $E_7A_1$ has center $(\ZZ/2)^2$. Which part of this acts
   trivially on $E_8$? We look at the $E_8$ Lie algebra as a representation of
   $E_7\times A_1$. You can read off how it splits form the picture above: $E_8\cong
   E_7\oplus A_1 \oplus 56 \otimes 2$, where $56$ and $2$ are irreducible, and the
   centers of $E_7$ and $A_1$ both act as $-1$ on them. So the maximal compact
   subgroup of this form of $E_8$ is the simply connected compact form of $E_7\times
   A_1/(-1,-1)$. This means that $\pi_1(E_8)$ is the same as $\pi_1$ of the compact
   subgroup, which is $(\ZZ/2)^2/(-1,-1)\cong \ZZ/2$. So this simple group has a
   nontrivial double cover (which is non-algebraic).

   \item For the other (split) form of $E_8$ with signature 8, the maximal compact
   subgroup is $\spin_{16}(\RR)/(\ZZ/2)$, and $\pi_1(E_8)$ is $\ZZ/2$.
  \end{enumerate}
 You can compute any other homotopy invariants with this method.
 \end{example}

 Let's look at the $56$ dimensional representation of $E_7$ in more detail. We had the
 picture
 \[\begin{tabular}{c|c}
   $(\alpha,\beta)$ & \# of $\alpha$'s\\
   \hline
   2 & 1\\
   1 & 56\\
   0 & 126\\
   -1 & 56\\
   -2 & 1\\
 \end{tabular}\]
 The Lie algebra $E_7$ fixes these 5 spaces of $E_8$ of dimensions $1,56,126+8,56,1$.
 From this we can get some representations of $E_7$. The $126+8$ splits as
 $1+(126+7)$. But we also get a 56 dimensional representation of $E_7$. Let's show
 that this is actually an irreducible representation. Recall that in calculating
 $W(E_8)$, we showed that $W(E_7)$ acts transitively on this set of $56$ roots of
 $E_8$, which can be considered as weights of $E_7$.

 An irreducible representation is called \emph{minuscule} if the Weyl group acts
 transitively on the weights. This kind of representation is particularly easy to work
 with. It is really easy to work out the character for example: just translate the
 1 at the highest weight around, so every weight has multiplicity 1.

 So the 56 dimensional representation of $E_7$ must actually be the irreducible
 representation with whatever highest weight corresponds to one of the vectors.

 \subsektion{Every possible simple Lie group} We will construct them as follows:
  Take an involution $\sigma$ of the compact form $L=L^+ + L^-$ of the Lie
  algebra, and form $L^+ + iL^-$. The way we constructed these was to first construct
  $A_n$, $D_n$, $E_6$, and $E_7$ as for $E_8$. Then construct the involution
  $\w:e^\alpha\mapsto -e^{-\alpha}$. We get $B_n$, $C_n$, $F_4$, and $G_2$ as fixed
  points of the involution $\w$.

 Kac classified all automorphisms of finite order of any compact simple Lie group. The
 method we'll use to classify involutions is extracted from his method. We can
 construct lots of involutions as follows:
 \begin{enumerate}
 \item Take any Dynkin diagram, say $E_8$, and select some of its vertices,
 corresponding to simple roots. Get an involution by taking $e^\alpha\mapsto \pm
 e^\alpha$ where the sign depends on whether $\alpha$ is one of the simple roots we've
 selected. However, this is not a great method. For one thing, you get a lot of
 repeats (recall that there are only 3, and we've found $2^8$ this way).
 \[
 \begin{xy}
   (0,0) *\cir<2pt>{};
   p+(1,0)="a" *\cir<2pt>{} **@{-};
   p+(1,0) *\cir<2pt>{} **@{-};
   p+(1,0)="b" *\cir<2pt>{} **@{-};
   p+(1,0) *\cir<2pt>{} **@{-};
   p+(0,1)="c" *\cir<2pt>{} **@{-},
   p+(1,0) *\cir<2pt>{} **@{-};
   p+(1,0) *\cir<2pt>{} **@{-};
   "a" *\cir<5pt>{} *+++!D{1};
   "b" *\cir<5pt>{} *+++!D{1};
   "c" *\cir<5pt>{} *+++!L{1};
 \end{xy}
 \]
 \item Take any diagram automorphism of order 2, such as
 \[\begin{xy}
   (0,0)="1" *\cir<2pt>{};
   (1,0)="2"  *\cir<2pt>{} **@{-};
   p+(1,0) *\cir<2pt>{} **@{-};
   p+(0,-1) *\cir<2pt>{} **@{-},
   p+(1,0)="22" *\cir<2pt>{} **@{-};
   p+(1,0)="11" *\cir<2pt>{} **@{-};
   "1" *+{\ };"11" *+{\ } **\crv{(2,1.5)} ?<*@{<} ?>*@{>},
   "2" *+{\ };"22" *+{\ } **\crv{(2,1)} ?<*@{<} ?>*@{>},
 \end{xy}\]
 This gives you more involutions.
 \end{enumerate}

 Next time, we'll see how to cut down this set of involutions.
