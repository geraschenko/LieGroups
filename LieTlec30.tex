 \stepcounter{lecture}
 \setcounter{lecture}{30}
 \sektion{Lecture 30 - Irreducible unitary representations of
                       \texorpdfstring{$SL_2(\RR)$}{SL(2,R)}}

 $SL_2(\RR)$ is non-compact. For compact Lie groups, all unitary representations are
 finite dimensional, and are all known well. For non-compact groups, the theory is
 much more complicated. Before doing the infinite dimensional representations, we'll
 review finite dimensional (usually not unitary) representations of $SL_2(\RR)$.

 \subsektion{Finite dimensional representations} Finite dimensional complex
 representations of the following are much the same: $SL_2(\RR)$, $\sl_2\RR$, $\sl_2
 \CC$ [branch $SL_2(\CC)$ as a complex Lie group] (as a complex Lie algebra),
 $\mathfrak{su}_2\RR$ (as a real Lie algebra), and $SU_2$ (as a real Lie group). This
 is because finite dimensional representations of a simply connected Lie group are in
 bijection with representations of the Lie algebra. Complex representations of a REAL
 Lie algebra $L$ correspond to complex representations of its complexification
 $L\otimes \CC$ considered as a COMPLEX Lie algebra.

 Note: Representations of a COMPLEX Lie algebra $L\otimes \CC$ are not the same
 as representations of the REAL Lie algebra $L\otimes \CC\cong L+ L$. The
 representations of the real Lie algebra correspond roughly to (reps of
 $L)\otimes$(reps of $L$).

 Strictly speaking, $SL_2(\RR)$ is not simply connected, which is not important for
 finite dimensional representations.

 Recall the main results for representations of $SU_2$:
 \begin{enumerate}
   \item For each positive integer $n$, there is one irreducible representation of
   dimension $n$.

   \item The representations are completely reducible (every representation is a sum
   of irreducible ones). This is perhaps the most important fact.

   The finite dimensional representation theory of $SU_2$ is EASIER than the
   representation theory of the ABELIAN Lie group $\RR^2$, and that is because
   representations of $SU_2$ are completely reducible.

   For example, it is very difficult to classify pairs of commuting nilpotent
   matrices.
 \end{enumerate}

 Completely reducible representations:
 \begin{enumerate}
   \item Complex representations of finite groups.

   \item Representations of compact groups (Weyl character formula)

   \item More generally, unitary representations of anything (you can take orthogonal
   complements of subrepresentations)

   \item Finite dimensional representations of semisimple Lie groups.
 \end{enumerate}
 Representations which are not completely reducible:
 \begin{enumerate}
   \item Representations of a finite group $G$  over fields of characteristic
   $p|\,|G|$.

   \item Infinite dimensional representations of non-compact Lie groups (even if they
   are semisimple).
 \end{enumerate}

 We'll work with the Lie algebra $\sl_2\RR$, which has basis $H=\matrix 100{-1}$,
 $E=\matrix 0100$, and $F=\matrix 0010$. $H$ is a basis for the Cartan subalgebra
 $\matrix a00{-a}$. $E$ spans the root space of the simple root. $F$ spans the root
 space of the negative of the simple root. We find that $[H,E]=2E$, $[H,F]=-2F$ (so
 $E$ and $F$ are eigenvectors of $H$), and you can check that $[E,F]=H$.
 \[\begin{xy}
   (1,0) *++!D{0} *++!U{H} *{\bullet};
   \ar@/^2.5em/@{<->}^{\text{Weyl group of order 2}}(0,0) *++!D{-2} *{\bullet} *++!U{F};
   (2,0) *++!D{2} *{\bullet} *++!U{E};
   \ar (3,.3) *+!L{\text{weights $=$ eigenvalues under }H};(2.2,.3)
 \end{xy}\]
 The Weyl group is generated by $\w = \matrix 01{-1}0$ and $\w^2=\matrix {-1}00{-1}$.

 Let $V$ be a finite dimensional irreducible complex representation of $\sl_2\RR$.
 First decompose $V$ into eigenspaces of the Cartan subalgebra (weight spaces) (i.e.\
 eigenspaces of the element $H$). Note that eigenspaces of $H$ exist because $V$ is
 FINITE-DIMENSIONAL (remember this is a complex representation). Look at the LARGEST
 eigenvalue of $H$ (exists since $V$ is finite dimensional), with eigenvector $v$. We
 have that $Hv=nv$ for some $n$. Compute
 \begin{align*}
   H(Ev) &= [H,E]v + E(Hv)\\
        &= 2Ev + Env = (n+2)Ev
 \end{align*}
 So $Ev=0$ (lest it be an eigenvector of $H$ with higher eigenvalue). $[E,-]$
 increases weights by 2 and $[F,-]$ decreases weights by 2, and $[H,-]$ fixes weights.

 We have that $E$ kills $v$, and $H$ multiplies it by $n$. What does $F$ do to $v$?
 \[\xymatrix{
  & nv & (n-2)Fv & (n-4)F^2v & (n-6)F^3v & \dots\\
  0 & v\ar@/^/[l]^E \ar[u]_H \ar@/^/[r]^F &
  Fv \ar@/^/[l]^{\substack{E\\ \times n}} \ar[u]_H \ar@/^/[r]^F &
  F^2v \ar@/^/[l]^{\substack{E \\ \times (2n-2)}} \ar[u]_H \ar@/^/[r]^F &
  F^3v \ar@/^/[l]^{\substack{E\\ \times (3n-6)}} \ar[u]_H & \dots
 }\]
 What is $E(Fv)$? Well,
 \begin{align*}
   EFv &= FEv + [E,F]v\\
        &= 0 + Hv = nv
 \end{align*}
 In general, we have
 \begin{align*}
   H(F^i v) &= (n-2i)F^i v\\
   E(F^i v) &= (ni-i(i-1))F^{i-1}v\\
   F(F^i v) &= F^{i+1} v
 \end{align*}
 So the vectors $F^i v$ span $V$ because they span an invariant subspace. This gives
 us an infinite number of vectors in distinct eigenspaces of $H$, and $V$ is
 finite dimensional. Thus, $F^k v=0$ for some $k$. Suppose $k$ is the SMALLEST integer
 such that $F^kv=0$. Then
 \[
    0 = E(F^k v) = (nk-k(k-1))\underbrace{EF^{k-1}v}_{\neq 0}
 \]
 So $nk-k(k-1)=0$, and $k\neq 0$, so $n-(k-1)=0$, so \fbox{$k=n+1$}\,. So $V$ has a
 basis consisting of $v,Fv,\dots, F^n v$. The formulas become a little better if we
 use the basis $w_n=v,w_{n-2}=Fv, w_{n-4}=\frac{F^2v}{2!}, \frac{F^3 v}{3!}, \dots,
 \frac{F^nv}{n!}$.
 \[\xymatrix{
  w_{-6} \ar@/^1.25em/[r]^1 \ar@/_1.25em/@{<-}[r]_6 &
  w_{-4} \ar@/^1.25em/[r]^2 \ar@/_1.25em/@{<-}[r]_5 &
  w_{-2} \ar@/^1.25em/[r]^3 \ar@/_1.25em/@{<-}[r]_4 &
  w_{0}  \ar@/^1.25em/[r]^4 \ar@/_1.25em/@{<-}[r]_3 &
  w_{2}  \ar@/^1.25em/[r]^5 \ar@/_1.25em/@{<-}[r]_2 &
  w_{4}  \ar@/^1.25em/[r]^6 \ar@/_1.25em/@{<-}[r]_1 &
  w_{6} & \hspace{-3.75em} \shortstack{$E$\\ \vspace{1.75em} \\$F$}
 }\]
 This says that $E(w_2)=5w_4$ for example. So we've found a complete description of
 all finite dimensional irreducible complex representations of $\sl_2 \RR$. This is as
 explicit as you could possibly want.

 These representations all lift to the group $SL_2(\RR)$: $SL_2(\RR)$ acts on
 homogeneous polynomials of degree $n$ by $\matrix abcd f(x,y)=f(ax+by,cx+dy)$. This
 is an $n+1$ dimensional space, and you can check that the eigenspaces are $x^i
 y^{n-i}$.

 We have implicitly constructed VERMA MODULES. We have a basis $w_n,w_{n-2},\dots,
 w_{n-2i},\dots$ with relations $H(w_{n-2i})=(n-2i)w_{n-2i}$,
 $Ew_{n-2i} = (n-i+1)w_{n-2i+2}$, and $Fw_{n-2i} = (i+1)w_{n-2i-2}$. These are
 obtained by copying the formulas from the finite dimensional case, but allow it to be
 infinite dimensional. This is the universal representation generated by the highest
 weight vector $w_n$ with eigenvalue $n$ under $H$ (highest weight just means
 $E(w_n)=0$).

 Let's look at some things that go wrong in infinite dimensions.
 \begin{warning}
   Representations corresponding to the Verma modules do NOT lift to representations of
   $SL_2(\RR)$, or even to its universal cover. The reason: look at the Weyl group
   (generated by $\matrix 01{-1}0$) of $SL_2(\RR)$ acting on $\langle H\rangle$; it
   changes $H$ to $-H$. It maps eigenspaces with eigenvalue $m$ to eigenvalue $-m$.
   But if you look at the Verma module, it has eigenspaces $n,n-2,n-4,\dots$, and this
   set is obviously not invariant under changing sign. The usual proof that
   representations of the Lie algebra lifts uses the exponential map of matrices,
   which doesn't converge in infinite dimensions.
 \end{warning}
 \begin{remark}
   The universal cover $\widetilde{SL_2(\RR)}$ of $SL_2(\RR)$, or even the double
   cover $Mp_2(\RR)$, has NO faithful finite dimensional representations.
   \begin{proof}
     Any finite dimensional representation comes from a
     finite dimensional representation of the Lie algebra $\sl_2\RR$. All such
     finite dimensional representations factor through $SL_2(\RR)$.
   \end{proof}
 \end{remark}
 All finite dimensional representations of $SL_2(\RR)$ are completely reducible. Weyl
 did this by Weyl's unitarian trick:

 Notice that finite dimensional representations of $SL_2(\RR)$ are isomorphic (sort
 of) to finite dimensional representations of the COMPACT group $SU_2$ (because they
 have the same complexified Lie algebras. Thus, we just have to show it for $SU_2$.
 But representations of ANY compact group are completely reducible. Reason:
 \begin{enumerate}
   \item All unitary representations are completely reducible (if $U\subseteq V$, then
   $V=U\oplus U^\perp$).

   \item Any representation $V$ of a COMPACT group $G$ can be made unitary: take any
   unitary form on $V$ (not necessarily invariant under $G$), and average it over $G$
   to get an invariant unitary form. We can average because $G$ is compact, so we can
   integrate any continuous function over $G$. This form is positive definite since it
   is the average of positive definite forms (if you try this with non-(positive
   definite) forms, you might get zero as a result).
 \end{enumerate}

 \subsektion{The Casimir operator}\index{Casimir operator|idxbf}Set $\W = 2EF + 2FE +
 H^2\in U(\sl_2\RR)$. The main point is that $\W$ commutes with $\sl_2\RR$. You can
 check this by brute force:
 \begin{align*}
   [H,\W] &= 2\underbrace{([H,E]F+E[H,F])}_0+\cdots\\
   [E,\W] &= 2[E,E]F + 2E[F,E] + 2[E,F]E \\
          & \qquad + 2F[E,E] + [E,H]H + H[E,H] = 0\\
   [F,\W] &= \text{Similar}
 \end{align*}
 Thus, $\W$ is in the center of $U(\sl_2\RR)$. In fact, it generates the center. This
 doesn't really explain where $\W$ comes from.
 \begin{remark}
   Why does $\W$ exist? The answer is that it comes from a symmetric invariant
   bilinear form on the Lie algebra $\sl_2\RR$ given by $(E,F)=1$,
   $(E,E)=(F,F)=(F,H)=(E,H)=0$, $(H,H)=2$. This bilinear form is an invariant map
   $L\otimes L\to \CC$, where $L=\sl_2\RR$, which by duality gives an invariant
   element in $L\otimes L$, which turns out to be $2E\otimes F + 2F\otimes E +
   H\otimes H$. The invariance of this element corresponds to $\W$ being in the center
   of $U(\sl_2\RR)$.
 \end{remark}
 Since $\W$ is in the center of $U(\sl_2\RR)$, it acts on each irreducible representation
 as multiplication by a constant. We can work out what this constant is for the
 finite dimensional representations.
 Apply $\W$ to the highest vector $w_n$:
 \begin{align*}
   (2EF + 2FE + HH)w_n &= (2n+0+n^2)w_n\\
            &= (2n+n^2)w_n
 \end{align*}
 So $\W$ has eigenvalue $2n+n^2$ on the irreducible representation of dimension $n+1$.
 Thus, $\W$ has DISTINCT eigenvalues on different irreducible representations, so it
 can be used to separate different irreducible representations. The main use of $\W$
 will be in the next lecture, where we'll use it to deal with infinite dimensional
 representation.

 To finish today's lecture, let's look at an application of $\W$. We'll sketch an
 algebraic argument that the representations of $\sl_2 \RR$ are completely reducible.
 Given an exact sequence of representations
 \[
    0\to U\to V\to W\to 0
 \]
 we want to find a splitting $W\to V$, so that $V=U\oplus W$.

 \underline{Step 1}: Reduce to the case where $W=\CC$. The idea is to look at
 \[
    0\to \hom_\CC(W,U)\to \hom_\CC(W,V)\to \hom_\CC(W,W)\to 0
 \]
 and $\hom_\CC(W,W)$ has an obvious one dimensional subspace, so we can get a smaller
 exact sequence
 \[
    0\to \hom_\CC(W,U)\to \text{subspace of }\hom_\CC(W,V)\to \CC \to 0
 \]
 and if we can split this, the original sequence splits.

 \underline{Step 2}: Reduce to the case where $U$ is irreducible. This is an easy
 induction on the number of irreducible components of $U$.
 \begin{exercise}
   Do this.
 \end{exercise}

 \underline{Step 3}: This is the key step. We have
 \[
    0\to U\to V\to \CC\to 0
 \]
 with $U$ irreducible. Now apply the Casimir operator $\W$. $V$ splits as eigenvalues
 of $\W$, so is $U\oplus \CC$ UNLESS $U$ has the same eigenvalue as $\CC$ (i.e.\
 unless $U=\CC$).

 \underline{Step 4}: We have reduced to
 \[
    0\to \CC\to V\to \CC\to 0
 \]
 which splits because $\sl_2(\RR)$ is perfect\footnote{$L$ is \emph{perfect} if
 $[L,L]=L$} (no homomorphisms to the abelian algebra $\matrix 0{\ast}00$).

 Next time, in the final lecture, we'll talk about infinite dimensional unitary
 representations.
