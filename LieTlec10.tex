 \stepcounter{lecture}
 \setcounter{lecture}{10}
 \sektion{Lecture 10}

 Here is the take-home exam, it's due on Tuesday:

 \begin{itemize}
 \item[(1)] $B\subset SL_{2}(\CC)$ are upper triangular matrices, then

 \begin{itemize}
 \item Describe $X=SL_{2}(\CC)/B$

 \item $SL_{2}(\CC)$ acts on itself via left multiplication implies that it acts on
 $X$. Describe the action.
 \end{itemize}

 \item[(2)] Find $\exp\left(
 \begin{array}
 [c]{cccc}%
 0 & x_{1} &  & 0\\
 & \ddots & \ddots & \\
 &  & 0 & x_{n-1}\\
 0 &  &  & 0
 \end{array}
 \right)  $

 \item[(3)] Prove that if $V,W$ are filtered vector spaces (with increasing filtration)
 and $\phi:V\to W$ satisfies $\phi(V_{i})\subseteq W_{i}$, and
 $Gr(\phi):Gr(V)\xrightarrow{\sim} Gr(W)$ an isomorphism, then $\phi$ is a linear
 isomorphism of filtered spaces.
 \end{itemize}

 \subsektion{Lie algebra cohomology}

 Recall $C^{\udot}(\g,M)$ from the previous lecture, for $M$ a finite dimensional
 representation of $\g$ (and $\g$ finite dimensional). There is a book by D.~Fuchs,
 \textsl{Cohomology of $\infty$ dimensional Lie algebras} \cite{Fuchs}.

 We computed that $H^{0}(\g,\g)=Z(\g)\simeq\g/[\g,\g]$ and that $H^{1}(\g,\g)$ is the
 space of exterior derivations of $\g$. Say $c\in Z^{1}(\g,\g)$,\footnote{$Z^n(\g,M)$
 is the space of $n$-cocycles, i.e.\ the kernel of $d:C^n(\g,M)\to C^{n+1}(\g,M)$.} so
 $[c]\in H^{1}(\g,\g)$. Define $\tilde\g_{c}=\g\oplus\CC\partial_{c}$ with the bracket
 $[(x,t),(y,s)]=([x,y]-tc(y)+sc(x),0)$. So if $e_{1},\dots,e_{n}$ is a basis in $\g$
 with the usual relations $[e_{i},e_{j}]=c_{ij}^{k}e_{k}$, then we get one more
 generator $\partial_{c}$ such that $[\partial _{c},x]=c(x)$. Then $H^{1}(\g,\g)$ is
 the space of equivalence classes of extensions
 \[
 0\rightarrow\g\rightarrow\tilde{\g}\rightarrow\CC\rightarrow0
 \]
 up to the equivalences $f$ such that the diagram commutes:
 \[
 \xymatrix{
 0 \ar[r] &\g \ar[r]\ar[d]^\id & \tilde \g \ar[r] \ar[d]^f & \CC \ar[d]^\id \ar[r] & 0\\
 0 \ar[r] &\g \ar[r] & \tilde\g' \ar[r] & \CC \ar[r] & 0\\
 }
 \]
 This is the same as the space of exterior derivations.

 \subsektion{\texorpdfstring{$H^{2}(\g,\g)$}{H2(g,g)} and Deformations of Lie algebras}

 A \emph{deformation}\index{deformation!of a Lie algebra} of $\g$ is the vector space
 $\g[[h]]$ with a bracket $[a,b]_{h} =[a,b]+\sum_{n\geq1}h^{n}m_{n}(a,b)$ such that
 $m_{n}(a,b)=-m_{n}(b,a)$ and
 \[
 [a,[b,c]_{h}]_{h}+[b,[c,a]_{h}]_{h}+[c,[a,b]_{h}]_{h}=0.
 \]
 The $h^{N}$ order term of the Jacobi identity yields equation \ref{lec09Eq:h^N},
 which was
 \[
 [a,m_{N}(b,c)]+m_{N}(a,[b,c])+\sum_{k=1}^{N-1}m_{k}(a,m_{N-k}(b,c))+\text{cycle}=0
 \]
 where ``cycle'' is the same thing, with $a$, $b$, and $c$ permuted cyclically.
 For $\mu\in C^{2}(\g,\g)$, we compute
 \[
 d\mu(a,b,c)=-[a,\mu(b,c)]-\mu(a,[b,c])+\text{cycle}.
 \]
 Define
 \[
 \{m_{k},m_{N-k}\}(a,b,c)\overset{def}{=}m_{k}(a,m_{N-k}(b,c))+\text{cycle}%
 \]\mpar[\anton{More Gerstenhaber, inconsistent notation}]{}
 This is called the Gerstenhaber bracket ... do a Google search for it if you like ...
 it is a tiny definition from a great big theory.

 Then we can rewrite equation \ref{lec09Eq:h^N} as equation \ref{lec09Eq:2}, which was
 \[
 dm_{N} = \sum_{k=1}^{N-1} \{m_{k},m_{N-k}\}. %\tag{$\ddag$}
 \]
 In partiular, $dm_{1}=0$, so $m_{1}$ is in $Z^{2}(\g,\g)$.

 Equivalences: $[a,b]'_{h}\simeq[a,b]_{h}$ if $[a,b]'_{h}=
 \phi^{-1}([\phi(a),\phi(b)]_{h})$ for some $\phi(a) = a+\sum_{n\ge1} h^{n}
 \phi_{n}(a)$. then
 \[
  m'_{1}(a,b) = m_{1}(a,b) - \phi_{1}([a,b]) + [a,\phi_{1}(b)] + [\phi _{1}(a),b].
 \]
 which we can write as $m'_{1} = m_{1}+d\phi_{1}$. From this we can conclude

 \begin{claim}
 The space of equivalence classes of possible $m_{1}$ is exactly
 $H^{2}(\g,\g)$.\mpar[\anton{not justified}]{}
 \end{claim}

 \begin{claim}
 [was HW]If $m_{1}$ is a 2-cocycle, and $m_{N-1},\dots, m_{2}$ satisfy the equations we
 want, then
 \[
 d\left(  \sum_{k=1}^{N-1} \{m_{k},m_{N-k}\} \right)  = 0.
 \]
 \end{claim}

 This is not enough; we know that $\sum_{k=1}^{N+1}\{m_{k},m_{N-k}\}$ is in
 $Z^{3}(\g,\g)$, but to find $m_{N}$, we need it to be trivial in $H^{3}(\g,\g)$
 because of equation \ref{lec09Eq:2}. If the cohomology class of
 $\sum_{k=1}^{N+1}\{m_{k},m_{N-k}\}$ is non-zero, it's class in $H^3(\g,\g)$ is called
 an \emph{obstruction} to $n$-th order deformation. If $H^{3}(\g,\g)$ is zero, then
 any first order deformation (element of $H^{2}(\g,\g)$) extends to a deformation, but
 if $H^{3}(\g,\g)$ is non-zero, then we don't know that we can always extend. Thus,
 $H^3(\g,\g)$ is the space of all possible obstructions to extending a deformation.

 Let's keep looking at cohomology spaces. Consider $C^{\udot}(\g,\CC)$, where $\CC$ is
 a one dimensional trivial representation of $\g$ given by $x\mapsto 0$ for any
 $x\in\g$.
 %Note that it is neat that such a representation exists.

 First question: take $U\g$, with the corresponding 1 dimensional representation
 $\varepsilon:U\g\rightarrow\CC$ given by $\varepsilon(x)=0$ for $x\in\g$.

 \begin{exercise}
   Show that $(U\g,\varepsilon,\Delta,S)$ is a Hopf algebra with the $\e$ above,
   $\Delta(x)=1\otimes x+x\otimes 1$, and $S(x)=-x$  for $x\in\g$. Remember that
   $\Delta$ and $\e$ are algebra homomorphisms, and that $S$ is an anti-homomorphism.
 \end{exercise}

 Let's compute $H^{1}(\g,\CC)$ ($H^{0}$ is boring, just a point). This is
 $\ker(C^{1}\xrightarrow{d} C^{2})$. Well, $C^{1}(\g,\CC)=\hom(\g,\CC)$,
 $C^{2}(\g,\CC)=\hom (\Lambda^{2}\g,\CC)$, and
 \[
 dc(x,y) = c([x,y]).
 \]
 So the kernel is the set of $c$ such that $c([x,y])=0$ for all $x,y\in\g$. Thus,
 $\ker(d)\subseteq C^{1}(\g,\CC)$ is the space of $\g$-invariant linear functionals. Recall
 that $\g$ acts on $\g$ by the adjoint action, and on
 $\g^{*}=C^{1}(\g,\g)$ by the coadjoint action ($x:l\mapsto l_{x}$ where $l_{x}%
 (y)=l([x,y])$). Under the coadjoint action, $l\in \g^*$ is $\g$-invariant if
 $l_{x}=0$. Note that $C^{0}$ is just one point, so its image doesn't have anything in
 it.

 Now let's compute $H^{2}(\g,\CC)=\ker(d:C^{2}\rightarrow C^{3})/\im(d:C^{1}%
 \rightarrow C^{2})$. Let $c\in Z^{2}$, then
 \[
 dc(x,y,z)=c([x,y],z)-c([x,z],y)+c([y,z],x)=0
 \]
 for all $x,y,z\in\g$. Now let's find the image of $d:C^{1}\rightarrow C^{2}$: it is
 the set of functions of the form $dl(x,y)=l([x,y])$ where $l\in\g^{\ast }$. It is
 clear that $l([x,y])$ are (trivial) 2-cocycles because of the Jacobi identity. Let's
 see what can we cook with this $H^{2}$.

 \begin{definition}
 A \emph{central extension}\index{central extension|idxbf} of $\g$ is a short exact sequence
 \[
 0\to \CC \to \tilde{\g}\to \g\to 0.
 \]
 Two such extensions are equivalent if there is a Lie algebra isomorphism
 $f:\tilde{\g}\to \tilde \g'$ such that the diagram commutes:
 \[
 \xymatrix{
 0 \ar[r] &\CC \ar[r]\ar[d]^\id & \tilde \g \ar[r] \ar[d]^f & \g \ar[d]^\id \ar[r] & 0\\
 0 \ar[r] &\CC \ar[r] & \tilde\g' \ar[r] & \g \ar[r] & 0\\
 }
 \]
 \end{definition}

 \begin{theorem} \label{lec10ThmCextns}
 $H^{2}(\g,\CC)$ is isomorphic to the space of equivalence classes of central
 extensions of $\g$.
 \end{theorem}

 \begin{proof}
 Let's describe the map in one direction. If $c\in Z^{2}$, then consider $\tilde\g =
 \g\oplus\CC$ with the bracket $[(x,t),(y,s)] = ([x,y],c(x,y))$. Equivalences of
 extensions boil down to $c(x,y)\mapsto c(x,y)+l([x,y])$.
 \begin{exercise}
   Finish this proof. \anton{This shouldn't be an exercise}
 \end{exercise}
 \end{proof}
 Let's do some (infinite dimensional) examples of central extensions.
 \begin{example}\index{central extension|idxit}
 [Affine Kac-Moody algebras]\index{Kac-Moody algebra} If $\g\subseteq \gl_n$, then we
 define the \emph{loop space}\index{loop space|idxit} or \emph{loop
 algebra}\index{loop algebra} $\mathcal{L}\g$ to be the set of maps $S^1\to \g$. To
 make the space more manageable, we only consider Laurent polynomials, $z\mapsto
 \sum_{m\in \ZZ}a_m z^m$ for $a_m\in \g$ with all but finitely many of the $a_m$ equal
 to zero. The bracket is given by $[f,g]_{\mathcal{L}\g}(z)=[f(z),g(z)]_\g$.

 Since $\g\subseteq \gl_n$, there is an induced trace $tr:\g\to \CC$. This gives a
 non-degenerate inner product on $\mathcal{L}\g$:
 \[
    (f,g):= \oint_{|z|=1} tr\bigl(f(z^{-1})g(z)\bigr) \frac{dz}{z}.
 \]
 There is a natural 2-cocylce on $\mathcal{L}\g$, given by
 \[
    c(f,g) = \frac{1}{2\pi i} \oint_{|z|=1} tr\bigl(f(z)g'(z)\bigr) \frac{dz}{z} =
           \underset{z=0}{Res}\biggl(tr\bigl(f(z)g^{\prime}(z)\bigr)\biggr),
 \]
 and a natural outer derivation $\partial: \mathcal{L}\g\to \mathcal{L}\g$ given by
 $\partial x(z) = \pder{x(z)}{z}$.

 The Kac-Moody algebra \anton{corresponding to the inner product?} is
 $\mathcal{L}\g\oplus \CC \partial\oplus \CC c$. A second course on Lie theory should
 have some discussion of the representation theory of this algebra.
 \end{example}

 \begin{example}
 Let $\gl_{\infty}$\index{gl(infinity)@$\gl_\infty$} be the algebra of matrices with
 finitely many non-zero entries. It is not very interesting. Let $\gl_{\infty}^{1}$ be
 the algebra of matrices with finitely many non-zero diagonals. $\gl_\infty^1$ is
 ``more infinite dimensional'' than $\gl_\infty$, and it is more interesting.
 \begin{exercise} Define
  \[
  J=\left(
  \begin{array}
  [c]{c|c}%
  I & 0\\\hline 0 & -I
  \end{array}
  \right).
  \]
  \mpar[\anton{Who calls that $J$?}]{}
   For $x,y\in\gl_{\infty}$, show that
   \[
   c(x,y)=tr(x[J,y])
   \]
   is well defined (i.e.\ is a finite sum).
 \end{exercise}
 This $c$ is a non-trivial 1-cocycle, i.e.\ $[c]\in H^{2}(\gl_{\infty}^{1},\CC)$ is
 non-zero. By the way, instead of just using linear maps, we require that the maps
 $\Lambda^{2}\gl_{\infty} ^{1}\rightarrow\CC$ are graded linear maps. This is
 $H_{\text{graded}}^{2}$.

 Notice that in $\gl_{n}$, $tr(x[J,y]) = tr(J[x,y])$ is a trivial cocycle (it is $d$ of
 $l(x)=tr(Jx)$. So we have that $H^{2}(\gl_{n},\CC)=\{0\}$.

 We can define $a_{\infty}= \gl_{\infty}\oplus\CC c$. This is some non-trivial central
 extension.
 \mpar[\small \anton{how do we show $H^2(\gl_n,\CC)=0$?, what is this $a_\infty$, and
 what are the relations to Conformal Field Theory and $\mathrm{Vect}(S^{1})\to$
 Virasoro.}]{\anton{what is the grading on $\CC$? It is all in degree 0, but
 $\Lambda \gl_{\infty}$ has positive and negative graded components}}
 \end{example}

 To summarize the last lectures:
 \begin{enumerate}
 \item We related Lie algebras and Lie Groups. If you're interested in
 representations of Lie Groups, looking at Lie algebras is easier.

 \item From a Lie algebra $\g$, we constructed $U\g$, the universal enveloping algebra.
 This got us thinking about associative algebras and Hopf algebras.

 \item We learned about dual pairings of Hopf algebras. For example, $\CC[\Gamma]$ and
 $C(\Gamma)$ are dual, and $U\g$ and $C(G)$ are dual (if $G$ is affine algebraic and
 we are looking at polynomial functions). This pairing is a starting point for many
 geometric realizations of representations of $G$. Conceptually, the notion of the
 universal enveloping algebra is closely related to the notion of the group algebra
 $\CC[\Gamma]$.

 \item Finally, we talked about deformations.
 \end{enumerate}

 \index{Reshetikhin, Nicolai|)}
