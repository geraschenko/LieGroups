 \stepcounter{lecture}
 \setcounter{lecture}{1}
 \sektion{Lecture 1} \index{Reshetikhin, Nicolai|(}

 \begin{definition}
   A \emph{Lie group}\index{Lie group} is a smooth manifold $G$ with a group structure such that the
   multiplication $\mu:G\times G\to G$ and inverse map $\iota:G\to G$ are smooth maps.
 \end{definition}
 \begin{exercise}
   If we assume only that $\mu$ is smooth, does it follow that $\iota$ is smooth?
   \begin{solution}
     Yes. Consider $\mu^{-1}(e)\subseteq G\times G$. We would like to use the implicit
     function theorem to show that there is a function $f$ (which is as smooth as
     $\mu$) such that $(h,g)\in \mu^{-1}(e)$ if and only if $g=f(h)$. This function
     will be $\iota$. You need to check that for every $g$, the derivative of left
     multiplication by $g$ at $g^{-1}$ is non-singular (i.e.\ that $dl_{g}(g^{-1})$ is
     a non-singular matrix). This is obvious because we have an inverse, namely
     $dl_{g^{-1}}(e)$.
   \end{solution}
 \end{exercise}

 \begin{example}
   The group of invertible endomorphisms of $\CC^n$, $GL_n(\CC)$, is a Lie group. The
   automorphisms of determinant 1, $SL_n(\CC)$, is also a Lie group.
 \end{example}
 \begin{example}
   If $B$ is a bilinear form on $\CC^n$, then we can consider the Lie group
   \[
    \{ A\in GL_n(\CC)| B(Av,Aw)=B(v,w) \text{ for all }v,w\in \CC^n \}.
   \]
   If we take $B$ to be the usual dot product, then we get the group $O_n(\CC)$. If we
   let $n=2m$ be even and set $B(v,w)= v^T \matrix 0{1_m}{-1_m}0 w$, then we get
   $Sp_{2m}(\CC)$.
 \end{example}
% \begin{example}
%  $SL_n, O_n, Sp_{2n}$ (for $O_n$ and $Sp_{2n}$, you can describe them as those
%  linear transformations preserving some quadratic form).
% \end{example}
 \begin{example}
   $SU_n\subseteq SL_n(\CC)$ is a \emph{real form}\index{real form} (look in lectures
   27,28, and 29 for more on real forms).
 \end{example}
 \begin{example}
   We'd like to consider infinite matrices, but the multiplication wouldn't make
 sense, so we can think of $GL_n\subseteq GL_{n+1}$ via $A\mapsto
 \matrix{A}{0}{0}{1}$, then define $GL_{\infty}$ as $\bigcup_n GL_n$. That is,
 invertible infinite matrices which look like the identity almost everywhere.
 \end{example}
 Lie groups are hard objects to work with because they have global characteristics,
 but we'd like to know about representations of them. Fortunately, there are things
 called Lie algebras, which are easier to work with, and representations of Lie
 algebras tell us about representations of Lie groups.

 \begin{definition}
   A \emph{Lie algebra}\index{Lie algebra} is a vector space $V$ equipped with a
   \emph{Lie bracket} $[\ ,\,]:V\times V\to V$, which satisfies
   \begin{enumerate}
     \item Skew symmetry: $[a,a]=0$ for all $a\in V$, and

     \item Jacobi identity\index{Jacobi identity}: $[a,[b,c]]+[b,[c,a]]+[c,[a,b]]=0$
     for all $a,b,c\in V$.
   \end{enumerate}
   A \emph{Lie subalgebra} of a Lie algebra $V$ is a subspace $W\subseteq V$ which is
   closed under the bracket: $[W,W]\subseteq W$.
 \end{definition}

 \begin{example}
   If $A$ is a finite dimensional associative algebra, you can set $[a,b]=ab-ba$. If
   you start with $A=\MM_n$, the algebra of $n\times n$ matrices, then you get the Lie
   algebra $\gl_n$\index{gl(n)@$\gl(n)$}. If you let $A\subseteq \MM_n$ be the algebra
   of matrices preserving a fixed flag $V_0\subset V_1\subset \cdots V_k \subseteq
   \CC^n$, then you get \emph{parabolic}index{parabolic subalgebras} Lie sub-algebras
   of $\gl_n$.
 \end{example}
% \underline{Examples}:
% \begin{itemize}
% \item[(a)] $A$, a finite dimensional associative algebra with $[a,b]=ab-ba$.
% \item[(b)] $A=\MM_n \rightsquigarrow \gl_n$
% \item[(c)] $A\subseteq \MM_n$ matrices of the form ... $\rightsquigarrow$ parabolic Lie sub-algebras of $\gl_n$
% \end{itemize}

 \begin{example}
   Consider the set of vector fields on $\RR^n$, $\mathrm{Vect}(\RR^n) = \{
   \ell=\Sigma e^i(x)\pder{}{x_i} | [\ell_1,\ell_2]=\ell_1\circ
   \ell_2-\ell_2\circ\ell_1\}$.
   \begin{exercise}
     Check that $[\ell_1,\ell_2]$ is a first order differential operator.
     \begin{solution}
       Just do it.
     \end{solution}
   \end{exercise}
 \end{example}
 \begin{example}
   If $A$ is an associative algebra, we say that $\partial:A\to A$ is a derivation if
   $\partial(ab)=(\partial a)b+a\partial b$. Inner derivations are those of the form
   $[d,\cdot ]$ for some $d\in A$; the others are called outer derivations. We denote
   the set of derivations of $A$ by $\D(A)$, and you can verify that it is a Lie algebra.
   Note that $\mathrm{Vect}(\RR^n)$ above is just $\D(C^\infty(\RR^n))$.
 \end{example}
 The first Hochschild cohomology\index{cohomology!Hochschild}, denoted $H^1(A,A)$, is
 the quotient $\D(A)/\{$inner derivations$\}$.

 \begin{definition}
   A \emph{Lie algebra homomorphism} is a linear map $\phi:L\to L'$ that takes the
   bracket in $L$ to the bracket in $L'$, i.e.\
   $\phi([a,b]_L)=[\phi(a),\phi(b)]_{L'}$. A \emph{Lie algebra isomorphism} is a
   morphism of Lie algebras that is a linear isomorphism.\footnote{The reader may
   verify that this implies that the inverse is also a morphism of Lie algebras.}
 \end{definition}

 A very interesting question is to classify Lie algebras (up to isomorphism) of
 dimension $n$ for a given $n$. For $n=2$, there are only two: the trivial bracket
 $[\ ,\,]=0$, and $[e_1,e_2]=e_2$. For $n=3$, it can be done without too much trouble.
 Maybe $n=4$ has been done, but in general, it is a very hard problem.

 If $\{e_i\}$ is a basis for $V$, with $[e_i,e_j]=c_{ij}^k e_k$ (the $c_{ij}^k$ are
 called the \emph{structure constants} of $V$), then the Jacobi identity is some
 quadratic relation on the $c_{ij}^k$, so the variety of Lie algebras is some
 quadratic surface in $\CC^{3n}$.\index{variety of Lie algebras}

 Given a smooth real manifold $M^n$ of dimension $n$, we can construct
 $\mathrm{Vect}(M^n)$, the set of smooth vector fields on $M^n$. For $X\in
 \mathrm{Vect}(M^n)$, we can define the \emph{Lie derivative}\index{Lie derivative}
 $L_X$ by $(L_X\cdot f)(m)=X_m\cdot f$, so $L_X$ acts on $C^\infty(M^n)$ as a
 derivation.
 \begin{exercise}
  Verify that $[L_X,L_Y]=L_X\circ L_Y-L_Y\circ L_X$ is of the form $L_Z$ for a unique
  $Z\in \mathrm{Vect}(M^n)$. Then we put a Lie algebra structure on
  $\mathrm{Vect}(M^n)=\D(C^\infty(M^n))$ by $[X,Y]=Z$.
 \end{exercise}
 There is a theorem (Ado's Theorem\footnote{Notice that if $\g$ has no center, then
 the adjoint representation\index{adjoint representation} $ad:\g\to \gl(\g)$ is
 already faithful. See Example \ref{lec07eg:adjoint} for more on the adjoint
 representation. For a proof of Ado's Theorem, see Appendix E of
 \cite{FulHar}})\index{Ado's Theorem} that any Lie algebra $\g$ is isomorphic to a Lie
 subalgebra of $\gl_n$, so if you understand everything about $\gl_n$, you're in
 pretty good shape.
