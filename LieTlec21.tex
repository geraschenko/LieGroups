 \stepcounter{lecture}
 \setcounter{lecture}{21}
 \sektion{Lecture 21 - An overview of Lie groups} \index{Borcherds, Richard E.|(}

 The (unofficial) goal of the last third of the course is to prove no theorems. We'll
 talk about
 \begin{enumerate}
 \item Lie groups in general,
 \item Clifford algebras and Spin groups,
 \item Construction of all Lie groups and all representations. You might say this
 is impossible, so let's just try to do all simple ones, and in particular
 $E_8,E_7,E_6$.
 \item Representations of $SL_2(\RR)$. % and of the Heisenberg group ... didn't happen
 \end{enumerate}

% Today we'll give an overview of Lie algebras and Lie groups. We'll compare with
% algebraic groups (matrix groups), and finite groups.

 \subsektion{Lie groups in general}

 In general, a Lie group $G$ can be broken up into a number of pieces.

 The connected component of the identity, $G_\text{conn}\subseteq G$, is a
 normal subgroup, and $G/G_\text{conn}$ is a discrete group.
 \[
      1\longrightarrow G_\text{conn} \longrightarrow G \longrightarrow
      G_\text{discrete}\longrightarrow 1
 \]

 The maximal connected normal solvable subgroup of $G_\text{conn}$ is called
 $G_\text{sol}$. Recall that a group is \emph{solvable}\index{solvable!group} if there
 is a chain of subgroups $G_\text{sol}\supseteq \cdots\supseteq 1$, where consecutive
 quotients are abelian. The Lie algebra of a solvable group is solvable (by Exercise
 \ref{lec11hardEx}), so Lie's theorem (Theorem \ref{lec11Lie}) tells us that
 $G_\text{sol}$ is isomorphic to a subgroup of the group of upper triangular
 matrices.\anton{almost ... it tells us that $G_\text{sol}/$(some discrete subgroup)
 is a subgroup of upper triangular matrices}

 Every normal solvable subgroup of $G_\text{conn}/G_\text{sol}$ is discrete, and
 therefore in the center (which is itself discrete). We call the pre-image of the
 center $G_*$. Then $G/G_*$ is a product of simple groups (groups with no normal
 subgroups).\anton{is this obvious? it is clear that this is the adjoint form of a
 group with semisimple Lie algebra}
 \[
    G_\text{sol}\subseteq
     \left\{ \mat{\ast & & & \smash{\raisebox{-1ex}{\llap{\LARGE $\ast$}}} \\
                   & \ddots &  & \\
                   & & \ast & \\
                 \smash{\rlap{\LARGE 0}} & & & \ast} \right\}
    \qquad
    G_\text{nil}\subseteq
     \left\{ \mat{1 & & & \smash{\raisebox{-1ex}{\llap{\LARGE $\ast$}}} \\
                   & \ddots &  & \\
                   & & 1 & \\
                 \smash{\rlap{\LARGE 0}} & & & 1} \right\}
 \]
 Since $G_\text{sol}$ is solvable, $G_\text{nil}:=[G_\text{sol},G_\text{sol}]$ is
 nilpotent,\index{nilpotent!group} i.e.\ there is a chain of subgroups
 $G_\text{nil}\supseteq G_1\supseteq\cdots \supseteq G_k=1$ such that $G_i/G_{i+1}$ is
 in the center of $G_\text{nil}/G_{i+1}$. In fact, $G_\text{nil}$ must be isomorphic
 to a subgroup of the group of upper triangular matrices with ones on the diagonal.
 Such a group is called \emph{unipotent}.\index{unipotent group|idxbf}

 We have the picture
 \[ \vfuzz=5pt
 \begin{xy}
   (0,5) *+{G};
   (0,4) *+{G_\text{conn}} **@{-};
   (0,3) *+{G_{*}} **@{-};
   (0,2) *+{G_\text{sol}} **@{-};
   (0,1) *+{G_\text{nil}} **@{-};
   (0,0) *+{1} **@{-};
   (.7,4.5) *=<0em,1.75em>\frm{)} *+!L{\text{discrete; classification hopeless}};
   (.7,3.5) *=<0em,1.75em>\frm{)} *+!L{\prod \text{connected simples; classified}};
   (.7,2.5) *=<0em,1.75em>\frm{)} *+!L{\text{abelian discrete}};
   (.7,1.5) *=<0em,1.75em>\frm{)} *+!L{\text{abelian}};
        (3.7,2) *=<0em,2.5em>\frm{\}} *++!L{\text{classification trivial}};
   (.7,0.5) *=<0em,1.75em>\frm{)} *+!L{\text{nilpotent; classification a mess}};
   (-1,2) *=<0em,10em>\frm{\{} *++!R{\text{connected}}
 \end{xy}
 \]
 The classification of connected simple Lie groups\index{connected} is quite long.
 There are many infinite series and a lot of exceptional cases. Some infinite series
 are $PSU(n)$, $PSL_n(\RR)$, and $PSL_n(\CC)$.\footnote{The $P$ means ``mod out by the
 center''.}

 One way to get many connected simple Lie groups is not observe that there is a unique
 connected simple Lie group for each simple Lie algebra. We've already classified
 complex Lie algebras, and it turns out that there a finite number of real Lie
 algebras which complexify to any given complex Lie algebra. We will classify all such
 real forms\index{real form} in Lecture~29.

 For example, $\sl_2(\RR)\not\simeq \mathfrak{su}_2(\RR)$, but $\sl_2(\RR)\otimes \CC
 \simeq \mathfrak{su}_2(\RR)\otimes \CC \simeq \sl_2(\CC)$. By the way, $\sl_2(\CC)$
 is simple as a \emph{real} Lie algebra, but its complexification is $\sl_2(\CC)\oplus
 \sl_2(\CC)$, which is not simple. Thus, we cannot obtain all connected simple groups
 this way.
 \begin{example}
   Let $G$ be the group of all shape-preserving transformations of $\RR^4$ (i.e.\
   translations, reflections, rotations, and scaling). It is sometimes called
   $\RR^4\cdot\nobreak GO_4(\RR)$. The $\RR^4$ stands for translations, the $G$ means
   that you can multiply by scalars, and the $O$ means that you can reflect and
   rotate. The $\RR^4$ is a normal subgroup.  In this case, we have
   \[
     \renewcommand\arraycolsep{.3ex}
     \raisebox{-.4\baselineskip}{\shortstack{$G_\text{conn}/G_\text{sol}$\\ $=SO_4(\RR)$}}
     \raisebox{.1\baselineskip}{
       $\left\{ \rule{0pt}{2.6\baselineskip} \right.$}
     \begin{array}{rl}
      \RR^4\cdot GO_4(\RR) &= G\\ \\
      \RR^4\cdot GO_4^+(\RR) &= G_\text{conn}\\ \\
      \RR^4\cdot \RR^\times &= G_*\\ \\
      \RR^4\cdot \RR^+ &= G_\text{sol} \\ \\
      \RR^4 &= G_\text{nil}
    \end{array}
    \begin{array}{rl}
      G/G_\text{conn} &= \ZZ/2\ZZ\\ \\
      G_\text{conn}/G_* &= PSO_4(\RR) \\
       & \quad \big(\simeq SO_3(\RR)\times SO_3(\RR)\big)\\
      G_*/G_\text{sol} &= \ZZ/2\ZZ \\ \\
      G_\text{sol}/G_\text{nil} &=\RR^+
    \end{array}
  \]
  where $GO_4^+(\RR)$ is the connected component of the identity (those
  transformations that preserve orientation), $\RR^\times$ is scaling by something
  other than zero, and $\RR^+$ is scaling by something positive. Note that
  $SO_3(\RR) = PSO_3(\RR)$ is simple.

  $SO_4(\RR)$ is ``almost'' the product $SO_3(\RR)\times SO_3(\RR)$. To see this,
  consider the associative (but not commutative) algebra of quaternions, $\HH$. Since
  $q\bar q = a^2+b^2+c^2+d^2 >0$ whenever $q\neq 0$, any non-zero quaternion has an
  inverse (namely, $\bar q/q\bar q$). Thus, $\HH$ is a division algebra. Think of
  $\HH$ as $\RR^4$ and let $S^3$ be the unit sphere, consisting of the quaternions
  such that $\|q\|=q\bar q=1$. It is easy to check that $\|pq\|=\|p\|\cdot \|q\|$,
  from which we get that left (right) multiplication by an element of $S^3$ is a
  norm-preserving transformation of $\RR^4$. So we have a map $S^3\times S^3\to
  O_4(\RR)$. Since $S^3\times S^3$ is connected, the image must lie in $SO_4(\RR)$. It
  is not hard to check that $SO_4(\RR)$ is the image. The kernel is
  $\{(1,1),(-1,-1)\}$. So we have $S^3\times S^3/\{(1,1),(-1,-1)\}\simeq SO_4(\RR)$.

  Conjugating a purely imaginary quaternion by some $q\in S^3$ yields a purely
  imaginary quaternion of the same norm as the original, so we have a homomorphism
  $S^3\to O_3(\RR)$. Again, it is easy to check that the image is $SO_3(\RR)$ and that
  the kernel is $\pm 1$, so $S^3/\{\pm 1\}\simeq SO_3(\RR)$.

  So the universal cover of $SO_4(\RR)$ (a double cover) is the cartesian square of
  the universal cover of $SO_3(\RR)$ (also a double cover). Orthogonal groups in
  dimension 4 have a strong tendency to split up like this.\anton{Really? especially
  in dimension 4?} Orthogonal groups in general tend to have these double covers, as
  we shall see in Lectures 23~and~24. These double covers are important if you want
  to study fermions.\index{fermions}
 \end{example}

% \begin{example}
% Let $G$ be the group of all shape-preserving (translations, rotations, reflections,
% dilations) transformations of $\RR^4$, sometimes called $\RR^4\cdot\nobreak
% GO_4(\RR)$.\footnote{$\RR^4$ is translations, $G$ is multiplication by scalers, $O$
% is reflections and rotations} This is pretty much the smallest possible Lie group with
% all the properties of a general Lie group.
% \begin{itemize}
% \item[(1)] $G$ has a connected component $G_\text{conn}$, which is a normal subgroup, and
% $G/G_\text{conn}$ is a discrete group.
%
% Any Lie group can be built from a connected Lie group and a discrete group
% (0 dimensional group). Classifying discrete groups is completely hopeless.
%
% \item[(2)] We let $G_\text{sol}$ be the maximal normal connected solvable subgroup of
% $G_\text{conn}$. Recall that solvable means that there is a chain
% $G_\text{sol}\supseteq G_\text{sol}' \supseteq \cdots \supset 1$ where all
% consecutive quotients are abelian. So you build solvable groups out of abelian
% groups.
%
% $G_\text{conn}/G_\text{sol}$ is connected and has no solvable normal subgroups. It
% turns out that it is much easier to deal with the group if you kill off all the
% solvable subgroups like this. Groups like this are semisimple. They are almost a
% product of simple groups. A semisimple group $G$ modulo its discrete center is a
% product of simple groups.
%
% The nice thing is that connected simple Lie groups are all completely classified. The
% list is quite long, and it has a lot of infinite series and a lot of exceptional
% guys. Some examples are $PSU(n)$, $PSL_n(\RR)$, $PSL_n(\CC)$. You can go from simple
% Lie groups to simple Lie algebras over $\RR$, then complexify to get Lie algebras
% over $\CC$. This last step is finite-to-one, and they can be classified fairly
% easily. For example, the Lie algebras of $\sl_2(\RR)$ and $\mathfrak{su}_2(\RR)$ are
% different, but they have the same complexification, $\sl_2(\CC)$, which, by the way,
% is actually simple as a \emph{real} Lie algebra. Its complexification is not simple,
% it is $\sl_2(\CC)\oplus \sl_2(\CC)$.
%
% Suppose $G_\text{sol}$ is solvable. The only thing you can say is that its derived
% subgroup $G'_\text{sol}$ is nilpotent. That is, $G'\supseteq [G',G']\supseteq
% [G',[G',G']]\supseteq \cdots\supseteq 1$, or rather there is a chain such that
% $G_i/G_{i+1}\subseteq $ the center of $G_1/G_{i+1}$. Thus, nilpotent groups can be
% built up by taking central extensions\index{central extension}. Nilpotent groups are in practice not
% classifiable. The simply connected ones are all isomorphic to subgroups of
% upper-triangular matrices
% \[
%%    \left\{ \mat{1 & & & \smash{\raisebox{-1ex}{\llap{\Huge $\ast$}}} \\
%%                   & \ddots &  & \\
%%                   & & 1 & \\
%%                 \smash{\rlap{\Huge 0}} & & & 1} \right\}
%    \left\{ \mat{1 & & & \smash{\raisebox{-1ex}{\llap{\LARGE $\ast$}}} \\
%                   & \ddots &  & \\
%                   & & 1 & \\
%                 \smash{\rlap{\LARGE 0}} & & & 1} \right\}
%%    \left\{ \mat{1 & & & \ast \\
%%                   & \ddots &  & \\
%%                   & & 1 & \\
%%                 0 & & & 1} \right\}
% \]
% If you put arbitrary things on the diagonal, that is what the solvable groups look
%like.
%
% \end{itemize}
%
% In our case, $G$ is $\RR^4\cdot\nobreak GO_4(\RR)$. The connected component
% $G_\text{conn}$ is $\RR^4\cdot\nobreak GO^+_4(\RR)$, which is the stuff that
% preserves orientation. $G_\text{sol}$ is $\RR^4\cdot\nobreak \RR^+$ (rescaling).
% $G_\text{nil}$ is $\RR^4$. $G_\text{conn}/G_\text{sol}$ is $SO_4(\RR)$,
% $G_\text{sol}/G_\text{nil}$ is $\RR^+$. $G_\text{conn}/G_*$ is $PSO_4(\RR)\simeq
% SO_3(\RR)\times SO_3(\RR)$. $G_*/G_\text{sol}$ is $\ZZ/2\ZZ$.
%
% Note: What is $O_4(\RR)$? It almost is the product $O_3(\RR)\times O_3(\RR)$. To see
% this is to look at the quaternions $\HH$, which is an associative, but not
% commutative algebra. Notice that $q\bar q = a^2+b^2+c^2+d^2$, which is $>0$ if $q\neq
% 0$, so any non-zero quaternion has an inverse. Thus, $\HH$ is a division algebra. Put
% $\RR^4$ to be the quaternions, and $S^3$ is the unit sphere in this $\RR^4$
% consisting of the quaternions such that $\|q\|=q\bar q=1$. It is easy to check that
% $\|pq\|=\|p\|\, \|q\|$. So left or right multiplication by an element of $S^3$ give
% rotations of $\RR^4$. Thus, we have that $S^3\times S^3\to O_4(\RR)$, and the image
% is $SO_4(\RR)$. This is not quite injective because the kernel is
% $\{(1,1),(-1,-1)\}$. So you get $S^3\times S^3/(-1,-1)\simeq SO_4(\RR)$.
%
% Let $\RR^3$ be the purely imaginary quaternions. Then $q\in S^3$ acts by conjugation
% on $\RR^3$, so we get $S^3\to O_3(\RR)$, and again it is not quite injective or
% surjective. The image is $SO_3(\RR)$ and the kernel is $\{1,-1\}$, so
% $S^3/\{1,-1\}\simeq SO_3(\RR)$. So we almost have that $SO_4$ isomorphic to
% $SO_3\times SO_3$. If you take a double cover of $SO_4(\RR)$, it is the product of
% two copies of a double cover of $SO_3(\RR)$. Orthogonal groups in dimension 4 have a
% strong tendency to split up like this.
% \end{example}

 \subsektion{Lie groups and Lie algebras}

 Let $\g$ be a Lie algebra. We can set $\g_\text{sol} = \text{rad}\, \g$ to be the
 maximal solvable ideal (normal subalgebra), and $\g_\text{nil} =
 [\g_\text{sol},\g_\text{sol}]$. Then we get the chain
 \[\vfuzz=5pt
 \begin{xy}
   (0,3) *+{\g};
   (0,2) *+{\g_\text{sol}} **@{-};
   (0,1) *+{\g_\text{nil}} **@{-};
   (0,0) *+{0} **@{-};
   (.5,2.7);(.5,2.3) **\frm{)}; (.5,2.5) *+!L{\prod\text{simples; classification known}};
   (.5,1.7);(.5,1.3) **\frm{)}; (.5,1.5) *+!L{\text{abelian; easy to classify}};
   (.5,0.7);(.5,0.3) **\frm{)}; (.5,0.5) *+!L{\text{nilpotent; classification a mess}};
 \end{xy}
 \]
 We have an equivalence of categories between simply connected Lie groups and Lie
 algebras. The correspondence cannot detect
 \begin{itemize}
   \item Non-trivial components of $G$. For example, $SO_n$ and $O_n$ have the same
   Lie algebra.

   \item Discrete normal (therefore central, Lemma \ref{lec05L:discCentral}) subgroups
   of $G$. If $Z\subseteq G$ is any discrete normal subgroup, then $G$ and $G/Z$ have
   the same Lie algebra. For example, $SU(2)$ has the same Lie algebra as
   $PSU(2)\simeq SO_3(\RR)$.
 \end{itemize}
 If $\tilde G$ is a connected and simply connected Lie group with Lie algebra $\g$,
 then any other connected group $G$ with Lie algebra $\g$ must be isomorphic to
 $\tilde G/Z$, where $Z$ is some discrete subgroup of the center. Thus, if you know
 all the discrete subgroups of the center of $\tilde G$, you can read off all the
 connected Lie groups with the given Lie algebra.

 Let's find all the groups with the algebra $\so_4(\RR)$. First let's find a simply
 connected group with this Lie algebra. You might guess $SO_4(\RR)$, but that isn't
 simply connected. The simply connected one is $S^3\times S^3$ as we saw earlier (it
 is a product of two simply connected groups, so it is simply connected). The center
 of $S^3$ is generated by $-1$, so the center of $S^3\times S^3$ is $(\ZZ/2\ZZ)^2$,
 the Klein four group. There are three subgroups of order 2
 \[\xymatrix @!0 @R=3em @C=4em {
   & (\ZZ/2\ZZ)^2 \ar@{-}[dl] \ar@{-}[d] \ar@{-}[dr]\\
   (-1,1) \ar@{-}[dr]& (-1,-1) \ar@{-}[d]& (1,-1) \ar@{-}[dl]\\
   & 1
 }\qquad
 \xymatrix @!0 @R=3em @C=5.8em{
   & PSO_4(\RR) \ar@{-}[dl] \ar@{-}[d] \ar@{-}[dr]\\
   SO_3(\RR)\times S^3 \ar@{-}[dr]& SO_4(\RR) \ar@{-}[d]& S^3\times SO_3(\RR) \ar@{-}[dl]\\
   & S^3\times S^3
 }\]
 Therefore, there are 5 groups with Lie algebra $\so_4$.

 \subsektion{Lie groups and finite groups}
 \begin{enumerate}
 \item The classification of finite simple groups resembles the classification of
 connected simple Lie groups when $n\ge 2$. \anton{what is $n$?}

 For example, $PSL_n(\RR)$ is a simple Lie group, and $PSL_n(\FF_q)$ is a finite
 simple group except when $n=q=2$ or $n=2,q=3$. Simple finite groups form about 18
 series similar to Lie groups, and 26 or 27 exceptions, called sporadic groups, which
 don't seem to have any analogues for Lie groups.\anton{what about exceptional Lie
 groups?}

 \item Finite groups and Lie groups are both built up from simple and abelian groups.
 However, the way that finite groups are built is much more complicated than the way
 Lie groups are built. Finite groups can contain simple subgroups in very complicated
 ways; not just as direct factors.

 For example, there are \emph{wreath products}.\index{wreath product|idxbf} Let $G$
 and $H$ be finite simple groups with an action of $H$ on a set of $n$ points. Then
 $H$ acts on $G^n$ by permuting the factors. We can form the semi-direct product
 $G^n\ltimes H$, sometimes denoted $G\wr H$. There is no analogue for (finite
 dimensional) Lie groups. There \emph{is} an analogue for infinite dimensional Lie
 groups, which is why the theory becomes hard in infinite dimensions.

 \item The commutator subgroup of a solvable finite group need not be a nilpotent
 group. For example, the symmetric group $S_4$ has commutator subgroup $A_4$, which is
 not nilpotent.

% \item[(4)] (non-trivial) Nilpotent finite groups are never usually subgroups of upper
% triangular matrices (with ones on the diagonal).
 \end{enumerate}

 \subsektion{Lie groups and Algebraic groups (over \texorpdfstring{$\RR$}{the reals})}
 By algebraic group, we mean an algebraic variety which is also a group, such as
 $GL_n(\RR)$. Any algebraic group is a Lie group. Probably all the Lie groups you've
 come across have been algebraic groups. Since they are so similar, we'll list some
 differences.
 \begin{enumerate}
 \item Unipotent and semisimple abelian algebraic groups are totally different, but
 for Lie groups they are nearly the same. For example $\RR\simeq
 \left\{\matrix{1}{\ast}{0}{1}\right\}$ is unipotent and $\RR^\times \simeq
 \left\{\matrix{a}00{a^{-1}}\right\}$ is semisimple. As Lie groups, they are closely
 related (nearly the same), but the Lie group homomorphism $\exp: \RR\to \RR^\times$
 is not algebraic (polynomial), so they look quite different as algebraic groups.

 \item Abelian varieties are different from affine algebraic groups. For example,
 consider the (projective) elliptic curve $y^2=x^3+x$ with its usual group operation
 and the group of matrices of the form $\matrix{a}b{-b}a$ with $a^2+b^2=1$. Both are
 isomorphic to $S^1$ as Lie groups, but they are completely different as algebraic
 groups; one is projective and the other is affine.

 \item Some Lie groups do not correspond to ANY algebraic group. We give two examples
 here.

 The \emph{Heisenberg group}\index{Heisenberg group|idxit} is the subgroup of
 symmetries of $L^2(\RR)$ generated by translations ($f(t)\mapsto f(t+x)$),
 multiplication by $e^{2\pi ity}$ ($f(t)\mapsto e^{2\pi ity} f(t)$), and
 multiplication by $e^{2\pi iz}$ ($f(t)\mapsto e^{2\pi iz}f(t)$). The general element
 is of the form $f(t)\mapsto e^{2\pi i(yt+z)}f(t+x)$. This can also be modelled as
 \[
 \left.\left\{\mat{1 & x & z \\ 0 & 1 & y \\ 0 & 0 & 1}\right\}\right/
 \left\{\left.\mat{1 & 0 & n \\ 0 & 1 & 0 \\ 0 & 0 & 1} \right| n\in \ZZ\right\}
 \]
 It has the property that in any finite dimensional representation, the center
 (elements with $x=y=0$) acts trivially, so it cannot be isomorphic to any algebraic
 group.

 The \emph{metaplectic group}.\index{metaplectic group|idxbf} Let's try to find all
 connected groups with Lie algebra $\sl_2(\RR) = \{\matrix{a}{b}{c}{d}| a+d=0\}$.
 There are two obvious ones: $SL_2(\RR)$ and $PSL_2(\RR)$. There aren't any other ones
 that can be represented as groups of finite dimensional matrices. However, if you
 look at $SL_2(\RR)$, you'll find that it is not simply connected. To see this, we
 will use Iwasawa decomposition (without proof).
 \begin{theorem}[Iwasawa decomposition]\label{lec21T:Iwasawa}\index{Iwasawa
 decomposition!idxbf}
   If $G$ is a (connected) semisimple Lie group, then there are closed subgroups $K$,
   $A$, and $N$, with $K$ compact, $A$ abelian, and $N$ unipotent, such that the
   multiplication map $K\times A\times N\to G$ is a surjective diffeomorphism.
   Moreover, $A$ and $N$ are simply connected.
 \end{theorem}
 In the case of $SL_n$, this is the statement that any basis can be obtained uniquely
 by taking an orthonormal basis ($K=SO_n$), scaling by positive reals ($A$ is the
 group of diagonal matrices with positive real entries), and shearing ($N$ is the
 group $\Bigl( \begin{array}{cc}
        1 \smash{\raisebox{-1.5ex}{\rlap{\scriptsize $\ddots$}}}& \ast\\ 0 & 1
        \end{array}\Bigr)$). This is exactly the result of the Gram-Schmidt
 process.\index{Gram-Schmidt}

 The upshot is that $G\simeq K\times A \times N$ (topologically), and $A$ and $N$ do
 not contribute to the fundamental group, so the fundamental group of $G$ is the same
 as that of $K$. In our case, $K=SO_2(\RR)$ is isomorphic to a circle, so the
 fundamental group of $SL_2(\RR)$ is $\ZZ$.

 So the universal cover $\widetilde{SL_2(\RR)}$ has center $\ZZ$. Any
 finite dimensional representation of $\widetilde{SL_2(\RR)}$ factors through
 $SL_2(\RR)$, so none of the covers of $SL_2(\RR)$ can be written as a group of
 finite dimensional matrices. Representing such groups is a pain.

 The most important case is the metaplectic group $Mp_2(\RR)$, which is the connected
 double cover of $SL_2(\RR)$. It turns up in the theory of modular forms of
 half-integral weight and has a representation called the metaplectic representation.
 \end{enumerate}

 \subsektion{Important Lie groups}

 \underline{Dimension 1}: There are just $\RR$ and $S^1 = \RR/\ZZ$.

 \underline{Dimension 2}: The abelian groups are quotients of $\RR^2$ by some discrete
 subgroup; there are three cases: $\RR^2$, $\RR^2/\ZZ = \RR\times S^1$, and
 $\RR^2/\ZZ^2 = S^1\times S^1$.

 There is also a non-abelian group, the group of all matrices of the form $\matrix
 ab0{a^{-1}}$, where $a>0$. The Lie algebra is the subalgebra of $2\times 2$ matrices
 of the form $\matrix hx0{-h}$, which is generated by two elements $H$ and $X$, with
 $[H,X]=2X$.

 \underline{Dimension 3}: There are some boring abelian and solvable groups, such as
 $\RR^2\ltimes \RR^1$, or the direct sum of $\RR^1$ with one of the two dimensional
 groups. As the dimension increases, the number of boring solvable groups gets huge,
 and nobody can do anything about them, so we ignore them from here on.

 You get the group $SL_2(\RR)$, which is the most important Lie group of all. We saw
 earlier that $SL_2(\RR)$ has fundamental group $\ZZ$. The double cover $Mp_2(\RR)$ is
 important. The quotient $PSL_2(\RR)$ is simple, and acts on the open upper half plane
 by linear fractional transformations

 Closely related to $SL_2(\RR)$ is the compact group $SU_2$. We know that $SU_2\simeq
 S^3$, and it covers $SO_3(\RR)$, with kernel $\pm 1$. After we learn about Spin
 groups, we will see that $SU_2 \cong \spin_3(\RR)$. The Lie algebra $\mathfrak{su}_2$
 is generated by three elements $X$, $Y$, and $Z$ with relations $[X,Y]=2Z$,
 $[Y,Z]=2X$, and $[Z,X]=2Y$.\footnote{An explicit representation is given by
 $X=\matrix 01{-1}0$, $Y=\matrix 0ii0$, and $Z=\matrix i00{-i}$. The cross product on
 $\RR^3$ gives it the structure of this Lie algebra.}

 The Lie algebras $\sl_2(\RR)$ and
 $\mathfrak{su}_2$ are non-isomorphic, but when you complexify, they both become
 isomorphic to $\sl_2(\CC)$.

 There is another interesting 3 dimensional algebra. The Heisenberg algebra is the Lie
 algebra of the Heisenberg group. It is generated by $X,Y,Z$, with $[X,Y]=Z$ and $Z$
 central. You can think of this as strictly upper triangular matrices.

 \underline{Dimension 6}: (nothing interesting happens in dimensions 4,5) We get the
 group $SL_2(\CC)$. Later, we will see that it is also called $\spin_{1,3}(\RR)$.

 \underline{Dimension 8}: We have $SU_3(\RR)$ and $SL_3(\RR)$. This is the first time
 we get a non-trivial root system.

 \underline{Dimension 14}: $G_2$, which we will discuss a little.

 \underline{Dimension 248}: $E_8$, which we will discuss in detail.

 \smallskip
 This class is mostly about finite dimensional algebras, but let's mention some
 infinite dimensional Lie groups or Lie algebras.
 \begin{enumerate}
   \item Automorphisms of a Hilbert space form a Lie group.

   \item Diffeomorphisms of a manifold form a Lie group. There is some physics stuff
   related to this.

   \item \emph{Gauge groups}\index{Guage groups!idxbf} are (continuous, smooth,
   analytic, or whatever) maps from a manifold $M$ to a group $G$.

   \item The \emph{Virasoro algebra}\index{Virasoro algebra} is generated by $L_n$ for
   $n\in \ZZ$ and $c$, with relations $[L_n,L_m]=(n-m) L_{n+m} +
   \delta_{n+m,0}\frac{n^3-n}{12}c$, where $c$ is central (called the \emph{central
   charge}). If you set $c=0$, you get (complexified) vector fields on $S^1$, where we
   think of $L_n$ as $ie^{in\theta}\pder{}{\theta}$. Thus, the Virasoro algebra is a
   central extension
   \[
     0\to c\CC \to \text{Virasoro}\to \text{Vect}(S^1)\to 0.
   \]

   \item Affine Kac-Moody algebras, which are more or less central extensions of
   certain gauge groups over the circle.
 \end{enumerate}
