 \stepcounter{lecture}
 \setcounter{lecture}{20}
 \sektion{Lecture 20 - Compact Lie groups}\index{compact groups|(}
 \anton{Things that should be here:
 \begin{enumerate}
   \item Rep theory of reductive Lie \rlap{\rule[3pt]{3.6em}{.4pt}}algebras groups
   \item fundamental group of a compact group is finite?
   \item That one paragraph should be expanded. $|Z(G)|\le |P/Q|=\det($Cartan), with
   equality if $G$ simply connected.
   \item stuff about maximal tori? e.g.\ $\exp$ is surjective for a compact group
   \item uniqueness of the compact real form?
   \item Peter-Weyl theorem?
   \item Finite-dimensional representations of a compact group are unitary.
 \end{enumerate}}

 So far we classified semisimple Lie algebras over an algebraically closed field
 characteristic 0. Now we will discuss the connection to compact groups.
 Representations of Lie groups are always taken to be smooth.
 \begin{example}\index{SU(n)@$SU(n)$|idxit}
   $SU(n) = \{X\in GL(n,\CC)| \bar X^tX=\id \text{ and } \det X=1\}$ is a compact
   connected Lie group over $\RR$. It is the group of linear transformations of
   $\CC^n$ preserving some hermitian form.

   You may already know that $SU(2)$ is topologically a 3-sphere.
 \end{example}
 \begin{exercise}
   If $G$ is an abelian compact connected Lie group, then it is a product of circles,
   so it is $\mathbb{T}^n$.
   \begin{solution}
     Since $G$ is abelian, $\g$ is the abelian Lie algebra $\RR^n$, whose simply
     connected Lie group is $\RR^n$. Thus, $G$ is a quotient of $\RR^n$ by a discrete
     subgroup (i.e.\ a lattice). Since $G$ is compact, this lattice must be full rank,
     so $G\cong \mathbb{T}^n$.
   \end{solution}
 \end{exercise}
 There exists the $G$-invariant volume form\footnote{A \emph{volume form} is a
 non-vanishing top degree form.} $\w$ satisfying
 \begin{enumerate}
 \item The volume of $G$ is one: $\int_G \w =1$, and \item $\w$ is left invariant:
 $\int_G f \w = \int_G L_h^* f\,\w$ for all $h\in G$. Recall that $L_h^*f$ is defined
 by $(L_h^* f)(g) = f(hg)$.
 \end{enumerate}
 To construct $\w$ pick $\w_e\in \Lambda^\text{top} (T_eG)^*$ and define $\w_g =
 L_{g^{-1}}^* \w_e$.
 \begin{exercise}
   If $G$ is connected, show that this $\w$ is also right invariant. Even if $G$ is
   not connected, show that the measure obtained from a right invariant form agrees
   with the measure obtained from a left invariant form.
   \begin{solution}
     Consider the representation $G\to \End (\Lambda^\text{top}T_eG)\simeq \End(\RR)=
     \RR^\times$ given by $h\mapsto \Lambda^\text{top}Ad_h$. Since $G$ is compact, its
     image must also be compact, but the only compact subgroups of $\RR^\times$ are
     $\{1\}$ and $\{\pm 1\}$.

     If $G$ is connected, the image must be $\{1\}$, so the adjoint action on
     $\Lambda^\text{top}T_eG$ is trivial. It follows that $R_h^* \w_e =
     L_h^*\w_e=\w_h$, i.e.\ that $\w$ is right invariant.

     If $G$ is not connected, then we may have $R_h^*\w_e=-\w_h$. That is, the left
     invariant form agrees with the right invariant form up to sign. Since the volume
     form determines the orientation, changing it by a sign does not change the
     measure.
   \end{solution}
 \end{exercise}
 \begin{theorem}
   If $G$ is a compact group and $V$ is a real representation of $G$, then there
   exists a positive definite $G$-invariant inner product on $V$. That is,
   $(gv,gw)=(v,w)$.
 \end{theorem}
 \begin{proof}
   Pick any positive definite inner product\footnote{Pick any basis, and declare it to
   be orthonormal.} $\langle v,w\rangle$, and define
   \[
    (v,w) = \int_G \langle gv,gw\rangle \w
   \]
   which is positive definite and invariant.
 \end{proof}
 It follows that any finite dimensional representation of a compact group $G$ is
 completely reducible (i.e.\ splits into a direct sum of irreducibles) because the
 orthogonal complement to a subrepresentation is a subrepresentation.

 In particular, the representation $Ad: G\to GL(\g)$ is completely reducible, and the
 irreducible subrepresentations are exactly the irreducible subrepresentations of the
 derivative, $ad:\g\to \gl(\g)$. Thus, we get the decomposition $\g=\g_1\oplus\cdots
 \g_k\oplus \a$, with each $\g_i$ is a one dimensional or simple ideal. We dump all
 the one dimensional $\g_i$ into $\a$, which is then the center of $\g$. Thus, the Lie
 algebra of a compact group is the direct sum of its center and a semisimple Lie
 algebra. Such a Lie algebra is called \emph{reductive}.\index{reductive}

 If $G$ is simply connected, then I claim that $\a$ is trivial. This is because the
 simply connected group connected to $\a$ must be a torus, so a center gives you some
 fundamental group\anton{why can't the rest of $\g$ somehow ``fill in'' the torus?}.
 Thus, if $G$ is simply connected, then $\g$ is semisimple.

 \begin{theorem}
   If the Lie group $G$ of $\g$ is compact, then the Killing form $B$ on $\g$ is
   negative semi-definite. If the Killing form on $\g$ is negative definite, then
   there is some compact group $G$ with Lie algebra $\g$. \anton{In fact, since
   $\pi_1$ of a compact group is finite, all groups with Lie algebra $\g$ are
   compact. Should this be in this lecture?}
 \end{theorem}
 \begin{proof}
   If you have $\g\to \gl(\g)$, and you know that $\g$ has an $ad$-invariant positive
   definite product, so it lies in $\so(\g)$. Here you have $A^t=-A$, so you have to
   check that $tr(A^2)< 0$. It is not hard to check that the eigenvalues of $A$ are
   imaginary (as soon as $A^t=-A$), so we have that the trace of the square is
   negative (or zero).

   If $B$ is negative definite, then it is non-degenerate, so $\g$ is semisimple by
   Theorem \ref{lec12Cartan}, and $-B$ is an inner product. Moreover, we have that
   \[
      -B(ad_X Y,Z) = B(Y,ad_X Z)
   \]
   so $ad_X = -ad_X^t$ with respect to this inner product. That is, the image of $ad$
   lies in $\so(\g)$. It follows that the image under $Ad$ of the simply connected
   group $\tilde G$ with Lie algebra $\g$ lies in $SO(\g)$. Thus, the image is a
   closed subgroup of a compact group, so it is compact. Since $Ad$ has a discrete
   kernel, the image has the same Lie algebra.
 \end{proof}


 How to classify compact Lie algebras?  We know the classification over $\CC$, so we
 can always take $\g\rightsquigarrow \g_\CC=\g\otimes_\RR \CC$, which remains
 semisimple. However, this process might not be injective. For example, take $\mathfrak{su}(2)
 = \{\matrix{a}{b}{-\bar b}{a}| a\in \RR i, b\in \CC\}$ and $\sl(2,\RR)$, then they
 complexify to the same thing.

 $\g$ in this case is called a \emph{real form}\index{real form} of $\g_\CC$. So you
 can start with $\g_\CC$ and classify all real forms.
 \begin{theorem}[Cartan]\index{Cartan}
   Every semisimple Lie algebra has exactly one (up to isomorphism) compact real
   form.\index{real form!compact|idxbf}
 \end{theorem}
 For example, for $\sl(2)$ it is $\mathfrak{su}(2)$.

 Classical Lie groups: $SL(n,\CC), SO(n,\CC)$ ($SO$ has lots of real forms of this,
because in the real case, you get a signiture of a form; in the complex case, all
forms are isomorphic), $Sp(2n,\CC)$. What are the corresponding compact simple Lie
groups?

 \underline{Compact real forms}: $SU(n)=$ the group of linear operators on $\CC^n$
 preserving a positive definite Hermitian form. $SL(n)=$ the group of linear operators
 on $\RR^n$ preserving a positive definite symmetric bilinear form. $Sp(2n)=$the group
 of linear operators on $\mathbb{H}^n$ preserving a positive definite Hermitian form

 We're not going to prove this theorem because we don't have time, but let's show
 existence.

 \begin{proof}[Proof of existence]
   Suppose $\g_\CC = \g\otimes_\RR \CC = \g\oplus i\g$. Then you can construct
   $\sigma: \g_\CC\to \g_\CC$ ``complex conjugation''. Then $\sigma$ preserves the
   commutator, but it is an anti-linear involution. Classifying real forms amounts to
   classifying all anti-linear involutions. There should be one that corresponds to
   the compact algebra. Take $X_1,\dots, X_n, H_1,\dots, H_n,Y_1,\dots, Y_n$
   generators for the algebra. Then we just need to define $\sigma$ on the generators:
   $\sigma(X_i)=-Y_i, \sigma(Y_i)=-X_i, \sigma(H_i)=-H_i$, and extend anti-linearly.
   This particular $\sigma$ is called the \emph{Cartan
   involution}\index{Cartan!involution|idxbf}.

   Now we claim that $\g = (\g_\CC)^\sigma = \{X|\sigma(X)=X\}$ is a compact simple
   Lie algebra. We just have to check that the Killing form is negative definite. If
   you take $h\in \h$, written as $h = \sum a_iH_i$, then $\sigma(h)=h$ implies that
   all the $a_i$ are purely imaginary. This implies that the eigenvalues of $h$ are
   imaginary, which implies that $B(h,h)<0$. You also have to check it on $X_i,Y_i$.
   The fixed things will be of the form $(aX_i-\bar aY_i)\in \g$. The Weyl group
   action shows that $B$ is negative on all of the root space.
 \end{proof}

   Look at $\exp \h^\sigma \subset G$ (simply connected), which is called the maximal
   torus $T$. I'm going to tell you several facts now. You can always think of $T$ as
   $\RR^n/L$. The point is that $\RR^n$ can be identified with $\h_{re}$, and
   $\h_{re}^*$ has two natural lattices: $Q$ (the root lattice) and $P$ (the weight
   lattice). So one can identify $T = \RR^n/L = \h_{re}/\check P$, where $\check P$ is
   the natural dual lattice to $P$, the set of $h\in \h$ such that $\langle
   \w,h\rangle \in \ZZ$ for all $\w\in P$. $G$ is simply connected, and when you
   quotient by the center, you get $Ad\, G$, and all other groups with the same
\anton{make sense of this paragraph}
   algebra are in between. $Ad\, T = \h_{re}/\check Q$. We have the sequence
   $\{1\}\to Z(G)\to T \to Ad\, T \to \{1\}$. You can check that any element is
   semisimple in a compact group, so the center of $G$ is the quotient $P/Q\simeq
   \check Q/\check P$. Observe that $|P/Q| = $ the determinant of the Cartan matrix.
   For example, if $\g=\sl(3)$, then we have $\det\matrix 2{-1}{-1}2=3$, and the
   center of $SU(3)$ is the set of all elements of the form $diag(\w,\w,\w)$ where
   $\w^3=1$.

   $G_2$ has only one real form because the det is 1?

  Orthogonality relations for compact groups:
  \[
    \int_G \chi(g)\bar\psi(g^{-1}) \w = \delta_{\chi,\psi}
  \]
  where $\chi$ and $\psi$ are characters of irreducible representations. You know that
  the character is constant on conjugacy classes, so you can integrate over the
  conjugacy classes. There is a nice picture for $SU(2)$.

  \[\begin{xy}
    (0,0) *\xycircle(2,2){},
    (-2,0);(2,0) **\crv{(-2,-1)&(2,-1)}
        **\crv{~*=<2pt>@{.} (-2,1)&(2,1)},(.5,-1) *{T},
    (-1,1.732);(-1,-1.732) **\crv{(-1.5,1.6)&(-1.5,-1.6)}
        **\crv{~*=<2pt>@{.} (-.7,1.6)&(-.7,-1.6)},
    (1.4142136,1.4142136);(1.4142136,-1.4142136) **\crv{(1.2,1.4142136)&(1.2,-1.4142136)}
        **\crv{~*=<2pt>@{.} (1.6,1.4142136)&(1.6,-1.4142136)},
    (1.55,.49) *{\bullet}, (1.28,-.61) *{\bullet},
    (-1.35,-.57) *{\bullet}, (-.79,.68) *{\bullet},
  \end{xy}\]

  The integral can be written as
  \[
    \frac{1}{|\weyl|}\int_T \chi(t)\bar\psi(t)Vol(C(t))dt
  \]
  And $Vol(C(t)) = \D(t)\bar \D(t)$. You divide by $|\weyl|$ because that is how many
  times each class hits $T$.

 \index{Serganova, Vera|)}
