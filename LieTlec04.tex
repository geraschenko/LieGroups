 \stepcounter{lecture}
 \setcounter{lecture}{4}
 \sektion{Lecture 4}

\begin{theorem}
Suppose $G$ is a topological group.  Let $U \subset G$ be an open neighbourhood of $e
\in G$.  If $G$ is connected, then
\begin{eqnarray*}
G = \bigcup_{n \geq 1} U^n.
\end{eqnarray*}
\end{theorem}
\begin{proof}
Choose a non-empty open set $V \subset U$ such that $V = V^{-1}$, for example $V = U
\cap U^{-1}$. Define $H = \cup_{n \geq 1} V^n$, and observe $H$ is an abstract
subgroup, since $V^n V^m \subseteq V^{n+m}$.  $H$ is open since it is the union of
open sets.  If $\sigma \notin H$, then $\sigma H \not\subset H$, since otherwise if
$h_1, h_2 \in H$ satisfy $\sigma h_1 = h_2$, then $\sigma = h_2 h_1^{-1} \in H$.  Thus
$H$ is a complement of the union of all cosets not containing $H$.  Hence $H$ is
closed. Since $G$ is connected, $H = G$.
\end{proof}
\begin{theorem}
Let $f: G \to H$ be a Lie group homomorphism.  Then the following diagram commutes:
 \[\xymatrix{
 T_e \ar[r]^{(df)_e} \ar[d]_{\exp} & T_e H\ar[d]^{\exp}\\
 G \ar[r]^f & H
  }\]
Further, if $G$ is connected, $(df)_e$ determines $f$ uniquely.
\end{theorem}

\begin{proof}
1) Commutative diagram.  Fix $\tau \in T_eG$ and set $\eta = df_e \tau \in T_eH$.
Recall we defined the vector field $V_\tau(g) = (dl_g)(\tau)$, then if $\phi(t)$
solves
\begin{eqnarray*}
\frac{ d \phi}{dt} = V_\tau ( \phi(t)) \in T_{\phi(t)}G,
\end{eqnarray*}
we have $\exp(\tau) = \phi(1)$.  Let $\psi$ solve
\begin{eqnarray*}
\frac{d \psi}{dt} = V_\eta( \psi(t)),
\end{eqnarray*}
so that $\exp ( \eta) = \psi(1)$.  Observe $\tilde{\psi}(t) = f( \phi(t))$ satisfies
\begin{eqnarray*}
\frac{d \tilde{\psi}}{dt} = (df)\left( \frac{d\phi}{dt} \right) =
V_\eta(\tilde{\psi}),
\end{eqnarray*}
so by uniqueness of solutions to ordinary differential equations, $\psi =
\tilde{\psi}$.

2) Uniqueness of $f$.  The exponential map is an isomorphism of a neighborhood of $0
\in \g$ and a neighborhood of $e \in G$.  But if $G$ is connected, $G = \cup_{n \geq
1} (\text{nbd } e)^n$.
\end{proof}

\begin{theorem}
Suppose $G$ is a topological group, with $G^0 \subset G$ the connected component of
$e$.  Then 1) $G^0$ is normal and 2) $G / G^0$ is discrete.
\end{theorem}

\begin{proof}
2) $G^0 \subset G$ is open implies $\text{pr}^{-1}( [e]) = e G^0$ is open in $G$,
which in turn implies $\text{pr}^{-1}([g]) \in G/G^0$ is open for every $g \in G$.
Thus each coset is both open and closed, hence $G/G^0$ is discrete.

1) Fix $g \in G$ and consider the map $G \to G$ defined by $x \mapsto g x g^{-1}$.
This map fixes $e$ and is continuous, which implies it maps $G^0$ into $G^0$.  In
other words, $g G^0 g^{-1} \subset G^0$, or $G^0$ is normal.
\end{proof}

We recall some basic notions of algebraic topology.  Suppose $M$ is a connected
topological space.  Let $x, y \in M$, and suppose $\gamma(t): [0,1] \to M$ is a path
from $x$ to $y$ in $M$.  We say $\tilde{\gamma}(t)$ is {\it homotopic} to $\gamma$ if
there is a continuous map $h(s,t): [0,1]^2 \to M$ satisfying
\begin{eqnarray*}
\bullet h(s,0) = x, \,\, h(s,1) = y \\
\bullet h(0,t) = \gamma(t), \,\, h(1,t) = \tilde{\gamma}(t).
\end{eqnarray*}
We call $h$ the {\it homotopy}.  On a smooth manifold, we may replace $h$ with a
smooth homotopy.  Now fix $x_0 \in M$.  We define the first fundamental group of $M$
\begin{eqnarray*}
\pi_1(M, x_0) = \left\{ \text{ homotopy classes of loops based at }x_0 \right\}.
\end{eqnarray*}
It is clear that this is a group with group multiplication composition of paths.  It
is also a fact that the definition does not depend on the base point $x_0$:
\begin{eqnarray*}
\pi_1(M, x_0) \simeq \pi_1(M, x_0').
\end{eqnarray*}
By $\pi_1(M)$ we denote the isomorphism class of $\pi_1(M, \cdot)$. Lastly, we say $M$
is {\it simply connected} if $\pi_1(M) = \{e\}$, that is if all closed paths can be
deformed to the trivial one.

\begin{theorem}
\label{lec04T:4} Suppose $G$ and $H$ are Lie groups with Lie algebras $\g, \h$
respectively. If $G$ is simply connected, then any Lie algebra homomorphism $\rho: \g
\to \h$ lifts to a Lie group homomorphism $R : G \to H$.
\end{theorem}

In order to prove this theorem, we will need the following lemma.
\begin{lemma}
Let $\xi : \mathbb{R} \to \g$ be a smooth mapping.  Then
\begin{eqnarray*}
\frac{dg}{dt} = (dl_g)(\xi(t))
\end{eqnarray*}
has a unique solution on all of $\mathbb{R}$ with $g(t_0) = g_0$.
\end{lemma}
For convenience, we will write $g \xi:= (dl_g)(\xi)$.
\begin{proof}
Since $\g$ is a vector space, we identify it with $\mathbb{R}^n$ and for sufficiently
small $r>0$, we identify $B_r(0) \subset \g$ with a small neighbourhood of $e$,
$U_e(r) \subset G$, under the exponential map.  Here $B_r(0)$ is measured with the
usual Euclidean norm $\| \cdot \|$.  Note for any $g \in U_e(r)$ and $|t-t_0|$
sufficiently small, we have $\|g \xi(t) \| \leq C$.  Now according to Exercise
\ref{ex-l4-1}, the solution with $g(t_0) = e$ exists for sufficiently small $|t-t_0|$
and
\begin{eqnarray*}
g(t) \in U_e(r) \,\, \forall |t-t_0| < \frac{r}{C'}.
\end{eqnarray*}
Now define $h(t) = g(t) g_0$ so that $h(t) \in U_{g_0}(r)$ for $|t-t_0| < r/C'$.  That
is, $r$ and $C'$ do not depend on the choice of initial conditions, and we can cover
$\mathbb{R}$ by intervals of length, say $r/C'$.
\end{proof}
\begin{exercise}
\label{ex-l4-1} Verify that there is a constant $C'$ such that if $|t-t_0|$ is
sufficiently small, we have
\begin{eqnarray*}
\|g(t)\| \leq C' |t-t_0|.
\end{eqnarray*}
  \begin{solution}
  We calculate:
  \begin{eqnarray*}
  \frac{d}{dt} \| g(t) \|^2 & = & 2 \langle \frac{d}{dt} g, g \rangle \\
  & \leq & 2\left\| \frac{d}{dt} g \right\|  \| g \|\\
  & \leq & 2\| \xi\| \| g \|^2.
  \end{eqnarray*}
  That is, $\eta(t) := \|g(t)\|^2$ satisfies the differential inequality:
  \begin{eqnarray*}
  \frac{d}{dt} \eta(t) \leq \|\xi\| \eta(t),
  \end{eqnarray*}
  which in turn implies (Gronwall's inequality) that
  \begin{eqnarray*}
  \eta(t)  \leq  e^{2\int_{t_0}^t \| \xi(s) \| ds}
  \end{eqnarray*}
  so that
  \begin{eqnarray*}
  \|g\| &  \leq & e^{\int_{t_0}^t \| \xi(s) \| ds} \\
  & \leq & C'|t - t_0|
  \end{eqnarray*}
  since for $|t-t_0|$ sufficiently small, exponentiation is Lipschitz.
  \end{solution}
\end{exercise}


\begin{proof}[Proof of Theorem \ref{lec04T:4}]
We will construct $R: G \to H$.  Beginning with $g(t): [0,1] \to G$ satisfying $g(0) =
e$, $g(1) = g$, define $\xi(t) \in \g$ for each $t$ by
\begin{eqnarray*}
g(t) \xi(t) = \frac{d}{dt} g(t).
\end{eqnarray*}
Let $\eta(t) = \rho(\xi(t))$, and let $h(t):[0,1] \to H$ satisfy
\begin{eqnarray*}
\frac{d}{dt} h(t) = h(t) \eta(t), \,\,\,\, h(0) = e.
\end{eqnarray*}
Define $R(g) = h(1)$.

{\bf Claim:} $h(1)$ does not depend on the path $g(t)$, only on $g$.

\noindent {\it Proof of Claim}.  Suppose $g_1(t)$ and $g_2(t)$ are two different paths
connecting $e$ to $g$.  Then there is a smooth homotopy $g(t,s)$ satisfying $g(t,0) =
g_1(t)$, $g(t,1) = g_2(t)$. Define $\xi(t,s)$ and $\eta(t,s)$ by
\begin{eqnarray*}
\frac{\partial g}{ \partial t} & = & g(t,s) \xi(t,s); \\
\frac{\partial g}{ \partial s} & = & g(t,s) \eta(t,s).
\end{eqnarray*}
Observe
\begin{eqnarray}
\frac{\partial^2 g}{ \partial s \partial t} & = & g \eta \circ \xi + g
\frac{\partial \xi }{\partial t} \,\,\, \text{and } \label{l4-eq-3}\\
\frac{\partial^2 g}{ \partial t \partial s} & = & g \xi \circ \eta + g \frac{\partial
\eta}{\partial s} \label{l4-eq-4},
\end{eqnarray}
and \eqref{l4-eq-3} is equal to \eqref{l4-eq-4} since $g$ is smooth. Consequently
\begin{eqnarray*}
\frac{\partial \eta}{\partial t} - \frac{\partial \xi}{ \partial s} =
     [ \eta, \xi ].
\end{eqnarray*}
Now define an $s$ dependent family of solutions $h(\cdot, s)$ to the equations
\begin{eqnarray*}
\frac{\partial h}{\partial t} (t,s) = h(t,s) \rho(\xi(t,s)), \,\,\, h(0,s) = e.
\end{eqnarray*}
Define $\theta(t,s)$ by
\begin{eqnarray}
\label{l4-eq-5} \left\{ \begin{array}{c} \displaystyle{\frac{\partial \theta}{\partial
t}} - \displaystyle{\frac{
    \partial \rho(\xi)}{\partial s}} = \left[ \rho(\xi), \theta
    \right], \\
  \theta(0,s) = 0. \end{array} \right.
\end{eqnarray}
Observe $\tilde{\theta}(t,s) = \rho(\eta(t,s))$ also satisfies equation
\eqref{l4-eq-5}, so that $\theta = \tilde{\theta}$ by uniqueness of solutions to ODEs.
Finally,
\begin{eqnarray*}
g \eta(1,s) = \frac{\partial g}{\partial s} (1,s) = 0 \implies \theta(1,s) = 0
\implies \frac{\partial h }{\partial s}(1,s) = 0,
\end{eqnarray*}
justifying the claim.

We need only show $R:G \to H$ is a homomorphism.  Let $g_1, g_2 \in G$ and set $g =
g_1g_2$.  Let $\tilde{g}_i(t)$ be a path from $e$ to $g_i$ in $G$ for each $i = 1,2$.
Then the path $\tilde{g}(t)$ defined by
\begin{eqnarray*}
\tilde{g}(t) = \left\{ \begin{array}{l} \tilde{g}_1(2t), \,\, 0 \leq t
  \leq \frac{1}{2}, \\ g_1 \tilde{g}_2(2t-1), \,\, \frac{1}{2} \leq t
  \leq 1 \end{array} \right.
\end{eqnarray*}
goes from $e$ to $g$.  Let $\tilde{h}_i$ for $i = 1,2$ and $\tilde{h}$ be the paths in
$H$ corresponding to $\tilde{g}_1$, $\tilde{g}_2$, and $\tilde{g}$ respectively and
calculate
\begin{eqnarray*}
R(g_1 g_2) = R(g) = \tilde{h}(1) = \tilde{h}_1(1) \tilde{h}_2(1) = R(g_1) R(g_2).
\end{eqnarray*}
\end{proof}
