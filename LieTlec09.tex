 \stepcounter{lecture}
 \setcounter{lecture}{9}
 \sektion{Lecture 9}

 \newcommand{\diagramsize}{1.25em}

 Let's summarize what has happened in the last couple of lectures.
 \begin{enumerate}
 \item We talked about $T(\g)$, and then constructed three algebras:
 \begin{itemize}
   \item $U\g =  T(\g)/\langle x\otimes y-y\otimes x-[x,y]\rangle$, with $U\g=S_1(\g)
     \simeq S_\varepsilon (\g)$ as filtered associative algebras, for all non-zero
     $\varepsilon \in \CC$.

   \item $S_\varepsilon (\g) = T(\g)/\langle x\otimes y-y\otimes
      x-\varepsilon[x,y]\rangle$ is a family of associative algebras, with $S_\varepsilon
      (\g) \simeq S_0(\g)$ as filtered vector spaces.

   \item $S_0(\g)\cong Pol(\g^*) = T(\g)/\langle x\otimes y-y\otimes x\rangle = S_0(\g)$
      is an associative, commutative algebra with a Poisson structure defined by the Lie
      bracket.
 \end{itemize}

 \item We have two ``pictures'' of deformations of an algebra
    \begin{enumerate}
    \item There is a simple ``big'' algebra $B$ (such as $B=T(\g)$) and a family of
    ideals $I_\varepsilon$. Then we get a family $B/I_\varepsilon=A_\varepsilon$. This
    becomes a deformation family of the associative algebra $A_0$ if we identify
    $A_\varepsilon \simeq A_0$ as vector spaces (these are called \emph{torsion free}
    deformations). Fixing this isomorphism gives a family of associative products on
    $A_0$.

    We can think of this geometrically as a family of (embedded) varieties.
    \item Alternatively, we can talk about deformations intrinsically (i.e., without referring to some bigger $B$).
    Suppose we have $A_0$ and a family of associative products $a\ast_\varepsilon  b$
    on $A_0$.
    \begin{example}
      Let $Pol(\g^*) \xrightarrow{\phi} S_\varepsilon (\g)$ be the isomorphism of the PBW theorem. Then
      define \( f\ast g = \phi^{-1} (\phi(f)\cdot \phi(g)) = fg + \sum_{n\ge 1}
      \varepsilon^n m_n(f,g)\).
    \end{example}
    \end{enumerate}
    Understanding deformations makes a connection between representation theory and
    Poisson geometry. A second course on Lie theory should discuss symplectic leaves
    of $Pol(\g^*)$, which happen to be coadjoint orbits and correspond to
    representations.  This is why deformations are relevant to representation theory.
 \end{enumerate}

 Let $A$ be a Poisson algebra with bracket $\{\ ,\,\}$, so it is a commutative
 algebra, and a Lie algebra, with the bracket acting by derivations. Typically,
 $A=C^\infty(M)$. Equivalence classes of formal (i.e., formal power series) symmetric
 (i.e.,$m_n(f,g)=(-1)^n m_n(g,f)$ ) star products on $C^\infty(M)$ are in bijection with
 equivalence classes of formal deformations of $\{\ ,\,\}$ on $C^\infty(M)[[h]]$.

 Apply this to the case $A=C^\infty(\g^*)$. The associative product on
 $S_\varepsilon(\g)$ comes from the product on $T(\g)$. The question is, ``how many
 equivalence classes of star products are there on $A$?'' Any formal deformation of
 the Poisson structure on $(A,\{\ ,\,\}_\g)$ is a PBW deformation\mpar[\anton{What is a PBW
 deformation? What's going one here?}]{} of some formal deformation of the Lie algebra
 $C^\infty(\g^*)$ (with Lie bracket $\{f,g\}(x)=x(df\wedge dg)$). Such a deformation
 is equivalent to a formal deformation of the Lie algebra structure on $\g$. This is
 one of the reasons that deformations of Lie algebras are important --- they describe
 deformations of certain associative algebras. When one asks such questions, some
 cohomology theory always shows up.

 \subsektion{Lie algebra cohomology}\index{Lie algebra
 cohomology|idxbf}\index{cohomology!of Lie algebras|idxbf} Recall that $(M,\phi)$ is a
 $\g$-module if $\phi:\g\to \End(M)$ is a Lie algebra homomorphism. We will write $xm$
 for $\phi(x)m$. Define $C^\udot (\g,M) = \bigoplus_{q\ge 0} C^q(\g,M)$ where
 $C^q(\g,M) = \hom(\Lambda^q \g, M)$ (linear maps). We define $d:C^q\to C^{q+1}$ by
 \begin{align*}
  \iflilbook
    &dc(x_1\wedge\cdots\wedge x_{q+1}) =\\
  \else
    dc(x_1&\wedge\cdots\wedge x_{q+1}) =\\
  \fi
     &= \sum_{1\le s< t\le q+1} (-1)^{s+t-1} c([x_s,x_t]\wedge x_1 \wedge\cdots \wedge
     \hat x_s \wedge \cdots \wedge \hat x_t \wedge \cdots \wedge x_{q+1}) \\
  &\qquad + \sum_{s=1}^{q+1} (-1)^s x_s c(x_1\wedge \cdots \wedge \hat x_s \wedge
  \cdots \wedge x_{q+1})
 \end{align*}
 \begin{exercise}
   Show that $d^2=0$.
 \end{exercise}

 \underline{Motivation}: If $\g=\mathrm{Vect}(\mathcal{M})$,
 $M=C^\infty(\mathcal{M})$, then $C^q(\g,M) = \W^q(\mathcal{M})$, with the Cartan
 formula
 \begin{align*}
  \iflilbook
    &(d\w)(\xi_1 \wedge\cdots \wedge \xi_{q+1}) = \\
  \else
    (d\w)(\xi_1 &\wedge\cdots \wedge \xi_{q+1}) = \\
  \fi
     &= \sum_{1\le s< t\le q+1} (-1)^{s+t-1} \w ([\xi_s,\xi_t]\wedge \xi_1 \wedge\cdots
     \wedge \hat \xi_s \wedge \cdots \wedge \hat \xi_t \wedge \cdots \wedge\xi_{q+1}) \\
  &\qquad + \sum_{s=1}^{q+1} (-1)^s \xi_s \w (\xi_1\wedge \cdots \wedge \hat \xi_s \wedge
  \cdots \wedge \xi_{q+1})
\end{align*}
 for vector fields $\xi_i$.

 Another motivation comes from the following proposition.
 \begin{proposition}
   $C^\udot (\g,\CC) \simeq \W^\udot_R(G) \subseteq \W^\udot(G)$ where $\CC$ is the
   1 dimensional trivial module over $\g$ (so $xm=0$).
 \end{proposition}
 \begin{exercise}
   Prove it.
 \end{exercise}
 \begin{remark}
   This was Cartan's original motivation for Lie algebra cohomology. It turns out that
   the inclusion $\W^\udot_R(G)\hookrightarrow \W^\udot(G)$ is a homotopy equivalence
   of complexes (i.e.\ the two complexes have the same homology), and the proposition
   above tells us that $C^\udot(\g,\CC)$ is homotopy equivalent to $\W_R(G)$. Thus, by
   computing the Lie algebra cohomology of $\g$ (the homology of the complex
   $C^\udot(\g,\CC)$), one obtains the De Rham cohomology of $G$ (the homology of the
   complex $\W^\udot(G)$).
 \end{remark}

 Define $H^q(\g,M) = \ker(d:C^q\to C^{q+1})/\im(d:C^{q-1}\to C^q)$ as always. Let's
 focus on the case $M=\g$, the adjoint representation\index{adjoint representation|idxit}:
 $x\cdot m = [x,m]$.
 \begin{itemize}
 \item[$H^0(\g,\g)$] We have that $C^0 = \hom(\CC,\g)\cong \g$, and
 \[
    dc(y) = y\cdot c = [y,c].
 \]
  so $\ker (d:C^0\to C^1)$ is the set of $c\in \g$ such that $[y,c]=0$ for all $y\in \g$.
  That is, the kernel is the center of $\g$, $Z(\g)$. So $H^0(\g,\g)=Z(\g)$.

 \mpar[\anton{these can be more general ... $H^i(\g,V)$. See \cite{HiltonStammbach}}]{}
 \item[$H^1(\g,\g)$]\label{lec09H1(g,g)} The kernel of $d:C^1(\g,\g)\to C^2(\g,\g)$ is
 \[\!\!
 \{\mu:\g\to \g| d\mu(x,y)=\mu([x,y])-[x,\mu(y)]-[\mu(x),y]=0 \text{ for all }x,y\in
 \g\},
 \]
 which is exactly the set of derivations of $\g$.
%
% : those $\mu$ such that
% \[
%  \mu([x,y]) = [\mu(x),y]+[x,\mu(y)]
% \]
%
 The image of $d:C^0(\g,\g)\to C^1(\g,\g)$ is the set of \emph{inner derivations},
 $\{dc:\g\to \g|dc(y)=[y,c]\}$.  The Liebniz rule is satisfied because of the Jacobi
 identity. So
 \[
    H^1(\g,\g) = \{\text{derivations}\}/\{\text{inner derivations}\} =: \text{\em
    outer derivations}.
 \]

 \item[$H^2(\g,\g)$] Let's compute $H^2(\g,\g)$. Suppose $\mu\in C^2$, so
 $\mu:\g\wedge\g\to \g$ is a linear map. What does $d\mu=0$ mean?
 \begin{align*}
    d\mu(x_1,x_2,x_3) &= \mu([x_1,x_2],x_3) - \mu([x_1,x_3],x_2) + \mu([x_2,x_3],x_1)
    \\ &\qquad - [x_1,\mu(x_2,x_3)]
    + [x_2,\mu(x_1,x_3)] - [x_3,\mu(x_1,x_2)]\\
     &= -\mu(x_1,[x_2,x_3]) - [x_1,\mu(x_2,x_3)] + \text{cyclic permutations}
 \end{align*}
 Where does this kind of thing show up naturally?

 Consider deformations of Lie algebras:
 \[
    [x,y]_h = [x,y] + \sum_{n\ge 1} h^n m_n(x,y)
 \]
 where the $m_n:\g\times \g\to \g$ are bilinear. The deformed bracket $[\ ,\,]_h$ must
 satisfy the Jacobi identity,
 \[
    [a,[b,c]_h]_h + [b,[c,a]_h]_h + [c,[a,b]_h]_h = 0
 \]
 which gives us relations on the $m_n$. In degree $h^N$, we get
 \begin{align}
    [a,m_N(b,c)] + m_N(a,[b,c]) + \sum_{k=1}^{N-1} m_k(a,m_{N-k}(b,c)) +& \nonumber \\
    [b,m_N(c,a)] + m_N(b,[c,a]) + \sum_{k=1}^{N-1} m_k(b,m_{N-k}(c,a)) +& \nonumber \\
    [c,m_N(a,b)] + m_N(c,[a,b]) + \sum_{k=1}^{N-1} m_k(c,m_{N-k}(a,b)) &=0
    \label{lec09Eq:h^N}
 \end{align}
 \begin{exercise} \label{lec09Ex2}
   Derive equation \ref{lec09Eq:h^N}.
   \begin{solution}
     We have $[a,b]_h := \sum_{n=0}^\infty h^n m_n(a,b)$, where $m_0(a,b)=[a,b]$.
     Now we compute
     \begin{align*}
       [a,[b,c]_h]_h &= [a,\sum_{l\ge 0} h^l m_l(b,c)]_h\\
                &= \sum_{l\ge 0} h^l \sum_{k\ge 0} h^k m_k(a,m_l(b,c))\\
                &= \sum_{N\ge 0} h^N m_k(a,m_{N-k}(b,c)) & (N=k+l)
     \end{align*}
     Adding the cyclic permutations and looking at the coefficient of $h^N$, we get
     the desired result.
   \end{solution}
 \end{exercise}
 Define $[m_K,m_{N-K}](a,b,c)$ as \mpar[\anton{Gerstenhaber}]{}
 \[
  m_K\bigl(a,m_{N-K}(b,c)\bigr) + m_K\bigl(b,m_{N-K}(c,a)\bigr) + m_K\bigl(c,m_{N-K}(a,b)\bigr).\]
  Then equation \ref{lec09Eq:h^N} can be written as
 \begin{equation}\label{lec09Eq:2}
    dm_N = \sum_{k=1}^{N-1} [m_k,m_{N-k}] %\tag{$\ddag$}
 \end{equation}

 \begin{theorem}
  Assume that for all $n \leq N-1$, we have the relation $dm_n = \sum_{k=1}^{n-1}
  [m_k,m_{n-k}]$.  Then $d(\sum_{k=1}^{N-1} [m_k,m_{N-k}])=0$.
 \end{theorem}
 \begin{exercise} \label{lec09Ex3}
    Prove it.
 \end{exercise}
 The theorem tells us that if we have a ``partial deformation'' (i.e.\ we have found
 $m_1,\dots, m_{N-1}$), then the expression $\sum_{k=1}^{N-1} [m_k,m_{N-k}]$ is a
 3-cocycle. Furthermore, equation \ref{lec09Eq:2} tells us that if we are to extend
 our deformation to one higher order, $\sum_{k=1}^{N-1} [m_k,m_{N-k}]$ must represent
 zero in $H^3(\g,\g)$.

  Taking $N=1$, we get $dm_1=0$, so   $\ker(d:C^2\to C^3) = $ space of first
  coefficients of formal deformations of $[\ ,\,]$. It will turn out that $H^2$ is the
  space of equivalence classes of $m_1$.\mpar[\anton{State clearly what $H^2$ is,
  please, or ref}]{}
 \end{itemize}

It is worth noting that the following ``pictorial calculus" may make some of the above
computations easier.  In the following pictures, arrows are considered to be oriented
downwards, and trivalent vertices with two lines coming in and one going out represent
the Lie bracket.  So, for example, the antisymmetry of the Lie bracket is expressed as

  \[\begin{xy}<\diagramsize,0em>:
   (0,.5);(.5,-.5) **\crv{(1,0)},
   (1,.5);(.5,-.5) **\crv{(0,0)},
   (.5,-.5);(.5,-1) **@{-},
 \end{xy}
 = -
 \begin{xy}<\diagramsize,0em>:
   (0,.5);(.5,-.5) **@{-},
   (1,.5);(.5,-.5) **@{-},
   (.5,-.5);(.5,-1) **@{-},
 \end{xy}
 \]
and the Jacobi identity is
 % \[\begin{xy}<\diagramsize,0em>:
 %  (0,1);(.5,0) **@{-} ?*@{>},
 %  (1,1);(.5,0) **@{-} ?*@{>},
 %  (.5,0);(1,-1) **@{-} ?*@{>},
 %  (2,1);(1,-1) **@{-} ?*@{>},
 %  (1,-1);(1,-1.5) **@{-} ?*@{>},
 %\end{xy}\]

 \begin{align*}
   {\begin{xy}<\diagramsize,0em>:
     (0,1);(.5,0) **@{-},
     (1,1);(.5,0) **@{-},
     (.5,0);(1,-1) **@{-},
     (2,1);(1,-1) **@{-},
     (1,-1);(1,-1.5) **@{-},
   \end{xy}} +
   {\begin{xy}<\diagramsize,0em>:
     (0,1);(1,-1) **\crv{(0,-1)&(2,0)},
     (1,1);(1,0) **@{-},
     (2,1);(1,0) **@{-},
     (1,0);(1,-1) **\crv{(0,-.5)},
     (1,-1);(1,-1.5) **@{-},
   \end{xy}} +
   {\begin{xy}<\diagramsize,0em>:
     (0,1);(.5,0) **@{-},
     (1,1);(1.3,-1) **@{-},
     (2,1);(.5,0) **@{-},
     (.5,0);(1.3,-1) **@{-},
     (1.3,-1);(1.3,-1.5) **@{-},
   \end{xy}} &= \\
   {\begin{xy}<\diagramsize,0em>:
     (0,1);(.5,0) **@{-},
     (1,1);(.5,0) **@{-},
     (.5,0);(1,-1) **@{-},
     (2,1);(1,-1) **@{-},
     (1,-1);(1,-1.5) **@{-},
   \end{xy}} -
   {\begin{xy}<\diagramsize,0em>:
     (0,1);(1,-1) **@{-},
     (1,1);(1.5,0) **@{-},
     (2,1);(1.5,0) **@{-},
     (1.5,0);(1,-1) **@{-},
     (1,-1);(1,-1.5) **@{-},
   \end{xy}} +
   {\begin{xy}<\diagramsize,0em>:
     (0,1);(.5,0) **@{-},
     (1,1);(1.3,-1) **@{-},
     (2,1);(.5,0) **@{-},
     (.5,0);(1.3,-1) **@{-},
     (1.3,-1);(1.3,-1.5) **@{-},
   \end{xy}} &= 0
 \end{align*}
We can also use pictures to represent cocycles.  Take $\mu \in H^n(\g, \g)$.  Then we
draw $\mu$ as
\[{\begin{xy}<\diagramsize,0em>:
   (1,0) *+{\mu} *\cir{};
   (0,1) **@{-}, (2,1) **@{-}, (1,-1) **@{-},
   (1,.8) *{\dots}
 \end{xy}}\]
 with $n$ lines going in.  Then, the Cartan formula\index{Cartan!formula} for the differential says that

 \[ d \left(
 {\begin{xy}<\diagramsize,0em>:
   (1,0) *+{\mu} *\cir{};
   (0,1) **@{-}, (2,1) **@{-}, (1,-1) **@{-},
   (1,.8) *{\dots}
 \end{xy}} \right)
 =
 \sum_{1\le i\le j\le n+1} (-1)^{i+j+1}
 {\begin{xy}<\diagramsize,0em>:
   (.7,1) *+!D{i};(0,0) **@{-},
   (1.6,1) *+!D{j};(0,0) **@{-};
   (1,-1) *+{\mu} *\cir{} **@{-};
   (0,1) **@{-}, (2,1) **@{-}, (1,-2) **@{-},
 \end{xy}}
 + \sum_{1\le i\le n+1}
 {\begin{xy}<\diagramsize,0em>:
   (1,0) *+{\mu} *\cir{};
   (0,1) **@{-}, (2,1) **@{-}, (.5,-1) **@{-},
   (1,1) *+!D{i}; (.5,-1) **\crv{(0,0)};
   (.5,-1.5) **@{-}
 \end{xy}}
 \]
and the bracket of two cocycles $\mu \in H^m$ and $\nu \in H^n$ is
 \[ [\mu, \nu] = \sum_{1\le i\le n}
 {\begin{xy}<\diagramsize,0em>:
   (1,.5) *+{\mu} *\cir{};
   (.3,1.5) *+!D{i} **@{-}, (1.7,1.5) **@{-},
   (1,-1) *+{\nu} *\cir{} **@{-};
   (-1,1.5) **@{-}, (-.3,1.5) **@{-}, (2.3,1.5) **@{-}, (3,1.5) **@{-}, (1,-2) **@{-},
 \end{xy}}
 - \sum_{1\le i\le m}
 {\begin{xy}<\diagramsize,0em>:
   (1,.5) *+{\nu} *\cir{};
   (.3,1.5) *+!D{i} **@{-}, (1.7,1.5) **@{-},
   (1,-1) *+{\mu} *\cir{} **@{-};
   (-1,1.5) **@{-}, (-.3,1.5) **@{-}, (2.3,1.5) **@{-}, (3,1.5) **@{-}, (1,-2) **@{-},
 \end{xy}}
 \]\mpar{\anton{This is not consistent with the Gerstenhaber bracket on the previous page}}
 \begin{exercise}
   Use pictures to show that $d[\mu,\nu]=\pm [d\mu, \nu] \pm [\mu, d\nu]$.
 \end{exercise}
 Also, these pictures can be used to do the calculations in Exercises \ref{lec09Ex2}
 and \ref{lec09Ex3}.
