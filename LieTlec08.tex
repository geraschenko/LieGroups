 \stepcounter{lecture}
 \setcounter{lecture}{8}
 \sektion{Lecture 8 - The PBW Theorem and Deformations}

 Last time, we introduced the universal enveloping algebra $U\g$ of a Lie algebra
 $\g$, with its universality property. We discussed graded and filtered spaces and
 algebras. We showed that under some condition on a filtered algebra $A$, the graded
 algebra $Gr(A)$ is a Poisson algebra. We also checked that $U\g$ satisfies this
 condition, and that $Gr(U\g)\simeq S(\g)$ as graded commutative algebras. The latter
 space can be understood as the space $Pol(\g^*)$ of polynomial functions on $\g^*$.
 It turns out that the Poisson bracket on $Gr(U\g)$, expressed in $Pol(\g^*)$, is
 given by
 \[
    \{f,g\}(x) = x([df_x,dg_x])
 \]
 for $f,g\in Pol(\g^*)$ and $x\in \g^*$. Note that $f$ is a function on $\g^*$ and $x$
 an element of $\g^*$, so $df_x$ is a linear form on $T_x\g^*=\g^*$, that is,
 $df_x\in\g$.

 Suppose that $V$ admits a filtration $V_0\subset V_1\subset V_2\subset\cdots $. Then,
 the associated graded space $Gr(V)=V_0\oplus \bigoplus_{n\ge 1} (V_n/V_{n+1})$ is also
 filtered. (Indeed, every graded space $W=\bigoplus_{n\ge 0}W_n$ admits the filtration
 $W_0\subset W_0\oplus W_1\subset W_0\oplus W_1\oplus W_2\subset\cdots$) A natural
 question is:
 When do we have $V\simeq Gr(V)$ as filtered spaces ?

 For the filtered space $U\g$, the answer is a consequence of the following theorem.
 \begin{theorem}[Poincar\'e-Birkhoff-Witt] \index{Poincar\'e-Birkhoff-Witt|see{PBW}}
 \index{PBW|idxbf}
   Let $e_1,\dots, e_n$ be any linear basis for $\g$. Let us also denote by
   $e_1,\dots, e_n$ the image of this basis in the universal enveloping algebra $U\g$.
   Then the monomials $e_1^{m_1}\cdots e_n^{m_n}$ form a basis for $U\g$.
 \end{theorem}
 \begin{corollary}
 There is an isomorphism of filtered spaces $U\g\simeq Gr(U\g)$.
 \end{corollary}
 \begin{proof}[Proof of the corollary]

In $S(\g)$, $e_1^{m_1}\cdots e_n^{m_n}$ also forms a basis, so we get an isomorphism
$U\g\simeq S(\g)$ of filtered vector spaces by simple identification of the bases.
Since $Gr(U\g)\simeq S(\g)$ as graded algebras, the corollary is proved.
 \end{proof}
 \begin{remark}
 The point is that these spaces are isomorphic as {\em filtered} vector spaces. Saying
 that two infinite dimensional vector spaces are isomorphic is totally useless.
 \end{remark}
 \begin{proof}[Proof of the theorem]
  By definition, the unordered monomials $e_{i_1}\cdots e_{i_k}$ for $k\le p$ span the
  subspace $T_0\oplus \cdots \oplus T_p$ of $T(\g)$, where $T_i=\g^{\otimes i}$. Hence,
  they also span the quotient $(U\g)_p := T_0\oplus \cdots \oplus T_p/\langle x\otimes
  y-y\otimes x -[x,y]\rangle$. The goal is now to show that the {\em ordered} monomials
  $e^{m_1}_1\cdots e^{m_n}_n$ for $m_1+\dots+m_n\le p$ still span $(U\g)_p$. Let's prove
  this by induction on $p\ge 0$. The case $p=0$ being trivial, consider $e_{i_1}\cdots
  e_{i_a}\cdots e_{i_k}$, with $k\le p$, and assume that $i_a$ has the smallest value
  among the indices $i_1,\dots, i_k$. We can move $e_{i_a}$ to the left as follows
  \begin{align*}
  e_{i_1}\cdots e_{i_a}\cdots e_{i_k}=e_{i_a}e_{i_1}\cdots \hat e_{i_a}\cdots e_{i_k}
  + \sum_{b=1}^{a-1} e_{i_1}\cdots e_{i_{b-1}} [e_{i_b},e_{i_a}]\cdots\hat
  e_{i_a}\cdots e_{i_k}.
  \end{align*}
  Using the commutation relations $[e_{i_b},e_{i_a}]=\sum_\ell c_{i_bi_a}^\ell
  e_\ell$, we see that the term to the right belongs to $(U\g)_{k-1}$. Iterating this
  procedure leads to an equation of the form
  \[
  e_{i_1}\cdots e_{i_a}\cdots e_{i_k}=e_1^{m_1}\cdots e_n^{m_n} + \hbox{terms in
    $(U\g)_{k-1}$,}
  \]
  with $m_1+\dots+m_n=k\le p$. We are done by induction. The proof of the theorem is
  completed by the following homework.[This should really be done here]
 \end{proof}
 \begin{exercise}
   Prove that these ordered monomials are linearly independant.
 \end{exercise}

 Let's ``generalize'' the situation. We have $U\g$ and $S(\g)$, both of which are
 quotients of $T(\g)$, with kernels $\langle x\otimes y-y\otimes x -[x,y]\rangle$ and
 $\langle x\otimes y-y\otimes x\rangle$. For any $\varepsilon \in \CC$, consider the
 associative algebra $S_\varepsilon(\g) = T(\g)/\langle x\otimes y-y\otimes x
 -\varepsilon[x,y]\rangle$. By construction, $S_0(\g)=S(g)$ and $S_1(\g)=U\g$. Recall
 that they are isomorphic as filtered vector spaces.
 \begin{remark}
   If $\varepsilon\neq 0$, the linear map $\phi_\varepsilon: S_\varepsilon(\g) \to
   U\g$ given by $\phi_\varepsilon(x)=\varepsilon x$ for all $x\in \g$ is an
   isomorphism of filtered algebras. So, we have nothing new here.
 \end{remark}
 We can think of $S_\varepsilon (\g)$ as a non-commutative deformation of the
 associative commutative algebra $S(\g)$. (Note that commutative deformations of the
 algebra of functions on a variety correspond to deformations of the variety.)

 \index{universal enveloping algebra|)}
 \subsektion{Deformations of associative algebras}\index{deformations!of associative
 algebras|idxbf}
 Let $(A,m:A\otimes A\to A)$ be an associative algebra, that is, the linear map $m$
 satisfies the quadratic equation
 \begin{equation}\label{lec08Eq:1}
 m(m(a,b),c)=m(a,m(b,c)). %\tag{$\dag$}
 \end{equation}
 Note that if $\varphi:A\to A$ is a linear automorphism, the multiplication
 $m_\varphi$ given by $m_\varphi(a,b)=\varphi^{-1}(m(\varphi(a),\varphi(b)))$ is also
 associative. We like to think of $m$ and $m_\varphi$ as equivalent associative
 algebra structures on $A$. The ``moduli space'' of associative algebras on the vector
 space $A$ is the set of solutions to equation \ref{lec08Eq:1} modulo this equivalence
 relation.

 One can come up with a notion of deformation for almost any kind of object. In these
 deformation theories, we are interested in some cohomology theories because they
 parameterize obstructions to deformations. The knowledge of the cohomology of a
 given Lie algebra $\g$, enables us say a lot about the deformations of $\g$. We'll
 come back to this question in the next lecture.

  Let us turn to our original example: the family of associative algebras
  $S_\varepsilon(\g)$. Recall that by the PBW theorem, we have an isomorphism of
  filtered vector spaces $S_\varepsilon(\g)\stackrel{\psi}{\to}S(\g)= Pol(\g^*)$, but
  this is not an isomorphisms of associative algebras. Therefore, the multiplication
  defined by $f\ast g:=\psi(\psi^{-1}(f)\cdot \psi^{-1}(g))$ is not the normal
  multiplication on $S(\g)$. We claim that the result is of the form
  \[
     f\ast g = fg + \sum_{n\ge 1}\varepsilon^n m_n(f,g) ,
  \]
  where $m_n$ is a bidifferential operator of order $n$, that is, it is of the form
  \[
  m_n(f,g)=\sum_{I,J}p_n^{I,J} \partial^If\partial^Jg,
  \]
  where $I$ and $J$ are multi-indices of length $n$, and $p_n^{I,J}\in Pol(\g^*)$. The
  idea of the proof is to check this for $f=\psi(e_1^{r_1}\cdots e_n^{r_n}) $ and
  $g=\psi(e_1^{l_1}\cdots e_n^{l_n})$ by writing
  \[
  e_1^{r_1}\cdots e_n^{r_n} \cdot e_1^{l_1}\cdots e_n^{l_n}=e_1^{l_1+r_1}\cdots
  e_n^{l_n+r_n} + \sum_{k\ge 1} \varepsilon^k m_k(e_1^{r_1}\cdots
  e_n^{r_n},e_1^{l_1}\cdots e_n^{l_n})
  \]
  in $S_\varepsilon(\g)$ using the commuting relations.

  \begin{exercise}
   Compute the $p_n^{I,J}$ for the Lie algebra $\g$ generated by $X$, $Y$, and $H$
   with bracket $[X,Y]=H, [H,X]=[H,Y]=0$. This is called the Heisenberg Lie
   algebra.\index{Heisenberg algebra}
   \begin{solution}
     We would like to compute the coefficients of the product $(X^aH^bY^r)(X^sH^cY^d)$
     once it is rewritten in the PBW basis by repeatedly applying the relations
     $XY-YX=\e H$, $HX=XH$, and $HY=YH$. Check by induction that
     \[
        Y^r X^s = \sum_{n=0}^\infty \e^n (-1)^n n! \binom{r}{n}\binom{s}{n}
        X^{r-n}H^lY^{s-n}.
     \]
     It follows that $p_n^{I,J}$ is zero unless $I=(Y,\dots, Y)$ and $J=(X,\dots,X)$,
     in which case $p_n^{I,J}=\frac{(-1)^n}{n!} H^n $.
   \end{solution}
  \end{exercise}

  So we have a family of products on $Pol(\g^*)$ which depend on $\varepsilon$ in the
  following way:
  \[
   f\ast g = fg + \sum_{n\ge 1}\varepsilon^n m_n(f,g)
  \]
  Since $f,g$ are polynomials and $m_n$ is a bidifferential operator of order $n$,
  this series terminates, so it is a polynomial in $\varepsilon$. If we try to extend
  this product to $C^\infty(\g^*)$, then there are questions about the convergence of
  the product $\ast$. There are two ways to deal with this problem. The first one is
  to take these matters of convergence seriously, consider some topology on
  $C^\infty(\g^*)$ and demand that the series converges. The other solution is to
  forget about convergence and just think in terms of formal power series in
  $\varepsilon$. This is the so-called ``formal deformation'' approach. As we shall
  see, there are interesting things to say with this seemingly rudimentary point
  of view.

  \subsektion{Formal deformations of associative algebras} Let $(A,m_0)$ be an
  associative algebra over $\CC$. Then, a formal deformation of $(A,m_0)$ is a
  $\CC[[h]]$-linear map $m:A[[h]]\otimes_{\CC[[h]]} A[[h]] \to A[[h]]$ such that
  \[
   m(a,b) = m_0(a,b) + \sum_{n\ge 1} h^n m_n(a,b)
  \]
  for all $a,b\in A$, and such that $(A[[h]],m)$ is an associative algebra. We say
  that two formal deformations $m$ and $\tilde m$ are equivalent if there is a
  $\CC[[h]]$-automorphism $A[[h]]\xrightarrow{\varphi} A[[h]]$ such that $\tilde m =
  m_\varphi$, with $\varphi(x) = x + \sum_{n\ge 1} h^n \varphi_n(x)$ for all $x\in
  A$, where $\varphi_n$ is an endomorphism of $A$.

  \smallskip
  \noindent \underline{Question}: Describe the equivalence classes of formal
  deformations of a given associative algebra.

   When $(A,m_0)$ is a commutative algebra, the answer is known. Philosophically and
   historically, this case is relevant to quantum mechanics. In classical mechanics,
   observables are smooth functions on a phase space $M$, i.e\ they form a commutative
   associative algebra $C^\infty(M)$. But when you quantize this system (which is
   needed to describe something on the order of the Planck scale), you cannot think of
   observables as functions on phase space anymore. You need to deform the commutative
   algebra $C^\infty(M)$ to a noncommutative algebra. And it works...

   From now on, let $(A,m_0)$ be a commutative associative algebra. Let's write
   $m_0(a,b)=ab$, and $m(a,b)=a\ast b$. (This is called a star product\index{star
   product}, and the terminology goes back to the sixties and the work of J.~Vey).
   Then we have
   \[
    a\ast b = ab + \sum_{n\ge 1} h^n m_n(a,b).
   \]
   Demanding the associativity of $\ast$ imposes an infinite number of equations for
   the $m_n$'s, one for each order:
   \begin{itemize}
   \item[$h^0$:] $a(bc)=(ab)c$
   \item[$h^1$:] $am_1(b,c) + m_1(a,bc) = m_1(a,b)c+m_1(ab,c)$
   \item[$h^2$:] $\dots$ \anton{we should compute this one}
   \item[$\vdots$]
   \end{itemize}

   \begin{exercise}
   Show that the bracket $\{a,b\} = m_1(a,b)-m_1(b,a)$ defines a Poisson structure on
   $A$. This means that we can think of a Poisson structure on an algebra as the
   remnants of a deformed product where $a\ast b-b\ast a = h\{a,b\} + O(h)$.
   \end{exercise}

   One easily checks that if two formal deformations $m$ and $\tilde m$ are equivalent
   via $\varphi$ (i.e: $\tilde m=m_\varphi$), then the associated $m_1,\tilde m_1$ are
   related by $m_1(a,b) = \tilde m_1(a,b)+\varphi_1(ab)-\varphi_1(a)b-a\varphi_1(b)$.
   In particular, two equivalent formal deformations induce the same Poisson
   structure. Also, it is possible to choose a representative in an equivalence class
   such that $m_1$ is skew-symmetric (and then, $m_1(a,b)=\frac{1}{2}\{a,b\}$). This
   leads to the following program for the classification problem:
   \begin{enumerate}
   \item Classify all Poisson structures on $A$.

   \item Given a Poisson algebra $(A,\{\ ,\,\})$, classify all equivalence
   classes of star products on $A$ such that $m_1(a,b)=\frac{1}{2}\{a,b\}$.
   \end{enumerate}

   Under some mild assumption, it can be assumed that a star product is symmetric,
   i.e.\ that it satisfies the equation $m_n(a,b)=(-1)^n m_n(b,a)$ for all $n$. The
   program given above was completed by Maxim Kontsevitch \index{Kontsevitch, Maxim}
   for the algebra of smooth functions on a manifold $M$. Recall that Poisson
   structures on $C^\infty(M)$ are given by bivector fields on $M$ that satisfy the
   Jacobi identity.

  \begin{theorem}[Kontsevich, 1994]
  Let $A$ be the commutative associative algebra $C^\infty(M)$, and let us fix a
  Poisson bracket $\{\ ,\,\}$ on $A$. Equivalence classes of symmetric star products
  on $A$ with $m_1(a,b)=\frac{1}{2}\{a,b\}$ are in bijection with formal deformations
  of $\{\ ,\,\}$ modulo formal diffeomorphisms of $M$.
  \end{theorem}
  A \emph{formal deformation} of $\{\ ,\,\}$ is a Poisson bracket $\{\ ,\,\}_h$ on $A[[h]]$
  such that
  \[
    \{a,b\}_h = \{a,b\} + \sum_{n\ge 1} h^n \mu_n(a,b)
  \]
  for all $a,b$ in $A$. A \emph{formal diffeomorphism} of $M$ is an automorphism
  $\varphi$ of $A[[h]]$ such that $\varphi(f) = f+\sum_{n\ge 1} h^n \varphi_n(f)$
  and $\varphi(fg)=\varphi(f)\varphi(g)$ for all $f,g$ in $A$.

  We won't prove the theorem (it would take about a month) \anton{find a reference}.
  As Poisson algebras are Lie algebras, it relates deformations of associative
  algebras to deformations of Lie algebras.

  \subsektion{Formal deformations of Lie algebras}
  Given a Lie algebra $(\g, [\ ,\,])$, you want to know how many formal deformations
  of $\g$ there are. Sometimes, there are none (like in the case of $\sl_n$, as we
  will see later). Sometimes, there are plenty (as for triangular matrices). The
  goal is now to construct some invariants of Lie algebras which will tell you whether
  there are deformations, and how many of them there are. In order to do this, we
  should consider cohomology theories for Lie algebras. We will focus first on the
  standard complex $C^\udot (\g,\g) = \bigoplus_{n\ge 0} C^n(\g,\g)$, where
  $C^n(\g,\g) = \hom(\Lambda^n \g,\g)$.
